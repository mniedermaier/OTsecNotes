% ============================================================================
%  Introduction to IEC 62443 - OT Security Learning Resource
% ============================================================================

\documentclass[11pt,a4paper]{article}
\usepackage{otsec-template}

\hypersetup{
    pdftitle={Introduction to IEC 62443},
    pdfsubject={Industrial Cybersecurity Standards},
}

\begin{document}

% ----------------------------------------------------------------------------
%  TITLE PAGE
% ----------------------------------------------------------------------------

\maketitlepage
    {Introduction to IEC 62443}
    {The International Standard for Industrial Cybersecurity}
    {OT Security Learning Series}
    {Document 110 \quad|\quad January 2026}
    {Matthias Niedermaier}

% ----------------------------------------------------------------------------
%  TABLE OF CONTENTS
% ----------------------------------------------------------------------------

\tableofcontents
\newpage

% ----------------------------------------------------------------------------
%  INTRODUCTION
% ----------------------------------------------------------------------------

\section{What is IEC 62443?}

\textbf{IEC 62443} is the international standard series for cybersecurity in Industrial Automation and Control Systems (IACS). Developed by the International Electrotechnical Commission (IEC) in collaboration with ISA (International Society of Automation), it provides a comprehensive framework for securing industrial systems.

\begin{infobox}
IEC 62443 is often referred to as \textbf{ISA/IEC 62443} because it originated from the ISA99 committee's work. The standards are technically equivalent -- ISA-62443 and IEC 62443 can be used interchangeably.
\end{infobox}

\subsection{Why IEC 62443?}

Traditional IT security standards (like ISO 27001) don't fully address the unique requirements of OT environments:

\begin{itemize}
    \item \textbf{Availability over Confidentiality:} In OT, system uptime is critical
    \item \textbf{Safety Requirements:} Industrial systems can cause physical harm
    \item \textbf{Legacy Systems:} 20+ year old equipment can't be easily patched
    \item \textbf{Real-time Constraints:} Security can't introduce latency
    \item \textbf{Different Lifecycles:} OT systems run for decades, not years
\end{itemize}

\begin{successbox}
IEC 62443 provides a \textbf{risk-based approach} that balances security requirements with operational needs, covering the entire lifecycle from design to decommissioning.
\end{successbox}

% ----------------------------------------------------------------------------
%  STRUCTURE
% ----------------------------------------------------------------------------

\section{Standard Structure}

IEC 62443 is organized into four main series, each addressing different stakeholders and aspects of industrial cybersecurity.

\subsection{The Four Series}

\begin{center}
\begin{tikzpicture}[
    series/.style={
        rectangle,
        rounded corners=5pt,
        minimum width=3.5cm,
        minimum height=2cm,
        text=white,
        font=\small\bfseries,
        align=center,
        text width=3.2cm
    }
]

\node[series, fill=primary] (s1) at (0,0) {62443-1-x\\[0.3em]\footnotesize General};
\node[series, fill=secondary] (s2) at (4.5,0) {62443-2-x\\[0.3em]\footnotesize Policies \&\\Procedures};
\node[series, fill=accent] (s3) at (9,0) {62443-3-x\\[0.3em]\footnotesize System};
\node[series, fill=zone3] (s4) at (13.5,0) {62443-4-x\\[0.3em]\footnotesize Component};

\node[font=\scriptsize, text=darkgray, align=center, text width=3.5cm] at (0,-1.8) {Concepts, models,\\terminology};
\node[font=\scriptsize, text=darkgray, align=center, text width=3.5cm] at (4.5,-1.8) {Asset owners,\\service providers};
\node[font=\scriptsize, text=darkgray, align=center, text width=3.5cm] at (9,-1.8) {System integrators,\\architects};
\node[font=\scriptsize, text=darkgray, align=center, text width=3.5cm] at (13.5,-1.8) {Product vendors,\\developers};

\end{tikzpicture}
\end{center}

\subsection{Key Documents}

\begin{definitionbox}{Most Referenced Standards}
\begin{description}[leftmargin=!,labelwidth=2.5cm]
    \item[62443-1-1] Terminology, concepts, and models
    \item[62443-2-1] Security program requirements for asset owners
    \item[62443-2-4] Security program requirements for service providers
    \item[62443-3-2] Security risk assessment and zone/conduit design
    \item[62443-3-3] System security requirements and security levels
    \item[62443-4-1] Secure product development lifecycle
    \item[62443-4-2] Technical security requirements for components
\end{description}
\end{definitionbox}

% ----------------------------------------------------------------------------
%  SECURITY LEVELS
% ----------------------------------------------------------------------------

\section{Security Levels (SL)}

One of the most important concepts in IEC 62443 is the \textbf{Security Level} (SL). Security levels define the degree of protection required against different threat actors.

\subsection{The Four Security Levels}

\begin{conceptbox}{\slone\ Security Level 1 -- Casual/Coincidental}
\textbf{Threat Actor:} Unintentional errors, accidental exposure

\textbf{Protection Against:}
\begin{itemize}
    \item Accidental or unintentional violations
    \item Casual exploration by curious individuals
    \item Basic automated tools and scripts
\end{itemize}

\textbf{Example:} Employee accidentally clicking a phishing link
\end{conceptbox}

\vspace{0.5em}

\begin{conceptbox}{\sltwo\ Security Level 2 -- Intentional/Simple Means}
\textbf{Threat Actor:} Low motivation, limited resources, general skills

\textbf{Protection Against:}
\begin{itemize}
    \item Intentional attacks using simple techniques
    \item Low-resource hackers and script kiddies
    \item Common malware and known exploits
\end{itemize}

\textbf{Example:} Opportunistic attacker using publicly available tools
\end{conceptbox}

\vspace{0.5em}

\begin{conceptbox}{\slthree\ Security Level 3 -- Intentional/Sophisticated Means}
\textbf{Threat Actor:} Moderate motivation, sophisticated tools, IACS-specific skills

\textbf{Protection Against:}
\begin{itemize}
    \item Sophisticated attacks with IACS knowledge
    \item Organized crime groups
    \item Targeted attacks on specific organizations
\end{itemize}

\textbf{Example:} Targeted ransomware attack on industrial facility
\end{conceptbox}

\vspace{0.5em}

\begin{conceptbox}{\slfour\ Security Level 4 -- Intentional/State-Sponsored}
\textbf{Threat Actor:} High motivation, extended resources, nation-state capabilities

\textbf{Protection Against:}
\begin{itemize}
    \item Nation-state actors and APT groups
    \item Extended campaigns with significant resources
    \item Zero-day exploits and custom malware
\end{itemize}

\textbf{Example:} Stuxnet-style targeted attack on critical infrastructure
\end{conceptbox}

\subsection{Security Level Types}

IEC 62443 defines three types of security levels:

\begin{center}
\rowcolors{2}{lightgray}{white}
\begin{tabular}{p{3cm} p{9cm}}
\rowcolor{primary}
\textcolor{white}{\bfseries Type} & \textcolor{white}{\bfseries Description} \\
\midrule
\textbf{SL-T} (Target) & The desired security level based on risk assessment \\
\textbf{SL-A} (Achieved) & The actual security level measured/tested \\
\textbf{SL-C} (Capability) & The maximum level a component/system can achieve \\
\end{tabular}
\end{center}

\begin{warningbox}
The goal is to ensure: \textbf{SL-A $\geq$ SL-T}

If your achieved security level is lower than your target, you have a security gap that must be addressed through compensating controls or system upgrades.
\end{warningbox}

% ----------------------------------------------------------------------------
%  ZONES AND CONDUITS
% ----------------------------------------------------------------------------

\section{Zones and Conduits}

IEC 62443 uses the concept of \textbf{zones} and \textbf{conduits} for network segmentation and risk management.

\subsection{Zones}

\begin{definitionbox}{Zone Definition}
A \textbf{zone} is a logical or physical grouping of assets that share common security requirements based on:
\begin{itemize}
    \item Criticality and consequence of compromise
    \item Required security level
    \item Physical or logical location
    \item Responsible organization
\end{itemize}
\end{definitionbox}

Each zone has a single \textbf{target security level} (SL-T). Assets within a zone should have similar security requirements.

\subsection{Conduits}

\begin{definitionbox}{Conduit Definition}
A \textbf{conduit} is a logical grouping of communication channels that share common security requirements and connect two or more zones.
\end{definitionbox}

Conduits must provide security controls appropriate to protect both connected zones. The conduit's security level should match or exceed the highest SL-T of the connected zones.

\subsection{Zone and Conduit Diagram}

\begin{center}
\begin{tikzpicture}[
    zone/.style={
        rectangle,
        rounded corners=8pt,
        minimum width=3cm,
        minimum height=2cm,
        draw=#1,
        line width=2pt,
        fill=#1!10
    },
    conduit/.style={
        ->,
        line width=3pt,
        color=accent
    }
]

% Zones
\node[zone=zone4] (enterprise) at (0,0) {\begin{tabular}{c}\textbf{Enterprise}\\SL-T: 2\end{tabular}};
\node[zone=zone35] (dmz) at (5,0) {\begin{tabular}{c}\textbf{DMZ}\\SL-T: 3\end{tabular}};
\node[zone=zone2] (control) at (10,0) {\begin{tabular}{c}\textbf{Control}\\SL-T: 3\end{tabular}};
\node[zone=zone0] (safety) at (10,-3) {\begin{tabular}{c}\textbf{Safety}\\SL-T: 4\end{tabular}};

% Conduits
\draw[conduit] (enterprise) -- node[above, font=\scriptsize, text=darkgray] {C1} (dmz);
\draw[conduit] (dmz) -- node[above, font=\scriptsize, text=darkgray] {C2} (control);
\draw[conduit] (control) -- node[right, font=\scriptsize, text=darkgray] {C3} (safety);

\end{tikzpicture}
\end{center}

% ----------------------------------------------------------------------------
%  FOUNDATIONAL REQUIREMENTS
% ----------------------------------------------------------------------------

\section{Foundational Requirements (FR)}

IEC 62443-3-3 defines seven \textbf{Foundational Requirements} (FRs) that form the basis of all security controls.

\begin{center}
\rowcolors{2}{lightgray}{white}
\begin{tabular}{c p{4cm} p{7cm}}
\rowcolor{primary}
\textcolor{white}{\bfseries FR} & \textcolor{white}{\bfseries Name} & \textcolor{white}{\bfseries Description} \\
\midrule
\textbf{FR 1} & Identification \& Authentication & Control who and what can access the system \\
\textbf{FR 2} & Use Control & Control what authenticated users can do \\
\textbf{FR 3} & System Integrity & Ensure the system operates correctly \\
\textbf{FR 4} & Data Confidentiality & Protect sensitive data from disclosure \\
\textbf{FR 5} & Restricted Data Flow & Segment networks and control data flow \\
\textbf{FR 6} & Timely Response & Respond to security violations \\
\textbf{FR 7} & Resource Availability & Ensure system availability against DoS \\
\end{tabular}
\end{center}

Each FR contains multiple \textbf{System Requirements} (SRs) and \textbf{Requirement Enhancements} (REs) that specify detailed controls for each security level.

\begin{tipbox}
When implementing IEC 62443, start with the FRs and map them to your existing controls. This gap analysis helps prioritize security improvements.
\end{tipbox}

% ----------------------------------------------------------------------------
%  ROLES AND RESPONSIBILITIES
% ----------------------------------------------------------------------------

\section{Roles and Responsibilities}

IEC 62443 defines clear responsibilities for different stakeholders:

\subsection{Asset Owner}

\begin{itemize}
    \item Defines security requirements (SL-T) based on risk assessment
    \item Implements and maintains security program (62443-2-1)
    \item Responsible for overall OT security governance
    \item Verifies that SL-A meets SL-T
\end{itemize}

\subsection{System Integrator}

\begin{itemize}
    \item Designs systems to meet SL-T requirements
    \item Implements zones, conduits, and security controls
    \item Follows secure integration practices (62443-2-4)
    \item Documents achieved security level (SL-A)
\end{itemize}

\subsection{Product Vendor}

\begin{itemize}
    \item Develops products following secure lifecycle (62443-4-1)
    \item Documents product security capabilities (SL-C)
    \item Provides security patches and updates
    \item Certifies products against 62443-4-2
\end{itemize}

% ----------------------------------------------------------------------------
%  CERTIFICATION
% ----------------------------------------------------------------------------

\section{Certification}

IEC 62443 certification is offered by several organizations including TUV, Exida, and ISASecure.

\subsection{Certification Types}

\begin{itemize}
    \item \textbf{Component Certification} (62443-4-2): Individual products
    \item \textbf{SDLC Certification} (62443-4-1): Development processes
    \item \textbf{System Certification} (62443-3-3): Complete systems
    \item \textbf{Capability Certification}: Organization processes
\end{itemize}

\begin{infobox}
Certification provides independent verification of security claims but is not mandatory. Many organizations use IEC 62443 as a framework without formal certification.
\end{infobox}

% ----------------------------------------------------------------------------
%  SUMMARY
% ----------------------------------------------------------------------------

\section{Summary}

\begin{definitionbox}{Key Takeaways}
\begin{description}[leftmargin=!,labelwidth=3cm]
    \item[IEC 62443] Comprehensive standard for industrial cybersecurity
    \item[Security Levels] SL 1-4 define protection against threat actors
    \item[Zones/Conduits] Network segmentation methodology
    \item[7 FRs] Foundational requirements covering all security aspects
    \item[Stakeholders] Asset owners, integrators, and vendors have defined roles
\end{description}
\end{definitionbox}

\begin{successbox}
IEC 62443 provides a \textbf{common language} for discussing OT security requirements between asset owners, integrators, and vendors. Even partial adoption improves security posture.
\end{successbox}

% ----------------------------------------------------------------------------
%  NEXT STEPS
% ----------------------------------------------------------------------------

\section{Next Steps}

To start implementing IEC 62443 in your organization:

\begin{enumerate}
    \item \textbf{Inventory:} Document all IACS assets and their criticality
    \item \textbf{Risk Assessment:} Determine SL-T for each zone (62443-3-2)
    \item \textbf{Gap Analysis:} Compare current state (SL-A) to target (SL-T)
    \item \textbf{Roadmap:} Prioritize improvements based on risk
    \item \textbf{Procurement:} Require 62443-4-2 compliance for new products
\end{enumerate}

% ----------------------------------------------------------------------------
%  FURTHER READING
% ----------------------------------------------------------------------------

\section{Further Reading}

\subsection*{Standards}
\begin{itemize}
    \item \textbf{ISA/IEC 62443 Series} -- Complete Standards Collection\\
          \url{https://www.isa.org/standards-and-publications/isa-standards/isa-iec-62443-series-of-standards}
    \item \textbf{IEC 62443-2-1} -- Security Program Requirements for Asset Owners
    \item \textbf{IEC 62443-3-3} -- System Security Requirements and Security Levels
    \item \textbf{IEC 62443-4-2} -- Technical Security Requirements for IACS Components
\end{itemize}

\subsection*{Resources}
\begin{itemize}
    \item \textbf{NIST SP 800-82 Rev. 3} -- Guide to OT Security\\
          \url{https://csrc.nist.gov/publications/detail/sp/800-82/rev-3/final}
    \item \textbf{CISA ICS-CERT} -- Industrial Control Systems Advisories\\
          \url{https://www.cisa.gov/news-events/ics-advisories}
    \item \textbf{ISASecure} -- IEC 62443 Certification Information\\
          \url{https://isasecure.org/}
\end{itemize}

\subsection*{Books}
\begin{itemize}
    \item Knapp, E. \& Langill, J. -- \textit{Industrial Network Security} (Syngress)
    \item Macaulay, T. \& Singer, B. -- \textit{Cybersecurity for Industrial Control Systems} (CRC Press)
\end{itemize}

\vfill
\begin{center}
\textcolor{mediumgray}{\rule{0.5\textwidth}{0.5pt}}\\[1em]
\textcolor{mediumgray}{\small Part of the OT Security Learning Series}
\end{center}

\end{document}
