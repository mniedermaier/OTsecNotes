% ============================================================================
%  The Purdue Model - OT Security Learning Resource
% ============================================================================

\documentclass[11pt,a4paper]{article}
\usepackage{otsec-template}

\hypersetup{
    pdftitle={The Purdue Model},
    pdfsubject={Industrial Network Architecture},
}

\begin{document}

% ----------------------------------------------------------------------------
%  TITLE PAGE
% ----------------------------------------------------------------------------

\maketitlepage
    {The Purdue Model}
    {Understanding Industrial Network Architecture and Segmentation}
    {OT Security Learning Series}
    {Document 010 \quad|\quad January 2026}
    {Matthias Niedermaier}

% ----------------------------------------------------------------------------
%  TABLE OF CONTENTS
% ----------------------------------------------------------------------------

\tableofcontents
\newpage

% ----------------------------------------------------------------------------
%  INTRODUCTION
% ----------------------------------------------------------------------------

\section{Introduction}

The \textbf{Purdue Enterprise Reference Architecture} (PERA), commonly known as the \textbf{Purdue Model}, is a reference model for industrial control system (ICS) network segmentation. Originally developed in the 1990s at Purdue University, it has become the de facto standard for designing secure OT network architectures.

\begin{infobox}
The Purdue Model provides a hierarchical framework that separates industrial networks into distinct levels, each with specific functions and security requirements. This separation is fundamental to implementing defense-in-depth strategies in OT environments.
\end{infobox}

\subsection{Why the Purdue Model Matters}

In modern industrial environments, the convergence of IT and OT systems creates significant security challenges:

\begin{itemize}
    \item Legacy OT systems were designed for reliability, not security
    \item Flat networks allow lateral movement during attacks
    \item IT-based threats can now reach physical processes
    \item Regulatory frameworks (IEC 62443, NIST) reference this model
\end{itemize}

\begin{successbox}
The Purdue Model helps organizations understand \textbf{where} security controls should be placed and \textbf{what} traffic should flow between different parts of the industrial network.
\end{successbox}

% ----------------------------------------------------------------------------
%  THE LEVELS
% ----------------------------------------------------------------------------

\section{The Purdue Model Levels}

The model defines six primary levels (0-5), plus a demilitarized zone (DMZ) between the OT and IT networks.

\subsection{Level 0 -- Physical Process}

\begin{conceptbox}{\zonezero\ Physical Process}
\textbf{Function:} The actual physical equipment and processes

\textbf{Components:}
\begin{itemize}
    \item Sensors (temperature, pressure, flow, level)
    \item Actuators (valves, motors, pumps)
    \item Physical machinery and production equipment
\end{itemize}

\textbf{Security Considerations:}
\begin{itemize}
    \item No network connectivity at this level
    \item Physical security is paramount
    \item Tampering detection mechanisms
\end{itemize}
\end{conceptbox}

\subsection{Level 1 -- Basic Control}

\begin{conceptbox}{\zoneone\ Basic Control}
\textbf{Function:} Direct control of the physical process

\textbf{Components:}
\begin{itemize}
    \item Programmable Logic Controllers (PLCs)
    \item Remote Terminal Units (RTUs)
    \item Intelligent Electronic Devices (IEDs)
    \item Variable Frequency Drives (VFDs)
\end{itemize}

\textbf{Protocols:} Modbus, PROFINET, EtherNet/IP, DNP3

\textbf{Security Considerations:}
\begin{itemize}
    \item Often lacks authentication mechanisms
    \item Firmware updates require careful planning
    \item Network isolation is critical
\end{itemize}
\end{conceptbox}

\subsection{Level 2 -- Area Supervisory Control}

\begin{conceptbox}{\zonetwo\ Area Supervisory Control}
\textbf{Function:} Supervising and controlling the physical process

\textbf{Components:}
\begin{itemize}
    \item Human-Machine Interfaces (HMIs)
    \item SCADA systems
    \item Engineering Workstations
    \item Local operator consoles
\end{itemize}

\textbf{Security Considerations:}
\begin{itemize}
    \item User authentication required
    \item Application whitelisting recommended
    \item USB and removable media controls
\end{itemize}
\end{conceptbox}

\subsection{Level 3 -- Site Operations}

\begin{conceptbox}{\zonethree\ Site Operations}
\textbf{Function:} Site-wide monitoring, optimization, and data collection

\textbf{Components:}
\begin{itemize}
    \item Data Historians
    \item OPC servers
    \item Batch management systems
    \item Manufacturing Execution Systems (MES)
    \item Asset management systems
\end{itemize}

\textbf{Security Considerations:}
\begin{itemize}
    \item Database security and access controls
    \item Secure remote access configuration
    \item Integration point with business systems
\end{itemize}
\end{conceptbox}

\subsection{Level 3.5 -- Industrial DMZ}

\begin{conceptbox}{\zonedmz\ Industrial Demilitarized Zone}
\textbf{Function:} Buffer zone between OT and IT networks

\textbf{Components:}
\begin{itemize}
    \item Data diodes (unidirectional gateways)
    \item Jump servers / Bastion hosts
    \item Patch management servers
    \item Antivirus update servers
    \item Remote access gateways
\end{itemize}

\textbf{Security Considerations:}
\begin{itemize}
    \item No direct connectivity between IT and OT
    \item All traffic must be proxied through DMZ
    \item Strict firewall rules on both sides
\end{itemize}
\end{conceptbox}

\begin{warningbox}
The DMZ is \textbf{critical} for protecting OT networks. Never allow direct connections from Level 4/5 to Levels 0-3. All data exchange should pass through DMZ services.
\end{warningbox}

\subsection{Levels 4 \& 5 -- Enterprise}

\begin{conceptbox}{\zonefour\ Enterprise Network}
\textbf{Function:} Business operations and enterprise IT

\textbf{Level 4 -- Site Business:}
\begin{itemize}
    \item Site email and intranet
    \item ERP system interfaces
    \item Business reporting
\end{itemize}

\textbf{Level 5 -- Enterprise Network:}
\begin{itemize}
    \item Corporate network
    \item Cloud services
    \item Internet connectivity
    \item External business partners
\end{itemize}
\end{conceptbox}

% ----------------------------------------------------------------------------
%  VISUAL DIAGRAM
% ----------------------------------------------------------------------------

\section{Purdue Model Diagram}

\begin{center}
\begin{tikzpicture}[
    scale=0.85,
    every node/.style={transform shape},
    level/.style={
        rectangle,
        rounded corners=3pt,
        minimum width=9cm,
        minimum height=0.9cm,
        text=white,
        font=\small\bfseries,
        align=center
    },
    label/.style={
        font=\scriptsize,
        text=darkgray
    }
]

% Levels
\node[level, fill=zone4] (l5) at (0,0) {Level 5: Enterprise Network};
\node[level, fill=zone4!80] (l4) at (0,-1.2) {Level 4: Site Business Planning};

\node[level, fill=zone35, minimum height=0.7cm] (dmz) at (0,-2.4) {DMZ (Level 3.5)};

\node[level, fill=zone3] (l3) at (0,-3.6) {Level 3: Site Operations};
\node[level, fill=zone2] (l2) at (0,-4.8) {Level 2: Area Supervisory Control};
\node[level, fill=zone1] (l1) at (0,-6) {Level 1: Basic Control};
\node[level, fill=zone0] (l0) at (0,-7.2) {Level 0: Physical Process};

% Labels on right
\node[label, anchor=west] at (5,0) {Corporate IT, Internet};
\node[label, anchor=west] at (5,-1.2) {ERP, Email, Business Apps};
\node[label, anchor=west] at (5,-2.4) {Data Diodes, Jump Servers};
\node[label, anchor=west] at (5,-3.6) {Historians, MES, OPC};
\node[label, anchor=west] at (5,-4.8) {HMI, SCADA, Eng. Workstations};
\node[label, anchor=west] at (5,-6) {PLCs, RTUs, Controllers};
\node[label, anchor=west] at (5,-7.2) {Sensors, Actuators, Motors};

% Zone labels on left
\node[font=\scriptsize\bfseries, text=zone4, anchor=east] at (-5,-0.6) {IT ZONE};
\node[font=\scriptsize\bfseries, text=zone35, anchor=east] at (-5,-2.4) {BUFFER};
\node[font=\scriptsize\bfseries, text=zone1, anchor=east] at (-5,-5.4) {OT ZONE};

% Firewall indicators
\draw[red, line width=1.5pt, dashed] (-4.5,-1.75) -- (4.5,-1.75);
\draw[red, line width=1.5pt, dashed] (-4.5,-3.05) -- (4.5,-3.05);
\node[font=\tiny, text=red] at (4,-1.75) {Firewall};
\node[font=\tiny, text=red] at (4,-3.05) {Firewall};

\end{tikzpicture}
\end{center}

% ----------------------------------------------------------------------------
%  IMPLEMENTATION
% ----------------------------------------------------------------------------

\section{Implementing the Purdue Model}

\subsection{Key Principles}

\begin{enumerate}
    \item \textbf{Network Segmentation:} Each level should be on separate network segments/VLANs
    \item \textbf{Traffic Control:} Firewalls between levels with strict rule sets
    \item \textbf{Data Flow:} Information flows up; commands flow down
    \item \textbf{No Bypass:} Never skip levels (e.g., no direct Level 5 to Level 1 connection)
\end{enumerate}

\subsection{Common Mistakes to Avoid}

\begin{dangerbox}
\textbf{Flat Networks:} Many legacy OT environments have flat networks where all devices can communicate directly. This allows attackers to move laterally from compromised IT systems directly to PLCs.
\end{dangerbox}

\begin{warningbox}
\textbf{Direct Remote Access:} Allowing VPN connections directly into Level 2 or below bypasses the DMZ protection and creates significant risk.
\end{warningbox}

\subsection{Firewall Rules Example}

Basic firewall policy between DMZ and Level 3:

\begin{itemize}
    \item \textbf{Allow:} Historian replication (specific ports, specific hosts)
    \item \textbf{Allow:} Patch downloads from DMZ server to Level 3
    \item \textbf{Deny:} All inbound connections from IT to OT
    \item \textbf{Deny:} Direct database queries from IT
    \item \textbf{Log:} All denied traffic for analysis
\end{itemize}

% ----------------------------------------------------------------------------
%  SUMMARY
% ----------------------------------------------------------------------------

\section{Summary}

\begin{definitionbox}{Key Takeaways}
\begin{description}[leftmargin=!,labelwidth=2.5cm]
    \item[Level 0] Physical process -- sensors and actuators
    \item[Level 1] Basic control -- PLCs, RTUs
    \item[Level 2] Supervisory -- HMI, SCADA
    \item[Level 3] Operations -- Historians, MES
    \item[DMZ] Buffer zone -- data diodes, jump servers
    \item[Level 4-5] Enterprise -- IT and business systems
\end{description}
\end{definitionbox}

\begin{tipbox}
When assessing an OT environment, start by mapping the existing network to the Purdue Model. This helps identify gaps in segmentation and areas where security controls are missing.
\end{tipbox}

% ----------------------------------------------------------------------------
%  REFERENCES
% ----------------------------------------------------------------------------

\section{Further Reading}

\subsection*{Standards and Guidelines}
\begin{itemize}
    \item \textbf{NIST SP 800-82 Rev. 3} -- Guide to OT Security\\
          \url{https://csrc.nist.gov/publications/detail/sp/800-82/rev-3/final}
    \item \textbf{IEC 62443 Series} -- Industrial Automation and Control Systems Security\\
          \url{https://www.isa.org/standards-and-publications/isa-standards/isa-iec-62443-series-of-standards}
    \item \textbf{ISA-95 / IEC 62264} -- Enterprise-Control System Integration\\
          \url{https://www.isa.org/isa95}
\end{itemize}

\subsection*{Resources}
\begin{itemize}
    \item \textbf{CISA} -- ICS Security Recommended Practices\\
          \url{https://www.cisa.gov/topics/industrial-control-systems}
    \item \textbf{SANS ICS} -- Industrial Control Systems Security Resources\\
          \url{https://www.sans.org/industrial-control-systems-security/}
    \item \textbf{Purdue Enterprise Reference Architecture} -- Original Williams Paper (1992)\\
          Reference: Williams, T.J. "The Purdue Enterprise Reference Architecture"
\end{itemize}

\subsection*{Books}
\begin{itemize}
    \item Knapp, E. \& Langill, J. -- \textit{Industrial Network Security} (Syngress)
    \item Macaulay, T. \& Singer, B. -- \textit{Cybersecurity for Industrial Control Systems} (CRC Press)
\end{itemize}

\vfill
\begin{center}
\textcolor{mediumgray}{\rule{0.5\textwidth}{0.5pt}}\\[1em]
\textcolor{mediumgray}{\small Part of the OT Security Learning Series}
\end{center}

\end{document}
