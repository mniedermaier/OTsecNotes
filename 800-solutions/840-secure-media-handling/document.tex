% ============================================================================
%  840-secure-media-handling - OT Security Learning Resource
% ============================================================================

\documentclass[11pt,a4paper]{article}
\usepackage{otsec-template}

% Define colors for TikZ (matching template colors)
\colorlet{otprimary}{primary}
\colorlet{otaccent}{accent}
\colorlet{otsuccess}{success}
\colorlet{otwarning}{warning}
\colorlet{otdanger}{danger}
\colorlet{otinfo}{info}

\begin{document}

\maketitlepage
    {Secure Removable Media Handling}
    {Controlling USB drives, optical media, and portable storage in OT environments}
    {OT Security Learning Series}
    {Document 840 \quad|\quad January 2026}
    {Matthias Niedermaier}

\tableofcontents
\newpage

% ============================================================================
\section{Introduction}
% ============================================================================

\begin{infobox}
Removable media -- USB drives, CD-ROMs, and other portable storage -- represent one of the most significant attack vectors for OT environments. Stuxnet spread via USB drives, and many ransomware incidents trace back to infected portable media crossing the air gap.
\end{infobox}

Despite the risks, removable media remains necessary in OT environments for:
\begin{itemize}
    \item Vendor software and firmware updates
    \item PLC program transfers
    \item Diagnostic tool delivery
    \item Data export for analysis
    \item Emergency recovery procedures
\end{itemize}

This document covers secure handling procedures for all types of removable media in industrial environments.

% ============================================================================
\section{Media Types and Risks}
% ============================================================================

\begin{table}[h]
\centering
\begin{tabular}{|l|l|l|}
\hline
\textbf{Media Type} & \textbf{Common Use} & \textbf{Risk Level} \\
\hline
USB Flash Drives & General file transfer & \riskhigh \\
External HDDs/SSDs & Large data transfer, backups & \riskhigh \\
SD/MicroSD Cards & Embedded systems, cameras & \riskmedium \\
CD/DVD-ROM & Software distribution & \riskmedium \\
Floppy Disks & Legacy systems & \riskmedium \\
Memory Cards (CF, etc.) & PLCs, industrial cameras & \riskmedium \\
\hline
\end{tabular}
\caption{Removable Media Types in OT Environments}
\end{table}

\begin{dangerbox}
\textbf{USB Weaponization:} Attackers can weaponize USB devices beyond just storing malware:
\begin{itemize}
    \item \textbf{BadUSB} -- Firmware-level attacks that make drives appear as keyboards
    \item \textbf{USB Killers} -- Devices that physically destroy ports via electrical surge
    \item \textbf{Rubber Ducky} -- Keystroke injection devices disguised as flash drives
    \item \textbf{Data exfiltration} -- Covert channels via modified firmware
\end{itemize}
\end{dangerbox}

% ============================================================================
\section{Scanning Kiosk Architecture}
% ============================================================================

\begin{figure}[h]
\centering
\begin{tikzpicture}[
    zone/.style={rectangle, draw, thick, minimum height=2.5cm, minimum width=3cm, rounded corners=5pt},
    kiosk/.style={rectangle, draw, thick, fill=otwarning!30, minimum height=1.5cm, minimum width=2.5cm, rounded corners=3pt, font=\small\bfseries},
    arrow/.style={->, very thick, >=stealth}
]

% Zones
\node[zone, fill=otdanger!10] (untrusted) at (-4,0) {};
\node[font=\scriptsize\bfseries, otdanger] at (-4,1.6) {Untrusted Media};
\node[font=\tiny, text width=2.5cm, align=center] at (-4,0) {Vendor USB\\Personal drives\\Unknown media};

\node[zone, fill=otwarning!10] (dmz) at (0,0) {};
\node[font=\scriptsize\bfseries, otwarning] at (0,1.6) {Scanning Station};

\node[zone, fill=otsuccess!10] (ot) at (4,0) {};
\node[font=\scriptsize\bfseries, otsuccess] at (4,1.6) {OT Environment};
\node[font=\tiny, text width=2.5cm, align=center] at (4,0) {PLCs\\HMIs\\Engineering\\Workstations};

% Kiosk in DMZ
\node[kiosk] (kiosk) at (0,0) {Scan Kiosk};

% Arrows
\draw[arrow, otdanger] (-2.3,0) -- (-1.5,0);
\draw[arrow, otsuccess] (1.5,0) -- (2.3,0);

% Labels
\node[font=\tiny, otwarning] at (0,-0.6) {\faIcon{shield-alt} Multi-AV};
\node[font=\tiny, otwarning] at (0,-0.95) {\faIcon{file-alt} File Check};

\end{tikzpicture}
\caption{Media Scanning Kiosk Architecture}
\end{figure}

\subsection{Kiosk Requirements}

\begin{definitionbox}{Scanning Kiosk Specifications}
\begin{itemize}
    \item \textbf{Isolation} -- No network connection to OT or corporate networks
    \item \textbf{Multiple AV engines} -- At least 2-3 different vendors
    \item \textbf{Regular updates} -- Signature updates via secure, one-way channel
    \item \textbf{File type filtering} -- Block executables, scripts by default
    \item \textbf{Logging} -- Record all scans, results, and user actions
    \item \textbf{Regular reimaging} -- Weekly or after any detection
\end{itemize}
\end{definitionbox}

\subsection{Physical Placement}

\begin{itemize}
    \item Located at controlled entry points to OT areas
    \item Visible to security cameras
    \item Secure storage for clean media nearby
    \item Clear signage with procedures
\end{itemize}

% ============================================================================
\section{Media Handling Procedures}
% ============================================================================

\begin{figure}[h]
\centering
\begin{tikzpicture}[
    procstep/.style={rectangle, draw, thick, fill=otaccent!20, minimum height=1cm, minimum width=2cm, rounded corners=3pt, font=\tiny\bfseries, align=center},
    arrow/.style={->, thick, >=stealth}
]

\node[procstep] (receive) at (0,0) {1. Receive\\Media};
\node[procstep] (register) at (2.5,0) {2. Register\\in Log};
\node[procstep] (scan) at (5,0) {3. Scan at\\Kiosk};
\node[procstep] (transfer) at (7.5,0) {4. Transfer\\to Clean};
\node[procstep] (use) at (10,0) {5. Use in\\OT};

\draw[arrow, otprimary] (receive) -- (register);
\draw[arrow, otprimary] (register) -- (scan);
\draw[arrow, otprimary] (scan) -- (transfer);
\draw[arrow, otprimary] (transfer) -- (use);

\end{tikzpicture}
\caption{Media Handling Process Flow}
\end{figure}

\subsection{Step-by-Step Procedure}

\begin{enumerate}
    \item \textbf{Receive and register} -- Log media source, owner, purpose, date
    \item \textbf{Visual inspection} -- Check for physical tampering or damage
    \item \textbf{Scan at kiosk} -- Full scan with multiple AV engines
    \item \textbf{Review results} -- Verify clean scan, check file types
    \item \textbf{Transfer to clean media} -- Copy approved files to company-owned media
    \item \textbf{Secure original} -- Store or return original media
    \item \textbf{Deploy to OT} -- Use clean media in target system
    \item \textbf{Post-use handling} -- Wipe or secure-store media after use
\end{enumerate}

\begin{warningbox}
\textbf{Never} insert untrusted media directly into OT systems. Even ``quick checks'' bypass security controls and have caused major incidents.
\end{warningbox}

% ============================================================================
\section{USB Port Controls}
% ============================================================================

\subsection{Disabling USB Ports}

\begin{table}[h]
\centering
\begin{tabular}{|l|l|l|}
\hline
\textbf{Method} & \textbf{Effectiveness} & \textbf{Notes} \\
\hline
BIOS/UEFI disable & High & Requires physical access to re-enable \\
Group Policy (Windows) & Medium & Can be bypassed with admin rights \\
Endpoint protection & Medium-High & Allows granular control \\
Physical port blockers & High & Visible deterrent, tamper-evident \\
Epoxy/hot glue & Very High & Permanent, not recommended \\
\hline
\end{tabular}
\caption{USB Port Control Methods}
\end{table}

\subsection{Selective USB Access}

For systems requiring occasional USB access:
\begin{itemize}
    \item \textbf{Device whitelisting} -- Only allow specific, registered devices
    \item \textbf{Read-only mode} -- Allow reading but block writing
    \item \textbf{Time-limited access} -- Enable ports only during maintenance windows
    \item \textbf{Supervised access} -- Require two-person authorization
\end{itemize}

% ============================================================================
\section{Company-Issued Media Program}
% ============================================================================

\begin{successbox}
\textbf{Best Practice:} Maintain a pool of company-owned, managed removable media. Personal devices should never enter OT environments.
\end{successbox}

\subsection{Media Inventory Management}

\begin{itemize}
    \item Unique identifiers (serial numbers, asset tags)
    \item Check-out/check-in tracking system
    \item Assigned custodians for each device
    \item Regular audits of media location and condition
    \item Secure storage when not in use
\end{itemize}

\subsection{Media Lifecycle}

\begin{enumerate}
    \item \textbf{Procurement} -- Purchase from trusted sources, verify authenticity
    \item \textbf{Initialization} -- Format, scan, assign ID, register in inventory
    \item \textbf{Active use} -- Track assignments, scan before/after each use
    \item \textbf{Retirement} -- Secure wipe (multiple passes) or physical destruction
\end{enumerate}

\subsection{Hardware-Encrypted Drives}

For sensitive data transfers:
\begin{itemize}
    \item Hardware encryption with PIN/password
    \item Auto-wipe after failed access attempts
    \item FIPS 140-2 certified devices for regulated environments
    \item No software installation required on host
\end{itemize}

% ============================================================================
\section{Vendor and Contractor Media}
% ============================================================================

\begin{dangerbox}
Vendor USB drives are a common attack vector. Require all vendors to submit files through your controlled transfer process rather than bringing their own media.
\end{dangerbox}

\subsection{Vendor Media Policy}

\begin{itemize}
    \item \textbf{Advance notification} -- Vendors must declare file transfer needs before arrival
    \item \textbf{Pre-transfer option} -- Encourage vendors to send files electronically in advance
    \item \textbf{No direct insertion} -- Vendor media never connects directly to OT systems
    \item \textbf{Company media provided} -- Transfer scanned files to company-owned media
    \item \textbf{Supervision required} -- Escort vendors during all media handling
\end{itemize}

\subsection{Contractual Requirements}

Include in vendor agreements:
\begin{itemize}
    \item Compliance with site media handling policies
    \item Liability for malware introduced via vendor media
    \item Right to scan and inspect all media
    \item Prohibition on unauthorized media use
\end{itemize}

% ============================================================================
\section{Legacy Media Considerations}
% ============================================================================

\subsection{Floppy Disks}

Still used in some legacy OT systems:
\begin{itemize}
    \item Maintain dedicated, isolated floppy drive for scanning
    \item Stock supply of new, sealed diskettes
    \item Consider USB floppy emulators with logging capability
\end{itemize}

\subsection{Optical Media (CD/DVD)}

\begin{itemize}
    \item Write-once media (CD-R, DVD-R) preferred for software distribution
    \item Verify disc authenticity (holograms, vendor packaging)
    \item Scan contents before use even from ``trusted'' vendors
    \item Disable autorun/autoplay on all systems
\end{itemize}

% ============================================================================
\section{Logging and Auditing}
% ============================================================================

All media handling activities should be logged:

\begin{itemize}
    \item Date, time, and user identity
    \item Media type, serial number, and source
    \item Files transferred (names, sizes, hashes)
    \item Scan results from all engines
    \item Destination system and purpose
    \item Any anomalies or policy exceptions
\end{itemize}

\begin{successbox}
\textbf{Forensic Value:} Media handling logs are critical for incident investigation. They help trace infection sources and identify compromised systems.
\end{successbox}

% ============================================================================
\section{Summary}
% ============================================================================

\begin{definitionbox}{Key Takeaways}
\begin{itemize}
    \item \textbf{Media is an attack vector} -- USB drives enabled Stuxnet and many other attacks
    \item \textbf{Scan everything} -- Use dedicated kiosks with multiple AV engines
    \item \textbf{Disable USB ports} -- Block by default, enable selectively
    \item \textbf{Company media only} -- No personal or vendor devices in OT
    \item \textbf{Control vendors} -- Require advance notice, scan all vendor files
    \item \textbf{Log all transfers} -- Audit trails support incident response
    \item \textbf{Include legacy media} -- Floppies and CDs still exist in OT
\end{itemize}
\end{definitionbox}

% ============================================================================
\section{Further Reading}
% ============================================================================

\subsection*{Standards and Guidelines}

\begin{itemize}
    \item \textbf{NIST SP 800-82 Rev. 3} -- Guide to OT Security (Section 6.2.5 Media Protection)\\
          \url{https://csrc.nist.gov/pubs/sp/800/82/r3/final}
    \item \textbf{IEC 62443-2-1} -- Security program requirements\\
          \url{https://webstore.iec.ch/publication/7030}
\end{itemize}

\subsection*{Resources}

\begin{itemize}
    \item \textbf{CISA -- Using Caution with USB Drives}\\
          \url{https://www.cisa.gov/news-events/news/using-caution-usb-drives}
    \item \textbf{SANS -- Securing Removable Media}\\
          \url{https://www.sans.org/white-papers/}
\end{itemize}

\subsection*{Books}

\begin{itemize}
    \item Knapp, E. \& Langill, J. -- \textit{Industrial Network Security} (Syngress)
    \item Macaulay, T. \& Singer, B. -- \textit{Cybersecurity for Industrial Control Systems} (CRC Press)
\end{itemize}

\vfill
\begin{center}
\textit{Part of the OT Security Learning Series}
\end{center}

\end{document}
