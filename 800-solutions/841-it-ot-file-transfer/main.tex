% ============================================================================
%  841-it-ot-file-transfer - OT Security Learning Resource
% ============================================================================

\documentclass[11pt,a4paper]{article}
\usepackage{otsec-template}

% Define colors for TikZ (matching template colors)
\colorlet{otprimary}{primary}
\colorlet{otaccent}{accent}
\colorlet{otsuccess}{success}
\colorlet{otwarning}{warning}
\colorlet{otdanger}{danger}
\colorlet{otinfo}{info}

\begin{document}

\maketitlepage
    {Secure IT/OT File Transfer}
    {DMZ-based architectures for network file exchange between IT and OT}
    {OT Security Learning Series}
    {Document 841 \quad|\quad January 2026}
    {Matthias Niedermaier}

\tableofcontents
\newpage

% ============================================================================
\section{Introduction}
% ============================================================================

\begin{infobox}
Network-based file transfer between IT and OT networks requires careful architectural design. A properly implemented DMZ-based file transfer system provides secure, auditable exchange while maintaining network segmentation and preventing direct connectivity between zones.
\end{infobox}

Common file transfer requirements include:
\begin{itemize}
    \item Software patches and updates from IT to OT
    \item Production data export from OT to IT (historians, reports)
    \item Configuration files and documentation
    \item Vendor files received via email or download
    \item Backup data replication
\end{itemize}

This document covers secure network-based file transfer architectures and procedures for IT/OT environments.

% ============================================================================
\section{Architecture Overview}
% ============================================================================

\begin{figure}[h]
\centering
\begin{tikzpicture}[
    zone/.style={rectangle, draw, thick, minimum height=3cm, minimum width=2.5cm, rounded corners=5pt},
    server/.style={rectangle, draw, thick, fill=otaccent!20, minimum height=0.7cm, minimum width=1.8cm, rounded corners=2pt, font=\tiny\bfseries, align=center},
    firewall/.style={rectangle, draw, thick, fill=otdanger!30, minimum height=2.5cm, minimum width=0.4cm},
    arrow/.style={->, thick, >=stealth}
]

% Zones
\node[zone, fill=otinfo!10] (it) at (0,0) {};
\node[font=\scriptsize\bfseries, otinfo] at (0,1.8) {IT Network};
\node[font=\tiny, text width=2cm, align=center] at (0,0.5) {Corporate\\systems};
\node[font=\tiny, text width=2cm, align=center] at (0,-0.5) {User\\workstations};

\node[zone, fill=otwarning!10, minimum width=4.5cm] (dmz) at (5,0) {};
\node[font=\scriptsize\bfseries, otwarning] at (5,1.8) {File Transfer DMZ};

\node[zone, fill=otsuccess!10] (ot) at (10,0) {};
\node[font=\scriptsize\bfseries, otsuccess] at (10,1.8) {OT Network};
\node[font=\tiny, text width=2cm, align=center] at (10,0.5) {SCADA\\PLCs};
\node[font=\tiny, text width=2cm, align=center] at (10,-0.5) {Engineering\\stations};

% Firewalls
\node[firewall] (fw1) at (2,0) {};
\node[font=\tiny, rotate=90, white] at (2,0) {FW};
\node[firewall] (fw2) at (8,0) {};
\node[font=\tiny, rotate=90, white] at (8,0) {FW};

% Servers in DMZ (spread apart)
\node[server] (upload) at (3.8,0.8) {Upload\\Server};
\node[server, fill=otdanger!20] (scan) at (5,-0.3) {Scan\\Engine};
\node[server] (download) at (6.2,0.8) {Download\\Server};

% Arrows - IT pushes to DMZ, OT pulls from DMZ
\draw[arrow, otinfo] (1.3,0.5) -- (2.8,0.5);
\node[font=\tiny] at (2,0.9) {Push};

\draw[arrow, otwarning] (upload) -- (scan);
\draw[arrow, otwarning] (scan) -- (download);

\draw[arrow, otsuccess] (7.2,0.5) -- (8.7,0.5);
\node[font=\tiny] at (8,0.9) {Pull};

\end{tikzpicture}
\caption{DMZ-Based File Transfer Architecture}
\end{figure}

\subsection{Key Design Principles}

\begin{definitionbox}{Secure File Transfer Principles}
\begin{itemize}
    \item \textbf{No direct connectivity} -- IT and OT never communicate directly
    \item \textbf{Defense in depth} -- Multiple security layers (firewalls, scanning, access control)
    \item \textbf{Pull model for OT} -- OT retrieves files; nothing pushed into OT
    \item \textbf{Scan everything} -- Multi-engine AV and content inspection
    \item \textbf{Audit trail} -- Complete logging of all transfers
    \item \textbf{Least privilege} -- Minimal permissions at each stage
\end{itemize}
\end{definitionbox}

% ============================================================================
\section{DMZ Components}
% ============================================================================

\subsection{Upload Server (IT-Facing)}

Receives files from IT network:
\begin{itemize}
    \item Web portal for manual uploads
    \item SFTP/SCP for automated transfers
    \item User authentication (AD integration optional)
    \item File size and type restrictions
    \item Quarantine storage pending scan
\end{itemize}

\subsection{Scanning Engine}

Processes all incoming files:
\begin{itemize}
    \item Multiple AV engines (minimum 2-3 vendors)
    \item Content Disarm and Reconstruction (CDR)
    \item File type validation (magic numbers, structure)
    \item Signature updates via one-way channel
    \item Sandboxing for suspicious files (optional)
\end{itemize}

\subsection{Download Server (OT-Facing)}

Provides clean files to OT:
\begin{itemize}
    \item Separate server from upload (defense in depth)
    \item Pull-only access from OT network
    \item File integrity verification (hashes)
    \item Automatic expiration of unclaimed files
    \item Notification system for available files
\end{itemize}

% ============================================================================
\section{Transfer Protocols}
% ============================================================================

\begin{table}[h]
\centering
\begin{tabular}{|l|l|l|l|}
\hline
\textbf{Protocol} & \textbf{Direction} & \textbf{Use Case} & \textbf{Security} \\
\hline
SFTP & IT$\rightarrow$DMZ & Automated uploads & Encrypted, key auth \\
HTTPS & IT$\rightarrow$DMZ & Web portal uploads & TLS, cert validation \\
SCP & DMZ$\rightarrow$OT & Pull from OT & Encrypted, key auth \\
SMB/CIFS & Internal & Windows shares & Requires Kerberos \\
\hline
\end{tabular}
\caption{Recommended Transfer Protocols}
\end{table}

\begin{warningbox}
\textbf{Avoid These Protocols:}
\begin{itemize}
    \item FTP (unencrypted credentials and data)
    \item Telnet (unencrypted)
    \item HTTP (unencrypted)
    \item NFS without Kerberos (weak authentication)
\end{itemize}
\end{warningbox}

% ============================================================================
\section{Firewall Rules}
% ============================================================================

\subsection{IT to DMZ Firewall}

\begin{table}[h]
\centering
\begin{tabular}{|l|l|l|l|}
\hline
\textbf{Source} & \textbf{Dest} & \textbf{Port} & \textbf{Purpose} \\
\hline
IT Users & Upload Server & 443/tcp & HTTPS portal \\
IT Systems & Upload Server & 22/tcp & SFTP uploads \\
\hline
\end{tabular}
\caption{IT$\rightarrow$DMZ Firewall Rules}
\end{table}

\subsection{DMZ to OT Firewall}

\begin{table}[h]
\centering
\begin{tabular}{|l|l|l|l|}
\hline
\textbf{Source} & \textbf{Dest} & \textbf{Port} & \textbf{Purpose} \\
\hline
OT Systems & Download Server & 22/tcp & SFTP pull \\
OT Systems & Download Server & 443/tcp & HTTPS download \\
\hline
\end{tabular}
\caption{DMZ$\rightarrow$OT Firewall Rules (OT Initiates)}
\end{table}

\begin{successbox}
\textbf{Key Point:} OT systems initiate connections to the DMZ. The DMZ never initiates connections into OT. This prevents attackers who compromise the DMZ from directly accessing OT systems.
\end{successbox}

% ============================================================================
\section{Content Disarm and Reconstruction}
% ============================================================================

\begin{definitionbox}{CDR Technology}
CDR sanitizes files by deconstructing them, removing potentially malicious active content, and rebuilding clean versions. This neutralizes threats that bypass signature-based detection.
\end{definitionbox}

\subsection{CDR Capabilities}

\begin{itemize}
    \item \textbf{Office documents} -- Remove macros, embedded objects, OLE
    \item \textbf{PDFs} -- Strip JavaScript, forms, embedded files
    \item \textbf{Images} -- Flatten layers, remove metadata, steganography
    \item \textbf{Archives} -- Extract, scan contents, repackage
    \item \textbf{Conversion} -- Convert to safe formats (e.g., PDF/A)
\end{itemize}

\subsection{File Type Policies}

\begin{table}[h]
\centering
\begin{tabular}{|l|l|l|}
\hline
\textbf{File Type} & \textbf{Action} & \textbf{Notes} \\
\hline
.exe, .msi, .bat & Block or quarantine & Require approval workflow \\
.pdf, .docx & CDR processing & Strip active content \\
.csv, .xml, .txt & Allow after scan & Low risk \\
.zip, .7z & Extract and scan & Recursive scanning \\
.bin, .hex (firmware) & Manual review & High-risk, requires approval \\
\hline
\end{tabular}
\caption{File Type Processing Policies}
\end{table}

% ============================================================================
\section{Workflow and Approval}
% ============================================================================

\begin{figure}[h]
\centering
\begin{tikzpicture}[
    procstep/.style={rectangle, draw, thick, fill=otaccent!20, minimum height=0.9cm, minimum width=1.8cm, rounded corners=3pt, font=\tiny\bfseries, align=center},
    decision/.style={diamond, draw, thick, fill=otwarning!20, minimum height=0.8cm, minimum width=0.8cm, font=\tiny\bfseries, align=center, aspect=2},
    arrow/.style={->, thick, >=stealth}
]

\node[procstep] (upload) at (0,0) {Upload};
\node[procstep] (scan) at (2.2,0) {Scan};
\node[decision] (clean) at (4.2,0) {Clean?};
\node[procstep] (approve) at (6.4,0) {Approve};
\node[procstep] (avail) at (8.6,0) {Available};
\node[procstep] (pull) at (10.8,0) {OT Pull};

\node[procstep, fill=otdanger!20] (quarantine) at (4.2,-1.5) {Quarantine};

\draw[arrow, otprimary] (upload) -- (scan);
\draw[arrow, otprimary] (scan) -- (clean);
\draw[arrow, otsuccess] (clean) -- node[above, font=\tiny] {Yes} (approve);
\draw[arrow, otprimary] (approve) -- (avail);
\draw[arrow, otprimary] (avail) -- (pull);
\draw[arrow, otdanger] (clean) -- node[right, font=\tiny] {No} (quarantine);

\end{tikzpicture}
\caption{File Transfer Workflow}
\end{figure}

\subsection{Standard Workflow}

\begin{enumerate}
    \item User uploads file via portal or SFTP
    \item File quarantined, scan initiated
    \item Multi-AV scan and CDR processing
    \item Clean files queued for approval (if required)
    \item Approver reviews and approves
    \item File moved to download area
    \item OT user notified, retrieves file
    \item Unclaimed files expire after defined period
\end{enumerate}

\subsection{Approval Requirements}

\begin{itemize}
    \item \textbf{Automatic approval} -- Low-risk file types (text, CSV, sanitized docs)
    \item \textbf{Single approval} -- Standard files (patches, configs)
    \item \textbf{Dual approval} -- High-risk files (executables, firmware)
    \item \textbf{Emergency bypass} -- Documented exception with senior approval
\end{itemize}

% ============================================================================
\section{Data Diode Integration}
% ============================================================================

For high-security environments, data diodes provide hardware-enforced one-way transfer:

\begin{itemize}
    \item \textbf{Inbound diode} -- IT$\rightarrow$OT for patches and updates
    \item \textbf{Outbound diode} -- OT$\rightarrow$IT for historian data, logs
    \item Eliminates any possibility of reverse channel
    \item Requires protocol proxies for TCP-based applications
\end{itemize}

\begin{warningbox}
Data diodes require careful protocol handling. TCP acknowledgments cannot traverse the diode, so proxy applications must handle protocol conversion.
\end{warningbox}

% ============================================================================
\section{Logging and Monitoring}
% ============================================================================

All file transfers must be logged with:

\begin{itemize}
    \item Timestamp and unique transfer ID
    \item Source user/system identity
    \item File metadata (name, size, hash before/after)
    \item Scan results from all engines
    \item Approval workflow (who, when)
    \item Destination (who retrieved, when)
    \item Any alerts or anomalies
\end{itemize}

\begin{successbox}
\textbf{SIEM Integration:} Forward file transfer logs to your security monitoring platform. Alert on failed scans, unusual file types, high volumes, or after-hours transfers.
\end{successbox}

% ============================================================================
\section{Operational Procedures}
% ============================================================================

\subsection{Patch Tuesday Process}

\begin{enumerate}
    \item IT downloads patches from vendor
    \item Upload to file transfer system
    \item Scan and CDR processing
    \item OT team reviews and approves
    \item Files staged for maintenance window
    \item OT retrieves and tests in lab
    \item Deploy to production during outage
\end{enumerate}

\subsection{Emergency File Transfer}

\begin{itemize}
    \item Pre-approved emergency contacts authorized to bypass
    \item Documented justification required
    \item Post-incident review mandatory
    \item Scan still performed (blocking only bypassed)
\end{itemize}

% ============================================================================
\section{Summary}
% ============================================================================

\begin{definitionbox}{Key Takeaways}
\begin{itemize}
    \item \textbf{Use a DMZ} -- Never allow direct IT/OT file transfer
    \item \textbf{Pull model} -- OT retrieves files; nothing pushed to OT
    \item \textbf{Scan everything} -- Multiple AV engines plus CDR
    \item \textbf{Strict firewall rules} -- Minimal ports, OT initiates only
    \item \textbf{Approval workflows} -- Human review for sensitive files
    \item \textbf{Complete audit trail} -- Log all transfers for forensics
    \item \textbf{Consider data diodes} -- Hardware-enforced for high security
\end{itemize}
\end{definitionbox}

% ============================================================================
\section{Further Reading}
% ============================================================================

\subsection*{Standards and Guidelines}

\begin{itemize}
    \item \textbf{IEC 62443-3-3} -- System security requirements for IACS\\
          \url{https://webstore.iec.ch/publication/7033}
    \item \textbf{NIST SP 800-82 Rev. 3} -- Guide to OT Security\\
          \url{https://csrc.nist.gov/publications/detail/sp/800-82/rev-3/final}
\end{itemize}

\subsection*{Resources}

\begin{itemize}
    \item \textbf{CISA -- Securing Industrial Control Systems}\\
          \url{https://www.cisa.gov/topics/industrial-control-systems}
    \item \textbf{SANS ICS -- Network Security Monitoring}\\
          \url{https://www.sans.org/cyber-security-courses/industrial-control-system-network-security-monitoring/}
\end{itemize}

\subsection*{Books}

\begin{itemize}
    \item Knapp, E. \& Langill, J. -- \textit{Industrial Network Security} (Syngress)
    \item Macaulay, T. \& Singer, B. -- \textit{Cybersecurity for Industrial Control Systems} (CRC Press)
\end{itemize}

\vfill
\begin{center}
\textit{Part of the OT Security Learning Series}
\end{center}

\end{document}
