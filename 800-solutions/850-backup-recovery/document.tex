% ============================================================================
%  850-backup-recovery - OT Security Learning Resource
% ============================================================================

\documentclass[11pt,a4paper]{article}
\usepackage{otsec-template}

% Define colors for TikZ (matching template colors)
\colorlet{otprimary}{primary}
\colorlet{otaccent}{accent}
\colorlet{otsuccess}{success}
\colorlet{otwarning}{warning}
\colorlet{otdanger}{danger}
\colorlet{otinfo}{info}

\begin{document}

\maketitlepage
    {OT Backup and Recovery}
    {Strategies for protecting and restoring industrial control systems}
    {OT Security Learning Series}
    {Document 850 \quad|\quad January 2026}
    {Matthias Niedermaier}

\tableofcontents
\newpage

% ============================================================================
\section{Introduction}
% ============================================================================

\begin{infobox}
Backup and recovery capabilities are essential for OT resilience. Unlike IT systems where data is the primary asset, OT environments must protect and restore \textbf{control logic}, \textbf{configurations}, and \textbf{operational state} -- often for systems that cannot tolerate extended downtime.
\end{infobox}

Effective OT backup strategies address:
\begin{itemize}
    \item PLC and controller program backups
    \item HMI/SCADA configuration preservation
    \item Network device configurations
    \item Historian and operational data
    \item System images for rapid restoration
\end{itemize}

Without proper backups, ransomware attacks or system failures can result in weeks of downtime while engineers recreate control logic from scratch.

% ============================================================================
\section{OT vs IT Backup Challenges}
% ============================================================================

\begin{table}[h]
\centering
\begin{tabular}{|l|l|l|}
\hline
\textbf{Aspect} & \textbf{IT Backup} & \textbf{OT Backup} \\
\hline
Primary asset & Data files & Control logic, configs \\
Backup frequency & Daily/hourly & After changes \\
Restore window & Hours acceptable & Minutes critical \\
Agents & Standard backup agents & Often not possible \\
Network access & Always connected & Air-gapped common \\
Change rate & Frequent & Rare but critical \\
Validation & File integrity & Functional testing \\
\hline
\end{tabular}
\caption{IT vs OT Backup Considerations}
\end{table}

\begin{warningbox}
\textbf{OT-Specific Challenge:} Many industrial devices cannot run backup agents, don't support standard protocols, and may require proprietary tools for program extraction. Generic IT backup solutions often fail in OT environments.
\end{warningbox}

% ============================================================================
\section{What to Back Up}
% ============================================================================

\subsection{Control System Components}

\begin{definitionbox}{Critical OT Assets for Backup}
\begin{itemize}
    \item \textbf{PLC/RTU Programs} -- Ladder logic, function blocks, structured text
    \item \textbf{HMI Projects} -- Graphics, scripts, tag databases, alarm configs
    \item \textbf{SCADA Configurations} -- Communication settings, historian configs
    \item \textbf{DCS Configurations} -- Controller configs, I/O assignments, tuning parameters
    \item \textbf{Safety System Logic} -- SIS programs (with strict change control)
    \item \textbf{Network Devices} -- Switch/firewall/router configurations
    \item \textbf{Engineering Workstations} -- Project files, licensing, tools
\end{itemize}
\end{definitionbox}

\subsection{Operational Data}

\begin{itemize}
    \item Historian databases (process data, trends)
    \item Alarm and event logs
    \item Batch records and production data
    \item Audit trails and access logs
\end{itemize}

\subsection{Supporting Documentation}

\begin{itemize}
    \item Network diagrams and IP assignments
    \item I/O lists and wiring documentation
    \item Setpoint and tuning parameter records
    \item Recovery procedures
\end{itemize}

% ============================================================================
\section{Backup Strategies}
% ============================================================================

\begin{figure}[h]
\centering
\begin{tikzpicture}[
    box/.style={rectangle, draw, thick, minimum height=1.2cm, minimum width=3cm, rounded corners=3pt, font=\small\bfseries},
    arrow/.style={->, thick, >=stealth}
]

% Backup types
\node[box, fill=otaccent!20] (full) at (0,2) {Full Backup};
\node[box, fill=otsuccess!20] (incr) at (4,2) {Incremental};
\node[box, fill=otwarning!20] (diff) at (8,2) {Differential};

% Descriptions
\node[font=\tiny, text width=2.8cm, align=center] at (0,0.6) {Complete copy\\of all data\\Longest time\\Self-contained};
\node[font=\tiny, text width=2.8cm, align=center] at (4,0.6) {Changes since\\last backup\\Fastest backup\\Complex restore};
\node[font=\tiny, text width=2.8cm, align=center] at (8,0.6) {Changes since\\last full backup\\Medium speed\\Simpler restore};

% OT recommendation
\node[font=\scriptsize\bfseries, otsuccess] at (4,-0.8) {OT Recommendation: Full backups after every change};

\end{tikzpicture}
\caption{Backup Strategy Types}
\end{figure}

\subsection{Change-Triggered Backups}

In OT environments where changes are infrequent but critical:

\begin{successbox}
\textbf{Best Practice:} Perform a full backup immediately before and after any change to control logic. This creates a known-good state to restore and documents what changed.
\end{successbox}

\subsection{Scheduled Backups}

Even without changes, periodic backups catch:
\begin{itemize}
    \item Unauthorized modifications
    \item Runtime parameter drift
    \item Configuration corruption
    \item Comparison baselines for integrity monitoring
\end{itemize}

% ============================================================================
\section{Backup Methods}
% ============================================================================

\subsection{Native Vendor Tools}

Most control system vendors provide backup utilities:
\begin{itemize}
    \item Upload/download functions in programming software
    \item Project archive features
    \item Export to portable formats
\end{itemize}

\begin{warningbox}
\textbf{Limitation:} Vendor tools often require manual operation and don't integrate with enterprise backup systems. Automation may require scripting or third-party solutions.
\end{warningbox}

\subsection{Online vs Offline Backups}

\begin{table}[h]
\centering
\begin{tabular}{|l|l|l|}
\hline
\textbf{Method} & \textbf{Advantages} & \textbf{Disadvantages} \\
\hline
Online (running) & No downtime & May miss runtime state \\
Offline (stopped) & Complete capture & Requires outage \\
Memory card copy & Direct, complete & Physical access needed \\
Network upload & Remote capable & Protocol limitations \\
\hline
\end{tabular}
\caption{Online vs Offline Backup Methods}
\end{table}

\subsection{Image-Based Backups}

For Windows-based OT systems (HMIs, historians, engineering workstations):
\begin{itemize}
    \item Full disk images capture OS, applications, and data
    \item Bare-metal restore capability
    \item Must validate application functionality after restore
    \item Consider licensing implications
\end{itemize}

% ============================================================================
\section{Backup Storage and Protection}
% ============================================================================

\begin{dangerbox}
\textbf{Critical Rule:} Backups must be stored separately from production systems. If ransomware encrypts your OT network, it should not reach your backups.
\end{dangerbox}

\subsection{Storage Locations}

\begin{figure}[h]
\centering
\begin{tikzpicture}[
    loc/.style={rectangle, draw, thick, minimum height=1cm, minimum width=2.5cm, rounded corners=3pt, font=\tiny\bfseries, align=center},
    arrow/.style={->, thick, >=stealth}
]

\node[loc, fill=otdanger!20] (ot) at (0,0) {OT Network};
\node[loc, fill=otwarning!20] (dmz) at (3.5,0) {DMZ Backup\\Server};
\node[loc, fill=otsuccess!20] (offline) at (7,0) {Offline/\\Air-gapped};
\node[loc, fill=otinfo!20] (offsite) at (10.5,0) {Offsite\\Storage};

\draw[arrow, otprimary] (ot) -- (dmz);
\draw[arrow, otprimary] (dmz) -- (offline);
\draw[arrow, otprimary] (offline) -- (offsite);

\node[font=\tiny, otprimary] at (1.75,0.8) {Automated};
\node[font=\tiny, otprimary] at (5.25,0.8) {Manual};
\node[font=\tiny, otprimary] at (8.75,0.8) {Periodic};

\end{tikzpicture}
\caption{Backup Storage Tiers}
\end{figure}

\subsection{The 3-2-1 Rule for OT}

\begin{definitionbox}{3-2-1 Backup Rule}
\begin{itemize}
    \item \textbf{3} copies of critical data
    \item \textbf{2} different storage media/types
    \item \textbf{1} copy offsite or air-gapped
\end{itemize}
For OT, add: at least one copy that is \textbf{offline} and immune to network-based attacks.
\end{definitionbox}

\subsection{Backup Security}

\begin{itemize}
    \item \textbf{Encryption} -- Protect backups at rest and in transit
    \item \textbf{Access control} -- Limit who can read/write/delete backups
    \item \textbf{Integrity verification} -- Hash validation, checksums
    \item \textbf{Immutable storage} -- Write-once media or immutable cloud storage
    \item \textbf{Version retention} -- Keep multiple generations
\end{itemize}

\subsection{Backup Encryption}

Encrypting backups protects sensitive control logic and configurations from unauthorized access, even if backup media is lost or stolen.

\subsubsection{What to Encrypt}

\begin{table}[h]
\centering
\begin{tabular}{|l|l|l|}
\hline
\textbf{Backup Type} & \textbf{Encryption Priority} & \textbf{Rationale} \\
\hline
PLC/RTU programs & High & Proprietary control logic \\
Credentials/certificates & Critical & Direct security impact \\
Network configurations & High & Reveals architecture \\
Historian data & Medium & May contain process secrets \\
System images & Medium & Contains configurations \\
Documentation & Low & Often less sensitive \\
\hline
\end{tabular}
\caption{Encryption priority by backup type}
\end{table}

\subsubsection{Encryption Methods}

\begin{itemize}
    \item \textbf{File-level encryption} -- Encrypt individual backup files (GPG, 7-Zip AES)
    \item \textbf{Volume encryption} -- Encrypt entire backup volumes (LUKS, BitLocker)
    \item \textbf{Backup software encryption} -- Built-in encryption in backup tools
    \item \textbf{Hardware encryption} -- Self-encrypting drives (SEDs) for backup storage
\end{itemize}

\begin{successbox}
\textbf{Recommendation:} Use AES-256 encryption for backups. Prefer backup software with built-in encryption to ensure consistent protection across all backup operations.
\end{successbox}

\subsubsection{Key Management for Backups}

\begin{dangerbox}
\textbf{Critical:} Encrypted backups are useless without the decryption keys. Key loss equals data loss. Store encryption keys separately from the encrypted backups.
\end{dangerbox}

Key management best practices:

\begin{itemize}
    \item \textbf{Key separation} -- Never store keys on the same media as encrypted backups
    \item \textbf{Key escrow} -- Maintain secure copies of keys in multiple locations
    \item \textbf{Key rotation} -- Change encryption keys periodically; retain old keys for archived backups
    \item \textbf{Offline key storage} -- Keep master keys in air-gapped, physical secure storage
    \item \textbf{Key documentation} -- Document which keys decrypt which backups
\end{itemize}

\subsubsection{Recovery Considerations}

Encryption adds complexity to recovery procedures:

\begin{itemize}
    \item \textbf{Key availability} -- Ensure decryption keys are accessible during emergencies
    \item \textbf{Recovery time impact} -- Decryption adds time to restore operations
    \item \textbf{Offline recovery} -- Plan for scenarios without network access to key servers
    \item \textbf{Testing} -- Include decryption in recovery drills
\end{itemize}

\begin{warningbox}
During a ransomware incident, key management systems may also be compromised. Maintain offline copies of backup encryption keys that are immune to network-based attacks.
\end{warningbox}

% ============================================================================
\section{Recovery Planning}
% ============================================================================

\subsection{Recovery Time Objectives}

Define acceptable downtime for each system:

\begin{table}[h]
\centering
\begin{tabular}{|l|l|l|}
\hline
\textbf{System Type} & \textbf{Typical RTO} & \textbf{Recovery Priority} \\
\hline
Safety systems & Minutes & \riskcritical \\
Critical PLCs & 1-4 hours & \riskhigh \\
HMI/SCADA & 4-8 hours & \riskhigh \\
Historians & 24 hours & \riskmedium \\
Engineering stations & Days & \risklow \\
\hline
\end{tabular}
\caption{Recovery Time Objectives by System Type}
\end{table}

\subsection{Recovery Procedures}

Document step-by-step procedures for each system type:
\begin{enumerate}
    \item Prerequisites (tools, access, credentials)
    \item Locate and verify backup integrity
    \item Restoration steps
    \item Validation and testing
    \item Return to production checklist
\end{enumerate}

\begin{successbox}
\textbf{Critical:} Recovery procedures must be usable by operators under stress during an incident. Keep them simple, tested, and accessible (printed copies in control rooms).
\end{successbox}

% ============================================================================
\section{Testing and Validation}
% ============================================================================

\begin{warningbox}
\textbf{Untested backups are not backups.} Many organizations discover their backups are corrupt, incomplete, or unusable only when they need them most.
\end{warningbox}

\subsection{Backup Validation}

\begin{itemize}
    \item Verify backup completed successfully (no errors)
    \item Check file integrity (hashes match)
    \item Confirm backup is readable and not corrupted
    \item Validate backup contains expected content
\end{itemize}

\subsection{Recovery Testing}

\begin{itemize}
    \item \textbf{Tabletop exercises} -- Walk through procedures
    \item \textbf{Lab restoration} -- Test in non-production environment
    \item \textbf{Partial recovery} -- Restore individual components
    \item \textbf{Full recovery drill} -- Complete system restoration (during maintenance)
\end{itemize}

\subsection{Testing Frequency}

\begin{itemize}
    \item Backup verification: Every backup
    \item Procedure review: Quarterly
    \item Lab recovery test: Semi-annually
    \item Full recovery drill: Annually
\end{itemize}

% ============================================================================
\section{Summary}
% ============================================================================

\begin{definitionbox}{Key Takeaways}
\begin{itemize}
    \item \textbf{OT backups differ from IT} -- Focus on control logic and configurations, not just data
    \item \textbf{Back up everything} -- PLCs, HMIs, network devices, documentation
    \item \textbf{Change-triggered backups} -- Full backup before and after every change
    \item \textbf{Isolate backup storage} -- Keep backups separate from production networks
    \item \textbf{Follow 3-2-1 rule} -- Three copies, two media types, one offline/offsite
    \item \textbf{Document recovery procedures} -- Simple, tested, accessible
    \item \textbf{Test regularly} -- Untested backups provide false confidence
    \item \textbf{Define RTOs} -- Know acceptable downtime for each system
\end{itemize}
\end{definitionbox}

% ============================================================================
\section{Further Reading}
% ============================================================================

\subsection*{Standards and Guidelines}

\begin{itemize}
    \item \textbf{IEC 62443-2-1} -- Security program requirements (backup and recovery)\\
          \url{https://webstore.iec.ch/publication/7030}
    \item \textbf{NIST SP 800-82 Rev. 3} -- Guide to OT Security (Section 6.2.9 Contingency Planning)\\
          \url{https://csrc.nist.gov/pubs/sp/800/82/r3/final}
    \item \textbf{NIST SP 800-184} -- Guide for Cybersecurity Event Recovery\\
          \url{https://csrc.nist.gov/publications/detail/sp/800-184/final}
\end{itemize}

\subsection*{Resources}

\begin{itemize}
    \item \textbf{CISA -- ICS Recommended Practices}\\
          \url{https://www.cisa.gov/topics/industrial-control-systems}
    \item \textbf{SANS ICS -- Incident Response and Recovery}\\
          \url{https://www.sans.org/blog/}
\end{itemize}

\subsection*{Books}

\begin{itemize}
    \item Knapp, E. \& Langill, J. -- \textit{Industrial Network Security} (Syngress)
    \item Stouffer, K. et al. -- \textit{Guide to Industrial Control Systems Security} (NIST)
\end{itemize}

\vfill
\begin{center}
\textit{Part of the OT Security Learning Series}
\end{center}

\end{document}
