% ============================================================================
%  835-epp-edr - OT Security Learning Resource
% ============================================================================

\documentclass[11pt,a4paper]{article}
\usepackage{otsec-template}
\usepackage{float}

% Define colors for TikZ
\colorlet{otprimary}{primary}
\colorlet{otaccent}{accent}
\colorlet{otsuccess}{success}
\colorlet{otwarning}{warning}
\colorlet{otdanger}{danger}
\colorlet{otinfo}{info}

\begin{document}

\maketitlepage
    {Endpoint Protection Platforms}
    {EPP and EDR Solutions for OT Environments}
    {OT Security Learning Series}
    {Document 835 \quad|\quad January 2026}
    {Matthias Niedermaier}

\tableofcontents
\newpage

\section{Introduction}

Endpoint Protection Platforms (EPP) have evolved significantly from traditional antivirus solutions. Modern EPP integrates multiple security capabilities into unified platforms, while Endpoint Detection and Response (EDR) adds advanced threat detection and investigation capabilities. Deploying these technologies in Operational Technology (OT) environments requires careful consideration of operational constraints.

\begin{infobox}
This document covers Endpoint Protection Platform (EPP) and Endpoint Detection and Response (EDR) concepts for OT environments. It addresses deployment considerations, architecture options, and the unique challenges of protecting industrial endpoints while maintaining system availability.
\end{infobox}

\section{EPP vs EDR vs XDR}

\subsection{Technology Evolution}

Endpoint security has evolved through several generations:

\begin{figure}[H]
\centering
\begin{tikzpicture}[
    box/.style={rectangle, draw, thick, rounded corners=3pt, minimum width=3cm, minimum height=1.2cm, align=center, font=\small},
    arrow/.style={->, thick, >=stealth}
]
% Timeline boxes
\node[box, fill=otinfo!20] (av) at (0,0) {Traditional\\Antivirus};
\node[box, fill=otaccent!20] (epp) at (4,0) {EPP\\Platform};
\node[box, fill=otsuccess!20] (edr) at (8,0) {EDR\\Detection};
\node[box, fill=otwarning!20] (xdr) at (12,0) {XDR\\Extended};

% Arrows
\draw[arrow] (av) -- (epp);
\draw[arrow] (epp) -- (edr);
\draw[arrow] (edr) -- (xdr);

% Labels below
\node[font=\scriptsize, align=center] at (0,-1.2) {Signature-based\\File scanning};
\node[font=\scriptsize, align=center] at (4,-1.2) {Multi-layered\\Prevention};
\node[font=\scriptsize, align=center] at (8,-1.2) {Behavior analysis\\Investigation};
\node[font=\scriptsize, align=center] at (12,-1.2) {Cross-domain\\Correlation};
\end{tikzpicture}
\caption{Evolution of endpoint security technologies}
\end{figure}

\subsection{Capability Comparison}

\begin{table}[H]
\centering
\small
\rowcolors{2}{lightgray}{white}
\begin{tabular}{p{3.5cm}p{2cm}p{2cm}p{2cm}p{2cm}}
\rowcolor{primary}
\textcolor{white}{\bfseries Capability} & \textcolor{white}{\bfseries AV} & \textcolor{white}{\bfseries EPP} & \textcolor{white}{\bfseries EDR} & \textcolor{white}{\bfseries XDR} \\
\midrule
Signature detection & \checkmark & \checkmark & \checkmark & \checkmark \\
Behavior analysis & -- & \checkmark & \checkmark & \checkmark \\
Machine learning & -- & \checkmark & \checkmark & \checkmark \\
Threat hunting & -- & -- & \checkmark & \checkmark \\
Forensic investigation & -- & -- & \checkmark & \checkmark \\
Network correlation & -- & -- & -- & \checkmark \\
Automated response & -- & Limited & \checkmark & \checkmark \\
\end{tabular}
\caption{Endpoint security capability comparison}
\end{table}

\section{EPP Core Components}

\subsection{Prevention Technologies}

Modern EPP platforms combine multiple prevention layers:

\begin{figure}[H]
\centering
\begin{tikzpicture}[
    layer/.style={rectangle, draw, thick, rounded corners=3pt, minimum width=10cm, minimum height=0.8cm, align=center, font=\small}
]
% Stacked layers
\node[layer, fill=otdanger!20] (l1) at (0,3) {Application Control / Whitelisting};
\node[layer, fill=otwarning!20] (l2) at (0,2) {Behavioral Analysis / HIPS};
\node[layer, fill=otaccent!20] (l3) at (0,1) {Machine Learning / AI Detection};
\node[layer, fill=otsuccess!20] (l4) at (0,0) {Signature-based Detection};
\node[layer, fill=otinfo!20] (l5) at (0,-1) {Exploit Prevention / Memory Protection};

% Label
\node[font=\small\bfseries, right, align=center] at (-8,1) {Defense\\in Depth};

% Arrow showing depth
\draw[very thick, ->, >=stealth] (-5.8,3.3) -- (-5.8,-1.3);
\end{tikzpicture}
\caption{EPP defense-in-depth layers}
\end{figure}

\subsection{Key EPP Features}

\begin{itemize}
    \item \textbf{Anti-malware:} Signature and heuristic detection of known threats
    \item \textbf{Application Control:} Whitelist/blacklist enforcement
    \item \textbf{Device Control:} USB and removable media policies
    \item \textbf{Host Firewall:} Local network traffic filtering
    \item \textbf{Exploit Prevention:} Memory protection and anti-exploit
    \item \textbf{Web Protection:} URL filtering and download scanning
\end{itemize}

\section{EDR Capabilities}

\subsection{Detection and Response}

EDR extends EPP with advanced detection and investigation:

\begin{table}[H]
\centering
\small
\rowcolors{2}{lightgray}{white}
\begin{tabular}{p{4cm}p{8.5cm}}
\rowcolor{primary}
\textcolor{white}{\bfseries Capability} & \textcolor{white}{\bfseries Description} \\
\midrule
Continuous monitoring & Records endpoint activity including process execution, file changes, network connections, and registry modifications \\
Threat detection & Behavioral rules and analytics identify suspicious activity patterns \\
Alert triage & Prioritizes alerts based on severity and context \\
Investigation tools & Timeline analysis, process trees, and artifact collection \\
Threat hunting & Proactive search for indicators of compromise (IOCs) \\
Response actions & Isolate host, kill process, quarantine file, collect forensics \\
\end{tabular}
\caption{EDR core capabilities}
\end{table}

\subsection{Telemetry Collection}

EDR agents collect extensive telemetry:

\begin{itemize}
    \item Process creation and termination events
    \item Network connections and DNS queries
    \item File system operations
    \item Registry modifications (Windows)
    \item User authentication events
    \item Loaded modules and drivers
\end{itemize}

\begin{warningbox}
EDR telemetry collection can impact system performance and generate significant network traffic. In OT environments, carefully evaluate resource requirements and consider limiting telemetry scope on constrained systems.
\end{warningbox}

\section{OT Deployment Considerations}

\subsection{Operational Constraints}

OT environments present unique challenges for endpoint protection:

\begin{table}[H]
\centering
\small
\rowcolors{2}{lightgray}{white}
\begin{tabular}{p{4cm}p{8.5cm}}
\rowcolor{primary}
\textcolor{white}{\bfseries Constraint} & \textcolor{white}{\bfseries Impact on EPP/EDR} \\
\midrule
Legacy operating systems & May not support modern agents; limited protection options \\
Real-time requirements & Agent overhead must not impact process timing \\
Change management & Agent updates require testing and approval cycles \\
Network isolation & Cloud-based management may not be feasible \\
Vendor support & Security software may void OT system warranties \\
Limited resources & Embedded systems lack CPU/memory for full agents \\
\end{tabular}
\caption{OT constraints affecting EPP/EDR deployment}
\end{table}

\subsection{System Classification}

Not all OT systems can support the same protection level:

\begin{figure}[H]
\centering
\begin{tikzpicture}[
    box/.style={rectangle, draw, thick, rounded corners=3pt, minimum width=3.8cm, minimum height=2cm, align=center, font=\small}
]
% Three categories
\node[box, fill=otsuccess!20] (full) at (0,0) {\textbf{Full EPP/EDR}\\[3pt]Windows servers\\HMI workstations\\Engineering stations};
\node[box, fill=otwarning!20] (light) at (5,0) {\textbf{Lightweight Agent}\\[3pt]Older Windows\\Limited resources\\Real-time systems};
\node[box, fill=otdanger!20] (none) at (10,0) {\textbf{No Agent}\\[3pt]PLCs/RTUs\\Embedded devices\\Safety systems};

% Labels
\node[font=\scriptsize\bfseries] at (0,1.5) {Standard IT-like};
\node[font=\scriptsize\bfseries] at (5,1.5) {OT-optimized};
\node[font=\scriptsize\bfseries] at (10,1.5) {Network-only};
\end{tikzpicture}
\caption{Endpoint protection tiers based on system capability}
\end{figure}

\subsection{Agent Deployment Modes}

\begin{itemize}
    \item \textbf{Full Protection:} All EPP/EDR features enabled
    \item \textbf{Detection Only:} Monitor and alert without blocking
    \item \textbf{Audit Mode:} Log what would be blocked for tuning
    \item \textbf{Passive:} Minimal footprint, scheduled scans only
\end{itemize}

\begin{successbox}
\textbf{Recommendation:} Start with detection-only or audit mode in OT environments. Analyze alerts and tune policies before enabling blocking to prevent operational disruptions.
\end{successbox}

\section{Architecture Options}

\subsection{Management Infrastructure}

EPP/EDR platforms require management infrastructure:

\begin{figure}[H]
\centering
\begin{tikzpicture}[
    box/.style={rectangle, draw, thick, rounded corners=3pt, minimum width=2.8cm, minimum height=1cm, align=center, font=\small},
    zone/.style={rectangle, draw, dashed, thick, rounded corners=5pt, fill=#1!10}
]
% Zones
\node[zone=otdanger, minimum width=6cm, minimum height=2.2cm] at (0,3.5) {};
\node[font=\small\bfseries, left] at (-4,3.5) {Enterprise};

\node[zone=otwarning, minimum width=6cm, minimum height=2.2cm] at (0,0.8) {};
\node[font=\small\bfseries, left] at (-4,0.8) {DMZ};

\node[zone=otsuccess, minimum width=6cm, minimum height=2.2cm] at (0,-2) {};
\node[font=\small\bfseries, left] at (-4,-2) {OT Zone};

% Components
\node[box, fill=otdanger!20] (cloud) at (0,3.5) {Cloud Console\\(Optional)};

\node[box, fill=otwarning!20] (mgmt) at (0,0.8) {On-Premise\\Management};

\node[box, fill=otsuccess!20] (ep1) at (-2,-2) {HMI\\Agent};
\node[box, fill=otsuccess!20] (ep2) at (2,-2) {Server\\Agent};

% Connections
\draw[thick, ->] (cloud) -- (mgmt) node[midway, right, font=\scriptsize] {Sync};
\draw[thick, ->] (mgmt) -- (ep1) node[midway, left, font=\scriptsize] {Policy};
\draw[thick, ->] (mgmt) -- (ep2);
\draw[thick, <-] (mgmt) -- ++(3.5,0) node[right, font=\scriptsize] {Telemetry};
\end{tikzpicture}
\caption{EPP/EDR management architecture for OT}
\end{figure}

\subsection{Cloud vs On-Premise}

\begin{table}[H]
\centering
\small
\rowcolors{2}{lightgray}{white}
\begin{tabular}{p{3cm}p{5cm}p{5cm}}
\rowcolor{primary}
\textcolor{white}{\bfseries Aspect} & \textcolor{white}{\bfseries Cloud-Managed} & \textcolor{white}{\bfseries On-Premise} \\
\midrule
Deployment & Quick, no infrastructure & Requires server infrastructure \\
Updates & Automatic, always current & Manual update cycles \\
Connectivity & Requires internet access & Air-gap compatible \\
Data residency & Data leaves network & Data stays on-site \\
Scalability & Unlimited & Capacity planning needed \\
Cost model & Subscription (OpEx) & License + infrastructure (CapEx) \\
\end{tabular}
\caption{Cloud vs on-premise management comparison}
\end{table}

\begin{dangerbox}
Cloud-managed EPP/EDR sends telemetry outside the OT network. For air-gapped or highly sensitive environments, on-premise management is required. Evaluate data sensitivity and regulatory requirements before choosing cloud management.
\end{dangerbox}

\section{Policy Configuration}

\subsection{OT-Specific Tuning}

EPP policies require OT-specific adjustments:

\begin{itemize}
    \item \textbf{Exclusions:} Whitelist OT application paths and processes
    \item \textbf{Scan Scheduling:} Avoid scans during critical operations
    \item \textbf{Update Windows:} Control signature updates to maintenance periods
    \item \textbf{Response Actions:} Disable automatic quarantine; alert only
    \item \textbf{Resource Limits:} Cap CPU and memory usage
\end{itemize}

\subsection{Application Whitelisting Integration}

\begin{tipbox}
Combine EPP with application whitelisting for defense in depth. Whitelisting prevents unauthorized executables while EPP detects threats within allowed applications. This layered approach is particularly effective in static OT environments.
\end{tipbox}

\section{Response Procedures}

\subsection{Alert Handling in OT}

Response to EPP/EDR alerts in OT requires modified procedures:

\begin{enumerate}
    \item \textbf{Assess Impact:} Determine if alert affects critical process
    \item \textbf{Validate Alert:} Confirm true positive vs false positive
    \item \textbf{Coordinate Response:} Involve OT operations before action
    \item \textbf{Controlled Isolation:} If needed, follow safe shutdown procedures
    \item \textbf{Investigate:} Use EDR tools to understand scope
    \item \textbf{Remediate:} Plan remediation during maintenance window
\end{enumerate}

\begin{warningbox}
Never automatically isolate or shut down OT endpoints without operational coordination. Abrupt disconnection of control systems can cause safety incidents or process disruptions more severe than the security threat.
\end{warningbox}

\section{Summary}

\begin{definitionbox}{Key Takeaways}
\begin{itemize}
    \item \textbf{Layered Protection:} EPP provides multiple prevention layers; EDR adds detection and response capabilities for advanced threats
    \item \textbf{Tiered Deployment:} Classify OT systems by capability and deploy appropriate protection levels---full agents, lightweight agents, or network-only monitoring
    \item \textbf{OT-Specific Configuration:} Tune policies for OT requirements including exclusions, scan scheduling, and disabled automatic responses
    \item \textbf{On-Premise Management:} Consider on-premise management servers for air-gapped or sensitive OT environments
    \item \textbf{Coordinated Response:} Integrate security response with OT operations to prevent disruptions from automated or hasty containment actions
\end{itemize}
\end{definitionbox}

\section{Further Reading}

\subsection*{Standards and Guidelines}

\begin{itemize}
    \item \textbf{NIST SP 800-82 Rev. 3} -- Guide to OT Security\\
          \url{https://csrc.nist.gov/pubs/sp/800/82/r3/final}
    \item \textbf{IEC 62443-2-1} -- Security Program Requirements for IACS Asset Owners\\
          \url{https://webstore.iec.ch/publication/7030}
    \item \textbf{CISA} -- Industrial Control Systems Security\\
          \url{https://www.cisa.gov/topics/industrial-control-systems}
\end{itemize}

\subsection*{Resources}

\begin{itemize}
    \item \textbf{MITRE ATT\&CK for ICS} -- Adversary Tactics and Techniques\\
          \url{https://attack.mitre.org/techniques/ics/}
    \item \textbf{SANS ICS} -- Industrial Control Systems Security\\
          \url{https://www.sans.org/cyber-security-courses/ics-scada-cyber-security-essentials}
\end{itemize}

\subsection*{Books}

\begin{itemize}
    \item Knapp, Eric D. -- \textit{Industrial Network Security} (Syngress)
    \item Hadnagy, Christopher -- \textit{Social Engineering: The Science of Human Hacking} (Wiley)
\end{itemize}

\vfill
\begin{center}
\textit{Part of the OT Security Learning Series}
\end{center}

\end{document}
