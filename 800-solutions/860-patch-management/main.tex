% ============================================================================
%  860-patch-management - OT Security Learning Resource
% ============================================================================

\documentclass[11pt,a4paper]{article}
\usepackage{otsec-template}
\usepackage{float}

% Define colors for TikZ
\colorlet{otprimary}{primary}
\colorlet{otaccent}{accent}
\colorlet{otsuccess}{success}
\colorlet{otwarning}{warning}
\colorlet{otdanger}{danger}
\colorlet{otinfo}{info}

\begin{document}

\maketitlepage
    {OT Patch Management}
    {Strategies for updating industrial control systems}
    {OT Security Learning Series}
    {Document 860 \quad|\quad January 2026}
    {AI Assistant}

\tableofcontents
\newpage

% ============================================================================
\section{Introduction}
% ============================================================================

\begin{infobox}
Patch management in OT environments is one of the most challenging security tasks. Unlike IT systems where patches can be applied quickly, OT systems often run 24/7, require vendor approval, and may have dependencies that make patching risky or impossible.
\end{infobox}

OT patching challenges:
\begin{itemize}
    \item \textbf{Uptime requirements} -- Systems cannot be rebooted easily
    \item \textbf{Vendor dependencies} -- Patches may void support
    \item \textbf{Legacy systems} -- No patches available for old OS
    \item \textbf{Testing requirements} -- Must validate in lab first
    \item \textbf{Change management} -- Strict approval processes
\end{itemize}

\begin{dangerbox}
Never apply patches to production OT systems without thorough testing. A failed patch can halt production and cause safety incidents.
\end{dangerbox}

% ============================================================================
\section{OT vs IT Patching}
% ============================================================================

\begin{table}[H]
\centering
\small
\begin{tabularx}{\textwidth}{|l|X|X|}
\hline
\textbf{Aspect} & \textbf{IT Patching} & \textbf{OT Patching} \\
\hline
Frequency & Weekly/Monthly & Quarterly/Annually \\
Downtime tolerance & Hours acceptable & Minutes may be unacceptable \\
Testing & Limited/Staged rollout & Extensive lab testing required \\
Vendor approval & Rarely needed & Often mandatory \\
Automation & Highly automated & Mostly manual \\
Rollback & Usually straightforward & May require full restore \\
Lifecycle & 3-5 years & 15-25 years \\
\hline
\end{tabularx}
\caption{IT vs OT Patching Comparison}
\end{table}

% ============================================================================
\section{Patch Management Process}
% ============================================================================

\subsection{OT Patch Lifecycle}

\begin{enumerate}
    \item \textbf{Identification} -- Monitor for new patches and vulnerabilities
    \item \textbf{Assessment} -- Evaluate relevance and risk to OT systems
    \item \textbf{Vendor Check} -- Verify vendor approval and compatibility
    \item \textbf{Lab Testing} -- Test in representative environment
    \item \textbf{Planning} -- Schedule during maintenance window
    \item \textbf{Backup} -- Full system backup before patching
    \item \textbf{Deployment} -- Apply patch with rollback plan ready
    \item \textbf{Validation} -- Verify system functionality post-patch
    \item \textbf{Documentation} -- Record changes and outcomes
\end{enumerate}

\begin{figure}[H]
\centering
\begin{tikzpicture}[
    process/.style={rectangle, draw, thick, rounded corners=3pt, fill=otinfo!20, minimum width=1.8cm, minimum height=0.7cm, align=center, font=\scriptsize},
    decision/.style={diamond, draw, thick, fill=otwarning!20, minimum width=1.2cm, minimum height=0.8cm, align=center, font=\scriptsize, aspect=2},
    arrow/.style={->, thick, >=stealth}
]

% Top row - using absolute positions with more spacing
\node[process] (identify) at (0,0) {Identify\\Patch};
\node[process] (assess) at (3,0) {Assess\\Risk};
\node[decision] (vendor) at (6,0) {Vendor\\OK?};
\node[process] (test) at (9,0) {Lab\\Test};
\node[decision] (pass) at (12,0) {Test\\Pass?};

% Bottom row - positioned lower
\node[process] (backup) at (12,-2.5) {Backup\\System};
\node[process] (deploy) at (9,-2.5) {Deploy\\Patch};
\node[decision] (valid) at (6,-2.5) {System\\OK?};
\node[process] (doc) at (3,-2.5) {Document};

% Compensating control - far left bottom
\node[process, fill=otdanger!20] (comp) at (0,-2.5) {Compensating\\Controls};

% Main flow arrows
\draw[arrow] (identify) -- (assess);
\draw[arrow] (assess) -- (vendor);
\draw[arrow] (vendor) -- node[above, font=\tiny] {Yes} (test);
\draw[arrow] (test) -- (pass);
\draw[arrow] (pass) -- node[right, font=\tiny] {Yes} (backup);
\draw[arrow] (backup) -- (deploy);
\draw[arrow] (deploy) -- (valid);
\draw[arrow] (valid) -- node[above, font=\tiny] {Yes} (doc);

% Test fail loop (above)
\draw[arrow] (pass.north) -- ++(0,0.5) -| node[above, font=\tiny, pos=0.25] {No} (test.north);

% Rollback loop (below)
\draw[arrow] (valid.south) -- ++(0,-0.5) -| node[below, font=\tiny, pos=0.25] {Rollback} (backup.south);

% Vendor No -> Compensating Controls (straight down then left)
\draw[arrow] (vendor.south) -- (6,-1.25) -- (0,-1.25) -- node[left, font=\tiny] {No} (comp.north);

\end{tikzpicture}
\caption{OT Patch Management Process Flow}
\end{figure}

\subsection{Risk-Based Prioritization}

\begin{table}[H]
\centering
\small
\begin{tabular}{|l|l|l|}
\hline
\textbf{Priority} & \textbf{Criteria} & \textbf{Timeline} \\
\hline
Critical & Active exploitation, safety impact & Next maintenance window \\
High & Remote exploit, no authentication & Within 30 days \\
Medium & Local exploit or complex attack & Within 90 days \\
Low & Theoretical or low impact & Next scheduled outage \\
\hline
\end{tabular}
\caption{Patch Prioritization Framework}
\end{table}

% ============================================================================
\section{Vendor Considerations}
% ============================================================================

\begin{warningbox}
\textbf{Before patching any OT system:}
\begin{itemize}
    \item Check vendor compatibility list for OS patches
    \item Verify if vendor has tested the specific patch
    \item Understand impact on support agreements
    \item Review vendor security bulletins
    \item Contact vendor support if uncertain
\end{itemize}
\end{warningbox}

\subsection{Vendor Patch Programs}

Major OT vendors provide patch guidance:
\begin{itemize}
    \item \textbf{Siemens} -- Monthly security bulletins, tested patch lists
    \item \textbf{Rockwell} -- Patch qualification reports
    \item \textbf{Schneider Electric} -- Security notifications
    \item \textbf{ABB} -- Cybersecurity advisories
    \item \textbf{Honeywell} -- Security update notifications
\end{itemize}

% ============================================================================
\section{Patching Strategies}
% ============================================================================

\subsection{Strategy by System Type}

\begin{table}[H]
\centering
\small
\begin{tabularx}{\textwidth}{|l|X|}
\hline
\textbf{System Type} & \textbf{Patching Strategy} \\
\hline
Engineering Workstations & Regular patching, similar to IT \\
HMI/Operator Stations & Vendor-approved patches only \\
Historians & Patch during maintenance windows \\
SCADA Servers & Vendor coordination required \\
PLCs/RTUs & Firmware updates during outages \\
Legacy Systems (XP, 2003) & Compensating controls instead \\
Safety Systems & Extreme caution, vendor mandatory \\
\hline
\end{tabularx}
\caption{Patching Strategy by System Type}
\end{table}

\subsection{Compensating Controls}

When patching is not possible:

\begin{successbox}
\textbf{Alternative protections for unpatched systems:}
\begin{itemize}
    \item Network segmentation and firewall rules
    \item Application whitelisting
    \item Disable unnecessary services and ports
    \item Enhanced monitoring and logging
    \item Virtual patching via IPS/IDS rules
    \item Physical access controls
\end{itemize}
\end{successbox}

% ============================================================================
\section{Testing Requirements}
% ============================================================================

\subsection{Lab Environment}

\begin{itemize}
    \item Mirror production configuration
    \item Include same software versions
    \item Test integration with connected systems
    \item Validate control logic execution
    \item Verify communication protocols
    \item Test failover and redundancy
\end{itemize}

\subsection{Testing Checklist}

\begin{enumerate}
    \item System boots correctly after patch
    \item All services start properly
    \item Control logic executes as expected
    \item HMI displays update correctly
    \item Alarms and events function
    \item Communication with field devices works
    \item Historian data collection continues
    \item Performance is not degraded
    \item Redundancy/failover still works
\end{enumerate}

% ============================================================================
\section{Deployment Best Practices}
% ============================================================================

\begin{enumerate}
    \item \textbf{Maintenance windows} -- Only patch during planned outages
    \item \textbf{Full backup} -- Image-level backup before any changes
    \item \textbf{Rollback plan} -- Document exact steps to restore
    \item \textbf{Staged deployment} -- Non-critical systems first
    \item \textbf{Monitor closely} -- Watch for issues after patching
    \item \textbf{Keep records} -- Document all patch activities
    \item \textbf{Communicate} -- Inform operations team of changes
\end{enumerate}

\subsection{Air-Gapped System Patching}

\begin{warningbox}
For systems without network connectivity:
\begin{itemize}
    \item Use dedicated, scanned USB media
    \item Verify patch integrity (checksums)
    \item Scan media at secure transfer station
    \item Maintain strict chain of custody
    \item Document media handling process
\end{itemize}
\end{warningbox}

% ============================================================================
\section{Legacy System Challenges}
% ============================================================================

For systems running unsupported operating systems:

\begin{table}[H]
\centering
\small
\begin{tabularx}{\textwidth}{|l|X|}
\hline
\textbf{Challenge} & \textbf{Mitigation} \\
\hline
No security patches & Network isolation, virtual patching \\
Vulnerable services & Disable or firewall unused services \\
No vendor support & Third-party extended support \\
Incompatible AV & Application whitelisting \\
End of life & Plan and budget for migration \\
\hline
\end{tabularx}
\caption{Legacy System Mitigations}
\end{table}

% ============================================================================
\section{Patch Management Tools}
% ============================================================================

\begin{itemize}
    \item \textbf{WSUS/SCCM} -- Microsoft patch distribution (IT-focused)
    \item \textbf{OT-specific tools} -- Vendor patch management solutions
    \item \textbf{Asset inventory} -- Know what needs patching
    \item \textbf{Vulnerability scanners} -- Identify missing patches
    \item \textbf{Change management} -- Track and approve changes
\end{itemize}

\begin{infobox}
Consider OT-specific patch management solutions that understand industrial protocols and can coordinate with maintenance schedules.
\end{infobox}

% ============================================================================
\section{Summary}
% ============================================================================

\begin{definitionbox}{Key Takeaways}
\begin{itemize}
    \item \textbf{Test first} -- Never patch production without lab testing
    \item \textbf{Vendor approval} -- Check compatibility before patching
    \item \textbf{Maintenance windows} -- Patch only during planned outages
    \item \textbf{Backup always} -- Full backup before any changes
    \item \textbf{Compensating controls} -- Protect what cannot be patched
    \item \textbf{Risk-based} -- Prioritize by criticality and exposure
    \item \textbf{Document everything} -- Maintain patch records
\end{itemize}
\end{definitionbox}

% ============================================================================
\section{Further Reading}
% ============================================================================

\subsection*{Standards}

\begin{itemize}
    \item \textbf{IEC 62443-2-3} -- Patch management in IACS\\
          \url{https://webstore.iec.ch/publication/7030}
    \item \textbf{NIST SP 800-82} -- Guide to ICS Security\\
          \url{https://csrc.nist.gov/publications/detail/sp/800-82/rev-3/final}
\end{itemize}

\subsection*{Resources}

\begin{itemize}
    \item \textbf{CISA -- ICS Patch Management}\\
          \url{https://www.cisa.gov/topics/industrial-control-systems}
\end{itemize}

\vfill
\begin{center}
\textit{Part of the OT Security Learning Series}
\end{center}

\end{document}
