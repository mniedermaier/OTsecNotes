% ============================================================================
%  420-ot-attack-vectors - OT Security Learning Resource
% ============================================================================

\documentclass[11pt,a4paper]{article}
\usepackage{otsec-template}
\usepackage{float}

% Define colors for TikZ
\colorlet{otprimary}{primary}
\colorlet{otaccent}{accent}
\colorlet{otsuccess}{success}
\colorlet{otwarning}{warning}
\colorlet{otdanger}{danger}
\colorlet{otinfo}{info}

\begin{document}

\maketitlepage
    {OT Attack Vectors}
    {Common attack paths targeting industrial control systems}
    {OT Security Learning Series}
    {Document 420 \quad|\quad January 2026}
    {Matthias Niedermaier}

\tableofcontents
\newpage

% ============================================================================
\section{Introduction}
% ============================================================================

\begin{infobox}
Attack vectors are the paths or methods attackers use to gain access to OT systems. Understanding these vectors is essential for implementing effective defenses. OT environments face both IT-style attacks and OT-specific threats targeting industrial protocols and processes.
\end{infobox}

Attack vector categories:
\begin{itemize}
    \item \textbf{Network-based} -- Exploiting network connectivity
    \item \textbf{Physical} -- Direct access to equipment
    \item \textbf{Supply chain} -- Compromised vendors or components
    \item \textbf{Human} -- Social engineering and insider threats
    \item \textbf{Wireless} -- Exploiting RF communications
\end{itemize}

\begin{figure}[H]
\centering
\begin{tikzpicture}[
    zone/.style={rectangle, draw, thick, rounded corners=5pt, minimum width=3.5cm, minimum height=1cm, align=center, font=\small},
    attack/.style={->, thick, >=stealth, color=otdanger},
    label/.style={font=\tiny, color=otdanger}
]

% OT Environment (center)
\node[zone, fill=otsuccess!20, minimum width=4cm, minimum height=2.5cm] (ot) at (0,0) {OT\\Environment\\(Target)};

% Attack vectors pointing to OT
% Network (top)
\node[font=\scriptsize] (net) at (0,3) {Network-based};
\draw[attack] (net) -- node[right, label] {VPN, Firewall} (0,1.4);

% Physical (left)
\node[font=\scriptsize] (phys) at (-4,0) {Physical};
\draw[attack] (phys) -- node[above, label] {USB, Serial} (-2.1,0);

% Supply Chain (right)
\node[font=\scriptsize] (supply) at (5,0) {Supply Chain};
\draw[attack] (supply) -- node[above, label, pos=0.4] {Vendor, Updates} (2.1,0);

% Human (bottom left)
\node[font=\scriptsize] (human) at (-3,-2.5) {Human};
\draw[attack] (human) -- node[below, label, sloped] {Phishing, Insider} (-1,-1.4);

% Wireless (bottom right)
\node[font=\scriptsize] (wireless) at (3,-2.5) {Wireless};
\draw[attack] (wireless) -- node[below, label, sloped] {WiFi, RF} (1,-1.4);

\end{tikzpicture}
\caption{Attack Vectors Targeting OT Environments}
\end{figure}

% ============================================================================
\section{Network-Based Vectors}
% ============================================================================

\subsection{IT/OT Boundary Attacks}

\begin{table}[H]
\centering
\small
\begin{tabularx}{\textwidth}{|l|X|}
\hline
\textbf{Vector} & \textbf{Description} \\
\hline
Firewall misconfiguration & Overly permissive rules allowing lateral movement \\
VPN compromise & Stolen credentials or vulnerabilities in VPN \\
Jump server exploitation & Compromising shared administration points \\
Historian pivoting & Using data historians as bridge to control networks \\
DMZ bypass & Exploiting poorly segmented DMZ architecture \\
\hline
\end{tabularx}
\caption{IT/OT Boundary Attack Vectors}
\end{table}

\subsection{Protocol Exploitation}

\begin{dangerbox}
\textbf{Industrial protocols lack authentication:}
\begin{itemize}
    \item Modbus -- No authentication, commands accepted from any source
    \item DNP3 -- Optional security features rarely enabled
    \item EtherNet/IP -- CIP protocol has limited security
    \item S7comm -- Legacy protocol with weak authentication
    \item OPC Classic -- DCOM-based with known vulnerabilities
\end{itemize}
\end{dangerbox}

\subsection{Man-in-the-Middle Attacks}

\begin{itemize}
    \item ARP spoofing to intercept OT traffic
    \item Modifying commands between HMI and PLCs
    \item Injecting false data into historian
    \item Intercepting engineering uploads/downloads
\end{itemize}

% ============================================================================
\section{Remote Access Vectors}
% ============================================================================

\begin{table}[H]
\centering
\small
\begin{tabularx}{\textwidth}{|l|X|l|}
\hline
\textbf{Vector} & \textbf{Attack Method} & \textbf{Risk} \\
\hline
VPN credentials & Phishing, credential stuffing & \riskhigh \\
RDP exposure & Brute force, BlueKeep exploits & \riskcritical \\
Vendor backdoors & Default/hardcoded credentials & \riskhigh \\
TeamViewer/AnyDesk & Compromised remote tools & \riskhigh \\
Cellular modems & Exposed management interfaces & \riskmedium \\
\hline
\end{tabularx}
\caption{Remote Access Attack Vectors}
\end{table}

\begin{warningbox}
Remote access is one of the most exploited vectors in OT attacks. The Ukraine power grid attacks (2015/2016) used stolen VPN credentials to gain initial access.
\end{warningbox}

% ============================================================================
\section{Physical Vectors}
% ============================================================================

\subsection{Direct Physical Access}

\begin{itemize}
    \item \textbf{USB devices} -- Malware delivery (Stuxnet spread via USB)
    \item \textbf{Serial ports} -- Direct connection to PLCs/RTUs
    \item \textbf{Network jacks} -- Connecting rogue devices
    \item \textbf{Exposed equipment} -- Substations, pump stations
    \item \textbf{Maintenance laptops} -- Infected contractor equipment
\end{itemize}

\subsection{Removable Media Risks}

\begin{table}[H]
\centering
\small
\begin{tabular}{|l|l|}
\hline
\textbf{Media Type} & \textbf{Risk Scenario} \\
\hline
USB flash drives & Malware autorun, BadUSB attacks \\
External hard drives & Infected backup restoration \\
SD cards & Compromised firmware updates \\
CDs/DVDs & Legacy systems without USB \\
\hline
\end{tabular}
\caption{Removable Media Attack Vectors}
\end{table}

% ============================================================================
\section{Supply Chain Vectors}
% ============================================================================

\begin{dangerbox}
\textbf{Supply chain compromises are difficult to detect:}
\begin{itemize}
    \item Trojanized software updates (SolarWinds-style)
    \item Compromised firmware from manufacturer
    \item Malicious code in third-party libraries
    \item Counterfeit hardware with backdoors
    \item Compromised vendor remote access
\end{itemize}
\end{dangerbox}

\subsection{Vendor Access Risks}

\begin{itemize}
    \item Persistent vendor VPN connections
    \item Shared credentials among vendor staff
    \item Unmonitored maintenance sessions
    \item Vendor laptops connecting to OT networks
    \item Cloud-based vendor management platforms
\end{itemize}

% ============================================================================
\section{Human Vectors}
% ============================================================================

\subsection{Social Engineering}

\begin{table}[H]
\centering
\small
\begin{tabularx}{\textwidth}{|l|X|}
\hline
\textbf{Technique} & \textbf{OT-Specific Example} \\
\hline
Phishing & Fake vendor security bulletin with malicious link \\
Spear phishing & Targeting control engineers with project files \\
Pretexting & Impersonating vendor support for credentials \\
Baiting & Dropping infected USB near control room \\
Tailgating & Following authorized person into secure area \\
\hline
\end{tabularx}
\caption{Social Engineering Techniques}
\end{table}

\subsection{Insider Threats}

\begin{itemize}
    \item Disgruntled employees with system access
    \item Contractors with excessive privileges
    \item Unintentional errors by operators
    \item Knowledge transfer during layoffs
    \item Credential sharing among staff
\end{itemize}

% ============================================================================
\section{Wireless Vectors}
% ============================================================================

\begin{table}[H]
\centering
\small
\begin{tabularx}{\textwidth}{|l|X|}
\hline
\textbf{Vector} & \textbf{Description} \\
\hline
Rogue access points & Unauthorized WiFi bridging IT and OT \\
WiFi attacks & WPA2 cracking, evil twin APs \\
Bluetooth & Exploiting industrial Bluetooth devices \\
Cellular/LTE & Attacking exposed cellular modems \\
RF protocols & ZigBee, LoRa, WirelessHART vulnerabilities \\
Radio jamming & Disrupting wireless control communications \\
\hline
\end{tabularx}
\caption{Wireless Attack Vectors}
\end{table}

% ============================================================================
\section{Attack Chain Example}
% ============================================================================

\begin{figure}[H]
\centering
\begin{tikzpicture}[
    node distance=0.3cm,
    stage/.style={rectangle, draw, thick, rounded corners=3pt, minimum width=2.8cm, minimum height=0.8cm, align=center, font=\scriptsize},
    arrow/.style={->, thick, >=stealth}
]

% Attack stages - top row
\node[stage, fill=otinfo!20] (phish) {1. Phishing\\Email};
\node[stage, fill=otinfo!20, right=0.4cm of phish] (foot) {2. Establish\\Foothold};
\node[stage, fill=otinfo!20, right=0.4cm of foot] (lateral) {3. Lateral\\Movement};
\node[stage, fill=otwarning!30, right=0.4cm of lateral] (boundary) {4. IT/OT\\Boundary};

% Attack stages - bottom row
\node[stage, fill=otwarning!30, below=0.8cm of boundary] (recon) {5. OT\\Recon};
\node[stage, fill=otdanger!20, left=0.4cm of recon] (target) {6. Target\\ID};
\node[stage, fill=otdanger!20, left=0.4cm of target] (develop) {7. Develop\\Payload};
\node[stage, fill=otdanger!40, left=0.4cm of develop] (execute) {8. Execute\\Attack};

% Arrows
\draw[arrow] (phish) -- (foot);
\draw[arrow] (foot) -- (lateral);
\draw[arrow] (lateral) -- (boundary);
\draw[arrow] (boundary) -- (recon);
\draw[arrow] (recon) -- (target);
\draw[arrow] (target) -- (develop);
\draw[arrow] (develop) -- (execute);

% Zone labels
\node[above=0.3cm of foot, font=\tiny\bfseries, color=otinfo] {IT Network};
\node[below=0.3cm of target, font=\tiny\bfseries, color=otdanger] {OT Network};

\end{tikzpicture}
\caption{Typical OT Attack Chain Progression}
\end{figure}

\begin{successbox}
\textbf{Typical OT attack progression:}
\begin{enumerate}
    \item \textbf{Initial Access} -- Phishing email to corporate user
    \item \textbf{Establish Foothold} -- Malware on IT workstation
    \item \textbf{Lateral Movement} -- Pivot through IT network
    \item \textbf{IT/OT Boundary} -- Exploit jump server or historian
    \item \textbf{OT Reconnaissance} -- Map industrial network
    \item \textbf{Target Identification} -- Find critical controllers
    \item \textbf{Develop Capability} -- Create OT-specific payload
    \item \textbf{Execute Attack} -- Manipulate physical process
\end{enumerate}
\end{successbox}

% ============================================================================
\section{Defensive Priorities}
% ============================================================================

\begin{table}[H]
\centering
\small
\begin{tabularx}{\textwidth}{|l|X|}
\hline
\textbf{Attack Vector} & \textbf{Primary Defense} \\
\hline
Remote access & MFA, jump servers, session monitoring \\
Network-based & Segmentation, firewalls, IDS \\
Protocol attacks & Network monitoring, protocol validation \\
Physical access & Access controls, USB restrictions \\
Supply chain & Vendor management, integrity verification \\
Social engineering & Security awareness training \\
Wireless & WPA3-Enterprise, WIDS, RF monitoring \\
\hline
\end{tabularx}
\caption{Defenses by Attack Vector}
\end{table}

% ============================================================================
\section{Summary}
% ============================================================================

\begin{definitionbox}{Key Takeaways}
\begin{itemize}
    \item \textbf{Remote access} -- Most commonly exploited initial vector
    \item \textbf{Protocol weakness} -- Industrial protocols lack authentication
    \item \textbf{Physical access} -- USB and direct connections remain threats
    \item \textbf{Supply chain} -- Trusted vendors can be attack paths
    \item \textbf{Defense in depth} -- No single control stops all vectors
    \item \textbf{Monitor boundaries} -- IT/OT interface is critical chokepoint
\end{itemize}
\end{definitionbox}

% ============================================================================
\section{Further Reading}
% ============================================================================

\subsection*{Resources}

\begin{itemize}
    \item \textbf{MITRE ATT\&CK for ICS}\\
          \url{https://attack.mitre.org/techniques/ics/}
    \item \textbf{CISA -- ICS Attack Vectors}\\
          \url{https://www.cisa.gov/topics/industrial-control-systems}
    \item \textbf{NIST SP 800-82} -- Guide to ICS Security\\
          \url{https://csrc.nist.gov/publications/detail/sp/800-82/rev-3/final}
\end{itemize}

\vfill
\begin{center}
\textit{Part of the OT Security Learning Series}
\end{center}

\end{document}
