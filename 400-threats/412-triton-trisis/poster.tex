% ============================================================================
%  TRITON/TRISIS - Poster / Cheat Sheet
% ============================================================================

\documentclass[9pt,a4paper]{extarticle}
\usepackage{otsec-poster}
\usepackage{float}

\begin{document}

\makepostertitle
    {TRITON / TRISIS}
    {The First Cyber Attack Targeting Safety Instrumented Systems}
    {Poster 412}
    {Matthias Niedermaier}

\begin{multicols}{2}

\section{\textcolor{accent}{\faIcon{info-circle}}\hspace{0.4em}Overview}

TRITON (also TRISIS/HatMan), discovered 2017, is the \textbf{first known malware targeting Safety Instrumented Systems (SIS)}. It crossed a critical line -- targeting systems designed to prevent catastrophic accidents and protect human life at a Middle Eastern petrochemical facility.

\section{\textcolor{accent}{\faIcon{list-ul}}\hspace{0.4em}Key Facts}

\begin{center}
\rowcolors{2}{lightgray}{white}
\begin{tabular}{p{2.2cm}p{4.3cm}}
\rowcolor{primary}
\textcolor{white}{\faIcon{list-ul}\hspace{0.2em}\bfseries Attribute} & \textcolor{white}{\faIcon{info-circle}\hspace{0.2em}\bfseries Details} \\
\midrule
Discovery & December 2017 \\
Target & Middle Eastern petrochemical \\
Systems & Schneider Triconex SIS \\
Goal & Disable safety systems \\
Outcome & Attack failed; safe shutdown \\
Attribution & CNIIHM (Russian govt.) \\
\end{tabular}
\end{center}

\posterdanger{
\textcolor{danger}{\faIcon{skull-crossbones}}\hspace{0.2em} TRITON demonstrates \textbf{willingness to potentially cause loss of life}. By targeting the last line of defense against industrial disasters, it represents a significant escalation in cyber attacks on industrial systems.
}

\section{\textcolor{accent}{\faIcon{hard-hat}}\hspace{0.4em}What is a SIS?}

A Safety Instrumented System is an \textbf{autonomous control system} that brings a process to a safe state when dangerous conditions are detected. It operates independently from the BPCS and is the last automated defense.

\subsection{\textcolor{accent}{\faIcon{cogs}}\hspace{0.3em}SIS Functions}

\begin{itemize}
    \item \textcolor{warning}{\faIcon{hard-hat}}\hspace{0.2em} \textbf{Emergency Shutdown (ESD)} -- Stop dangerous processes
    \item \textcolor{warning}{\faIcon{hard-hat}}\hspace{0.2em} \textbf{Fire \& Gas Detection} -- Trigger alarms/protection
    \item \textcolor{warning}{\faIcon{hard-hat}}\hspace{0.2em} \textbf{Burner Management} -- Safe startup/shutdown
    \item \textcolor{warning}{\faIcon{hard-hat}}\hspace{0.2em} \textbf{HIPPS} -- High Integrity Pressure Protection
\end{itemize}

\subsection{\textcolor{accent}{\faIcon{layer-group}}\hspace{0.3em}Protection Layers}

\begin{enumerate}
    \item \faIcon{drafting-compass}\hspace{0.2em}Process Design (inherently safer)
    \item \faIcon{sliders-h}\hspace{0.2em}Basic Process Control System (BPCS)
    \item \faIcon{bell}\hspace{0.2em}Alarms and Operator Response
    \item \textcolor{warning}{\faIcon{hard-hat}}\hspace{0.2em} \textbf{Safety Instrumented System (SIS)} $\leftarrow$ \textit{TRITON target}
    \item \faIcon{compress-arrows-alt}\hspace{0.2em}Physical Protection (relief valves, rupture discs)
\end{enumerate}

\posterwarning{
\textcolor{warning}{\faIcon{hard-hat}}\hspace{0.2em} The SIS is the \textbf{last automated barrier} before physical protection and potential disaster. Compromising it could allow dangerous conditions to escalate: explosions, toxic releases, loss of life.
}

\section{\textcolor{accent}{\faIcon{link}}\hspace{0.4em}Attack Chain}

\begin{center}
\begin{tikzpicture}[
    step/.style={rectangle, draw=#1!40, thick, fill=#1!5, minimum height=0.38cm, rounded corners=2pt, font=\scriptsize, text width=4.5cm, align=left},
    step/.default={otdanger},
    num/.style={circle, fill=#1, text=white, font=\scriptsize\bfseries, minimum size=0.33cm, inner sep=0pt},
    num/.default={otdanger},
]
    \node[num=otwarning] (n1) at (0,0) {1};
    \node[step=otwarning, right=3pt of n1] (s1) {Compromise corporate IT ($\sim$2014)};
    \node[num=otwarning, below=2pt of n1] (n2) {2};
    \node[step=otwarning, right=3pt of n2] (s2) {Lateral movement IT $\rightarrow$ OT};
    \node[num=otdanger, below=2pt of n2] (n3) {3};
    \node[step=otdanger, right=3pt of n3] (s3) {Access SIS engineering workstation};
    \node[num=otdanger, below=2pt of n3] (n4) {4};
    \node[step=otdanger, right=3pt of n4] (s4) {Study Triconex architecture};
    \node[num=otdanger, below=2pt of n4] (n5) {5};
    \node[step=otdanger, right=3pt of n5] (s5) {Deploy custom Triconex payload};
    \node[num=otsuccess, below=2pt of n5] (n6) {6};
    \node[step=otsuccess, right=3pt of n6] (s6) {\faIcon{check}\hspace{0.1em}Bug triggers safe shutdown};

    \foreach \i/\j in {1/2,2/3,3/4,4/5,5/6} {
        \draw[thick, otdanger!40] (n\i.south) -- (n\j.north);
    }
\end{tikzpicture}
\end{center}

\section{\textcolor{accent}{\faIcon{bug}}\hspace{0.4em}Malware Components}

\begin{center}
\rowcolors{2}{lightgray}{white}
\begin{tabular}{p{2.2cm}p{4.3cm}}
\rowcolor{primary}
\textcolor{white}{\faIcon{bug}\hspace{0.2em}\bfseries Component} & \textcolor{white}{\faIcon{cogs}\hspace{0.2em}\bfseries Purpose} \\
\midrule
trilog.exe & Main exec (disguised as legit) \\
library.zip & Python libraries for execution \\
inject.bin & Shellcode for Triconex controller \\
imain.bin & Main malicious logic \\
\end{tabular}
\end{center}

The malware communicated with Triconex via the \textbf{TriStation protocol}, could read/write controller memory, upload custom code, and potentially disable safety functions while hiding from operators.

\section{\textcolor{accent}{\faIcon{check-circle}}\hspace{0.4em}Why the Attack Failed}

\postersuccess{
The malicious code contained a \textbf{bug} that caused the safety controller to detect an invalid state and initiate a \textbf{safe shutdown}. This triggered an investigation that uncovered the intrusion. The SIS performed its designed function -- failing safely.
}

\section{\textcolor{accent}{\faIcon{skull-crossbones}}\hspace{0.4em}Potential Consequences}

Had the attack succeeded:

\begin{itemize}
    \item \textcolor{danger}{\faIcon{skull-crossbones}}\hspace{0.2em} Safety functions disabled silently
    \item \textcolor{danger}{\faIcon{skull-crossbones}}\hspace{0.2em} Hazardous conditions hidden from operators
    \item \textcolor{danger}{\faIcon{skull-crossbones}}\hspace{0.2em} Possible explosions, fires, or toxic releases
    \item \textcolor{danger}{\faIcon{skull-crossbones}}\hspace{0.2em} Environmental contamination
    \item \textcolor{danger}{\faIcon{skull-crossbones}}\hspace{0.2em} Potential loss of life
\end{itemize}

\section{\textcolor{accent}{\faIcon{search}}\hspace{0.4em}Detection Indicators}

\begin{itemize}
    \item \faIcon{network-wired}\hspace{0.2em}Unauthorized TriStation protocol communications
    \item \faIcon{file-alt}\hspace{0.2em}Unexpected files on SIS engineering workstations
    \item \faIcon{toggle-on}\hspace{0.2em}Safety controller in programming mode unexpectedly
    \item \faIcon{chart-bar}\hspace{0.2em}Anomalous network traffic to/from safety systems
    \item \faIcon{bell}\hspace{0.2em}Unexpected safety system shutdowns or alarms
    \item \faIcon{user-lock}\hspace{0.2em}Unauthorized access to SIS programming tools
\end{itemize}

\section{\textcolor{accent}{\faIcon{bullhorn}}\hspace{0.4em}Industry Response}

\begin{itemize}
    \item \faIcon{tools}\hspace{0.2em}Schneider Electric released security advisories and patches
    \item \faIcon{bullhorn}\hspace{0.2em}ICS-CERT issued alerts and recommended mitigations
    \item \faIcon{users}\hspace{0.2em}Industry groups developed SIS-specific security guidelines
    \item \faIcon{book}\hspace{0.2em}Increased focus on safety system cybersecurity in IEC 61511
    \item \faIcon{project-diagram}\hspace{0.2em}New emphasis on physical separation of SIS from BPCS
\end{itemize}

\section{\textcolor{accent}{\faIcon{shield-alt}}\hspace{0.4em}Defense Recommendations}

\begin{enumerate}
    \item \faIcon{lock}\hspace{0.2em}\textbf{Isolate SIS networks} -- Physical/strong logical separation
    \item \faIcon{user-shield}\hspace{0.2em}\textbf{Restrict engineering access} -- Limit who programs SIS
    \item \faIcon{eye}\hspace{0.2em}\textbf{Monitor SIS changes} -- Detect unauthorized mods
    \item \faIcon{key}\hspace{0.2em}\textbf{Hardware key switches} -- Physical programming mode
    \item \faIcon{clipboard-check}\hspace{0.2em}\textbf{Change management} -- Approve all SIS changes
    \item \faIcon{search}\hspace{0.2em}\textbf{Integrity checks} -- Verify logic against baseline
    \item \faIcon{ban}\hspace{0.2em}\textbf{No direct connectivity} -- SIS to business networks
\end{enumerate}

\postertip{
SIS engineering workstations should \textbf{not be connected to general OT networks}. Access to safety system programming should require physical presence and multi-person authorization. Use hardware key switches to control programming mode. Regularly compare SIS logic against approved baselines. Monitor for unauthorized TriStation protocol communications. The SIS saved the day here -- proper safety system design is the ultimate backstop.
}

\end{multicols}

\end{document}
