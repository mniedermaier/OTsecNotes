% ============================================================================
%  421-cyber-kill-chain - OT Security Learning Resource
% ============================================================================

\documentclass[11pt,a4paper]{article}
\usepackage{otsec-template}
\usepackage{float}

% Define colors for TikZ
\colorlet{otprimary}{primary}
\colorlet{otaccent}{accent}
\colorlet{otsuccess}{success}
\colorlet{otwarning}{warning}
\colorlet{otdanger}{danger}
\colorlet{otinfo}{info}

\begin{document}

\maketitlepage
    {Cyber Kill Chain for OT}
    {Understanding attack methodologies in industrial environments}
    {OT Security Learning Series}
    {Document 421 \quad|\quad January 2026}
    {Matthias Niedermaier}

\tableofcontents
\newpage

% ============================================================================
\section{Introduction}
% ============================================================================

\begin{infobox}
The Cyber Kill Chain is a framework for understanding the stages of a cyberattack. Originally developed for IT environments, specialized models have emerged for OT/ICS that account for the unique characteristics of industrial control systems.
\end{infobox}

Why kill chains matter for OT:
\begin{itemize}
    \item \textbf{Structured defense} -- Identify where to detect and disrupt attacks
    \item \textbf{Threat intelligence} -- Map adversary behavior to stages
    \item \textbf{Gap analysis} -- Find weaknesses in defensive coverage
    \item \textbf{Incident response} -- Understand attack progression
\end{itemize}

\begin{warningbox}
OT attacks typically require more stages than IT attacks because adversaries must understand physical processes before causing impact. This extended timeline provides more opportunities for detection.
\end{warningbox}

% ============================================================================
\section{Lockheed Martin Cyber Kill Chain}
% ============================================================================

The original Cyber Kill Chain, developed by Lockheed Martin, defines seven stages of an intrusion:

\begin{figure}[H]
\centering
\begin{tikzpicture}[
    stage/.style={rectangle, draw, thick, rounded corners=3pt, minimum width=2.2cm, minimum height=0.8cm, align=center, font=\scriptsize},
    arrow/.style={->, thick, >=stealth}
]

% Stages in a flow
\node[stage, fill=otinfo!30] (recon) at (0,0) {1. Recon};
\node[stage, fill=otinfo!30] (weap) at (2.5,0) {2. Weaponize};
\node[stage, fill=otwarning!30] (deliv) at (5,0) {3. Deliver};
\node[stage, fill=otwarning!30] (exploit) at (7.5,0) {4. Exploit};
\node[stage, fill=otdanger!30] (install) at (10,0) {5. Install};
\node[stage, fill=otdanger!30] (c2) at (5,-1.5) {6. Command \& Control};
\node[stage, fill=otdanger!40] (action) at (10,-1.5) {7. Actions on Objectives};

% Arrows
\draw[arrow] (recon) -- (weap);
\draw[arrow] (weap) -- (deliv);
\draw[arrow] (deliv) -- (exploit);
\draw[arrow] (exploit) -- (install);
\draw[arrow] (install) -- (10,-0.75) -- (c2);
\draw[arrow] (c2) -- (action);

\end{tikzpicture}
\caption{Lockheed Martin Cyber Kill Chain}
\end{figure}

\begin{table}[H]
\centering
\small
\begin{tabularx}{\textwidth}{|l|X|X|}
\hline
\textbf{Stage} & \textbf{Description} & \textbf{OT Example} \\
\hline
1. Reconnaissance & Gather information about target & Research SCADA vendors, job postings \\
2. Weaponization & Create malware/exploit & Develop PLC-specific payload \\
3. Delivery & Transmit weapon to target & Spear phishing, USB drop \\
4. Exploitation & Trigger vulnerability & Exploit unpatched HMI \\
5. Installation & Install persistent access & Deploy RAT on engineering WS \\
6. Command \& Control & Establish communication & C2 via encrypted tunnel \\
7. Actions & Achieve objectives & Manipulate process, steal data \\
\hline
\end{tabularx}
\caption{Kill Chain Stages with OT Examples}
\end{table}

% ============================================================================
\section{SANS ICS Kill Chain}
% ============================================================================

\begin{definitionbox}{Two-Stage Model}
The SANS ICS Kill Chain recognizes that attacks on industrial control systems require two distinct phases: initial intrusion (similar to IT) followed by ICS-specific attack development and execution.
\end{definitionbox}

\begin{figure}[H]
\centering
\begin{tikzpicture}[
    stage/.style={rectangle, draw, thick, rounded corners=3pt, minimum width=1.6cm, minimum height=0.7cm, align=center, font=\tiny},
    phase/.style={rectangle, draw, thick, rounded corners=5pt, minimum height=1.2cm, align=center, font=\small\bfseries},
    arrow/.style={->, thick, >=stealth}
]

% Stage 1 - IT Intrusion
\node[phase, fill=otinfo!20, minimum width=13cm] (phase1) at (6,2) {};
\node[above] at (6,2.7) {\small\bfseries Stage 1: IT/Corporate Intrusion};

\node[stage, fill=otinfo!40] (s1recon) at (0.5,2) {Recon};
\node[stage, fill=otinfo!40] (s1weap) at (2.7,2) {Weaponize};
\node[stage, fill=otinfo!40] (s1deliv) at (4.9,2) {Deliver};
\node[stage, fill=otinfo!40] (s1exploit) at (7.1,2) {Exploit};
\node[stage, fill=otinfo!40] (s1install) at (9.3,2) {Install};
\node[stage, fill=otinfo!40] (s1c2) at (11.5,2) {C2};

\draw[arrow] (s1recon) -- (s1weap);
\draw[arrow] (s1weap) -- (s1deliv);
\draw[arrow] (s1deliv) -- (s1exploit);
\draw[arrow] (s1exploit) -- (s1install);
\draw[arrow] (s1install) -- (s1c2);

% Stage 2 - ICS Attack
\node[phase, fill=otdanger!20, minimum width=13cm] (phase2) at (6,0) {};
\node[above] at (6,0.7) {\small\bfseries Stage 2: ICS Attack Development \& Execution};

\node[stage, fill=otdanger!40] (s2dev) at (1,0) {Develop};
\node[stage, fill=otdanger!40] (s2test) at (3.5,0) {Test};
\node[stage, fill=otdanger!40] (s2deliver) at (6,0) {Deliver};
\node[stage, fill=otdanger!40] (s2install) at (8.5,0) {Install};
\node[stage, fill=otdanger!40] (s2exec) at (11,0) {Execute};

\draw[arrow] (s2dev) -- (s2test);
\draw[arrow] (s2test) -- (s2deliver);
\draw[arrow] (s2deliver) -- (s2install);
\draw[arrow] (s2install) -- (s2exec);

% Connection between stages
\draw[arrow, dashed, thick] (s1c2.south) -- ++(0,-0.4) -| (s2dev.north);

\end{tikzpicture}
\caption{SANS ICS Kill Chain (Two-Stage Model)}
\end{figure}

\subsection{Stage 1: IT/Corporate Intrusion}

Follows the traditional kill chain to gain access to the corporate network:
\begin{itemize}
    \item Target IT systems, email, corporate network
    \item Goal: Establish foothold and pivot toward OT
    \item May take weeks to months
\end{itemize}

\subsection{Stage 2: ICS Attack Development}

\begin{table}[H]
\centering
\small
\begin{tabularx}{\textwidth}{|l|X|}
\hline
\textbf{Step} & \textbf{Description} \\
\hline
Develop & Create ICS-specific attack capability (requires process knowledge) \\
Test & Validate attack in lab environment or simulator \\
Deliver & Transfer attack tools to ICS network \\
Install/Modify & Deploy on ICS components, modify control logic \\
Execute & Trigger attack to cause physical impact \\
\hline
\end{tabularx}
\caption{Stage 2 ICS Attack Steps}
\end{table}

\begin{successbox}
\textbf{Key insight:} Stage 2 requires significant ICS knowledge. Attackers must understand the physical process, control logic, and potential impacts. This typically requires reconnaissance within the OT environment, extending the attack timeline.
\end{successbox}

% ============================================================================
\section{Real-World Examples}
% ============================================================================

\subsection{Stuxnet Kill Chain}

\begin{table}[H]
\centering
\small
\begin{tabularx}{\textwidth}{|l|X|}
\hline
\textbf{Stage} & \textbf{Stuxnet Activity} \\
\hline
Reconnaissance & Years of intelligence gathering on Iranian nuclear program \\
Weaponization & Custom malware with 4 zero-days, Siemens S7 payloads \\
Delivery & USB drives, possibly via contractors \\
Exploitation & Windows vulnerabilities, WinCC database exploit \\
Installation & Rootkit on Windows, inject code into PLCs \\
C2 & Peer-to-peer updates, limited external C2 \\
ICS Attack & Modified centrifuge speeds while hiding from operators \\
\hline
\end{tabularx}
\caption{Stuxnet Mapped to Kill Chain}
\end{table}

\subsection{Ukraine Power Grid (2015)}

\begin{table}[H]
\centering
\small
\begin{tabularx}{\textwidth}{|l|X|}
\hline
\textbf{Stage} & \textbf{Attack Activity} \\
\hline
Reconnaissance & Months of network mapping, learning SCADA systems \\
Weaponization & BlackEnergy malware, KillDisk wiper \\
Delivery & Spear phishing with malicious Word documents \\
Exploitation & Macro execution, credential theft \\
Installation & Persistent access on corporate and SCADA networks \\
C2 & VPN connections using stolen credentials \\
ICS Attack & Remote HMI access, opened breakers, wiped systems \\
\hline
\end{tabularx}
\caption{Ukraine Attack Mapped to Kill Chain}
\end{table}

% ============================================================================
\section{Defensive Strategies by Stage}
% ============================================================================

\begin{table}[H]
\centering
\small
\begin{tabularx}{\textwidth}{|l|X|}
\hline
\textbf{Stage} & \textbf{Defensive Measures} \\
\hline
Reconnaissance & Limit public information, monitor for scanning \\
Weaponization & Threat intelligence, malware analysis \\
Delivery & Email filtering, USB controls, web filtering \\
Exploitation & Patching, application whitelisting, hardening \\
Installation & Endpoint detection, integrity monitoring \\
C2 & Network monitoring, DNS analysis, egress filtering \\
ICS Attack & Process monitoring, anomaly detection, safety systems \\
\hline
\end{tabularx}
\caption{Defenses Mapped to Kill Chain Stages}
\end{table}

\begin{warningbox}
\textbf{Defense in depth:} No single control stops all attacks. Implement detection and prevention at multiple stages to increase the chances of disrupting an attack before impact.
\end{warningbox}

% ============================================================================
\section{MITRE ATT\&CK for ICS}
% ============================================================================

\begin{infobox}
MITRE ATT\&CK for ICS provides a more detailed framework with specific tactics, techniques, and procedures (TTPs) observed in real ICS attacks. It complements kill chain models with actionable threat intelligence.
\end{infobox}

Key ICS-specific tactics:
\begin{itemize}
    \item \textbf{Initial Access} -- How attackers enter OT networks
    \item \textbf{Execution} -- Running malicious code on ICS
    \item \textbf{Persistence} -- Maintaining access
    \item \textbf{Evasion} -- Avoiding detection
    \item \textbf{Discovery} -- Learning OT environment
    \item \textbf{Lateral Movement} -- Moving through OT network
    \item \textbf{Collection} -- Gathering process data
    \item \textbf{Command \& Control} -- Maintaining communication
    \item \textbf{Inhibit Response} -- Preventing operator action
    \item \textbf{Impair Process} -- Disrupting physical operations
    \item \textbf{Impact} -- Causing damage or safety incidents
\end{itemize}

% ============================================================================
\section{Summary}
% ============================================================================

\begin{definitionbox}{Key Takeaways}
\begin{itemize}
    \item \textbf{Traditional kill chain} -- 7 stages from recon to actions
    \item \textbf{ICS kill chain} -- Two-stage model (IT intrusion + ICS attack)
    \item \textbf{Extended timeline} -- OT attacks require process knowledge
    \item \textbf{Detection opportunities} -- More stages mean more chances to detect
    \item \textbf{Defense in depth} -- Layer controls across all stages
    \item \textbf{MITRE ATT\&CK} -- Detailed TTPs for threat-informed defense
\end{itemize}
\end{definitionbox}

% ============================================================================
\section{Further Reading}
% ============================================================================

\subsection*{Frameworks}

\begin{itemize}
    \item \textbf{MITRE ATT\&CK for ICS}\\
          \url{https://attack.mitre.org/techniques/ics/}
    \item \textbf{Lockheed Martin Cyber Kill Chain}\\
          \url{https://www.lockheedmartin.com/en-us/capabilities/cyber/cyber-kill-chain.html}
\end{itemize}

\subsection*{Resources}

\begin{itemize}
    \item \textbf{SANS ICS Kill Chain Paper}\\
          \url{https://www.sans.org/white-papers/36297/}
    \item \textbf{CISA ICS Security}\\
          \url{https://www.cisa.gov/topics/industrial-control-systems}
\end{itemize}

\vfill
\begin{center}
\textit{Part of the OT Security Learning Series}
\end{center}

\end{document}
