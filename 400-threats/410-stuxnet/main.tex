% ============================================================================
%  Stuxnet - OT Security Learning Resource
% ============================================================================

\documentclass[11pt,a4paper]{article}
\usepackage{otsec-template}
\usepackage{float}

% Define colors for TikZ
\colorlet{otprimary}{primary}
\colorlet{otaccent}{accent}
\colorlet{otsuccess}{success}
\colorlet{otwarning}{warning}
\colorlet{otdanger}{danger}
\colorlet{otinfo}{info}

\hypersetup{
    pdftitle={Stuxnet: The First Cyber Weapon},
    pdfsubject={Analysis of the Stuxnet Attack},
}

\begin{document}

% ----------------------------------------------------------------------------
%  TITLE PAGE
% ----------------------------------------------------------------------------

\maketitlepage
    {Stuxnet}
    {The First Cyber Weapon Targeting Industrial Control Systems}
    {OT Security Learning Series}
    {Document 410 \quad|\quad January 2026}
    {Matthias Niedermaier}

% ----------------------------------------------------------------------------
%  TABLE OF CONTENTS
% ----------------------------------------------------------------------------

\tableofcontents
\newpage

% ----------------------------------------------------------------------------
%  INTRODUCTION
% ----------------------------------------------------------------------------

\section{Introduction}

Stuxnet, discovered in 2010, is widely considered the first true cyber weapon---malware specifically designed to cause physical damage to industrial equipment. It targeted Iran's nuclear enrichment facility at Natanz, destroying centrifuges while hiding its activities from operators.

\begin{dangerbox}
Stuxnet demonstrated that cyber attacks can cause physical destruction of industrial equipment. It fundamentally changed the threat landscape for critical infrastructure worldwide.
\end{dangerbox}

\subsection{Key Facts}

\begin{center}
\rowcolors{2}{lightgray}{white}
\begin{tabular}{p{4cm}p{9cm}}
\rowcolor{primary}
\textcolor{white}{\bfseries Attribute} & \textcolor{white}{\bfseries Details} \\
\midrule
Discovery Date & June 2010 \\
Target & Natanz uranium enrichment facility, Iran \\
Target Systems & Siemens S7-300 PLCs controlling centrifuges \\
Attack Method & USB propagation, network spreading, PLC manipulation \\
Physical Impact & Approximately 1,000 centrifuges destroyed \\
Attribution & Widely attributed to US and Israel (unconfirmed) \\
Zero-Days Used & 4 Windows zero-day vulnerabilities \\
\end{tabular}
\end{center}

% ----------------------------------------------------------------------------
%  ATTACK OVERVIEW
% ----------------------------------------------------------------------------

\section{Attack Overview}

\subsection{Propagation Methods}

Stuxnet used multiple methods to spread and reach its target:

\begin{enumerate}
    \item \textbf{USB Drives:} Primary initial infection vector---exploited Windows autorun and LNK file vulnerabilities
    \item \textbf{Network Shares:} Spread via Windows network shares using multiple vulnerabilities
    \item \textbf{Print Spooler:} Exploited Windows Print Spooler vulnerability (MS10-061)
    \item \textbf{Siemens WinCC/Step 7:} Spread through shared project files
\end{enumerate}

\begin{warningbox}
Stuxnet crossed the ``air gap'' through infected USB drives carried by contractors and employees. Physical isolation alone is not sufficient protection.
\end{warningbox}

\begin{figure}[H]
\centering
\begin{tikzpicture}[
    box/.style={rectangle, draw, thick, rounded corners=3pt, minimum width=2.2cm, minimum height=1cm, align=center, font=\small},
    arrow/.style={->, very thick, >=stealth}
]

% Attack path
\node[box, fill=otdanger!20] (usb) at (0,0) {Infected\\USB Drive};
\node[box, fill=otwarning!20] (win) at (3.5,0) {Windows\\Workstation};
\node[box, fill=otwarning!20] (step7) at (7,0) {Step 7\\Project};
\node[box, fill=otdanger!20] (plc) at (10.5,0) {Siemens\\S7-300 PLC};

\draw[arrow, otdanger] (usb) -- (win);
\draw[arrow, otdanger] (win) -- (step7);
\draw[arrow, otdanger] (step7) -- (plc);

% Labels
\node[font=\tiny, below] at (1.75,-0.7) {Zero-day exploits};
\node[font=\tiny, below] at (5.25,-0.7) {Project infection};
\node[font=\tiny, below] at (8.75,-0.7) {PLC rootkit};

\end{tikzpicture}
\caption{Stuxnet Attack Path}
\end{figure}

\subsection{Target Identification}

Stuxnet was highly selective in its targeting:

\begin{itemize}
    \item Searched for Siemens Step 7 software installations
    \item Identified specific PLC configurations (S7-315 and S7-417)
    \item Verified presence of specific frequency converter drives (Vacon and Fararo Paya)
    \item Checked for drive frequencies between 807--1210 Hz (centrifuge operating range)
    \item Only activated payload when all conditions matched
\end{itemize}

\begin{tipbox}
The extreme specificity of Stuxnet's targeting criteria indicates detailed intelligence about the Natanz facility's configuration was available to the attackers.
\end{tipbox}

% ----------------------------------------------------------------------------
%  TECHNICAL ANALYSIS
% ----------------------------------------------------------------------------

\section{Technical Analysis}

\subsection{Zero-Day Vulnerabilities}

Stuxnet exploited four previously unknown Windows vulnerabilities:

\begin{center}
\small
\rowcolors{2}{lightgray}{white}
\begin{tabular}{p{2.5cm}p{4cm}p{6.5cm}}
\rowcolor{primary}
\textcolor{white}{\bfseries CVE} & \textcolor{white}{\bfseries Component} & \textcolor{white}{\bfseries Description} \\
\midrule
CVE-2010-2568 & Windows Shell & LNK file vulnerability for USB propagation \\
CVE-2010-2729 & Print Spooler & Remote code execution via shared printers \\
CVE-2010-2772 & Windows Server Service & Network propagation vulnerability \\
CVE-2010-3338 & Task Scheduler & Privilege escalation \\
\end{tabular}
\end{center}

\subsection{PLC Payload}

The core payload targeted Siemens S7-300 PLCs:

\begin{conceptbox}{PLC Attack Mechanism}
\begin{enumerate}
    \item \textbf{Infection:} Malicious code injected into Step 7 project files
    \item \textbf{Rootkit:} PLC rootkit hid modifications from engineering software
    \item \textbf{Recording:} Captured 21 days of normal centrifuge operation data
    \item \textbf{Replay:} Played back normal data to operators while attacking
    \item \textbf{Manipulation:} Varied centrifuge speeds to cause mechanical stress
    \item \textbf{Destruction:} Centrifuges destroyed through metal fatigue
\end{enumerate}
\end{conceptbox}

\subsection{Man-in-the-Middle on PLCs}

\begin{dangerbox}
Stuxnet implemented a man-in-the-middle attack between the HMI/SCADA system and the PLCs. Operators saw normal readings while centrifuges were being destroyed. This ``lying to the operator'' technique has been replicated in subsequent attacks.
\end{dangerbox}

% ----------------------------------------------------------------------------
%  ATTACK TIMELINE
% ----------------------------------------------------------------------------

\section{Attack Timeline}

\begin{figure}[H]
\centering
\begin{tikzpicture}[scale=0.9]
    % Timeline
    \draw[->, line width=1pt] (0,0) -- (13,0);

    % Years
    \foreach \x/\year in {1/2007, 4/2008, 7/2009, 10/2010, 13/2011} {
        \draw[line width=0.5pt] (\x,-0.15) -- (\x,0.15);
        \node[font=\small] at (\x,-0.5) {\year};
    }

    % Events
    \node[fill=otwarning!30, rounded corners, font=\scriptsize, align=center, anchor=south, text width=2cm] at (1,0.3) {Development\\begins};
    \node[fill=otdanger!30, rounded corners, font=\scriptsize, align=center, anchor=south, text width=2cm] at (4,0.3) {First infections\\detected};
    \node[fill=otdanger!30, rounded corners, font=\scriptsize, align=center, anchor=south, text width=2cm] at (7,0.3) {Major variant\\deployed};
    \node[fill=otinfo!30, rounded corners, font=\scriptsize, align=center, anchor=south, text width=2cm] at (10,0.3) {Publicly\\discovered};
    \node[fill=otsuccess!30, rounded corners, font=\scriptsize, align=center, anchor=south, text width=2cm] at (13,0.3) {Analysis\\complete};
\end{tikzpicture}
\caption{Stuxnet Timeline (2007--2011)}
\end{figure}

\begin{itemize}
    \item \textbf{2005--2007:} Development phase (estimated)
    \item \textbf{June 2009:} First known variant deployed
    \item \textbf{March 2010:} More aggressive variant released
    \item \textbf{June 2010:} Discovered by VirusBlokAda (Belarus)
    \item \textbf{July 2010:} Siemens acknowledges targeting of their systems
    \item \textbf{September 2010:} Full analysis published by Symantec
\end{itemize}

% ----------------------------------------------------------------------------
%  IMPACT AND CONSEQUENCES
% ----------------------------------------------------------------------------

\section{Impact and Consequences}

\subsection{Immediate Impact}

\begin{figure}[H]
\centering
\begin{tikzpicture}[
    impact/.style={rectangle, draw, thick, fill=otdanger!15, minimum width=6cm, minimum height=0.7cm, rounded corners=3pt, font=\small, align=left}
]

\node[impact] at (0,3) {\faIcon{cog} ~1,000 centrifuges destroyed};
\node[impact] at (0,2) {\faIcon{clock} Nuclear program delayed 1--2 years};
\node[impact] at (0,1) {\faIcon{globe} 100,000+ systems infected worldwide};
\node[impact] at (0,0) {\faIcon{code} 4 zero-day vulnerabilities burned};
\node[impact] at (0,-1) {\faIcon{shield-alt} Changed OT threat landscape forever};

\end{tikzpicture}
\caption{Stuxnet Impact Summary}
\end{figure}

\begin{itemize}
    \item Approximately 1,000 IR-1 centrifuges destroyed at Natanz
    \item Iranian nuclear program delayed by an estimated 1--2 years
    \item Stuxnet spread beyond intended target to systems worldwide
    \item Over 100,000 systems infected globally (most not affected by payload)
\end{itemize}

\subsection{Long-Term Consequences}

\begin{warningbox}
\textbf{Stuxnet's legacy extends far beyond its immediate impact:}
\begin{itemize}
    \item Demonstrated feasibility of cyber-physical attacks
    \item Lowered the barrier for future ICS attacks
    \item Code and techniques studied and reused by other actors
    \item Sparked global investment in OT security
    \item Led to development of ICS-specific security standards
\end{itemize}
\end{warningbox}

% ----------------------------------------------------------------------------
%  LESSONS LEARNED
% ----------------------------------------------------------------------------

\section{Lessons Learned}

\subsection{Defense Recommendations}

\begin{successbox}
\textbf{Key takeaways for OT security:}
\begin{enumerate}
    \item \textbf{Air gaps are not absolute:} USB drives and contractor laptops can bridge isolated networks
    \item \textbf{Monitor PLC integrity:} Detect unauthorized changes to PLC logic
    \item \textbf{Verify operator displays:} Cross-check HMI data with independent measurements
    \item \textbf{Control removable media:} Implement strict USB and media policies
    \item \textbf{Segment networks:} Limit lateral movement within OT environments
    \item \textbf{Update systems:} Apply security patches where safely possible
\end{enumerate}
\end{successbox}

\subsection{Detection Indicators}

Stuxnet could have been detected through:

\begin{itemize}
    \item Monitoring for unauthorized Step 7 project file modifications
    \item Detecting anomalous PLC communication patterns
    \item Identifying unexpected DLL injections in SCADA software
    \item Monitoring centrifuge performance deviations
    \item Network traffic analysis for C2 communications
\end{itemize}

% ----------------------------------------------------------------------------
%  FURTHER READING
% ----------------------------------------------------------------------------

\section{Further Reading}

\subsection*{Technical Reports}
\begin{itemize}
    \item \textbf{Symantec} -- W32.Stuxnet Dossier\\
          \url{https://docs.broadcom.com/docs/security-response-w32-stuxnet-dossier-11-en}
    \item \textbf{Langner Communications} -- To Kill a Centrifuge\\
          \url{https://www.langner.com/to-kill-a-centrifuge/}
    \item \textbf{ICS-CERT} -- Advisory ICSA-10-272-01\\
          \url{https://www.cisa.gov/news-events/ics-advisories}
\end{itemize}

\subsection*{Books}
\begin{itemize}
    \item Zetter, K. -- \textit{Countdown to Zero Day} (Crown, 2014)
    \item Langner, R. -- \textit{Robust Control System Networks} (Momentum Press, 2011)
\end{itemize}

\subsection*{Documentary}
\begin{itemize}
    \item \textit{Zero Days} (2016) -- Documentary by Alex Gibney
\end{itemize}

\vfill
\begin{center}
\textcolor{mediumgray}{\rule{0.5\textwidth}{0.5pt}}\\[1em]
\textcolor{mediumgray}{\small Part of the OT Security Learning Series}
\end{center}

\end{document}
