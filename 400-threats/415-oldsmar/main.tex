% ============================================================================
%  Oldsmar Water Treatment Attack - OT Security Learning Resource
% ============================================================================

\documentclass[11pt,a4paper]{article}
\usepackage{otsec-template}
\usepackage{float}

% Define colors for TikZ
\colorlet{otprimary}{primary}
\colorlet{otaccent}{accent}
\colorlet{otsuccess}{success}
\colorlet{otwarning}{warning}
\colorlet{otdanger}{danger}
\colorlet{otinfo}{info}

\begin{document}

\maketitlepage
    {Oldsmar Water Treatment Attack}
    {Remote access intrusion targeting public water supply}
    {OT Security Learning Series}
    {Document 415 \quad|\quad January 2026}
    {Matthias Niedermaier}

\tableofcontents
\newpage

% ============================================================================
\section{Introduction}
% ============================================================================

In February 2021, an attacker remotely accessed the water treatment plant in Oldsmar, Florida, and attempted to increase sodium hydroxide (lye) levels from 100 parts per million to 11,100 ppm---a potentially lethal concentration. An alert operator observed the intrusion in real-time and immediately reversed the changes.

\begin{dangerbox}
This attack demonstrated that small water utilities are vulnerable targets. The attacker used legitimate remote access software (TeamViewer) to directly manipulate chemical dosing---a direct threat to public health.
\end{dangerbox}

\subsection{Key Facts}

\begin{center}
\rowcolors{2}{lightgray}{white}
\begin{tabular}{p{4cm}p{9cm}}
\rowcolor{primary}
\textcolor{white}{\bfseries Attribute} & \textcolor{white}{\bfseries Details} \\
\midrule
Date & February 5, 2021 \\
Target & Oldsmar Water Treatment Plant, Florida \\
Population Served & Approximately 15,000 residents \\
Attack Vector & TeamViewer remote access software \\
Attack Method & Increased sodium hydroxide (NaOH) setpoint \\
Change Attempted & 100 ppm to 11,100 ppm (111x increase) \\
Detection & Operator observed cursor movement in real-time \\
Impact & None---attack reversed immediately \\
Attribution & Initially unknown; later linked to former employee \\
\end{tabular}
\end{center}

% ============================================================================
\section{Attack Details}
% ============================================================================

\subsection{Attack Timeline}

\begin{figure}[H]
\centering
\begin{tikzpicture}[scale=0.9]
    % Timeline
    \draw[->, line width=1pt] (0,0) -- (13,0);

    % Time markers
    \foreach \x/\time in {1/8:00am, 4/1:30pm, 7/1:35pm, 10/3:00pm} {
        \draw[line width=0.5pt] (\x,-0.15) -- (\x,0.15);
        \node[font=\scriptsize] at (\x,-0.5) {\time};
    }

    % Events
    \node[fill=otwarning!30, rounded corners, font=\scriptsize, align=center, anchor=south, text width=2cm] at (1,0.3) {First remote\\access};
    \node[fill=otdanger!30, rounded corners, font=\scriptsize, align=center, anchor=south, text width=2cm] at (4,0.3) {Attacker\\returns};
    \node[fill=otdanger!30, rounded corners, font=\scriptsize, align=center, anchor=south, text width=2cm] at (7,0.3) {NaOH level\\changed};
    \node[fill=otsuccess!30, rounded corners, font=\scriptsize, align=center, anchor=south, text width=2cm] at (10,0.3) {Operator\\reverses};

    % Date label
    \node[font=\small\bfseries] at (6.5,-1.2) {February 5, 2021};

\end{tikzpicture}
\caption{Oldsmar Attack Timeline}
\end{figure}

\begin{itemize}
    \item \textbf{8:00 AM:} First brief remote access---operator noticed but assumed supervisor checking in
    \item \textbf{1:30 PM:} Attacker returns via TeamViewer
    \item \textbf{1:30--1:35 PM:} Attacker navigates SCADA interface, changes NaOH from 100 to 11,100 ppm
    \item \textbf{1:35 PM:} Operator observes changes and immediately reverses them
    \item \textbf{3:00 PM:} Plant supervisor notified, law enforcement contacted
\end{itemize}

\subsection{Attack Method}

\begin{figure}[H]
\centering
\begin{tikzpicture}[
    box/.style={rectangle, draw, thick, rounded corners=3pt, minimum width=2.5cm, minimum height=1cm, align=center, font=\small},
    arrow/.style={->, very thick, >=stealth}
]

% Attack path
\node[box, fill=otdanger!20] (internet) at (0,0) {Internet};
\node[box, fill=otwarning!20] (tv) at (4,0) {TeamViewer\\Connection};
\node[box, fill=otwarning!20] (hmi) at (8,0) {HMI/SCADA\\Workstation};
\node[box, fill=otdanger!20] (chem) at (12,0) {Chemical\\Dosing};

\draw[arrow, otdanger] (internet) -- (tv);
\draw[arrow, otdanger] (tv) -- (hmi);
\draw[arrow, otdanger] (hmi) -- (chem);

% Labels
\node[font=\tiny, below] at (2,-0.7) {Remote access};
\node[font=\tiny, below] at (6,-0.7) {Shared password};
\node[font=\tiny, below] at (10,-0.7) {Direct control};

\end{tikzpicture}
\caption{Oldsmar Attack Path}
\end{figure}

\begin{warningbox}
\textbf{Critical Security Failures:}
\begin{itemize}
    \item TeamViewer installed directly on SCADA workstation
    \item Shared password among all plant staff
    \item No multi-factor authentication
    \item Windows 7 (end-of-life operating system)
    \item No network segmentation between IT and OT
\end{itemize}
\end{warningbox}

% ============================================================================
\section{Sodium Hydroxide Danger}
% ============================================================================

\begin{dangerbox}
\textbf{Sodium Hydroxide (Lye/Caustic Soda):}
\begin{itemize}
    \item Normal use in water treatment: pH adjustment, corrosion control
    \item Safe level: 25--100 ppm for treatment purposes
    \item Attempted level: 11,100 ppm (extremely dangerous)
    \item Effects: Chemical burns, tissue damage, potentially fatal if ingested
\end{itemize}
\end{dangerbox}

\subsection{Safety Barriers}

While the attack was caught immediately, multiple safety barriers would have prevented harm:

\begin{table}[H]
\centering
\small
\rowcolors{2}{lightgray}{white}
\begin{tabular}{p{4cm}p{9cm}}
\rowcolor{primary}
\textcolor{white}{\bfseries Barrier} & \textcolor{white}{\bfseries Protection} \\
\midrule
Operator monitoring & Real-time observation caught the change \\
pH sensors & Would detect abnormal levels \\
Time delay & Changes take 24+ hours to reach customers \\
Residual monitoring & Downstream sensors check water quality \\
Manual sampling & Regular testing would detect anomalies \\
\end{tabular}
\caption{Defense-in-Depth Safety Barriers}
\end{table}

\begin{infobox}
Defense-in-depth means no single failure causes harm. Even if the cyber attack succeeded, multiple process and safety controls would likely have prevented contaminated water from reaching customers.
\end{infobox}

% ============================================================================
\section{Security Failures Analysis}
% ============================================================================

\subsection{Remote Access Issues}

\begin{figure}[H]
\centering
\begin{tikzpicture}[
    issue/.style={rectangle, draw, thick, fill=otdanger!15, minimum width=6cm, minimum height=0.7cm, rounded corners=3pt, font=\small, align=left}
]

\node[issue] at (0,3) {\faIcon{desktop} TeamViewer on SCADA workstation};
\node[issue] at (0,2) {\faIcon{key} Shared password for all operators};
\node[issue] at (0,1) {\faIcon{unlock} No multi-factor authentication};
\node[issue] at (0,0) {\faIcon{network-wired} Direct internet connectivity};
\node[issue] at (0,-1) {\faIcon{times-circle} TeamViewer left running unused};

\end{tikzpicture}
\caption{Remote Access Security Failures}
\end{figure}

\subsection{System Configuration Issues}

\begin{itemize}
    \item \textbf{Outdated OS:} Windows 7 past end-of-life (no security updates)
    \item \textbf{Flat network:} No segmentation between business and OT systems
    \item \textbf{No firewall:} Direct internet access to SCADA
    \item \textbf{Shared credentials:} No individual accountability
    \item \textbf{No logging/alerting:} Remote access not monitored
\end{itemize}

% ============================================================================
\section{Investigation Findings}
% ============================================================================

\subsection{Initial Response}

Investigations involved:
\begin{itemize}
    \item FBI (federal investigation)
    \item Pinellas County Sheriff's Office
    \item CISA (technical assistance)
    \item Secret Service (critical infrastructure)
\end{itemize}

\subsection{Attribution Challenges}

Initial attribution was difficult because:
\begin{itemize}
    \item Shared credentials meant no individual identification
    \item TeamViewer logs insufficient for definitive attribution
    \item Remote access could originate from anywhere
    \item Multiple parties knew the shared password
\end{itemize}

\begin{infobox}
In late 2023, a former Oldsmar employee was charged in connection with the attack. The case highlights how insider threats and weak access controls combine to create serious risks.
\end{infobox}

% ============================================================================
\section{Lessons Learned}
% ============================================================================

\subsection{Immediate Recommendations}

\begin{successbox}
\textbf{Critical actions for water utilities:}
\begin{enumerate}
    \item \textbf{Remove or secure remote access:} No direct internet-to-SCADA connections
    \item \textbf{Implement MFA:} Required for all remote access
    \item \textbf{Individual accounts:} No shared passwords
    \item \textbf{Update operating systems:} Replace Windows 7 and other EOL systems
    \item \textbf{Network segmentation:} Separate IT, OT, and internet
    \item \textbf{Monitor remote access:} Log and alert on connections
\end{enumerate}
\end{successbox}

\subsection{Secure Remote Access Architecture}

\begin{figure}[H]
\centering
\begin{tikzpicture}[
    box/.style={rectangle, draw, thick, rounded corners=3pt, minimum width=2cm, minimum height=1cm, align=center, font=\small},
    arrow/.style={->, thick, >=stealth}
]

% Correct architecture
\node[box, fill=otinfo!20] (internet2) at (0,0) {Internet};
\node[box, fill=otsuccess!20] (vpn) at (3,0) {VPN with\\MFA};
\node[box, fill=otsuccess!20] (jump) at (6,0) {Jump\\Server};
\node[box, fill=otsuccess!20] (fw) at (9,0) {OT\\Firewall};
\node[box, fill=otaccent!20] (scada2) at (12,0) {SCADA};

\draw[arrow, otsuccess] (internet2) -- (vpn);
\draw[arrow, otsuccess] (vpn) -- (jump);
\draw[arrow, otsuccess] (jump) -- (fw);
\draw[arrow, otsuccess] (fw) -- (scada2);

% Labels
\node[font=\tiny, below] at (1.5,-0.7) {Encrypted};
\node[font=\tiny, below] at (4.5,-0.7) {Logged};
\node[font=\tiny, below] at (7.5,-0.7) {Filtered};
\node[font=\tiny, below] at (10.5,-0.7) {Monitored};

% Title
\node[font=\small\bfseries, otsuccess] at (6,-1.5) {Secure Remote Access Architecture};

\end{tikzpicture}
\caption{Recommended Secure Remote Access}
\end{figure}

\subsection{Water Sector Specific Guidance}

\begin{itemize}
    \item Review CISA/EPA water sector guidance
    \item Participate in WaterISAC threat sharing
    \item Conduct regular security assessments
    \item Train operators on security awareness
    \item Develop incident response procedures
\end{itemize}

% ============================================================================
\section{Broader Implications}
% ============================================================================

\subsection{Small Utility Challenges}

\begin{warningbox}
Small water utilities face unique challenges:
\begin{itemize}
    \item Limited IT/security staff and expertise
    \item Budget constraints for security improvements
    \item Legacy systems difficult to upgrade
    \item Remote operations require remote access
    \item Thousands of potential targets nationwide
\end{itemize}
\end{warningbox}

\subsection{Regulatory Response}

The Oldsmar incident prompted:
\begin{itemize}
    \item \textbf{CISA alerts:} Guidance for water and wastewater utilities
    \item \textbf{EPA focus:} Increased attention on water sector cybersecurity
    \item \textbf{State actions:} Florida increased water utility oversight
    \item \textbf{WaterISAC:} Enhanced threat sharing and resources
\end{itemize}

% ============================================================================
\section{Summary}
% ============================================================================

\begin{definitionbox}{Key Takeaways}
\begin{itemize}
    \item \textbf{Direct OT manipulation:} Attacker changed chemical dosing in real-time
    \item \textbf{Remote access risk:} TeamViewer on SCADA without MFA
    \item \textbf{Shared credentials:} No accountability or traceability
    \item \textbf{Operator saved the day:} Human vigilance caught the attack
    \item \textbf{Defense-in-depth works:} Multiple barriers would have prevented harm
    \item \textbf{Small utilities at risk:} Limited resources, high vulnerability
\end{itemize}
\end{definitionbox}

% ============================================================================
\section{Further Reading}
% ============================================================================

\subsection*{Official Advisories}
\begin{itemize}
    \item \textbf{CISA} -- Compromise of U.S. Water Treatment Facility\\
          \url{https://www.cisa.gov/news-events/cybersecurity-advisories/aa21-042a}
    \item \textbf{FBI/CISA/EPA} -- Water Sector Cybersecurity Brief\\
          \url{https://www.cisa.gov/resources-tools/resources/water-and-wastewater-cybersecurity}
\end{itemize}

\subsection*{Resources}
\begin{itemize}
    \item \textbf{WaterISAC} -- Water Information Sharing and Analysis Center\\
          \url{https://www.waterisac.org/}
    \item \textbf{EPA} -- Water Sector Cybersecurity Resources\\
          \url{https://www.epa.gov/waterresilience/epa-cybersecurity-water-sector}
    \item \textbf{AWWA} -- Cybersecurity Guidance\\
          \url{https://www.awwa.org/Resources-Tools/Resource-Topics/Risk-Resilience/Cybersecurity-Guidance}
\end{itemize}

\vfill
\begin{center}
\textcolor{mediumgray}{\rule{0.5\textwidth}{0.5pt}}\\[1em]
\textcolor{mediumgray}{\small Part of the OT Security Learning Series}
\end{center}

\end{document}
