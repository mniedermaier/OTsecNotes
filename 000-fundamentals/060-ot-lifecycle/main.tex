% ============================================================================
%  060-ot-lifecycle - OT Security Learning Resource
% ============================================================================

\documentclass[11pt,a4paper]{article}
\usepackage{otsec-template}
\usepackage{float}

% Define colors for TikZ
\colorlet{otprimary}{primary}
\colorlet{otaccent}{accent}
\colorlet{otsuccess}{success}
\colorlet{otwarning}{warning}
\colorlet{otdanger}{danger}
\colorlet{otinfo}{info}

\begin{document}

\maketitlepage
    {OT System Lifecycle}
    {Security considerations across the industrial asset lifecycle}
    {OT Security Learning Series}
    {Document 060 \quad|\quad January 2026}
    {Matthias Niedermaier}

\tableofcontents
\newpage

% ============================================================================
\section{Introduction}
% ============================================================================

\begin{infobox}
The OT system lifecycle spans from initial procurement through decommissioning. Security must be integrated at every phase---not bolted on afterward. Unlike IT systems with 3--5 year lifecycles, OT assets often operate for 15--30 years, requiring long-term security planning.
\end{infobox}

Key challenges in OT lifecycle management:
\begin{itemize}
    \item \textbf{Extended lifespans} -- Equipment outlives vendor support
    \item \textbf{Legacy systems} -- Older devices lack security features
    \item \textbf{Availability requirements} -- Limited maintenance windows
    \item \textbf{Safety constraints} -- Changes require careful validation
    \item \textbf{Supply chain risks} -- Compromised components or firmware
\end{itemize}

% ============================================================================
\section{Lifecycle Phases Overview}
% ============================================================================

\begin{figure}[H]
\centering
\begin{tikzpicture}[
    phase/.style={rectangle, draw, thick, rounded corners=5pt, minimum width=2.2cm, minimum height=1.2cm, align=center, font=\small\bfseries},
    arrow/.style={->, very thick, >=stealth}
]

% Phases in a cycle
\node[phase, fill=otinfo!20] (proc) at (0,3) {Procurement};
\node[phase, fill=otaccent!20] (deploy) at (4,3) {Deployment};
\node[phase, fill=otsuccess!20] (ops) at (7,1.5) {Operations};
\node[phase, fill=otwarning!20] (maint) at (4,0) {Maintenance};
\node[phase, fill=otdanger!20] (decom) at (0,0) {Decommission};

% Arrows connecting phases
\draw[arrow, otprimary] (proc) -- (deploy);
\draw[arrow, otprimary] (deploy) -- (ops);
\draw[arrow, otprimary] (ops) -- (maint);
\draw[arrow, otprimary] (maint) -- (decom);

% Loop back from maintenance to operations
\draw[arrow, otsuccess, dashed] (maint.north east) to[bend right=20] (ops.south);

% Time indicators
\node[font=\tiny, otinfo] at (0,2.3) {Weeks};
\node[font=\tiny, otaccent] at (4,2.3) {Months};
\node[font=\tiny, otsuccess] at (7,0.7) {Years};
\node[font=\tiny, otwarning] at (4,-0.7) {Ongoing};
\node[font=\tiny, otdanger] at (0,-0.7) {End of Life};

\end{tikzpicture}
\caption{OT System Lifecycle Phases}
\end{figure}

\begin{table}[H]
\centering
\small
\rowcolors{2}{lightgray}{white}
\begin{tabular}{p{3cm}p{4cm}p{6cm}}
\rowcolor{primary}
\textcolor{white}{\bfseries Phase} & \textcolor{white}{\bfseries Duration} & \textcolor{white}{\bfseries Security Focus} \\
\midrule
Procurement & Weeks to months & Vendor assessment, security requirements \\
Deployment & Weeks to months & Secure configuration, hardening \\
Operations & 10--30 years & Monitoring, access control, patching \\
Maintenance & Ongoing & Change management, vendor access \\
Decommissioning & Weeks & Data sanitization, secure disposal \\
\end{tabular}
\caption{Lifecycle Phases and Security Focus}
\end{table}

% ============================================================================
\section{Procurement Phase}
% ============================================================================

\begin{successbox}
\textbf{Security starts at procurement.} Specify security requirements before purchase---retrofitting security is expensive and often impossible with OT equipment.
\end{successbox}

\subsection{Security Requirements}

Include in procurement specifications:
\begin{itemize}
    \item \textbf{Authentication} -- Support for strong authentication (no hardcoded credentials)
    \item \textbf{Encryption} -- TLS/secure protocols for communications
    \item \textbf{Logging} -- Security event logging capabilities
    \item \textbf{Patching} -- Vendor patch support commitment and timeline
    \item \textbf{Certifications} -- IEC 62443-4-1/4-2 compliance
    \item \textbf{Documentation} -- Security hardening guides, network diagrams
\end{itemize}

\subsection{Vendor Assessment}

\begin{table}[H]
\centering
\small
\rowcolors{2}{lightgray}{white}
\begin{tabular}{p{4cm}p{9cm}}
\rowcolor{primary}
\textcolor{white}{\bfseries Criteria} & \textcolor{white}{\bfseries Questions to Ask} \\
\midrule
Security development & Does vendor follow secure SDLC (IEC 62443-4-1)? \\
Vulnerability handling & How are CVEs disclosed and patched? \\
Support lifecycle & How long will security patches be provided? \\
Supply chain & Where are components sourced? Firmware integrity? \\
Incident response & How does vendor communicate security issues? \\
\end{tabular}
\caption{Vendor Security Assessment Criteria}
\end{table}

\begin{warningbox}
\textbf{Supply Chain Risk:} Counterfeit components and compromised firmware are real threats. Verify component authenticity and require firmware integrity verification (signed updates).
\end{warningbox}

% ============================================================================
\section{Deployment Phase}
% ============================================================================

\subsection{Secure Configuration}

\begin{figure}[H]
\centering
\begin{tikzpicture}[
    stepbox/.style={rectangle, draw, thick, fill=otaccent!15, minimum width=10cm, minimum height=0.7cm, rounded corners=3pt, font=\small},
    num/.style={circle, fill=otprimary, text=white, font=\small\bfseries, minimum size=0.6cm}
]

\node[stepbox] (s1) at (0,4) {Change default credentials immediately};
\node[num] at (-5.5,4) {1};

\node[stepbox] (s2) at (0,3) {Disable unnecessary services and ports};
\node[num] at (-5.5,3) {2};

\node[stepbox] (s3) at (0,2) {Configure network segmentation and firewall rules};
\node[num] at (-5.5,2) {3};

\node[stepbox] (s4) at (0,1) {Enable logging and connect to monitoring};
\node[num] at (-5.5,1) {4};

\node[stepbox] (s5) at (0,0) {Document baseline configuration and create backup};
\node[num] at (-5.5,0) {5};

\end{tikzpicture}
\caption{Secure Deployment Checklist}
\end{figure}

\subsection{Hardening Guidelines}

\begin{itemize}
    \item Follow vendor hardening guides (if available)
    \item Apply latest firmware/patches before deployment
    \item Configure role-based access control
    \item Disable remote access unless required
    \item Implement network segmentation per zone model
    \item Create configuration backup before going live
\end{itemize}

\subsection{Acceptance Testing}

Before operational handover:
\begin{enumerate}
    \item Verify security controls are implemented
    \item Test authentication and access controls
    \item Validate network segmentation
    \item Confirm logging is functioning
    \item Document as-built security configuration
\end{enumerate}

% ============================================================================
\section{Operations Phase}
% ============================================================================

\begin{infobox}
The operations phase is the longest---often 15--30 years for OT equipment. Security must be maintained throughout this extended period, even as threats evolve and vendor support ends.
\end{infobox}

\subsection{Ongoing Security Activities}

\begin{table}[H]
\centering
\small
\rowcolors{2}{lightgray}{white}
\begin{tabular}{p{4cm}p{5cm}p{4cm}}
\rowcolor{primary}
\textcolor{white}{\bfseries Activity} & \textcolor{white}{\bfseries Description} & \textcolor{white}{\bfseries Frequency} \\
\midrule
Vulnerability monitoring & Track CVEs for deployed assets & Continuous \\
Patch management & Evaluate and apply security updates & As released \\
Access reviews & Verify user access remains appropriate & Quarterly \\
Configuration audits & Check for drift from baseline & Annually \\
Backup verification & Test configuration backups & Monthly \\
Security monitoring & Review logs and alerts & Daily \\
\end{tabular}
\caption{Operational Security Activities}
\end{table}

\subsection{Managing Legacy Systems}

\begin{dangerbox}
\textbf{End-of-Support Systems:} When vendors stop providing patches, implement compensating controls:
\begin{itemize}
    \item Network isolation (firewall, VLAN, data diode)
    \item Application whitelisting on connected systems
    \item Enhanced monitoring for anomalies
    \item Plan for replacement or upgrade
\end{itemize}
\end{dangerbox}

% ============================================================================
\section{Maintenance Phase}
% ============================================================================

\subsection{Change Management}

\begin{figure}[H]
\centering
\begin{tikzpicture}[
    box/.style={rectangle, draw, thick, rounded corners=3pt, minimum width=2cm, minimum height=1cm, align=center, font=\small},
    arrow/.style={->, thick, >=stealth}
]

\node[box, fill=otinfo!20] (request) at (0,0) {Change\\Request};
\node[box, fill=otwarning!20] (review) at (3,0) {Security\\Review};
\node[box, fill=otaccent!20] (test) at (6,0) {Test in\\Lab};
\node[box, fill=otsuccess!20] (implement) at (9,0) {Implement\\Change};
\node[box, fill=otprimary!20] (verify) at (12,0) {Verify \&\\Document};

\draw[arrow] (request) -- (review);
\draw[arrow] (review) -- (test);
\draw[arrow] (test) -- (implement);
\draw[arrow] (implement) -- (verify);

% Rejection path
\draw[arrow, otdanger, dashed] (review.south) -- ++(0,-0.8) -| (request.south);
\node[font=\tiny, otdanger] at (1.5,-1.2) {Reject};

\end{tikzpicture}
\caption{OT Change Management Process}
\end{figure}

\subsection{Vendor and Contractor Access}

\begin{warningbox}
\textbf{Third-Party Access Risks:}
\begin{itemize}
    \item Require documented authorization for each access
    \item Escort or monitor vendor activities
    \item Use time-limited, audited remote access
    \item Verify changes made during maintenance
    \item Scan vendor media before connecting to OT
\end{itemize}
\end{warningbox}

\subsection{Patching Considerations}

\begin{itemize}
    \item \textbf{Test first} -- Validate patches in lab environment
    \item \textbf{Schedule carefully} -- Use maintenance windows
    \item \textbf{Have rollback plan} -- Ability to restore if issues occur
    \item \textbf{Document everything} -- Record patch levels and dates
    \item \textbf{Coordinate with vendor} -- Ensure patches don't void support
\end{itemize}

% ============================================================================
\section{Decommissioning Phase}
% ============================================================================

\begin{dangerbox}
\textbf{Security doesn't end at shutdown.} Decommissioned equipment may contain sensitive data, credentials, or network information that could be exploited if improperly disposed.
\end{dangerbox}

\subsection{Data Sanitization}

\begin{table}[H]
\centering
\small
\rowcolors{2}{lightgray}{white}
\begin{tabular}{p{4cm}p{9cm}}
\rowcolor{primary}
\textcolor{white}{\bfseries Data Type} & \textcolor{white}{\bfseries Sanitization Method} \\
\midrule
Configuration files & Factory reset, verify deletion \\
Credentials & Remove all accounts, reset to defaults \\
Process data/recipes & Secure deletion or physical destruction \\
Network settings & Clear IP addresses, firewall rules \\
Logs & Export for retention, then delete \\
Firmware & Consider reinstalling clean firmware \\
\end{tabular}
\caption{Data Sanitization Requirements}
\end{table}

\subsection{Disposal Options}

\begin{itemize}
    \item \textbf{Certified destruction} -- Physical destruction with certificate
    \item \textbf{Resale/reuse} -- Only after complete sanitization
    \item \textbf{Return to vendor} -- May be required for leased equipment
    \item \textbf{Recycling} -- Remove storage media first
\end{itemize}

\subsection{Documentation Updates}

\begin{itemize}
    \item Remove from asset inventory
    \item Update network diagrams
    \item Revoke access credentials
    \item Archive relevant documentation
    \item Update firewall rules to remove references
\end{itemize}

% ============================================================================
\section{Summary}
% ============================================================================

\begin{definitionbox}{Key Takeaways}
\begin{itemize}
    \item \textbf{Lifecycle thinking} -- Security at every phase, not just operations
    \item \textbf{Procurement} -- Specify security requirements before purchase
    \item \textbf{Extended lifespans} -- Plan for 15--30 year asset lifecycles
    \item \textbf{Vendor management} -- Assess security practices and support commitments
    \item \textbf{Change control} -- All modifications through documented process
    \item \textbf{Legacy systems} -- Compensating controls when patches unavailable
    \item \textbf{Decommissioning} -- Sanitize data before disposal
\end{itemize}
\end{definitionbox}

% ============================================================================
\section{Further Reading}
% ============================================================================

\subsection*{Standards}

\begin{itemize}
    \item \textbf{IEC 62443-2-4} -- Security program requirements for IACS service providers\\
          \url{https://webstore.iec.ch/publication/7031}
    \item \textbf{IEC 62443-4-1} -- Secure product development lifecycle requirements\\
          \url{https://webstore.iec.ch/publication/33615}
    \item \textbf{NIST SP 800-82 Rev. 3} -- Guide to OT Security\\
          \url{https://csrc.nist.gov/pubs/sp/800/82/r3/final}
\end{itemize}

\subsection*{Resources}

\begin{itemize}
    \item \textbf{CISA} -- Securing Industrial Control Systems: A Unified Initiative\\
          \url{https://www.cisa.gov/topics/industrial-control-systems}
    \item \textbf{ENISA} -- Good Practices for Security of IoT in Manufacturing\\
          \url{https://www.enisa.europa.eu/publications}
\end{itemize}

\vfill
\begin{center}
\textit{Part of the OT Security Learning Series}
\end{center}

\end{document}
