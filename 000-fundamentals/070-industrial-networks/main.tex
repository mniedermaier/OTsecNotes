% ============================================================================
%  070-industrial-networks - OT Security Learning Resource
% ============================================================================

\documentclass[11pt,a4paper]{article}
\usepackage{otsec-template}

% Define colors for TikZ (matching template colors)
\colorlet{otprimary}{primary}
\colorlet{otaccent}{accent}
\colorlet{otsuccess}{success}
\colorlet{otwarning}{warning}
\colorlet{otdanger}{danger}
\colorlet{otinfo}{info}

\begin{document}

\maketitlepage
    {Industrial Network Fundamentals}
    {Understanding OT network technologies, topologies, and fieldbus systems}
    {OT Security Learning Series}
    {Document 070 \quad|\quad January 2026}
    {AI Assistant}

\tableofcontents
\newpage

% ============================================================================
\section{Introduction}
% ============================================================================

\begin{infobox}
Industrial networks connect field devices, controllers, and supervisory systems in OT environments. Unlike IT networks optimized for data throughput, industrial networks prioritize \textbf{determinism}, \textbf{reliability}, and \textbf{real-time performance} for process control.
\end{infobox}

Understanding industrial networking is essential for OT security because:
\begin{itemize}
    \item Network architecture affects segmentation strategies
    \item Legacy protocols lack security features
    \item Physical layer differences impact monitoring capabilities
    \item Redundancy mechanisms must be preserved during security implementations
\end{itemize}

% ============================================================================
\section{Network Topologies}
% ============================================================================

Industrial networks use various topologies based on reliability, cost, and application requirements.

\begin{figure}[h]
\centering
\begin{tikzpicture}[
    device/.style={circle, draw, thick, fill=otaccent!20, minimum size=0.8cm, font=\tiny},
    switch/.style={rectangle, draw, thick, fill=otwarning!20, minimum size=0.6cm, font=\tiny},
    label/.style={font=\scriptsize\bfseries, otprimary}
]

% Star Topology
\node[label] at (-4,2.5) {Star};
\node[switch] (star-sw) at (-4,1) {SW};
\node[device] (s1) at (-5.5,1.8) {D};
\node[device] (s2) at (-4,2.2) {D};
\node[device] (s3) at (-2.5,1.8) {D};
\node[device] (s4) at (-5.2,0) {D};
\node[device] (s5) at (-2.8,0) {D};
\draw[thick, otprimary!60] (star-sw) -- (s1);
\draw[thick, otprimary!60] (star-sw) -- (s2);
\draw[thick, otprimary!60] (star-sw) -- (s3);
\draw[thick, otprimary!60] (star-sw) -- (s4);
\draw[thick, otprimary!60] (star-sw) -- (s5);

% Ring Topology
\node[label] at (0,2.5) {Ring};
\node[device] (r1) at (0,1.8) {D};
\node[device] (r2) at (1,1) {D};
\node[device] (r3) at (0.6,0) {D};
\node[device] (r4) at (-0.6,0) {D};
\node[device] (r5) at (-1,1) {D};
\draw[thick, otprimary!60] (r1) -- (r2) -- (r3) -- (r4) -- (r5) -- (r1);

% Bus Topology
\node[label] at (4,2.5) {Bus/Line};
\draw[very thick, otwarning] (2.5,1) -- (5.5,1);
\node[device] (b1) at (2.8,1.8) {D};
\node[device] (b2) at (3.6,1.8) {D};
\node[device] (b3) at (4.4,1.8) {D};
\node[device] (b4) at (5.2,1.8) {D};
\draw[thick, otprimary!60] (b1) -- (2.8,1);
\draw[thick, otprimary!60] (b2) -- (3.6,1);
\draw[thick, otprimary!60] (b3) -- (4.4,1);
\draw[thick, otprimary!60] (b4) -- (5.2,1);

% Labels
\node[font=\tiny, text width=2cm, align=center] at (-4,-1) {Central switch\\Single point of failure};
\node[font=\tiny, text width=2cm, align=center] at (0,-1) {Redundant path\\Self-healing};
\node[font=\tiny, text width=2cm, align=center] at (4,-1) {Simple, low cost\\Legacy fieldbus};

\end{tikzpicture}
\caption{Common Industrial Network Topologies}
\end{figure}

\begin{table}[h]
\centering
\begin{tabular}{|l|l|l|l|}
\hline
\textbf{Topology} & \textbf{Redundancy} & \textbf{Cost} & \textbf{Common Use} \\
\hline
Star & None (single switch) & Medium & Ethernet networks \\
Ring & High (dual path) & Higher & Critical systems \\
Bus/Line & None & Low & Legacy fieldbus \\
Mesh & Very high & Highest & Wireless, critical \\
\hline
\end{tabular}
\caption{Topology Comparison}
\end{table}

% ============================================================================
\section{Serial Communications}
% ============================================================================

Many legacy industrial systems use serial communications for device connectivity.

\subsection{RS-232}

\begin{itemize}
    \item Point-to-point communication only
    \item Short distance (typically $<$ 15 meters)
    \item Common for HMI-to-PLC connections, programming ports
    \item Speeds up to 115.2 kbps
\end{itemize}

\subsection{RS-485}

\begin{itemize}
    \item Multi-drop bus supporting up to 32 devices
    \item Longer distances (up to 1200 meters)
    \item Half-duplex or full-duplex operation
    \item Basis for Modbus RTU and many fieldbus protocols
    \item Speeds up to 10 Mbps (distance dependent)
\end{itemize}

\begin{warningbox}
\textbf{Security Note:} Serial protocols typically have \textbf{no authentication or encryption}. Physical access to the bus allows reading and injecting traffic. Serial-to-Ethernet converters can expose these insecure protocols to IP networks.
\end{warningbox}

% ============================================================================
\section{Fieldbus Technologies}
% ============================================================================

Fieldbus networks connect sensors, actuators, and controllers at the field level.

\begin{figure}[h]
\centering
\begin{tikzpicture}[
    block/.style={rectangle, draw, thick, minimum height=0.8cm, minimum width=2.8cm, rounded corners=3pt, font=\footnotesize},
    arrow/.style={->, thick, >=stealth}
]

% Timeline
\draw[very thick, otprimary!40] (0,0) -- (10,0);
\node[font=\tiny] at (0,-0.4) {1980s};
\node[font=\tiny] at (5,-0.4) {2000s};
\node[font=\tiny] at (10,-0.4) {Today};

% Fieldbus era
\node[block, fill=otwarning!20] at (1.5,1.2) {Modbus RTU};
\node[block, fill=otwarning!20] at (1.5,2.2) {PROFIBUS};
\node[block, fill=otwarning!20] at (4,1.7) {DeviceNet};
\node[font=\scriptsize\bfseries, otwarning] at (2.5,3) {Serial Fieldbus};

% Industrial Ethernet
\node[block, fill=otsuccess!20] at (6.5,1.2) {Modbus TCP};
\node[block, fill=otsuccess!20] at (6.5,2.2) {PROFINET};
\node[block, fill=otsuccess!20] at (9,1.2) {EtherNet/IP};
\node[block, fill=otsuccess!20] at (9,2.2) {OPC UA};
\node[font=\scriptsize\bfseries, otsuccess] at (7.5,3) {Industrial Ethernet};

\end{tikzpicture}
\caption{Evolution from Serial Fieldbus to Industrial Ethernet}
\end{figure}

\subsection{Legacy Fieldbus}

\begin{table}[h]
\centering
\begin{tabular}{|l|l|l|}
\hline
\textbf{Protocol} & \textbf{Physical Layer} & \textbf{Typical Use} \\
\hline
Modbus RTU & RS-485 & Universal, simple I/O \\
PROFIBUS DP & RS-485 & Siemens environments \\
DeviceNet & CAN bus & Rockwell/Allen-Bradley \\
Foundation Fieldbus & H1 (31.25 kbps) & Process instrumentation \\
HART & 4-20mA analog + digital & Smart instruments \\
\hline
\end{tabular}
\caption{Common Serial Fieldbus Protocols}
\end{table}

% ============================================================================
\section{Industrial Ethernet}
% ============================================================================

Industrial Ethernet adapts standard Ethernet for factory and process automation with enhanced reliability and real-time capabilities.

\subsection{Key Differences from IT Ethernet}

\begin{itemize}
    \item \textbf{Deterministic timing} -- Guaranteed message delivery within defined time
    \item \textbf{Ruggedized hardware} -- Extended temperature, vibration, EMI resistance
    \item \textbf{Ring redundancy} -- Sub-50ms failover (MRP, HSR, PRP)
    \item \textbf{Application layer protocols} -- Industrial-specific (not just TCP/IP)
\end{itemize}

\subsection{Common Industrial Ethernet Protocols}

\begin{table}[h]
\centering
\begin{tabular}{|l|l|l|}
\hline
\textbf{Protocol} & \textbf{Vendor/Org} & \textbf{Real-Time Method} \\
\hline
EtherNet/IP & ODVA (Rockwell) & Standard TCP/IP + CIP \\
PROFINET & Siemens/PI & IRT (Isochronous Real-Time) \\
Modbus TCP & Modbus.org & Standard TCP/IP \\
EtherCAT & Beckhoff/ETG & Processing on the fly \\
OPC UA & OPC Foundation & Pub/Sub, TSN \\
\hline
\end{tabular}
\caption{Industrial Ethernet Protocols}
\end{table}

% ============================================================================
\section{Network Redundancy}
% ============================================================================

Industrial networks implement redundancy to ensure availability of critical control functions.

\begin{definitionbox}{Common Redundancy Protocols}
\begin{itemize}
    \item \textbf{MRP} (Media Redundancy Protocol) -- Ring topology, $<$200ms recovery
    \item \textbf{RSTP} (Rapid Spanning Tree) -- Standard Ethernet, $<$2s recovery
    \item \textbf{HSR} (High-availability Seamless Redundancy) -- Zero recovery time
    \item \textbf{PRP} (Parallel Redundancy Protocol) -- Dual parallel networks
\end{itemize}
\end{definitionbox}

\begin{figure}[h]
\centering
\begin{tikzpicture}[
    switch/.style={rectangle, draw, thick, fill=otaccent!20, minimum height=0.8cm, minimum width=1.2cm, font=\tiny\bfseries},
    device/.style={rectangle, draw, thick, fill=otsuccess!20, minimum height=0.6cm, minimum width=1cm, font=\tiny}
]

% MRP Ring
\node[switch] (sw1) at (0,2) {SW1};
\node[switch] (sw2) at (2,2) {SW2};
\node[switch] (sw3) at (2,0) {SW3};
\node[switch] (sw4) at (0,0) {SW4};

\draw[very thick, otsuccess] (sw1) -- (sw2) -- (sw3) -- (sw4) -- (sw1);
\draw[thick, dashed, otdanger] (sw1) -- (sw3);
\node[font=\tiny, otdanger] at (1,1) {Blocked};

% Devices
\node[device] (d1) at (-1.2,2) {PLC};
\node[device] (d2) at (3.2,2) {HMI};
\node[device] (d3) at (3.2,0) {I/O};
\draw[thick, otprimary!60] (d1) -- (sw1);
\draw[thick, otprimary!60] (d2) -- (sw2);
\draw[thick, otprimary!60] (d3) -- (sw3);

\node[font=\scriptsize\bfseries, otprimary] at (1,-1) {MRP Ring with Blocked Port};

\end{tikzpicture}
\caption{Media Redundancy Protocol (MRP) Ring Topology}
\end{figure}

% ============================================================================
\section{Security Considerations}
% ============================================================================

\begin{dangerbox}
Industrial networks often lack basic security controls. Many protocols were designed decades ago without security considerations and cannot be easily updated due to availability requirements.
\end{dangerbox}

\subsection{Common Vulnerabilities}

\begin{itemize}
    \item \textbf{No authentication} -- Most fieldbus protocols accept any command
    \item \textbf{No encryption} -- Traffic readable by anyone on the network
    \item \textbf{Flat networks} -- No segmentation between zones
    \item \textbf{Legacy devices} -- Cannot be patched or upgraded
    \item \textbf{Broadcast traffic} -- Easy reconnaissance via passive sniffing
\end{itemize}

\subsection{Security Best Practices}

\begin{successbox}
\textbf{Network Security Measures:}
\begin{itemize}
    \item Segment networks by function and criticality (zones and conduits)
    \item Deploy industrial firewalls at zone boundaries
    \item Use managed switches with port security, VLANs
    \item Implement network monitoring and anomaly detection
    \item Disable unused ports and services
    \item Consider encrypted tunnels for sensitive traffic
\end{itemize}
\end{successbox}

% ============================================================================
\section{Summary}
% ============================================================================

\begin{definitionbox}{Key Takeaways}
\begin{itemize}
    \item \textbf{Topology matters} -- Star, ring, and bus have different reliability profiles
    \item \textbf{Serial still exists} -- RS-485 and fieldbus remain common in brownfield sites
    \item \textbf{Industrial Ethernet differs} -- Determinism and redundancy, not just TCP/IP
    \item \textbf{Redundancy is critical} -- MRP, HSR, PRP ensure availability
    \item \textbf{Legacy = insecure} -- Most industrial protocols lack authentication/encryption
    \item \textbf{Segment and monitor} -- Compensating controls for protocol weaknesses
\end{itemize}
\end{definitionbox}

% ============================================================================
\section{Further Reading}
% ============================================================================

\subsection*{Standards and Guidelines}

\begin{itemize}
    \item \textbf{IEC 62439} -- Industrial communication networks -- High availability\\
          \url{https://webstore.iec.ch/publication/6990}
    \item \textbf{IEC 62443-3-2} -- Security risk assessment for system design\\
          \url{https://webstore.iec.ch/publication/30727}
\end{itemize}

\subsection*{Resources}

\begin{itemize}
    \item \textbf{ODVA} -- EtherNet/IP specifications\\
          \url{https://www.odva.org/}
    \item \textbf{PROFIBUS/PROFINET International}\\
          \url{https://www.profibus.com/}
\end{itemize}

\subsection*{Books}

\begin{itemize}
    \item Zurawski, R. -- \textit{Industrial Communication Technology Handbook} (CRC Press)
    \item Galloway, B. \& Hancke, G. -- \textit{Introduction to Industrial Control Networks} (IEEE)
\end{itemize}

\vfill
\begin{center}
\textit{Part of the OT Security Learning Series}
\end{center}

\end{document}
