% ============================================================================
%  HMI and SCADA Basics - Poster / Cheat Sheet
% ============================================================================

\documentclass[9pt,a4paper]{extarticle}
\usepackage{otsec-poster}
\usepackage{float}

\begin{document}

\makepostertitle
    {HMI \& SCADA Fundamentals}
    {Operator Interfaces and Supervisory Control}
    {Poster 030}
    {Matthias Niedermaier}

\begin{multicols}{2}

\section{\textcolor{accent}{\faIcon{info-circle}}\hspace{0.4em}Overview}

HMIs and SCADA systems are the ``eyes and hands'' of operators. HMIs provide local visualization and control; SCADA extends this to geographically distributed assets. Both sit at \textbf{Purdue Level 2} and are critical for process oversight.

\posterinfo{
\textbf{HMI} = Local operator interface for a specific process or machine.\\
\textbf{SCADA} = Distributed system for monitoring and controlling remote assets across wide areas.
}

\section{\textcolor{accent}{\faIcon{server}}\hspace{0.4em}SCADA Architecture}

\begin{center}
\begin{tikzpicture}[
    comp/.style={rectangle, draw=#1!60, thick, fill=#1!12, rounded corners=2pt,
        minimum height=0.5cm, align=center, font=\scriptsize},
    comp/.default={otaccent},
    arr/.style={<->, thick, >=stealth, #1!60},
    arr/.default={otaccent},
    lnk/.style={font=\scriptsize, text=mediumgray},
]
    % MTU/Server at top
    \node[comp=otprimary, minimum width=3.2cm] (mtu) at (0,1.8) {\faIcon{server}\hspace{0.2em}\textbf{SCADA Server (MTU)}};
    \node[comp=otinfo, minimum width=1.4cm] (hmi1) at (-2.8,1.8) {\faIcon{desktop}\hspace{0.1em}HMI};
    \node[comp=otaccent, minimum width=1.4cm] (hist) at (2.8,1.8) {\faIcon{database}\hspace{0.1em}Historian};
    \draw[-, thick, otprimary!40] (hmi1) -- (mtu);
    \draw[-, thick, otprimary!40] (mtu) -- (hist);

    % Communication links
    \node[lnk] at (0,1.1) {\faIcon{broadcast-tower}\hspace{0.2em}WAN / Radio / Cellular / Satellite};

    % RTUs at bottom
    \node[comp=otsuccess, minimum width=1.5cm] (rtu1) at (-2,0.3) {\faIcon{broadcast-tower}\hspace{0.1em}RTU 1};
    \node[comp=otsuccess, minimum width=1.5cm] (rtu2) at (0,0.3) {\faIcon{broadcast-tower}\hspace{0.1em}RTU 2};
    \node[comp=otsuccess, minimum width=1.5cm] (rtu3) at (2,0.3) {\faIcon{broadcast-tower}\hspace{0.1em}RTU 3};

    \draw[arr=otaccent] (mtu.south) -- ++(0,-0.25) -| (rtu1.north);
    \draw[arr=otaccent] (mtu.south) -- (rtu2.north);
    \draw[arr=otaccent] (mtu.south) -- ++(0,-0.25) -| (rtu3.north);

    % Field devices
    \node[comp=otwarning, minimum width=1cm, font=\scriptsize] (f1) at (-2,-0.5) {\faIcon{cog}\hspace{0.1em}Field};
    \node[comp=otwarning, minimum width=1cm, font=\scriptsize] (f2) at (0,-0.5) {\faIcon{cog}\hspace{0.1em}Field};
    \node[comp=otwarning, minimum width=1cm, font=\scriptsize] (f3) at (2,-0.5) {\faIcon{cog}\hspace{0.1em}Field};
    \draw[-, thick, otsuccess!40] (rtu1) -- (f1);
    \draw[-, thick, otsuccess!40] (rtu2) -- (f2);
    \draw[-, thick, otsuccess!40] (rtu3) -- (f3);
\end{tikzpicture}
\end{center}

\section{\textcolor{accent}{\faIcon{desktop}}\hspace{0.4em}HMI Types}

\begin{center}
\rowcolors{2}{lightgray}{white}
\begin{tabular}{p{2cm}p{5cm}}
\rowcolor{primary}
\textcolor{white}{\bfseries Type} & \textcolor{white}{\bfseries Use Case} \\
\midrule
\faIcon{tablet-alt}\hspace{0.2em}Panel HMI & Dedicated touchscreen, machine-level control \\
\faIcon{desktop}\hspace{0.2em}PC-based & Windows/Linux workstation, complex processes \\
\faIcon{globe}\hspace{0.2em}Web HMI & Browser-based, remote monitoring \\
\faIcon{mobile-alt}\hspace{0.2em}Mobile HMI & Tablet/smartphone for field operations \\
\end{tabular}
\end{center}

\subsection{\textcolor{accent}{\faIcon{columns}}\hspace{0.3em}SCADA vs DCS}

\begin{center}
\rowcolors{2}{lightgray}{white}
\begin{tabular}{p{2cm}p{2.3cm}p{2.3cm}}
\rowcolor{primary}
\textcolor{white}{\bfseries Aspect} & \textcolor{white}{\bfseries SCADA} & \textcolor{white}{\bfseries DCS} \\
\midrule
Geography & Wide area & Single site \\
Control & Supervisory & Continuous \\
Comms & WAN/radio/cell & High-speed LAN \\
Latency & Seconds--min & Milliseconds \\
Industries & Utilities, water & Chemical, refining \\
\end{tabular}
\end{center}

\section{\textcolor{accent}{\faIcon{network-wired}}\hspace{0.4em}Protocols}

\begin{center}
\rowcolors{2}{lightgray}{white}
\begin{tabular}{p{2.3cm}p{2.3cm}p{2.3cm}}
\rowcolor{primary}
\textcolor{white}{\bfseries Protocol} & \textcolor{white}{\bfseries Common Use} & \textcolor{white}{\bfseries Security} \\
\midrule
Modbus TCP/RTU & Legacy HMI-PLC & \risklow None \\
DNP3 & Utilities SCADA & \riskmedium Optional auth \\
IEC 61850 & Substation auto. & \riskmedium TLS capable \\
OPC DA/UA & Multi-vendor HMI & \riskhigh UA built-in \\
IEC 60870-5-104 & Power grid SCADA & \riskmedium Optional auth \\
\end{tabular}
\end{center}

\section{\textcolor{accent}{\faIcon{exclamation-triangle}}\hspace{0.4em}Security Threats}

\posterdanger{
\textbf{Common vulnerabilities:}
\begin{itemize}
    \item Default/weak credentials on HMI and SCADA servers
    \item Unpatched Windows OS on HMI workstations (XP, 7 still common)
    \item Insecure protocols without authentication or encryption
    \item SQL/command injection in web-based HMI interfaces
    \item Exposed SCADA servers on corporate/internet networks
\end{itemize}
}

\subsection{\textcolor{accent}{\faIcon{crosshairs}}\hspace{0.3em}Attack Vectors}

\begin{itemize}
    \item \textcolor{danger}{\faIcon{network-wired}}\hspace{0.2em}\textbf{Network-based:} Exploiting exposed SCADA/HMI on corporate networks
    \item \textcolor{danger}{\faIcon{code}}\hspace{0.2em}\textbf{Protocol manipulation:} Injecting commands via insecure protocols
    \item \textcolor{warning}{\faIcon{user-secret}}\hspace{0.2em}\textbf{Social engineering:} Phishing operators for HMI credentials
    \item \textcolor{warning}{\faIcon{box}}\hspace{0.2em}\textbf{Supply chain:} Compromised HMI software or vendor access
    \item \textcolor{info}{\faIcon{user-shield}}\hspace{0.2em}\textbf{Insider threats:} Malicious or negligent authorized users
\end{itemize}

\subsection{\textcolor{accent}{\faIcon{history}}\hspace{0.3em}Notable Incidents}

\begin{center}
\rowcolors{2}{lightgray}{white}
\begin{tabular}{p{1.2cm}p{5.8cm}}
\rowcolor{primary}
\textcolor{white}{\bfseries Year} & \textcolor{white}{\bfseries Incident} \\
\midrule
2015/16 & Ukraine -- HMI access used to open breakers \\
2021 & Oldsmar -- Remote HMI access, chemical attack \\
2021 & Colonial Pipeline -- IT ransomware shut OT \\
\end{tabular}
\end{center}

\section{\textcolor{accent}{\faIcon{shield-alt}}\hspace{0.4em}Security Best Practices}

\postersuccess{
\textbf{Access Control:} RBAC for all HMI users, strong passwords, MFA for remote access, disable default accounts, log all operator actions.

\textbf{Network Security:} Isolate HMI/SCADA from corporate IT, firewalls between zones, IDS tuned for OT protocols, encrypt where possible.

\textbf{System Hardening:} Patch after testing, application whitelisting, disable USB ports, host-based firewalls, remove unnecessary software.
}

\postertip{
Compromised HMI/SCADA provides \textbf{direct control over physical processes}. Defense in depth is essential: access control, network segmentation, system hardening, and continuous monitoring.
}

\end{multicols}

\end{document}
