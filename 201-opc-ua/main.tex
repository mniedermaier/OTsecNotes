% ============================================================================
%  OPC UA - OT Security Learning Resource
% ============================================================================

\documentclass[11pt,a4paper]{article}
\usepackage{otsec-template}

\hypersetup{
    pdftitle={OPC UA},
    pdfsubject={Understanding OPC Unified Architecture},
}

\begin{document}

% ----------------------------------------------------------------------------
%  TITLE PAGE
% ----------------------------------------------------------------------------

\maketitlepage
    {OPC UA}
    {Understanding OPC Unified Architecture}
    {OT Security Learning Series}
    {Document 201 \quad|\quad January 2026}
    {Matthias Niedermaier}

% ----------------------------------------------------------------------------
%  TABLE OF CONTENTS
% ----------------------------------------------------------------------------

\tableofcontents
\newpage

% ----------------------------------------------------------------------------
%  INTRODUCTION
% ----------------------------------------------------------------------------

\section{Introduction}

OPC Unified Architecture (OPC UA) is a platform-independent, service-oriented architecture for industrial communication. Developed by the OPC Foundation, it was designed to replace legacy OPC Classic and address its limitations, including security.

\begin{infobox}
OPC UA is increasingly adopted as the standard for Industry 4.0 and IIoT applications. Unlike most legacy protocols, it includes built-in security features---but proper configuration is essential.
\end{infobox}

\subsection{OPC Classic vs OPC UA}

\begin{center}
\small
\rowcolors{2}{lightgray}{white}
\begin{tabular}{p{4cm}p{4.5cm}p{4.5cm}}
\rowcolor{primary}
\textcolor{white}{\bfseries Aspect} & \textcolor{white}{\bfseries OPC Classic} & \textcolor{white}{\bfseries OPC UA} \\
\midrule
Platform & Windows only (COM/DCOM) & Platform independent \\
Transport & DCOM (dynamic ports) & TCP, HTTPS, WebSockets \\
Security & Windows security only & Built-in encryption \& auth \\
Firewall & Difficult (DCOM) & Simple (single port) \\
Data Model & Separate specs (DA, HDA, A\&E) & Unified information model \\
Released & 1996 & 2008 \\
\end{tabular}
\end{center}

% ----------------------------------------------------------------------------
%  ARCHITECTURE
% ----------------------------------------------------------------------------

\section{Architecture Overview}

\subsection{Communication Stack}

OPC UA defines a layered communication architecture:

\begin{conceptbox}{OPC UA Stack Layers}
\begin{itemize}
    \item \textbf{Application Layer:} Client/server application logic
    \item \textbf{Service Layer:} OPC UA services (Read, Write, Subscribe, etc.)
    \item \textbf{Security Layer:} Authentication, encryption, signing
    \item \textbf{Transport Layer:} TCP binary, HTTPS, WebSockets
\end{itemize}
\end{conceptbox}

\subsection{Information Model}

OPC UA uses a rich, object-oriented information model:

\begin{itemize}
    \item \textbf{Nodes:} Basic building blocks (Objects, Variables, Methods)
    \item \textbf{References:} Relationships between nodes
    \item \textbf{Address Space:} Hierarchical namespace of all nodes
    \item \textbf{Companion Specs:} Industry-specific data models (e.g., for robotics, PLCs)
\end{itemize}

\subsection{Communication Patterns}

\begin{center}
\small
\rowcolors{2}{lightgray}{white}
\begin{tabular}{p{3.5cm}p{9.5cm}}
\rowcolor{primary}
\textcolor{white}{\bfseries Pattern} & \textcolor{white}{\bfseries Description} \\
\midrule
Client-Server & Traditional request-response model \\
Publish-Subscribe & Event-driven data distribution (OPC UA PubSub) \\
\end{tabular}
\end{center}

% ----------------------------------------------------------------------------
%  SECURITY FEATURES
% ----------------------------------------------------------------------------

\section{Security Features}

\subsection{Security Model}

\begin{successbox}
\textbf{OPC UA includes comprehensive security features:}
\begin{itemize}
    \item Authentication (user and application identity)
    \item Authorization (role-based access control)
    \item Confidentiality (encryption)
    \item Integrity (message signing)
    \item Auditability (event logging)
\end{itemize}
\end{successbox}

\subsection{Security Modes}

OPC UA defines three security modes:

\begin{center}
\small
\rowcolors{2}{lightgray}{white}
\begin{tabular}{p{3cm}p{2.5cm}p{2.5cm}p{4.5cm}}
\rowcolor{primary}
\textcolor{white}{\bfseries Mode} & \textcolor{white}{\bfseries Encryption} & \textcolor{white}{\bfseries Signing} & \textcolor{white}{\bfseries Use Case} \\
\midrule
None & No & No & Testing only (never in production) \\
Sign & No & Yes & Integrity without confidentiality \\
SignAndEncrypt & Yes & Yes & Full security (recommended) \\
\end{tabular}
\end{center}

\begin{dangerbox}
Security Mode ``None'' disables all security. Unfortunately, many deployments use this mode for convenience, negating OPC UA's security advantages.
\end{dangerbox}

\subsection{Authentication Methods}

\begin{itemize}
    \item \textbf{Anonymous:} No authentication (not recommended)
    \item \textbf{Username/Password:} Basic credential authentication
    \item \textbf{X.509 Certificates:} Strong application and user authentication
    \item \textbf{Kerberos:} Enterprise SSO integration
\end{itemize}

\subsection{Certificate Management}

OPC UA relies heavily on X.509 certificates:

\begin{conceptbox}{Certificate Trust Model}
\begin{itemize}
    \item Each application has its own certificate
    \item Clients and servers must trust each other's certificates
    \item Trust can be explicit (trusted list) or CA-based
    \item Certificate validation includes expiry, revocation, hostname
\end{itemize}
\end{conceptbox}

% ----------------------------------------------------------------------------
%  SECURITY CONCERNS
% ----------------------------------------------------------------------------

\section{Security Concerns}

\subsection{Common Misconfigurations}

\begin{warningbox}
\textbf{Security features don't help if not properly configured:}
\begin{itemize}
    \item Security Mode set to ``None''
    \item Anonymous authentication enabled
    \item Self-signed certificates without proper trust management
    \item Default or weak credentials
    \item Missing certificate revocation checks
\end{itemize}
\end{warningbox}

\subsection{Known Vulnerabilities}

OPC UA implementations have had security vulnerabilities:

\begin{itemize}
    \item \textbf{Stack implementations:} Buffer overflows, DoS vulnerabilities
    \item \textbf{Certificate handling:} Improper validation, path traversal
    \item \textbf{Authentication bypass:} Implementation-specific flaws
\end{itemize}

\begin{tipbox}
The OPC UA specification is complex. Implementation vulnerabilities are common. Keep OPC UA software updated and monitor security advisories.
\end{tipbox}

\subsection{Attack Scenarios}

\begin{enumerate}
    \item \textbf{Discovery abuse:} Enumerate servers and endpoints
    \item \textbf{Downgrade attacks:} Force use of weaker security modes
    \item \textbf{Certificate theft:} Steal private keys to impersonate applications
    \item \textbf{DoS attacks:} Exhaust server resources with malformed requests
    \item \textbf{Information disclosure:} Access sensitive process data
\end{enumerate}

% ----------------------------------------------------------------------------
%  BEST PRACTICES
% ----------------------------------------------------------------------------

\section{Security Best Practices}

\subsection{Configuration Hardening}

\begin{itemize}
    \item \textbf{Always use SignAndEncrypt} security mode
    \item \textbf{Disable anonymous access} -- require authentication
    \item \textbf{Use strong security policies} (Basic256Sha256 or better)
    \item \textbf{Implement proper certificate management} with PKI
    \item \textbf{Enable audit logging} for security events
    \item \textbf{Apply least privilege} through role-based access
\end{itemize}

\subsection{Network Security}

\begin{itemize}
    \item \textbf{Segment OPC UA traffic} in dedicated network zones
    \item \textbf{Use firewalls} to restrict access to OPC UA ports (4840)
    \item \textbf{Monitor traffic} for anomalous patterns
    \item \textbf{Consider additional encryption} (VPN) for cross-zone traffic
\end{itemize}

\subsection{Operational Security}

\begin{itemize}
    \item \textbf{Keep software updated} -- patch known vulnerabilities
    \item \textbf{Inventory all OPC UA endpoints} in your environment
    \item \textbf{Review security configurations} regularly
    \item \textbf{Test security settings} before deployment
\end{itemize}

% ----------------------------------------------------------------------------
%  FURTHER READING
% ----------------------------------------------------------------------------

\section{Further Reading}

\subsection*{Specifications}
\begin{itemize}
    \item \textbf{OPC Foundation} -- OPC UA Specifications\\
          \url{https://opcfoundation.org/developer-tools/specifications-unified-architecture}
    \item \textbf{OPC UA Security Analysis} -- BSI (German Federal Office)\\
          \url{https://www.bsi.bund.de/EN/}
\end{itemize}

\subsection*{Standards}
\begin{itemize}
    \item \textbf{IEC 62541} -- OPC Unified Architecture (international standard)
    \item \textbf{IEC 62443} -- Industrial Automation Security\\
          \url{https://www.isa.org/standards-and-publications/isa-standards/isa-iec-62443-series-of-standards}
\end{itemize}

\subsection*{Resources}
\begin{itemize}
    \item \textbf{CISA} -- ICS Advisories\\
          \url{https://www.cisa.gov/news-events/ics-advisories}
\end{itemize}

\vfill
\begin{center}
\textcolor{mediumgray}{\rule{0.5\textwidth}{0.5pt}}\\[1em]
\textcolor{mediumgray}{\small Part of the OT Security Learning Series}
\end{center}

\end{document}
