% ============================================================================
%  610-ot-forensics - OT Security Learning Resource
% ============================================================================

\documentclass[11pt,a4paper]{article}
\usepackage{otsec-template}

% Define colors for TikZ (matching template colors)
\colorlet{otprimary}{primary}
\colorlet{otaccent}{accent}
\colorlet{otsuccess}{success}
\colorlet{otwarning}{warning}
\colorlet{otdanger}{danger}
\colorlet{otinfo}{info}

\begin{document}

\maketitlepage
    {OT Forensics}
    {Evidence collection and analysis in industrial control system environments}
    {OT Security Learning Series}
    {Document 610 \quad|\quad January 2026}
    {Matthias Niedermaier}

\tableofcontents
\newpage

% ============================================================================
\section{Introduction}
% ============================================================================

\begin{infobox}
Digital forensics in OT environments requires specialized approaches that balance evidence preservation with operational continuity. Unlike IT forensics, OT investigators must work with proprietary systems, real-time constraints, and equipment that cannot be easily taken offline.
\end{infobox}

OT forensics challenges include:
\begin{itemize}
    \item Systems that cannot be shut down for imaging
    \item Proprietary file systems and data formats
    \item Limited or no logging capabilities
    \item Volatile evidence in controller memory
    \item Chain of custody across IT and OT boundaries
\end{itemize}

% ============================================================================
\section{OT vs IT Forensics}
% ============================================================================

\begin{table}[h]
\centering
\small
\begin{tabularx}{\textwidth}{|l|X|X|}
\hline
\textbf{Aspect} & \textbf{IT Forensics} & \textbf{OT Forensics} \\
\hline
System access & Can often image offline & Must preserve operations \\
Evidence sources & Disk, memory, logs & Controllers, network, historian \\
Tools & Standard forensic suites & Vendor-specific, custom tools \\
File systems & NTFS, ext4, APFS & Proprietary, embedded \\
Time sensitivity & Hours to days & Minutes to hours \\
Expertise & IT security analysts & OT engineers + forensics \\
\hline
\end{tabularx}
\caption{IT vs OT Forensics Comparison}
\end{table}

\begin{warningbox}
\textbf{Critical Difference:} In IT, you typically take systems offline for imaging. In OT, shutting down a PLC or HMI may halt production or create safety hazards. Live forensics is often the only option.
\end{warningbox}

% ============================================================================
\section{Evidence Sources}
% ============================================================================

\begin{figure}[h]
\centering
\begin{tikzpicture}[
    source/.style={rectangle, draw, thick, fill=otaccent!20, minimum height=1cm, minimum width=2.3cm, rounded corners=3pt, font=\tiny\bfseries, align=center},
    arrow/.style={->, thick, >=stealth}
]

\node[source] (network) at (0,0) {Network\\Traffic};
\node[source] (historian) at (2.8,0) {Historian\\Data};
\node[source] (logs) at (5.6,0) {System\\Logs};
\node[source] (plc) at (8.4,0) {Controller\\Memory};
\node[source] (hmi) at (11.2,0) {HMI/SCADA\\Systems};

\node[font=\tiny, text width=2cm, align=center] at (0,-1.2) {PCAPs\\Protocol data\\Flow records};
\node[font=\tiny, text width=2cm, align=center] at (2.8,-1.2) {Process values\\Alarms\\Trends};
\node[font=\tiny, text width=2cm, align=center] at (5.6,-1.2) {Event logs\\Auth logs\\Audit trails};
\node[font=\tiny, text width=2cm, align=center] at (8.4,-1.2) {Program logic\\Variables\\Diagnostics};
\node[font=\tiny, text width=2cm, align=center] at (11.2,-1.2) {Screenshots\\Configs\\User actions};

\end{tikzpicture}
\caption{OT Evidence Sources}
\end{figure}

\subsection{Network Evidence}

\begin{itemize}
    \item \textbf{Packet captures} -- Full PCAP from network taps or SPAN ports
    \item \textbf{NetFlow/IPFIX} -- Connection metadata and flow statistics
    \item \textbf{Firewall logs} -- Allowed and denied connections
    \item \textbf{IDS/IPS alerts} -- Detection events with packet samples
    \item \textbf{Protocol-specific logs} -- Modbus, DNP3, OPC UA transactions
\end{itemize}

\subsection{Historian Data}

\begin{successbox}
\textbf{Key Evidence:} Historians often contain the best timeline of process anomalies. Sudden setpoint changes, unusual values, or gaps in data collection can indicate compromise.
\end{successbox}

\begin{itemize}
    \item Process variable trends (before, during, after incident)
    \item Alarm and event sequences
    \item Operator actions and acknowledgments
    \item System health metrics
\end{itemize}

\subsection{Controller Evidence}

\begin{itemize}
    \item \textbf{Program files} -- Current logic vs known-good baseline
    \item \textbf{Runtime variables} -- Current values in memory
    \item \textbf{Diagnostic buffers} -- Error logs, communication stats
    \item \textbf{Firmware version} -- Check for unauthorized changes
    \item \textbf{Project files} -- Engineering workstation copies
\end{itemize}

\subsection{Endpoint Evidence}

\begin{itemize}
    \item HMI screenshots and session recordings
    \item Engineering workstation disk images
    \item USB device connection logs
    \item Remote access session logs
    \item Antivirus/EDR detection logs
\end{itemize}

% ============================================================================
\section{Evidence Collection Process}
% ============================================================================

\begin{figure}[h]
\centering
\begin{tikzpicture}[
    procstep/.style={rectangle, draw, thick, fill=otaccent!20, minimum height=0.9cm, minimum width=1.6cm, rounded corners=3pt, font=\tiny\bfseries, align=center},
    arrow/.style={->, thick, >=stealth}
]

\node[procstep] (scope) at (0,0) {1. Define\\Scope};
\node[procstep] (volatile) at (2.3,0) {2. Capture\\Volatile};
\node[procstep] (network) at (4.6,0) {3. Network\\Evidence};
\node[procstep] (storage) at (6.9,0) {4. Storage\\Imaging};
\node[procstep] (document) at (9.2,0) {5. Document\\Everything};

\draw[arrow, otprimary] (scope) -- (volatile);
\draw[arrow, otprimary] (volatile) -- (network);
\draw[arrow, otprimary] (network) -- (storage);
\draw[arrow, otprimary] (storage) -- (document);

\end{tikzpicture}
\caption{Evidence Collection Workflow}
\end{figure}

\subsection{Order of Volatility}

Collect evidence in order of how quickly it may be lost:

\begin{enumerate}
    \item \textbf{Network traffic} -- Capture immediately, flows constantly
    \item \textbf{Controller memory} -- Runtime state, may change with process
    \item \textbf{System memory} -- RAM on Windows/Linux systems
    \item \textbf{Running processes} -- Active connections, loaded modules
    \item \textbf{Historian data} -- May be overwritten by retention policies
    \item \textbf{Disk storage} -- Most persistent, collect last
\end{enumerate}

\subsection{Live Collection Techniques}

\begin{definitionbox}{Non-Disruptive Collection}
\begin{itemize}
    \item \textbf{Network tap} -- Passive copy of all traffic
    \item \textbf{SPAN/mirror port} -- Switch-based traffic copy
    \item \textbf{Read-only PLC upload} -- Extract program without stopping
    \item \textbf{Historian export} -- Query historical data via API
    \item \textbf{Log forwarding} -- Real-time copy to forensic server
\end{itemize}
\end{definitionbox}

\begin{dangerbox}
\textbf{Do Not:}
\begin{itemize}
    \item Stop or restart controllers during evidence collection
    \item Install forensic agents on production OT systems
    \item Run active network scans that may disrupt protocols
    \item Modify system configurations to enable logging
\end{itemize}
\end{dangerbox}

% ============================================================================
\section{Chain of Custody}
% ============================================================================

Maintain rigorous documentation:

\begin{itemize}
    \item \textbf{Who} collected the evidence
    \item \textbf{What} was collected (hashes, descriptions)
    \item \textbf{When} collection occurred (timestamps)
    \item \textbf{Where} evidence was stored
    \item \textbf{How} collection was performed (tools, methods)
\end{itemize}

\subsection{Evidence Integrity}

\begin{itemize}
    \item Calculate cryptographic hashes (SHA-256) immediately
    \item Use write-blockers for disk imaging
    \item Store evidence on encrypted, access-controlled media
    \item Maintain detailed logs of all access
    \item Create working copies for analysis
\end{itemize}

% ============================================================================
\section{Analysis Techniques}
% ============================================================================

\subsection{Timeline Analysis}

Correlate events across multiple sources:
\begin{itemize}
    \item Historian timestamps (process anomalies)
    \item Network capture timestamps (malicious traffic)
    \item Log timestamps (authentication, errors)
    \item Normalize all times to UTC
\end{itemize}

\subsection{PLC Program Analysis}

\begin{itemize}
    \item Compare current program to known-good baseline
    \item Identify unauthorized function blocks or logic changes
    \item Check for hidden routines or conditional triggers
    \item Analyze program upload/download history
\end{itemize}

\subsection{Network Traffic Analysis}

\begin{itemize}
    \item Identify unauthorized connections
    \item Analyze industrial protocol commands
    \item Look for reconnaissance (scans, enumeration)
    \item Detect data exfiltration patterns
    \item Check for C2 communication signatures
\end{itemize}

% ============================================================================
\section{Tools for OT Forensics}
% ============================================================================

\begin{table}[h]
\centering
\small
\begin{tabularx}{\textwidth}{|l|X|}
\hline
\textbf{Category} & \textbf{Tools} \\
\hline
Network capture & Wireshark, tcpdump, NetworkMiner \\
Protocol analysis & Wireshark dissectors, custom parsers \\
Disk imaging & FTK Imager, dd, Guymager \\
Memory analysis & Volatility, Rekall \\
Timeline & log2timeline/Plaso, Timesketch \\
PLC analysis & Vendor tools, custom scripts \\
\hline
\end{tabularx}
\caption{OT Forensics Tool Categories}
\end{table}

% ============================================================================
\section{Summary}
% ============================================================================

\begin{definitionbox}{Key Takeaways}
\begin{itemize}
    \item \textbf{Operational priority} -- Evidence collection must not disrupt operations
    \item \textbf{Multiple sources} -- Network, historian, controllers, endpoints
    \item \textbf{Order of volatility} -- Capture ephemeral evidence first
    \item \textbf{Live forensics} -- Often the only option in OT
    \item \textbf{Chain of custody} -- Document everything for legal proceedings
    \item \textbf{Baseline comparison} -- Compare against known-good configurations
    \item \textbf{Cross-domain expertise} -- Requires both OT and forensics skills
\end{itemize}
\end{definitionbox}

% ============================================================================
\section{Further Reading}
% ============================================================================

\subsection*{Standards and Guidelines}

\begin{itemize}
    \item \textbf{NIST SP 800-86} -- Guide to Integrating Forensic Techniques\\
          \url{https://csrc.nist.gov/publications/detail/sp/800-86/final}
    \item \textbf{IEC 62443-2-1} -- Security program requirements\\
          \url{https://webstore.iec.ch/publication/7030}
\end{itemize}

\subsection*{Resources}

\begin{itemize}
    \item \textbf{CISA -- ICS Forensics Resources}\\
          \url{https://www.cisa.gov/topics/industrial-control-systems}
    \item \textbf{SANS ICS -- Digital Forensics}\\
          \url{https://www.sans.org/cyber-security-courses/ics-digital-forensics/}
\end{itemize}

\subsection*{Books}

\begin{itemize}
    \item Knapp, E. \& Langill, J. -- \textit{Industrial Network Security} (Syngress)
    \item NIST -- \textit{Guide to Industrial Control Systems Security}
\end{itemize}

\vfill
\begin{center}
\textit{Part of the OT Security Learning Series}
\end{center}

\end{document}
