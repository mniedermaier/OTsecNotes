% ============================================================================
%  650-tabletop-exercises - OT Security Learning Resource
% ============================================================================

\documentclass[11pt,a4paper]{article}
\usepackage{otsec-template}
\usepackage{float}

% Define colors for TikZ
\colorlet{otprimary}{primary}
\colorlet{otaccent}{accent}
\colorlet{otsuccess}{success}
\colorlet{otwarning}{warning}
\colorlet{otdanger}{danger}
\colorlet{otinfo}{info}

\begin{document}

\maketitlepage
    {OT Tabletop Exercises}
    {Testing Incident Response Through Scenario-Based Discussion}
    {OT Security Learning Series}
    {Document 650 \quad|\quad January 2026}
    {Matthias Niedermaier}

\tableofcontents
\newpage

\section{Introduction}

\begin{infobox}
Tabletop exercises are discussion-based sessions where participants walk through simulated incident scenarios to test response plans, identify gaps, and improve coordination. For OT environments, these exercises are essential because real-world testing of incident response on production systems is often impossible without risking safety or operations.
\end{infobox}

Unlike IT environments where systems can often be taken offline for testing, OT systems typically run continuously. Tabletop exercises provide a safe way to validate that teams can respond effectively to cyber incidents affecting industrial processes, without disrupting operations or creating safety hazards.

\section{Exercise Types}

\begin{figure}[H]
\centering
\begin{tikzpicture}[
    box/.style={rectangle, draw=otprimary, thick, fill=otprimary!10,
                rounded corners=5pt, minimum width=5cm, minimum height=1.2cm,
                align=center, font=\small\bfseries},
    desc/.style={rectangle, draw=otaccent, thick, fill=otaccent!5,
                 rounded corners=3pt, minimum width=5cm, minimum height=2cm,
                 align=left, text width=4.8cm, font=\scriptsize}
]
    % Discussion-based
    \node[box] (disc) at (0,0) {Discussion-Based};
    \node[desc, below=0.3cm of disc] {
        \textbullet\ Tabletop exercises\\
        \textbullet\ Workshops\\
        \textbullet\ Seminars\\
        \textbullet\ Low cost, low risk\\
        \textbullet\ Focus on plans/procedures
    };

    % Operations-based
    \node[box] (ops) at (7,0) {Operations-Based};
    \node[desc, below=0.3cm of ops] {
        \textbullet\ Drills\\
        \textbullet\ Functional exercises\\
        \textbullet\ Full-scale exercises\\
        \textbullet\ Higher cost/complexity\\
        \textbullet\ Tests actual capabilities
    };
\end{tikzpicture}
\caption{Exercise types by complexity}
\end{figure}

\subsection{Tabletop Exercise Characteristics}

\begin{itemize}
    \item \textbf{Format} -- Facilitated group discussion around a scenario
    \item \textbf{Duration} -- Typically 2--4 hours
    \item \textbf{Resources} -- Minimal; requires facilitator and participants
    \item \textbf{Risk} -- None to operations; purely discussion-based
    \item \textbf{Output} -- Identified gaps, action items, improved awareness
\end{itemize}

\section{Planning and Preparation}

\subsection{Define Objectives}

Before designing an exercise, establish clear goals:

\begin{table}[H]
\centering
\small
\rowcolors{2}{lightgray}{white}
\begin{tabular}{p{4cm}p{9cm}}
\rowcolor{primary}
\textcolor{white}{\bfseries Objective Type} & \textcolor{white}{\bfseries Example} \\
\midrule
Test procedures & Validate OT incident response plan steps \\
Identify gaps & Find missing playbooks for specific attack types \\
Train personnel & Familiarize operators with security escalation \\
Test communication & Verify notification chains work as documented \\
Assess decisions & Evaluate criteria for OT system isolation \\
\end{tabular}
\caption{Example exercise objectives}
\end{table}

\subsection{Develop the Scenario}

Effective OT scenarios should be:

\begin{itemize}
    \item \textbf{Realistic} -- Based on actual threats to your industry
    \item \textbf{Relevant} -- Target systems participants actually manage
    \item \textbf{Challenging} -- Force difficult decisions, not obvious answers
    \item \textbf{Scoped} -- Achievable within the exercise timeframe
\end{itemize}

\begin{tipbox}
Base scenarios on real incidents from your sector. Adapt published case studies (Stuxnet, TRITON, Colonial Pipeline) to your specific environment and systems.
\end{tipbox}

\subsection{Prepare Materials}

\begin{figure}[H]
\centering
\begin{tikzpicture}[
    item/.style={rectangle, draw=otaccent, thick, fill=otaccent!10,
                 rounded corners=3pt, minimum width=5cm, minimum height=0.7cm,
                 align=center, font=\small},
    num/.style={circle, fill=otprimary, text=white, font=\small\bfseries,
                minimum size=0.6cm}
]
    \node[num] at (0,0) {1};
    \node[item, anchor=west] at (0.6,0) {Scenario narrative and injects};
    \node[num] at (0,-1.0) {2};
    \node[item, anchor=west] at (0.6,-1.0) {Network diagrams and system info};
    \node[num] at (0,-2.0) {3};
    \node[item, anchor=west] at (0.6,-2.0) {Discussion questions per phase};
    \node[num] at (0,-3.0) {4};
    \node[item, anchor=west] at (0.6,-3.0) {Reference documents (IR plans)};
    \node[num] at (0,-4.0) {5};
    \node[item, anchor=west] at (0.6,-4.0) {Participant roles and ground rules};
    \node[num] at (0,-5.0) {6};
    \node[item, anchor=west] at (0.6,-5.0) {Evaluation forms};
\end{tikzpicture}
\caption{Exercise preparation checklist}
\end{figure}

\section{Key Participants}

\subsection{Required Roles}

\begin{table}[H]
\centering
\small
\rowcolors{2}{lightgray}{white}
\begin{tabular}{p{3.5cm}p{9.5cm}}
\rowcolor{primary}
\textcolor{white}{\bfseries Role} & \textcolor{white}{\bfseries Responsibility} \\
\midrule
Facilitator & Guides discussion, presents injects, keeps time \\
OT Operations & Represents control room, field operations \\
OT Engineering & Provides technical system knowledge \\
IT Security & Brings cybersecurity detection/response expertise \\
Management & Makes escalation and business decisions \\
Safety & Ensures safety implications are considered \\
Communications & Handles internal/external messaging \\
\end{tabular}
\caption{Key exercise participants}
\end{table}

\begin{warningbox}
OT tabletop exercises must include operations and engineering staff, not just IT security. Decisions about isolating control systems or shutting down processes require input from those who understand operational consequences.
\end{warningbox}

\section{OT-Specific Scenarios}

\subsection{Scenario Categories}

\begin{table}[H]
\centering
\small
\rowcolors{2}{lightgray}{white}
\begin{tabular}{p{4cm}p{9cm}}
\rowcolor{primary}
\textcolor{white}{\bfseries Category} & \textcolor{white}{\bfseries Scenario Examples} \\
\midrule
Ransomware & Ransomware spreads from IT to historian, threatens HMI \\
Targeted Attack & APT compromises engineering workstation, modifies PLC logic \\
Insider Threat & Disgruntled employee with OT access sabotages process \\
Supply Chain & Vendor's compromised update deployed to RTUs \\
Safety System & Attacker attempts to disable safety instrumented system \\
Data Integrity & Process values manipulated to cause unsafe conditions \\
\end{tabular}
\caption{OT-specific scenario categories}
\end{table}

\subsection{Sample Scenario Structure}

A typical tabletop scenario progresses through phases:

\begin{enumerate}
    \item \textbf{Initial Detection} -- SOC alerts on suspicious activity
    \item \textbf{Investigation} -- Determine scope, affected systems
    \item \textbf{Escalation} -- IT/OT coordination, management notification
    \item \textbf{Containment} -- Decisions on isolation, operational impact
    \item \textbf{Eradication} -- Remove threat while maintaining operations
    \item \textbf{Recovery} -- Restore systems, verify integrity
    \item \textbf{Post-Incident} -- Lessons learned, improvements
\end{enumerate}

\section{Conducting the Exercise}

\subsection{Exercise Flow}

\begin{figure}[H]
\centering
\begin{tikzpicture}[
    phase/.style={rectangle, draw=otprimary, thick, fill=otprimary!10,
                  rounded corners=3pt, minimum width=2.2cm, minimum height=1cm,
                  align=center, font=\scriptsize\bfseries},
    arrow/.style={->, thick, >=stealth, otaccent}
]
    \node[phase] (intro) at (0,0) {Introduction\\(15 min)};
    \node[phase] (p1) at (3,0) {Phase 1\\Inject};
    \node[phase] (d1) at (5.5,0) {Discussion};
    \node[phase] (p2) at (8,0) {Phase 2\\Inject};
    \node[phase] (d2) at (10.5,0) {Discussion};
    \node[phase] (hot) at (13,0) {Hot Wash\\(30 min)};

    \draw[arrow] (intro) -- (p1);
    \draw[arrow] (p1) -- (d1);
    \draw[arrow] (d1) -- (p2);
    \draw[arrow] (p2) -- (d2);
    \draw[arrow] (d2) -- (hot);
\end{tikzpicture}
\caption{Typical tabletop exercise flow}
\end{figure}

\subsection{Facilitation Tips}

\begin{itemize}
    \item \textbf{Stay neutral} -- Don't lead participants to "correct" answers
    \item \textbf{Encourage participation} -- Draw out quieter team members
    \item \textbf{Manage time} -- Keep discussions focused, use parking lot for tangents
    \item \textbf{Capture insights} -- Assign a note-taker for key decisions and gaps
    \item \textbf{No blame} -- Focus on process improvement, not individual criticism
\end{itemize}

\subsection{Key Discussion Questions}

For each scenario phase, ask:

\begin{itemize}
    \item Who needs to be notified? When?
    \item What information do we need to make decisions?
    \item What are our options? Trade-offs of each?
    \item At what point do we isolate OT systems?
    \item How do we maintain safe operations during response?
    \item What if this happens during a critical production period?
\end{itemize}

\section{After-Action Review}

\subsection{Hot Wash}

Immediately after the exercise, conduct a brief debrief:

\begin{itemize}
    \item What went well?
    \item What was confusing or unclear?
    \item What gaps did we identify?
    \item What needs immediate attention?
\end{itemize}

\subsection{After-Action Report}

Document findings formally:

\begin{table}[H]
\centering
\small
\rowcolors{2}{lightgray}{white}
\begin{tabular}{p{3.5cm}p{9.5cm}}
\rowcolor{primary}
\textcolor{white}{\bfseries Section} & \textcolor{white}{\bfseries Content} \\
\midrule
Executive Summary & Key findings and recommendations \\
Exercise Overview & Objectives, scenario, participants \\
Strengths & What worked well, validated capabilities \\
Areas for Improvement & Gaps identified, with specific examples \\
Recommendations & Prioritized action items with owners \\
\end{tabular}
\caption{After-action report structure}
\end{table}

\begin{successbox}
The value of a tabletop exercise is in the improvements that follow. Track action items to completion and incorporate lessons learned into updated plans and procedures.
\end{successbox}

\section{Exercise Frequency}

\begin{table}[H]
\centering
\small
\rowcolors{2}{lightgray}{white}
\begin{tabular}{p{4cm}p{4cm}p{5cm}}
\rowcolor{primary}
\textcolor{white}{\bfseries Exercise Type} & \textcolor{white}{\bfseries Frequency} & \textcolor{white}{\bfseries Purpose} \\
\midrule
Basic tabletop & Quarterly & Maintain awareness, test updates \\
Cross-functional & Semi-annually & IT/OT coordination \\
Executive-level & Annually & Strategic decisions, resource allocation \\
Full-scale (if feasible) & Every 2--3 years & Validate end-to-end capabilities \\
\end{tabular}
\caption{Recommended exercise frequency}
\end{table}

\section{Summary}

\begin{definitionbox}{Key Takeaways}
\begin{itemize}
    \item \textbf{Safe Testing:} Tabletop exercises allow testing OT incident response without operational risk
    \item \textbf{Include OT Staff:} Operations and engineering must participate; decisions affect physical processes
    \item \textbf{Realistic Scenarios:} Base exercises on actual threats to your industry and systems
    \item \textbf{Structured Approach:} Progress through detection, containment, eradication, and recovery phases
    \item \textbf{Document Outcomes:} Capture gaps and improvements in after-action reports
    \item \textbf{Follow Through:} Track action items to completion; exercises without follow-up waste effort
\end{itemize}
\end{definitionbox}

\section{Further Reading}

\subsection*{Government Resources}
\begin{itemize}
    \item \textbf{CISA Tabletop Exercise Packages} -- Ready-made exercises for critical infrastructure\\
          \url{https://www.cisa.gov/resources-tools/services/cisa-tabletop-exercise-packages}
    \item \textbf{FEMA Homeland Security Exercise and Evaluation Program} -- Exercise design guidance\\
          \url{https://www.fema.gov/emergency-managers/national-preparedness/exercises/hseep}
\end{itemize}

\subsection*{Standards}
\begin{itemize}
    \item \textbf{NIST SP 800-84} -- Guide to Test, Training, and Exercise Programs\\
          \url{https://csrc.nist.gov/pubs/sp/800/84/final}
\end{itemize}

\subsection*{Books}
\begin{itemize}
    \item Bodeau \& Graubart -- \textit{Cyber Resiliency Engineering Framework} (MITRE)
    \item Cichonski et al. -- \textit{Computer Security Incident Handling Guide} (NIST)
\end{itemize}

\vfill
\begin{center}
\textit{Part of the OT Security Learning Series}
\end{center}

\end{document}
