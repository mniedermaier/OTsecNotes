% ============================================================================
%  Anomaly Detection in OT - OT Security Learning Resource
% ============================================================================

\documentclass[11pt,a4paper]{article}
\usepackage{otsec-template}
\usepackage{float}

% Define colors for TikZ
\colorlet{otprimary}{primary}
\colorlet{otaccent}{accent}
\colorlet{otsuccess}{success}
\colorlet{otwarning}{warning}
\colorlet{otdanger}{danger}
\colorlet{otinfo}{info}

\begin{document}

\maketitlepage
    {Anomaly Detection in OT}
    {Behavioral analysis for industrial security monitoring}
    {OT Security Learning Series}
    {Document 540 \quad|\quad January 2026}
    {Matthias Niedermaier}

\tableofcontents
\newpage

% ============================================================================
\section{Introduction}
% ============================================================================

\begin{infobox}
Anomaly detection identifies deviations from established normal behavior. OT environments are ideal for anomaly detection because industrial processes are predictable, repetitive, and change slowly---making unusual activity easier to spot.
\end{infobox}

Why anomaly detection is powerful for OT:
\begin{itemize}
    \item \textbf{Predictable traffic:} PLCs communicate in consistent patterns
    \item \textbf{Stable configurations:} Systems rarely change once deployed
    \item \textbf{Defined processes:} Operations follow predictable cycles
    \item \textbf{Novel attack detection:} Can find threats without signatures
    \item \textbf{Insider threat detection:} Identifies unusual authorized user activity
\end{itemize}

% ============================================================================
\section{Types of Anomaly Detection}
% ============================================================================

\subsection{Detection Categories}

\begin{figure}[H]
\centering
\begin{tikzpicture}[
    box/.style={rectangle, draw, thick, rounded corners=5pt, minimum width=3.5cm, minimum height=1.2cm, align=center, font=\small\bfseries},
    sub/.style={rectangle, draw, thick, rounded corners=3pt, minimum width=3cm, minimum height=0.7cm, align=center, font=\scriptsize}
]

% Main categories
\node[box, fill=otinfo!20] (network) at (0,3) {Network\\Anomalies};
\node[box, fill=otsuccess!20] (process) at (5,3) {Process\\Anomalies};
\node[box, fill=otwarning!20] (behavior) at (10,3) {Behavioral\\Anomalies};

% Sub-items
\node[sub, fill=otinfo!10] at (0,1.5) {Traffic patterns};
\node[sub, fill=otinfo!10] at (0,0.7) {Protocol violations};
\node[sub, fill=otinfo!10] at (0,-0.1) {New connections};

\node[sub, fill=otsuccess!10] at (5,1.5) {Sensor values};
\node[sub, fill=otsuccess!10] at (5,0.7) {Setpoint changes};
\node[sub, fill=otsuccess!10] at (5,-0.1) {Timing deviations};

\node[sub, fill=otwarning!10] at (10,1.5) {User actions};
\node[sub, fill=otwarning!10] at (10,0.7) {Access patterns};
\node[sub, fill=otwarning!10] at (10,-0.1) {Command sequences};

\end{tikzpicture}
\caption{Categories of Anomaly Detection in OT}
\end{figure}

\subsection{Network Anomalies}

\begin{definitionbox}{Network-Based Anomaly Detection}
Monitors network traffic patterns to identify unusual communications, new devices, unexpected protocols, or abnormal data volumes.
\end{definitionbox}

Detectable network anomalies:
\begin{itemize}
    \item New device on OT network (unauthorized asset)
    \item Communication between devices that never communicated before
    \item Unusual traffic volume or timing patterns
    \item Protocol violations or malformed packets
    \item Connections to external/internet addresses
    \item Port scans or reconnaissance activity
\end{itemize}

\subsection{Process Anomalies}

\begin{definitionbox}{Process-Based Anomaly Detection}
Analyzes physical process data (sensor values, control signals) to detect deviations that may indicate attacks or equipment problems.
\end{definitionbox}

Detectable process anomalies:
\begin{itemize}
    \item Sensor values outside normal operating ranges
    \item Unexpected setpoint modifications
    \item Control commands inconsistent with process state
    \item Timing anomalies in control loops
    \item Correlation breaks between related variables
\end{itemize}

\subsection{Behavioral Anomalies}

\begin{definitionbox}{Behavioral Anomaly Detection}
Tracks user and system behavior to identify actions that deviate from established patterns, potentially indicating compromise or insider threats.
\end{definitionbox}

Detectable behavioral anomalies:
\begin{itemize}
    \item User accessing systems outside normal hours
    \item Unusual command sequences from operators
    \item Access to assets outside normal job function
    \item Excessive data downloads or queries
    \item Login from unusual locations
\end{itemize}

% ============================================================================
\section{OT-Specific Advantages}
% ============================================================================

\subsection{Why OT is Ideal for Anomaly Detection}

\begin{figure}[H]
\centering
\begin{tikzpicture}[
    compare/.style={rectangle, draw, thick, minimum width=5cm, minimum height=0.7cm, rounded corners=3pt, font=\small}
]

% IT characteristics
\node[font=\small\bfseries] at (-5,3.5) {IT Environment};
\node[compare, fill=otdanger!15] at (-5,2.8) {Dynamic, changing traffic};
\node[compare, fill=otdanger!15] at (-5,2) {Many users, varied behavior};
\node[compare, fill=otdanger!15] at (-5,1.2) {Frequent software changes};
\node[compare, fill=otdanger!15] at (-5,0.4) {High baseline variance};

% OT characteristics
\node[font=\small\bfseries] at (5,3.5) {OT Environment};
\node[compare, fill=otsuccess!15] at (5,2.8) {Predictable, cyclic traffic};
\node[compare, fill=otsuccess!15] at (5,2) {Few users, defined roles};
\node[compare, fill=otsuccess!15] at (5,1.2) {Rare configuration changes};
\node[compare, fill=otsuccess!15] at (5,0.4) {Low baseline variance};

% Arrow
\draw[<->, very thick, otprimary] (-2,1.5) -- (2,1.5);
\node[font=\scriptsize] at (0,0.9) {Anomalies more detectable in OT};

\end{tikzpicture}
\caption{IT vs OT Anomaly Detection Suitability}
\end{figure}

\begin{successbox}
\textbf{OT Anomaly Detection Advantages:}
\begin{itemize}
    \item PLCs poll sensors at fixed intervals---timing anomalies are obvious
    \item Communication pairs are static---new connections stand out
    \item Process values follow physical laws---violations indicate problems
    \item Operators follow procedures---deviations may indicate compromise
\end{itemize}
\end{successbox}

% ============================================================================
\section{Detection Techniques}
% ============================================================================

\subsection{Statistical Methods}

\begin{table}[H]
\centering
\small
\rowcolors{2}{lightgray}{white}
\begin{tabular}{p{4cm}p{9cm}}
\rowcolor{primary}
\textcolor{white}{\bfseries Method} & \textcolor{white}{\bfseries Application} \\
\midrule
Threshold-based & Alert when values exceed defined limits \\
Standard deviation & Flag values outside N standard deviations \\
Moving average & Detect sudden changes from recent baseline \\
Seasonal patterns & Account for time-of-day, day-of-week variations \\
Correlation analysis & Identify breaks in related variable relationships \\
\end{tabular}
\caption{Statistical Anomaly Detection Methods}
\end{table}

\subsection{Machine Learning Approaches}

\begin{table}[H]
\centering
\small
\rowcolors{2}{lightgray}{white}
\begin{tabular}{p{4cm}p{9cm}}
\rowcolor{primary}
\textcolor{white}{\bfseries Approach} & \textcolor{white}{\bfseries Description} \\
\midrule
Clustering & Group similar behaviors, flag outliers \\
Autoencoders & Learn to reconstruct normal data, flag reconstruction errors \\
One-class SVM & Model normal class boundary, detect outside points \\
LSTM networks & Learn temporal sequences, detect sequence breaks \\
Isolation forests & Efficiently identify outliers in high-dimensional data \\
\end{tabular}
\caption{Machine Learning for Anomaly Detection}
\end{table}

\begin{warningbox}
\textbf{ML Model Considerations:}
\begin{itemize}
    \item Training data must represent true ``normal'' (not already compromised)
    \item Models need retraining after legitimate system changes
    \item Black-box models may be difficult to explain to operators
    \item False positive rates must be acceptable for OT operations
\end{itemize}
\end{warningbox}

\subsection{Rule-Based Detection}

\begin{itemize}
    \item \textbf{Whitelist approach:} Define allowed behaviors, alert on anything else
    \item \textbf{Communication matrix:} Specify valid source/destination/protocol combinations
    \item \textbf{Command validation:} Verify commands are appropriate for current process state
    \item \textbf{Sequence rules:} Define valid command sequences, flag violations
\end{itemize}

% ============================================================================
\section{Baseline Development}
% ============================================================================

\subsection{Creating Effective Baselines}

\begin{figure}[H]
\centering
\begin{tikzpicture}[
    stepbox/.style={rectangle, draw, thick, fill=otaccent!15, minimum width=9cm, minimum height=0.7cm, rounded corners=3pt, font=\small},
    num/.style={circle, fill=otprimary, text=white, font=\small\bfseries, minimum size=0.6cm}
]

\node[stepbox] (s1) at (0,4) {Verify network is not already compromised};
\node[num] at (-5,4) {1};

\node[stepbox] (s2) at (0,3) {Collect data across all operational modes};
\node[num] at (-5,3) {2};

\node[stepbox] (s3) at (0,2) {Include startup, shutdown, maintenance periods};
\node[num] at (-5,2) {3};

\node[stepbox] (s4) at (0,1) {Account for seasonal/cyclical variations};
\node[num] at (-5,1) {4};

\node[stepbox] (s5) at (0,0) {Validate baseline with OT personnel};
\node[num] at (-5,0) {5};

\end{tikzpicture}
\caption{Baseline Development Process}
\end{figure}

\subsection{Baseline Elements}

\begin{table}[H]
\centering
\small
\rowcolors{2}{lightgray}{white}
\begin{tabular}{p{4cm}p{9cm}}
\rowcolor{primary}
\textcolor{white}{\bfseries Element} & \textcolor{white}{\bfseries What to Capture} \\
\midrule
Communication pairs & Source IP, dest IP, ports, protocols \\
Traffic volume & Bytes/packets per time period \\
Timing patterns & Request/response intervals, polling rates \\
Protocol content & Function codes, register addresses \\
User behavior & Login times, systems accessed, commands used \\
Process values & Normal ranges, correlations, trends \\
\end{tabular}
\caption{Baseline Elements for OT Anomaly Detection}
\end{table}

\begin{tipbox}
\textbf{Baseline Duration:} Capture at least 2--4 weeks of normal operations, including any scheduled maintenance periods. For seasonal processes, longer baselines may be needed.
\end{tipbox}

% ============================================================================
\section{Implementation Architecture}
% ============================================================================

\subsection{Data Collection Points}

\begin{figure}[H]
\centering
\begin{tikzpicture}[
    zone/.style={rectangle, draw, thick, rounded corners=5pt, minimum width=2.5cm, minimum height=0.8cm, align=center, font=\small},
    sensor/.style={rectangle, fill=otsuccess, text=white, font=\tiny\bfseries, minimum width=0.5cm, minimum height=0.4cm, rounded corners=2pt}
]

% Network diagram - zones
\node[zone, fill=otinfo!15] (enterprise) at (0,3) {Enterprise};
\node[zone, fill=otwarning!15] (dmz) at (0,2) {DMZ};
\node[zone, fill=otsuccess!15] (control) at (0,1) {Control};
\node[zone, fill=otdanger!15] (field) at (0,0) {Field};

% Network sensors (right side)
\node[sensor] (s1) at (3,2.5) {N};
\node[sensor] (s2) at (3,1.5) {N};
\node[sensor] (s3) at (3,0.5) {N};

% Process sensor (left side)
\node[sensor, fill=otaccent] (p1) at (-2.5,0.5) {P};

% Anomaly detection engine
\node[rectangle, draw, thick, fill=otprimary!20, minimum width=2.2cm, minimum height=1.8cm, font=\small, align=center] (engine) at (6,1.5) {Anomaly\\Detection\\Engine};

% Zone to network sensor connections
\draw[thick, dashed] (enterprise.east) -| (s1);
\draw[thick, dashed] (dmz.east) -| (s2);
\draw[thick, dashed] (control.east) -| (s3);

% Network sensors to engine
\draw[->] (s1.east) -- (engine.west);
\draw[->] (s2.east) -- (engine.west);
\draw[->] (s3.east) -- (engine.west);

% Process sensor connections
\draw[thick, dashed] (control.west) -- (p1);
\draw[thick, dashed] (field.west) -- (p1);
\draw[->] (p1.south) |- (-2.5,-0.8) -| (engine.south);

% Legend
\node[font=\scriptsize, anchor=west] at (4.5,3.5) {N = Network sensor};
\node[font=\scriptsize, anchor=west] at (4.5,3) {P = Process sensor};

\end{tikzpicture}
\caption{Anomaly Detection Sensor Placement}
\end{figure}

\subsection{Integration with Other Systems}

\begin{itemize}
    \item \textbf{SIEM integration:} Send alerts for correlation with other events
    \item \textbf{Asset inventory:} Context about device type, criticality, owner
    \item \textbf{Change management:} Suppress alerts during planned changes
    \item \textbf{Historian:} Process data for process anomaly detection
    \item \textbf{IDS/IPS:} Complement signature-based detection
\end{itemize}

% ============================================================================
\section{Operational Considerations}
% ============================================================================

\subsection{Handling False Positives}

\begin{warningbox}
\textbf{False Positive Management:}
\begin{itemize}
    \item High false positive rates cause alert fatigue
    \item Operators may disable or ignore detection systems
    \item Tune thresholds carefully based on operational feedback
    \item Create exception rules for known acceptable deviations
    \item Continuously refine models based on feedback
\end{itemize}
\end{warningbox}

\subsection{Responding to Anomalies}

\begin{enumerate}
    \item \textbf{Alert triage:} Determine if anomaly is security-relevant
    \item \textbf{Context gathering:} Check for planned changes, maintenance
    \item \textbf{OT consultation:} Verify with operators if behavior is expected
    \item \textbf{Investigation:} If suspicious, investigate further
    \item \textbf{Feedback loop:} Update model if false positive confirmed
\end{enumerate}

\subsection{Model Maintenance}

\begin{itemize}
    \item Retrain models after significant system changes
    \item Periodically validate detection effectiveness
    \item Update baselines when legitimate changes occur
    \item Document all model changes and tuning decisions
    \item Test detection with red team exercises
\end{itemize}

% ============================================================================
\section{Summary}
% ============================================================================

\begin{definitionbox}{Key Takeaways}
\begin{itemize}
    \item \textbf{OT is ideal for anomaly detection:} Predictable, stable, cyclic
    \item \textbf{Three categories:} Network, process, and behavioral anomalies
    \item \textbf{Baseline is critical:} Must capture true normal before deployment
    \item \textbf{Multiple techniques:} Statistical, ML, and rule-based approaches
    \item \textbf{False positives matter:} High rates cause operational rejection
    \item \textbf{OT context required:} Operators must validate anomalies
    \item \textbf{Continuous refinement:} Models need ongoing maintenance
\end{itemize}
\end{definitionbox}

% ============================================================================
\section{Further Reading}
% ============================================================================

\subsection*{Standards}
\begin{itemize}
    \item \textbf{IEC 62443-3-3} -- System security requirements\\
          \url{https://webstore.iec.ch/publication/7033}
    \item \textbf{NIST SP 800-82 Rev. 3} -- Guide to OT Security\\
          \url{https://csrc.nist.gov/pubs/sp/800/82/r3/final}
\end{itemize}

\subsection*{Resources}
\begin{itemize}
    \item \textbf{CISA} -- Industrial Control Systems Monitoring\\
          \url{https://www.cisa.gov/topics/industrial-control-systems}
    \item \textbf{MITRE ATT\&CK for ICS} -- Adversary Behaviors\\
          \url{https://attack.mitre.org/matrices/ics/}
\end{itemize}

\vfill
\begin{center}
\textit{Part of the OT Security Learning Series}
\end{center}

\end{document}
