% ============================================================================
%  OT Intrusion Detection Systems - OT Security Learning Resource
% ============================================================================

\documentclass[11pt,a4paper]{article}
\usepackage{otsec-template}
\usepackage{float}

% Define colors for TikZ
\colorlet{otprimary}{primary}
\colorlet{otaccent}{accent}
\colorlet{otsuccess}{success}
\colorlet{otwarning}{warning}
\colorlet{otdanger}{danger}
\colorlet{otinfo}{info}

\begin{document}

\maketitlepage
    {Intrusion Detection for OT}
    {IDS/IPS deployment strategies for industrial environments}
    {OT Security Learning Series}
    {Document 520 \quad|\quad January 2026}
    {Matthias Niedermaier}

\tableofcontents
\newpage

% ============================================================================
\section{Introduction}
% ============================================================================

\begin{infobox}
Intrusion Detection Systems (IDS) and Intrusion Prevention Systems (IPS) monitor network traffic for malicious activity. In OT environments, these systems must understand industrial protocols and prioritize availability over aggressive blocking.
\end{infobox}

Key differences in OT intrusion detection:
\begin{itemize}
    \item \textbf{Protocol awareness:} Must parse Modbus, DNP3, OPC, EtherNet/IP, etc.
    \item \textbf{Availability focus:} False positives can disrupt operations
    \item \textbf{Passive preferred:} Active blocking requires careful consideration
    \item \textbf{Deterministic traffic:} OT traffic patterns are predictable
    \item \textbf{Long baselines:} Systems run unchanged for extended periods
\end{itemize}

% ============================================================================
\section{IDS vs IPS in OT}
% ============================================================================

\begin{table}[H]
\centering
\small
\rowcolors{2}{lightgray}{white}
\begin{tabular}{p{3cm}p{5cm}p{5cm}}
\rowcolor{primary}
\textcolor{white}{\bfseries Aspect} & \textcolor{white}{\bfseries IDS (Detection)} & \textcolor{white}{\bfseries IPS (Prevention)} \\
\midrule
Action & Alert only & Block malicious traffic \\
Deployment & Passive (SPAN/TAP) & Inline \\
Availability risk & Low & Higher (can block legitimate) \\
Response time & Human review required & Immediate automated response \\
OT suitability & Preferred for OT & Use with caution \\
\end{tabular}
\caption{IDS vs IPS Comparison}
\end{table}

\begin{warningbox}
\textbf{IPS Caution in OT:} Inline IPS can block legitimate traffic due to false positives, potentially disrupting critical processes. Most OT environments use IDS (detection) rather than IPS (prevention) to avoid availability impacts.
\end{warningbox}

\subsection{Deployment Modes}

\begin{figure}[H]
\centering
\begin{tikzpicture}[scale=0.85, every node/.style={scale=0.85},
    box/.style={rectangle, draw, thick, rounded corners=3pt, minimum width=2cm, minimum height=1cm, align=center, font=\small},
    ids/.style={rectangle, draw, thick, fill=otsuccess!20, rounded corners=3pt, minimum width=1.5cm, minimum height=0.8cm, align=center, font=\small},
    arrow/.style={->, thick, >=stealth}
]

% IDS Passive Mode
\node[font=\small\bfseries] at (-5,3) {IDS (Passive)};
\node[box, fill=otinfo!20] (sw1) at (-5,1.5) {Switch};
\node[ids] (ids1) at (-5,-0.5) {IDS};
\node[box, fill=otaccent!20] (plc1) at (-8,1.5) {PLC};
\node[box, fill=otaccent!20] (hmi1) at (-2,1.5) {HMI};

\draw[<->, thick] (plc1) -- (sw1);
\draw[<->, thick] (sw1) -- (hmi1);
\draw[->, thick, dashed, otsuccess] (sw1) -- (ids1);
\node[font=\tiny] at (-5,0.5) {SPAN/TAP};

% IPS Inline Mode
\node[font=\small\bfseries] at (5,3) {IPS (Inline)};
\node[box, fill=otinfo!20] (sw2) at (5,1.5) {Switch};
\node[ids, fill=otwarning!20] (ips) at (5,-0.5) {IPS};
\node[box, fill=otaccent!20] (plc2) at (2,1.5) {PLC};
\node[box, fill=otaccent!20] (hmi2) at (8,1.5) {HMI};

\draw[<->, thick] (plc2) -- (sw2);
\draw[<->, thick] (sw2) -- (hmi2);
\draw[<->, thick, otwarning] (sw2) -- (ips);
\node[font=\tiny] at (5,0.5) {All traffic};

\end{tikzpicture}
\caption{IDS (Passive) vs IPS (Inline) Deployment}
\end{figure}

% ============================================================================
\section{Detection Methods}
% ============================================================================

\subsection{Signature-Based Detection}

\begin{definitionbox}{Signature-Based Detection}
Compares network traffic against a database of known attack patterns (signatures). Effective for known threats but cannot detect novel attacks.
\end{definitionbox}

\begin{itemize}
    \item \textbf{Pros:} Low false positives, well-understood, fast
    \item \textbf{Cons:} Requires signature updates, misses zero-days
    \item \textbf{OT consideration:} Need OT-specific signatures (Modbus exploits, etc.)
\end{itemize}

\subsection{Anomaly-Based Detection}

\begin{definitionbox}{Anomaly-Based Detection}
Establishes a baseline of ``normal'' behavior and alerts on deviations. Can detect novel attacks but may generate more false positives.
\end{definitionbox}

\begin{itemize}
    \item \textbf{Pros:} Detects unknown attacks, learns environment
    \item \textbf{Cons:} Higher false positive rate, requires tuning
    \item \textbf{OT advantage:} OT traffic is predictable---anomalies stand out
\end{itemize}

\subsection{Protocol-Aware Detection}

\begin{figure}[H]
\centering
\begin{tikzpicture}[
    layer/.style={rectangle, draw, thick, minimum width=8cm, minimum height=0.8cm, align=center, font=\small}
]

\node[layer, fill=otdanger!15] (app) at (0,3) {Application Layer (OT protocol commands)};
\node[layer, fill=otwarning!15] (proto) at (0,2) {Protocol Layer (Modbus, DNP3, S7)};
\node[layer, fill=otinfo!15] (trans) at (0,1) {Transport Layer (TCP/UDP)};
\node[layer, fill=otsuccess!15] (net) at (0,0) {Network Layer (IP)};

% Annotation
\node[font=\scriptsize, align=left] at (6,3) {Validate commands};
\node[font=\scriptsize, align=left] at (6,2) {Parse industrial protocols};
\node[font=\scriptsize, align=left] at (6,1) {Port/connection analysis};
\node[font=\scriptsize, align=left] at (6,0) {IP reputation, geolocation};

\end{tikzpicture}
\caption{Deep Packet Inspection for OT Protocols}
\end{figure}

\begin{successbox}
\textbf{OT Protocol Inspection Examples:}
\begin{itemize}
    \item Detect unauthorized Modbus write commands to PLCs
    \item Alert on DNP3 commands from unexpected sources
    \item Identify firmware upload attempts via S7comm
    \item Flag changes to safety system parameters
\end{itemize}
\end{successbox}

% ============================================================================
\section{OT-Specific Considerations}
% ============================================================================

\subsection{Industrial Protocol Support}

Essential protocol support for OT IDS:

\begin{table}[H]
\centering
\small
\rowcolors{2}{lightgray}{white}
\begin{tabular}{p{3.5cm}p{9.5cm}}
\rowcolor{primary}
\textcolor{white}{\bfseries Protocol} & \textcolor{white}{\bfseries Detection Capabilities} \\
\midrule
Modbus TCP/RTU & Function codes, register addresses, read vs write \\
DNP3 & Object types, data link layer, secure authentication \\
EtherNet/IP & CIP commands, tag access, configuration changes \\
OPC UA/DA & Node access, browse requests, write operations \\
S7comm & Block transfers, program downloads, start/stop \\
IEC 61850/GOOSE & Substation communications, control commands \\
BACnet & Building automation, property writes \\
\end{tabular}
\caption{OT Protocol Detection Capabilities}
\end{table}

\subsection{Baseline Considerations}

\begin{itemize}
    \item \textbf{Learning period:} Allow sufficient time to capture normal operations
    \item \textbf{Operational modes:} Include startup, shutdown, maintenance periods
    \item \textbf{Seasonal variations:} Some processes vary by time of year
    \item \textbf{Update on changes:} Re-baseline after legitimate modifications
\end{itemize}

\begin{warningbox}
\textbf{Baseline Risks:} If the network is already compromised during baseline creation, malicious traffic may be learned as ``normal.'' Conduct security assessment before establishing baselines.
\end{warningbox}

% ============================================================================
\section{Deployment Architecture}
% ============================================================================

\subsection{Sensor Placement}

\begin{figure}[H]
\centering
\begin{tikzpicture}[
    zone/.style={rectangle, draw, thick, rounded corners=5pt, minimum width=3cm, minimum height=0.8cm, align=center, font=\small},
    sensor/.style={circle, fill=otsuccess, text=white, font=\tiny\bfseries, minimum size=0.6cm}
]

% Zones stacked vertically
\node[zone, fill=otinfo!15] (enterprise) at (0,5) {Enterprise};
\node[zone, fill=otwarning!15] (dmz) at (0,3.5) {DMZ};
\node[zone, fill=otsuccess!15] (control) at (0,2) {Control};
\node[zone, fill=otdanger!15] (field) at (0,0.5) {Field};

% Sensors at boundaries
\node[sensor] (s1) at (3.5,4.25) {};
\node[sensor] (s2) at (3.5,2.75) {};
\node[sensor] (s3) at (3.5,1.25) {};

% Connections (anchor to west of sensors)
\draw[thick] (enterprise.east) -- ++(0.5,0) |- (s1.west);
\draw[thick] (dmz.east) -- ++(0.5,0) |- (s1.west);
\draw[thick] (dmz.east) -- ++(0.5,0) |- (s2.west);
\draw[thick] (control.east) -- ++(0.5,0) |- (s2.west);
\draw[thick] (control.east) -- ++(0.5,0) |- (s3.west);
\draw[thick] (field.east) -- ++(0.5,0) |- (s3.west);

% Labels (positioned above sensors)
\node[font=\scriptsize, anchor=south] at (3.5,4.55) {IT/OT};
\node[font=\scriptsize, anchor=south] at (3.5,3.05) {DMZ};
\node[font=\scriptsize, anchor=south] at (3.5,1.55) {Control};

% Central management
\node[rectangle, draw, thick, fill=otaccent!20, minimum width=2cm, minimum height=0.6cm, font=\small] (mgmt) at (6,2.75) {IDS Manager};
\draw[dashed, ->] (s1) -- (mgmt);
\draw[dashed, ->] (s2) -- (mgmt);
\draw[dashed, ->] (s3) -- (mgmt);

\end{tikzpicture}
\caption{IDS Sensor Placement by Zone}
\end{figure}

\subsection{Key Monitoring Points}

\begin{table}[H]
\centering
\small
\rowcolors{2}{lightgray}{white}
\begin{tabular}{p{4cm}p{9cm}}
\rowcolor{primary}
\textcolor{white}{\bfseries Location} & \textcolor{white}{\bfseries Monitoring Focus} \\
\midrule
IT/OT boundary & North-south traffic, unauthorized access attempts \\
DMZ connections & Data transfers, remote access sessions \\
Control network & East-west OT traffic, protocol anomalies \\
Engineering workstations & Configuration changes, downloads to PLCs \\
Remote access points & VPN sessions, jump server activity \\
\end{tabular}
\caption{Key IDS Monitoring Points}
\end{table}

% ============================================================================
\section{Implementation Best Practices}
% ============================================================================

\subsection{Deployment Checklist}

\begin{enumerate}
    \item \textbf{Assess before deploying:} Understand network architecture first
    \item \textbf{Start in detection mode:} Alert only, no blocking initially
    \item \textbf{Allow adequate baseline:} Capture normal operational patterns
    \item \textbf{Tune aggressively:} Reduce false positives before expanding
    \item \textbf{Integrate with SIEM:} Correlate with other security data
    \item \textbf{Define response procedures:} Know what to do with alerts
    \item \textbf{Test regularly:} Verify detection capabilities
\end{enumerate}

\subsection{Alert Management}

\begin{tipbox}
\textbf{Reducing Alert Fatigue:}
\begin{itemize}
    \item Prioritize alerts by asset criticality
    \item Suppress known false positives
    \item Correlate related events into single incidents
    \item Route OT alerts to personnel who understand the context
    \item Establish clear escalation procedures
\end{itemize}
\end{tipbox}

\subsection{IPS Deployment (If Required)}

If inline prevention is required:
\begin{itemize}
    \item \textbf{Fail-open mode:} If IPS fails, traffic continues
    \item \textbf{Whitelist approach:} Block only confirmed threats
    \item \textbf{Test extensively:} Validate in lab before production
    \item \textbf{Bypass capability:} Ability to disable quickly if issues
    \item \textbf{High availability:} Redundant deployment for critical paths
\end{itemize}

% ============================================================================
\section{Integration Points}
% ============================================================================

\begin{figure}[H]
\centering
\begin{tikzpicture}[
    box/.style={rectangle, draw, thick, rounded corners=3pt, minimum width=2.2cm, minimum height=1cm, align=center, font=\small},
    arrow/.style={->, thick, >=stealth}
]

% Central IDS
\node[box, fill=otsuccess!20, minimum width=3cm, minimum height=1.5cm] (ids) at (0,0) {OT IDS/IPS};

% Integration points
\node[box, fill=otinfo!20] (siem) at (-4,2) {SIEM};
\node[box, fill=otaccent!20] (soar) at (0,2) {SOAR};
\node[box, fill=otwarning!20] (asset) at (4,2) {Asset\\Inventory};
\node[box, fill=otdanger!20] (threat) at (-4,-2) {Threat\\Intel};
\node[box, fill=otprimary!20] (fw) at (0,-2) {Firewalls};
\node[box, fill=otsuccess!15] (hist) at (4,-2) {Historian};

% Arrows
\draw[arrow] (ids) -- (siem);
\draw[arrow] (ids) -- (soar);
\draw[arrow] (asset) -- (ids);
\draw[arrow] (threat) -- (ids);
\draw[<->] (ids) -- (fw);
\draw[arrow] (ids) -- (hist);

\end{tikzpicture}
\caption{IDS Integration Architecture}
\end{figure}

% ============================================================================
\section{Summary}
% ============================================================================

\begin{definitionbox}{Key Takeaways}
\begin{itemize}
    \item \textbf{Detection over prevention:} IDS preferred over IPS in OT
    \item \textbf{Protocol awareness:} Must understand Modbus, DNP3, etc.
    \item \textbf{Anomaly detection:} OT traffic predictability is an advantage
    \item \textbf{Strategic placement:} Monitor zone boundaries and critical segments
    \item \textbf{Baseline carefully:} Ensure network is clean before learning
    \item \textbf{Tune thoroughly:} Reduce false positives before expanding
    \item \textbf{Integrate:} Connect to SIEM, asset inventory, threat intel
\end{itemize}
\end{definitionbox}

% ============================================================================
\section{Further Reading}
% ============================================================================

\subsection*{Standards}
\begin{itemize}
    \item \textbf{IEC 62443-3-3} -- System security requirements and security levels\\
          \url{https://webstore.iec.ch/publication/7033}
    \item \textbf{NIST SP 800-82 Rev. 3} -- Guide to OT Security\\
          \url{https://csrc.nist.gov/pubs/sp/800/82/r3/final}
\end{itemize}

\subsection*{Resources}
\begin{itemize}
    \item \textbf{CISA} -- Industrial Control Systems Detection and Response\\
          \url{https://www.cisa.gov/topics/industrial-control-systems}
    \item \textbf{SANS ICS} -- ICS Security Resources\\
          \url{https://www.sans.org/cybersecurity-focus-areas/industrial-control-systems-security}
\end{itemize}

\vfill
\begin{center}
\textit{Part of the OT Security Learning Series}
\end{center}

\end{document}
