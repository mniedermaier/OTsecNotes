% ============================================================================
%  OT Vulnerability Management - Poster / Cheat Sheet
% ============================================================================

\documentclass[9pt,a4paper]{extarticle}
\usepackage{otsec-poster}
\usepackage{float}

\begin{document}

\makepostertitle
    {OT Vulnerability Management}
    {Identifying, Prioritizing, and Remediating Industrial Vulnerabilities}
    {Poster 720}
    {Matthias Niedermaier}

\begin{multicols}{2}

\section{\textcolor{accent}{\faIcon{info-circle}}\hspace{0.4em}Overview}

Vulnerability management in OT requires balancing \textbf{security with operational continuity}. Unlike IT, OT systems may require extensive testing, vendor coordination, and scheduled downtime before vulnerabilities can be remediated.

\posterinfo{
\textbf{OT challenges:} Availability priority (downtime not acceptable), legacy systems (no security updates), testing requirements (patches must be validated), vendor dependencies, and long lifecycles (15--30 years).
}

\section{\textcolor{accent}{\faIcon{sync-alt}}\hspace{0.4em}Lifecycle}

\begin{center}
\begin{tikzpicture}[
    num/.style={circle, fill=#1, text=white, font=\scriptsize\bfseries,
                minimum size=0.35cm, inner sep=0pt},
    lbl/.style={font=\scriptsize},
    arr/.style={->, >=stealth, gray!60, thick},
]
    \def\R{1.1cm}
    \foreach \i/\angle/\col in {1/90/otprimary, 2/162/otaccent, 3/234/otwarning, 4/306/otdanger, 5/18/otsuccess}
        \node[num=\col] (n\i) at (\angle:\R) {\i};
    \node[lbl, above=2pt of n1] {Identify};
    \node[lbl, left=2pt of n2] {Assess};
    \node[lbl, below=2pt of n3] {Prioritize};
    \node[lbl, below=2pt of n4] {Remediate};
    \node[lbl, right=2pt of n5] {Verify};
    \foreach \a/\b in {1/2, 2/3, 3/4, 4/5, 5/1}
        \draw[arr] (n\a) -- (n\b);
\end{tikzpicture}
\end{center}

\section{\textcolor{accent}{\faIcon{search}}\hspace{0.4em}Vulnerability Identification}

\begin{center}
\rowcolors{2}{lightgray}{white}
\begin{tabular}{p{2.2cm}p{1.5cm}p{2.5cm}}
\rowcolor{primary}
\textcolor{white}{\faIcon{search}\hspace{0.2em}\bfseries Method} & \textcolor{white}{\faIcon{shield-alt}\hspace{0.2em}\bfseries Safety} & \textcolor{white}{\faIcon{sticky-note}\hspace{0.2em}\bfseries Notes} \\
\midrule
Passive scan & Preferred & Network traffic analysis \\
Agent-based & Moderate & May impact performance \\
Active scan & Caution & Can crash legacy devices \\
Manual inventory & Safe & Labor-intensive \\
Vendor alerts & Safe & Subscribe to advisories \\
\end{tabular}
\end{center}

\posterdanger{
\textbf{Active scanning risks:} \textcolor{danger}{\faIcon{bug}}\hspace{0.2em}Legacy PLCs may crash when scanned. \textcolor{danger}{\faIcon{bug}}\hspace{0.2em}Network scanning can disrupt real-time communications. \textcolor{danger}{\faIcon{bug}}\hspace{0.2em}Some devices reboot on unexpected packets. Always test scanning tools in lab first.
}

\section{\textcolor{accent}{\faIcon{sort-amount-down}}\hspace{0.4em}Risk-Based Prioritization}

\subsection{\textcolor{accent}{\faIcon{balance-scale}}\hspace{0.3em}OT-Specific Risk Factors}

\begin{center}
\rowcolors{2}{lightgray}{white}
\begin{tabular}{p{2.2cm}p{4.3cm}}
\rowcolor{primary}
\textcolor{white}{\faIcon{balance-scale}\hspace{0.2em}\bfseries Factor} & \textcolor{white}{\faIcon{industry}\hspace{0.2em}\bfseries OT Consideration} \\
\midrule
Safety impact & Could exploitation cause physical harm? \\
Production & Cost of downtime or damage? \\
Criticality & Safety system, core process, support? \\
Exposure & Internet-accessible? \\
Exploitability & Public exploit? Active exploitation? \\
Compensating & Isolated, monitored, protected? \\
\end{tabular}
\end{center}

\posterwarning{
\textbf{CVSS alone is insufficient for OT.} Standard scores don't reflect safety impact, production cost, or availability of compensating controls. Use OT-specific risk factors for prioritization.
}

\subsection{\textcolor{accent}{\faIcon{tasks}}\hspace{0.3em}Priority Categories}

\begin{itemize}
    \item \riskcritical\hspace{0.3em}Safety systems, actively exploited, internet-exposed
    \item \riskhigh\hspace{0.3em}Core process control, public exploit available
    \item \riskmedium\hspace{0.3em}Important systems, no active exploitation
    \item \risklow\hspace{0.3em}Low-impact, difficult to exploit
\end{itemize}

\section{\textcolor{accent}{\faIcon{wrench}}\hspace{0.4em}Remediation Strategies}

\begin{center}
\rowcolors{2}{lightgray}{white}
\begin{tabular}{p{1.5cm}p{2cm}p{2.7cm}}
\rowcolor{primary}
\textcolor{white}{\faIcon{wrench}\hspace{0.2em}\bfseries Option} & \textcolor{white}{\faIcon{clock}\hspace{0.2em}\bfseries When} & \textcolor{white}{\faIcon{industry}\hspace{0.2em}\bfseries OT Note} \\
\midrule
Patch & Vendor fix avail. & Requires test, downtime \\
Compensate & Patch unavailable & May not fully address \\
Isolate & Cannot patch & Limits connectivity \\
Replace & EOL, critical vuln & Expensive, slow \\
Accept & Low risk & Document decision \\
\end{tabular}
\end{center}

\subsection{\textcolor{accent}{\faIcon{download}}\hspace{0.3em}OT Patching Process}

\begin{enumerate}
    \item \faIcon{bell}\hspace{0.2em}Receive vendor patch notification
    \item \faIcon{clipboard-list}\hspace{0.2em}Assess applicability to your environment
    \item \faIcon{flask}\hspace{0.2em}Test in lab/staging environment
    \item \faIcon{calendar-alt}\hspace{0.2em}Schedule downtime with operations
    \item \faIcon{download}\hspace{0.2em}Apply patch with rollback plan ready
    \item \faIcon{check-double}\hspace{0.2em}Verify functionality and document
\end{enumerate}

\subsection{\textcolor{accent}{\faIcon{shield-alt}}\hspace{0.3em}Compensating Controls}

\postersuccess{
\textbf{When patching is impossible:} Network segmentation (firewall, VLAN isolation). Data diodes for one-way communication. Application whitelisting on connected systems. Enhanced monitoring and alerting. Physical access restrictions. Disable unnecessary services.
}

\section{\textcolor{accent}{\faIcon{server}}\hspace{0.4em}Legacy System Management}

\begin{center}
\rowcolors{2}{lightgray}{white}
\begin{tabular}{p{2.2cm}p{4.3cm}}
\rowcolor{primary}
\textcolor{white}{\faIcon{layer-group}\hspace{0.2em}\bfseries Layer} & \textcolor{white}{\faIcon{shield-alt}\hspace{0.2em}\bfseries Protection} \\
\midrule
Network & Firewall/VLAN isolation \\
Application & Whitelisting \\
Monitoring & Enhanced detection \\
Physical & Access control \\
Documentation & Formal risk acceptance \\
\end{tabular}
\end{center}

\section{\textcolor{accent}{\faIcon{chart-bar}}\hspace{0.4em}Metrics and Reporting}

\begin{center}
\rowcolors{2}{lightgray}{white}
\begin{tabular}{p{2.2cm}p{4.3cm}}
\rowcolor{primary}
\textcolor{white}{\faIcon{chart-bar}\hspace{0.2em}\bfseries Metric} & \textcolor{white}{\faIcon{info-circle}\hspace{0.2em}\bfseries Description} \\
\midrule
Open vulns & Count by severity level \\
MTTR & Mean time to remediate \\
Patch coverage & Systems at current patch level \\
Compensating & Systems with controls applied \\
Overdue & Past SLA for remediation \\
Risk trend & Exposure change over time \\
\end{tabular}
\end{center}

\postertip{
OT vulnerability management is a \textbf{continuous lifecycle, not a one-time event}. Accurate asset inventory is the foundation---you cannot match vulnerabilities without knowing what you have. Use OT-specific risk factors, not just CVSS scores. When patching isn't possible, implement defense-in-depth compensating controls. Track metrics to demonstrate risk reduction. Set realistic SLAs that account for OT patching constraints.
}

\end{multicols}

\end{document}
