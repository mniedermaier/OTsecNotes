% ============================================================================
%  OT Vulnerability Management - OT Security Learning Resource
% ============================================================================

\documentclass[11pt,a4paper]{article}
\usepackage{otsec-template}
\usepackage{float}

% Define colors for TikZ
\colorlet{otprimary}{primary}
\colorlet{otaccent}{accent}
\colorlet{otsuccess}{success}
\colorlet{otwarning}{warning}
\colorlet{otdanger}{danger}
\colorlet{otinfo}{info}

\begin{document}

\maketitlepage
    {OT Vulnerability Management}
    {Identifying, prioritizing, and remediating vulnerabilities in industrial systems}
    {OT Security Learning Series}
    {Document 720 \quad|\quad January 2026}
    {Matthias Niedermaier}

\tableofcontents
\newpage

% ============================================================================
\section{Introduction}
% ============================================================================

\begin{infobox}
Vulnerability management in OT environments requires balancing security with operational continuity. Unlike IT, where patches can often be applied quickly, OT systems may require extensive testing, vendor coordination, and scheduled downtime before vulnerabilities can be remediated.
\end{infobox}

OT vulnerability management challenges:
\begin{itemize}
    \item \textbf{Availability priority:} Downtime for patching may not be acceptable
    \item \textbf{Legacy systems:} Many systems no longer receive security updates
    \item \textbf{Testing requirements:} Patches must be validated before deployment
    \item \textbf{Vendor dependencies:} May need vendor approval or assistance
    \item \textbf{Long lifecycles:} Systems operate for 15--30 years
\end{itemize}

% ============================================================================
\section{Vulnerability Management Lifecycle}
% ============================================================================

\begin{figure}[H]
\centering
\begin{tikzpicture}[
    phase/.style={rectangle, draw, thick, rounded corners=5pt, minimum width=2cm, minimum height=1.2cm, align=center, font=\small\bfseries},
    arrow/.style={->, very thick, >=stealth}
]

% Phases in a cycle
\node[phase, fill=otinfo!20] (identify) at (0,2) {Identify};
\node[phase, fill=otaccent!20] (assess) at (3,2) {Assess};
\node[phase, fill=otwarning!20] (prioritize) at (6,2) {Prioritize};
\node[phase, fill=otsuccess!20] (remediate) at (9,2) {Remediate};
\node[phase, fill=otprimary!20] (verify) at (12,2) {Verify};

% Arrows
\draw[arrow, otprimary] (identify) -- (assess);
\draw[arrow, otprimary] (assess) -- (prioritize);
\draw[arrow, otprimary] (prioritize) -- (remediate);
\draw[arrow, otprimary] (remediate) -- (verify);

% Loop back
\draw[arrow, otprimary, dashed] (verify.south) -- ++(0,-0.8) -| (identify.south);

% Labels (below the loopback arrow)
\node[font=\tiny] at (0,-0.2) {Discovery};
\node[font=\tiny] at (3,-0.2) {Analysis};
\node[font=\tiny] at (6,-0.2) {Risk-based};
\node[font=\tiny] at (9,-0.2) {Action};
\node[font=\tiny] at (12,-0.2) {Validation};

\end{tikzpicture}
\caption{Vulnerability Management Lifecycle}
\end{figure}

% ============================================================================
\section{Vulnerability Identification}
% ============================================================================

\subsection{Discovery Methods}

\begin{table}[H]
\centering
\small
\rowcolors{2}{lightgray}{white}
\begin{tabular}{p{4cm}p{4cm}p{5cm}}
\rowcolor{primary}
\textcolor{white}{\bfseries Method} & \textcolor{white}{\bfseries OT Suitability} & \textcolor{white}{\bfseries Considerations} \\
\midrule
Passive scanning & \textcolor{otsuccess}{Preferred} & Network traffic analysis \\
Agent-based & \textcolor{otwarning}{Moderate} & May impact performance \\
Active scanning & \textcolor{otdanger}{Use caution} & Can crash legacy devices \\
Manual inventory & \textcolor{otsuccess}{Safe} & Labor-intensive \\
Vendor notifications & \textcolor{otsuccess}{Recommended} & Subscribe to advisories \\
\end{tabular}
\caption{Vulnerability Discovery Methods for OT}
\end{table}

\begin{warningbox}
\textbf{Active Scanning Risks:}
\begin{itemize}
    \item Legacy PLCs may crash when scanned
    \item Network scanning can disrupt real-time communications
    \item Some devices reboot when receiving unexpected packets
    \item Always test scanning tools in lab environment first
\end{itemize}
\end{warningbox}

\subsection{Vulnerability Sources}

\begin{itemize}
    \item \textbf{ICS-CERT/CISA advisories:} US government OT vulnerability alerts
    \item \textbf{Vendor security bulletins:} Direct from equipment manufacturers
    \item \textbf{CVE databases:} National Vulnerability Database (NVD)
    \item \textbf{Security researchers:} Published vulnerability disclosures
    \item \textbf{Internal assessments:} Penetration tests, security audits
\end{itemize}

\subsection{Asset Correlation}

\begin{figure}[H]
\centering
\begin{tikzpicture}[
    box/.style={rectangle, draw, thick, rounded corners=3pt, minimum width=2.5cm, minimum height=1cm, align=center, font=\small},
    arrow/.style={->, thick, >=stealth}
]

% Inputs (stacked vertically on left)
\node[box, fill=otinfo!20] (vulns) at (0,1) {Vulnerability\\Feeds};
\node[box, fill=otsuccess!20] (assets) at (0,-1) {Asset\\Inventory};

% Processing and output
\node[box, fill=otwarning!20] (match) at (5,0) {Correlation\\Engine};
\node[box, fill=otdanger!20] (report) at (10,0) {Affected\\Assets};

\draw[arrow] (vulns.east) -- (match.west);
\draw[arrow] (assets.east) -- (match.west);
\draw[arrow] (match) -- (report);

% Labels
\node[font=\tiny] at (0,0) {CVEs, advisories};
\node[font=\tiny] at (0,-2) {Vendor, model, version};
\node[font=\tiny] at (5,-0.9) {Match vuln to asset};
\node[font=\tiny] at (10,-0.9) {Prioritized list};

\end{tikzpicture}
\caption{Vulnerability-to-Asset Correlation Process}
\end{figure}

\begin{successbox}
\textbf{Accurate asset inventory is essential:} You cannot determine which vulnerabilities affect your environment without knowing exactly what hardware, software, and firmware versions are deployed.
\end{successbox}

% ============================================================================
\section{Risk-Based Prioritization}
% ============================================================================

\subsection{OT-Specific Risk Factors}

Standard CVSS scores may not reflect true OT risk. Consider additional factors:

\begin{table}[H]
\centering
\small
\rowcolors{2}{lightgray}{white}
\begin{tabular}{p{4cm}p{9cm}}
\rowcolor{primary}
\textcolor{white}{\bfseries Factor} & \textcolor{white}{\bfseries OT Consideration} \\
\midrule
Safety impact & Could exploitation cause physical harm? \\
Production impact & What is the cost of downtime or damage? \\
Asset criticality & Is this a safety system, core process, or support? \\
Exposure & Is the vulnerable system internet-accessible? \\
Exploitability & Is there a public exploit? Active exploitation? \\
Compensating controls & Is the system isolated, monitored, protected? \\
\end{tabular}
\caption{OT-Specific Risk Prioritization Factors}
\end{table}

\subsection{Prioritization Matrix}

\begin{figure}[H]
\centering
\begin{tikzpicture}[
    cell/.style={rectangle, draw, minimum width=2cm, minimum height=1cm, font=\small}
]

% Matrix
\node[cell, fill=otdanger!40] at (0,2) {\riskcritical};
\node[cell, fill=otdanger!30] at (2,2) {\riskhigh};
\node[cell, fill=otwarning!30] at (4,2) {\riskmedium};

\node[cell, fill=otdanger!30] at (0,1) {\riskhigh};
\node[cell, fill=otwarning!30] at (2,1) {\riskmedium};
\node[cell, fill=otsuccess!30] at (4,1) {\risklow};

\node[cell, fill=otwarning!30] at (0,0) {\riskmedium};
\node[cell, fill=otsuccess!30] at (2,0) {\risklow};
\node[cell, fill=otsuccess!20] at (4,0) {\risklow};

% Labels
\node[font=\small\bfseries, rotate=90] at (-2.5,1) {Exploitability};
\node[font=\small\bfseries] at (2,3.2) {Asset Criticality};

% Axis labels (row)
\node[font=\tiny, anchor=east] at (-1.2,2) {High};
\node[font=\tiny, anchor=east] at (-1.2,1) {Medium};
\node[font=\tiny, anchor=east] at (-1.2,0) {Low};

% Axis labels (column)
\node[font=\tiny] at (0,2.9) {Critical};
\node[font=\tiny] at (2,2.9) {High};
\node[font=\tiny] at (4,2.9) {Low};

\end{tikzpicture}
\caption{Risk Prioritization Matrix}
\end{figure}

\subsection{Priority Categories}

\begin{itemize}
    \item \riskcritical\ \textbf{Immediate action:} Safety systems, actively exploited, internet-exposed
    \item \riskhigh\ \textbf{Urgent (days):} Core process control, public exploit available
    \item \riskmedium\ \textbf{Planned (weeks):} Important systems, no active exploitation
    \item \risklow\ \textbf{Scheduled (months):} Low-impact systems, difficult to exploit
\end{itemize}

% ============================================================================
\section{Remediation Strategies}
% ============================================================================

\subsection{Remediation Options}

\begin{table}[H]
\centering
\small
\rowcolors{2}{lightgray}{white}
\begin{tabular}{p{3.5cm}p{4.5cm}p{5cm}}
\rowcolor{primary}
\textcolor{white}{\bfseries Option} & \textcolor{white}{\bfseries When to Use} & \textcolor{white}{\bfseries OT Consideration} \\
\midrule
Patch/Update & Vendor provides fix & Requires testing, downtime \\
Compensating control & Patch unavailable/risky & May not fully address risk \\
Network isolation & Cannot patch & Limits connectivity \\
System replacement & End-of-life, critical vuln & Expensive, time-consuming \\
Risk acceptance & Low risk, high cost & Document decision \\
\end{tabular}
\caption{Remediation Options for OT Vulnerabilities}
\end{table}

\subsection{Patching Process}

\begin{figure}[H]
\centering
\begin{tikzpicture}[
    stepbox/.style={rectangle, draw, thick, fill=otaccent!15, minimum width=8cm, minimum height=0.7cm, rounded corners=3pt, font=\small},
    num/.style={circle, fill=otprimary, text=white, font=\small\bfseries, minimum size=0.6cm}
]

\node[stepbox] (s1) at (0,5) {Receive vendor patch notification};
\node[num] at (-4.5,5) {1};

\node[stepbox] (s2) at (0,4) {Assess applicability to your environment};
\node[num] at (-4.5,4) {2};

\node[stepbox] (s3) at (0,3) {Test in lab/staging environment};
\node[num] at (-4.5,3) {3};

\node[stepbox] (s4) at (0,2) {Schedule downtime with operations};
\node[num] at (-4.5,2) {4};

\node[stepbox] (s5) at (0,1) {Apply patch with rollback plan ready};
\node[num] at (-4.5,1) {5};

\node[stepbox] (s6) at (0,0) {Verify functionality and document};
\node[num] at (-4.5,0) {6};

\end{tikzpicture}
\caption{OT Patching Process}
\end{figure}

\subsection{Compensating Controls}

When patching is not possible:

\begin{warningbox}
\textbf{Compensating Controls for Unpatchable Systems:}
\begin{itemize}
    \item Network segmentation (firewall, VLAN isolation)
    \item Data diodes for one-way communication
    \item Application whitelisting on connected systems
    \item Enhanced monitoring and alerting
    \item Physical access restrictions
    \item Disable unnecessary services and protocols
\end{itemize}
\end{warningbox}

% ============================================================================
\section{Managing Legacy Systems}
% ============================================================================

\subsection{End-of-Life Challenges}

\begin{dangerbox}
Many OT systems operate beyond vendor support. When patches are no longer available:
\begin{itemize}
    \item Document the risk formally
    \item Implement maximum compensating controls
    \item Plan for eventual replacement
    \item Monitor for exploitation attempts
    \item Consider third-party security support
\end{itemize}
\end{dangerbox}

\subsection{Legacy System Protection Layers}

\begin{figure}[H]
\centering
\begin{tikzpicture}[
    layer/.style={rectangle, draw, thick, minimum width=6cm, minimum height=0.8cm, align=center, font=\small}
]

\node[layer, fill=otinfo!15] (l1) at (0,4) {Network isolation (firewall/VLAN)};
\node[layer, fill=otaccent!15] (l2) at (0,3) {Application whitelisting};
\node[layer, fill=otsuccess!15] (l3) at (0,2) {Enhanced monitoring};
\node[layer, fill=otwarning!15] (l4) at (0,1) {Physical access control};
\node[layer, fill=otdanger!15] (l5) at (0,0) {Legacy System};

% Arrows
\draw[->, thick] (-3.5,4) -- (-3.5,0);
\node[font=\tiny, rotate=90] at (-4,2) {Defense in Depth};

\end{tikzpicture}
\caption{Protection Layers for Legacy OT Systems}
\end{figure}

% ============================================================================
\section{Vulnerability Metrics and Reporting}
% ============================================================================

\subsection{Key Metrics}

\begin{table}[H]
\centering
\small
\rowcolors{2}{lightgray}{white}
\begin{tabular}{p{5cm}p{8cm}}
\rowcolor{primary}
\textcolor{white}{\bfseries Metric} & \textcolor{white}{\bfseries Description} \\
\midrule
Open vulnerabilities by severity & Count of unresolved vulns by priority \\
Mean time to remediate (MTTR) & Average time from discovery to fix \\
Patch coverage & Percentage of systems at current patch level \\
Compensating control coverage & Systems protected when patching impossible \\
Overdue vulnerabilities & Vulns past SLA for remediation \\
Risk reduction trend & Change in overall risk exposure over time \\
\end{tabular}
\caption{Vulnerability Management Metrics}
\end{table}

\subsection{Reporting Cadence}

\begin{itemize}
    \item \textbf{Daily:} New critical vulnerabilities, active exploitation alerts
    \item \textbf{Weekly:} Vulnerability status, upcoming patch activities
    \item \textbf{Monthly:} Trends, metrics, risk posture summary
    \item \textbf{Quarterly:} Executive summary, program effectiveness
\end{itemize}

% ============================================================================
\section{Program Implementation}
% ============================================================================

\subsection{Success Factors}

\begin{successbox}
\textbf{Building an Effective OT Vulnerability Program:}
\begin{enumerate}
    \item \textbf{Executive support:} Security competes with production priorities
    \item \textbf{OT/IT collaboration:} Combined expertise required
    \item \textbf{Accurate asset inventory:} Foundation for everything
    \item \textbf{Realistic SLAs:} Account for OT patching constraints
    \item \textbf{Testing capability:} Lab environment for patch validation
    \item \textbf{Clear escalation:} Process for unacceptable risk decisions
\end{enumerate}
\end{successbox}

\subsection{Common Pitfalls}

\begin{itemize}
    \item Applying IT patching timelines to OT (unrealistic)
    \item Active scanning without understanding OT impact
    \item Ignoring compensating controls as valid remediation
    \item Lack of vendor coordination
    \item No lab environment for testing
\end{itemize}

% ============================================================================
\section{Summary}
% ============================================================================

\begin{definitionbox}{Key Takeaways}
\begin{itemize}
    \item \textbf{OT patching is different:} Requires testing, coordination, downtime
    \item \textbf{Asset inventory essential:} Cannot match vulns without it
    \item \textbf{Risk-based prioritization:} CVSS alone insufficient for OT
    \item \textbf{Multiple remediation options:} Patching, compensating controls, isolation
    \item \textbf{Legacy systems:} Protect with defense-in-depth when patches unavailable
    \item \textbf{Metrics matter:} Track progress and demonstrate risk reduction
\end{itemize}
\end{definitionbox}

% ============================================================================
\section{Further Reading}
% ============================================================================

\subsection*{Standards}
\begin{itemize}
    \item \textbf{IEC 62443-2-3} -- Patch management in the IACS environment\\
          \url{https://webstore.iec.ch/publication/7032}
    \item \textbf{NIST SP 800-82 Rev. 3} -- Guide to OT Security\\
          \url{https://csrc.nist.gov/publications/detail/sp/800-82/rev-3/final}
\end{itemize}

\subsection*{Resources}
\begin{itemize}
    \item \textbf{CISA ICS Advisories}\\
          \url{https://www.cisa.gov/news-events/ics-advisories}
    \item \textbf{National Vulnerability Database}\\
          \url{https://nvd.nist.gov/}
\end{itemize}

\vfill
\begin{center}
\textit{Part of the OT Security Learning Series}
\end{center}

\end{document}
