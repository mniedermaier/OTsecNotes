% ============================================================================
%  OT Penetration Testing - OT Security Learning Resource
% ============================================================================

\documentclass[11pt,a4paper]{article}
\usepackage{otsec-template}

\hypersetup{
    pdftitle={OT Penetration Testing},
    pdfsubject={Methodology and Considerations for OT Security Assessments},
}

\begin{document}

% ----------------------------------------------------------------------------
%  TITLE PAGE
% ----------------------------------------------------------------------------

\maketitlepage
    {OT Penetration Testing}
    {Methodology for Industrial Security Assessments}
    {OT Security Learning Series}
    {Document 710 \quad|\quad January 2026}
    {Matthias Niedermaier}

% ----------------------------------------------------------------------------
%  TABLE OF CONTENTS
% ----------------------------------------------------------------------------

\tableofcontents
\newpage

% ----------------------------------------------------------------------------
%  INTRODUCTION
% ----------------------------------------------------------------------------

\section{Introduction}

OT penetration testing evaluates the security of industrial control systems by simulating real-world attack scenarios. Unlike traditional IT pentesting, OT assessments require specialized knowledge, careful planning, and strict safety considerations to avoid disrupting critical processes.

\begin{dangerbox}
OT penetration testing carries significant risk. Improper testing can cause equipment damage, safety incidents, production outages, or environmental harm. Always prioritize safety over thoroughness.
\end{dangerbox}

\subsection{OT vs IT Pentesting}

\begin{center}
\small
\rowcolors{2}{lightgray}{white}
\begin{tabular}{p{4cm}p{4.5cm}p{4.5cm}}
\rowcolor{primary}
\textcolor{white}{\bfseries Aspect} & \textcolor{white}{\bfseries IT Pentesting} & \textcolor{white}{\bfseries OT Pentesting} \\
\midrule
Primary concern & Data confidentiality & Safety and availability \\
System tolerance & Resilient to testing & Fragile, crash-prone \\
Downtime impact & Business disruption & Physical/safety impact \\
Testing window & Often flexible & Maintenance windows only \\
Protocol knowledge & Standard (TCP/IP) & Industrial (Modbus, DNP3) \\
Recovery & Restore from backup & May require site visit \\
\end{tabular}
\end{center}

% ----------------------------------------------------------------------------
%  PLANNING AND SCOPING
% ----------------------------------------------------------------------------

\section{Planning and Scoping}

\subsection{Pre-Engagement Requirements}

\begin{successbox}
\textbf{Essential pre-engagement activities:}
\begin{itemize}
    \item Obtain written authorization from asset owner
    \item Identify all stakeholders (OT, IT, Safety, Operations)
    \item Define clear scope boundaries and exclusions
    \item Establish emergency contacts and rollback procedures
    \item Review safety documentation and process constraints
    \item Schedule during planned maintenance if possible
\end{itemize}
\end{successbox}

\subsection{Scope Considerations}

\begin{center}
\small
\rowcolors{2}{lightgray}{white}
\begin{tabular}{p{3.5cm}p{9.5cm}}
\rowcolor{primary}
\textcolor{white}{\bfseries Element} & \textcolor{white}{\bfseries Considerations} \\
\midrule
Network segments & Which zones are in scope (IT, DMZ, OT, Safety)? \\
System types & HMIs, PLCs, RTUs, historians, engineering workstations \\
Protocols & Which industrial protocols can be tested? \\
Active vs passive & Can active exploitation be performed? \\
Physical access & Is physical security testing included? \\
Time constraints & Testing windows, blackout periods \\
\end{tabular}
\end{center}

\subsection{Risk Assessment}

Before testing, assess potential impacts:

\begin{itemize}
    \item \textbf{Safety risks:} Could testing trigger safety systems or cause harm?
    \item \textbf{Production risks:} What is the cost of unplanned downtime?
    \item \textbf{Equipment risks:} Could commands damage physical equipment?
    \item \textbf{Cascading effects:} Could actions affect connected systems?
\end{itemize}

% ----------------------------------------------------------------------------
%  TESTING PHASES
% ----------------------------------------------------------------------------

\section{Testing Phases}

\subsection{Phase 1: Passive Reconnaissance}

Safe information gathering without active probing:

\begin{itemize}
    \item \textbf{OSINT:} Public information about systems, vendors, employees
    \item \textbf{Network capture:} Passive traffic analysis (span/tap port)
    \item \textbf{Protocol identification:} Identify industrial protocols in use
    \item \textbf{Asset inventory:} Map devices from observed traffic
    \item \textbf{Credential discovery:} Monitor for cleartext authentication
\end{itemize}

\begin{tipbox}
Passive reconnaissance provides significant intelligence with minimal risk. Spend adequate time in this phase before any active testing.
\end{tipbox}

\subsection{Phase 2: Active Reconnaissance}

Careful active discovery with OT-safe techniques:

\begin{itemize}
    \item \textbf{Controlled scanning:} Slow, targeted port scans
    \item \textbf{Service identification:} Banner grabbing on known ports
    \item \textbf{Protocol enumeration:} Query devices using native protocols
    \item \textbf{Vulnerability scanning:} OT-aware scanners only
\end{itemize}

\begin{warningbox}
Standard IT vulnerability scanners can crash OT devices. Use only scanners designed for industrial environments or perform manual testing.
\end{warningbox}

\subsection{Phase 3: Vulnerability Assessment}

Identify weaknesses without exploitation:

\begin{itemize}
    \item Default and weak credentials
    \item Unpatched vulnerabilities
    \item Insecure protocol configurations
    \item Network segmentation gaps
    \item Unnecessary services and ports
\end{itemize}

\subsection{Phase 4: Controlled Exploitation}

If authorized and safe, validate vulnerabilities:

\begin{itemize}
    \item \textbf{Test environment first:} Validate exploits on lab systems
    \item \textbf{Reversible actions only:} No destructive testing
    \item \textbf{Monitoring:} Continuous observation during exploitation
    \item \textbf{Immediate rollback:} Stop at first sign of issues
\end{itemize}

% ----------------------------------------------------------------------------
%  TESTING TECHNIQUES
% ----------------------------------------------------------------------------

\section{Testing Techniques}

\subsection{Network-Level Testing}

\begin{center}
\small
\rowcolors{2}{lightgray}{white}
\begin{tabular}{p{4cm}p{9cm}}
\rowcolor{primary}
\textcolor{white}{\bfseries Test} & \textcolor{white}{\bfseries Purpose} \\
\midrule
Segmentation validation & Verify zone boundaries are enforced \\
Firewall rule analysis & Identify overly permissive rules \\
VLAN hopping & Test for layer 2 isolation bypasses \\
Traffic interception & Assess encryption and authentication \\
Wireless assessment & Identify rogue or insecure wireless \\
\end{tabular}
\end{center}

\subsection{Protocol-Specific Testing}

\begin{itemize}
    \item \textbf{Modbus:} Read/write coils and registers, function code fuzzing
    \item \textbf{DNP3:} Command injection, authentication bypass
    \item \textbf{OPC:} Enumeration, unauthorized data access
    \item \textbf{EtherNet/IP:} Device discovery, configuration changes
\end{itemize}

\subsection{Application Testing}

\begin{itemize}
    \item HMI web interface vulnerabilities
    \item Historian database security
    \item Engineering software weaknesses
    \item Remote access portal testing
\end{itemize}

% ----------------------------------------------------------------------------
%  SAFETY GUIDELINES
% ----------------------------------------------------------------------------

\section{Safety Guidelines}

\subsection{Absolute Rules}

\begin{dangerbox}
\textbf{Never perform these actions without explicit approval:}
\begin{itemize}
    \item Write commands to PLCs or RTUs in production
    \item Modify safety system configurations
    \item Test during active production without operations present
    \item Use denial-of-service techniques on OT networks
    \item Exploit vulnerabilities that could cause physical impact
\end{itemize}
\end{dangerbox}

\subsection{Safe Testing Practices}

\begin{enumerate}
    \item \textbf{Start passive:} Observation before interaction
    \item \textbf{Test replicas first:} Use lab or staging environments
    \item \textbf{Coordinate continuously:} Real-time communication with operations
    \item \textbf{Monitor impacts:} Watch process variables during testing
    \item \textbf{Document everything:} Detailed logs for incident response
    \item \textbf{Have rollback plans:} Know how to undo any changes
\end{enumerate}

% ----------------------------------------------------------------------------
%  REPORTING
% ----------------------------------------------------------------------------

\section{Reporting}

\subsection{Report Structure}

\begin{infobox}
OT pentest reports should include:
\begin{itemize}
    \item Executive summary with business impact
    \item Methodology and scope description
    \item Findings with OT-specific risk ratings
    \item Attack path diagrams showing IT-to-OT routes
    \item Remediation recommendations prioritized by risk
    \item Compensating controls for unpatchable systems
\end{itemize}
\end{infobox}

\subsection{Risk Rating Considerations}

Traditional CVSS scores may not reflect OT risk. Consider:

\begin{itemize}
    \item \textbf{Safety impact:} Could exploitation cause physical harm?
    \item \textbf{Production impact:} What is the cost of downtime?
    \item \textbf{Cascading effects:} Could compromise spread?
    \item \textbf{Recovery difficulty:} How hard is restoration?
\end{itemize}

% ----------------------------------------------------------------------------
%  TOOLS
% ----------------------------------------------------------------------------

\section{Tools and Resources}

\subsection{OT-Specific Tools}

\begin{center}
\small
\rowcolors{2}{lightgray}{white}
\begin{tabular}{p{4cm}p{9cm}}
\rowcolor{primary}
\textcolor{white}{\bfseries Tool} & \textcolor{white}{\bfseries Purpose} \\
\midrule
Wireshark + dissectors & Protocol analysis (Modbus, DNP3, S7) \\
PLCScan & PLC discovery and enumeration \\
Redpoint (Nmap scripts) & ICS device identification \\
mbtget/modbus-cli & Modbus protocol testing \\
GRASSMARLIN & Passive OT network mapping \\
\end{tabular}
\end{center}

% ----------------------------------------------------------------------------
%  FURTHER READING
% ----------------------------------------------------------------------------

\section{Further Reading}

\subsection*{Standards and Guidelines}
\begin{itemize}
    \item \textbf{IEC 62443-4-1} -- Secure Product Development\\
          \url{https://www.isa.org/isa62443}
    \item \textbf{NIST SP 800-82 Rev. 3} -- OT Security Guide\\
          \url{https://csrc.nist.gov/publications/detail/sp/800-82/rev-3/final}
\end{itemize}

\subsection*{Resources}
\begin{itemize}
    \item \textbf{CISA} -- ICS Assessment Methodology\\
          \url{https://www.cisa.gov/resources-tools/services/cisa-assessments}
\end{itemize}

\subsection*{Books}
\begin{itemize}
    \item Pascal Ackerman -- \textit{Industrial Cybersecurity} (Packt)
    \item Clint Bodungen -- \textit{Hacking Exposed Industrial Control Systems} (McGraw-Hill)
\end{itemize}

\vfill
\begin{center}
\textcolor{mediumgray}{\rule{0.5\textwidth}{0.5pt}}\\[1em]
\textcolor{mediumgray}{\small Part of the OT Security Learning Series}
\end{center}

\end{document}
