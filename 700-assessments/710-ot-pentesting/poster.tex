% ============================================================================
%  OT Penetration Testing - Poster / Cheat Sheet
% ============================================================================

\documentclass[9pt,a4paper]{extarticle}
\usepackage{otsec-poster}
\usepackage{float}

\begin{document}

\makepostertitle
    {OT Penetration Testing}
    {Methodology for Industrial Security Assessments}
    {Poster 710}
    {Matthias Niedermaier}

\begin{multicols}{2}

\section{\textcolor{accent}{\faIcon{info-circle}}\hspace{0.4em}Overview}

OT penetration testing evaluates industrial control system security by simulating real-world attacks. Unlike IT pentesting, OT assessments require \textbf{specialized knowledge, careful planning, and strict safety considerations} to avoid disrupting critical processes.

\posterdanger{
\textbf{OT pentesting carries significant risk.} Improper testing can cause equipment damage, safety incidents, production outages, or environmental harm. Always prioritize safety over thoroughness.
}

\section{\textcolor{accent}{\faIcon{columns}}\hspace{0.4em}OT vs IT Pentesting}

\begin{center}
\rowcolors{2}{lightgray}{white}
\begin{tabular}{p{1.8cm}p{2.2cm}p{2.3cm}}
\rowcolor{primary}
\textcolor{white}{\faIcon{columns}\hspace{0.2em}\bfseries Aspect} & \textcolor{white}{\faIcon{desktop}\hspace{0.2em}\bfseries IT} & \textcolor{white}{\faIcon{industry}\hspace{0.2em}\bfseries OT} \\
\midrule
Concern & Confidentiality & Safety/availability \\
Tolerance & Resilient & Fragile, crash-prone \\
Downtime & Business impact & Physical/safety \\
Window & Often flexible & Maintenance only \\
Protocols & TCP/IP & Modbus, DNP3 \\
Recovery & Backup restore & May need site visit \\
\end{tabular}
\end{center}

\section{\textcolor{accent}{\faIcon{clipboard-check}}\hspace{0.4em}Pre-Engagement}

\postersuccess{
\textbf{Essential pre-engagement activities:}
\begin{itemize}
    \item \textcolor{success}{\faIcon{file-signature}}\hspace{0.2em}Obtain written authorization from asset owner
    \item \textcolor{info}{\faIcon{users}}\hspace{0.2em}Identify all stakeholders (OT, IT, Safety, Ops)
    \item \textcolor{accent}{\faIcon{border-all}}\hspace{0.2em}Define clear scope boundaries and exclusions
    \item \textcolor{danger}{\faIcon{phone}}\hspace{0.2em}Establish emergency contacts and rollback procedures
    \item \textcolor{warning}{\faIcon{hard-hat}}\hspace{0.2em}Review safety documentation and process constraints
    \item \textcolor{info}{\faIcon{calendar-alt}}\hspace{0.2em}Schedule during planned maintenance if possible
\end{itemize}
}

\subsection{\textcolor{accent}{\faIcon{crosshairs}}\hspace{0.3em}Scope Considerations}

\begin{center}
\rowcolors{2}{lightgray}{white}
\begin{tabular}{p{2.2cm}p{4.3cm}}
\rowcolor{primary}
\textcolor{white}{\faIcon{crosshairs}\hspace{0.2em}\bfseries Element} & \textcolor{white}{\faIcon{clipboard-list}\hspace{0.2em}\bfseries Considerations} \\
\midrule
Network & Which zones in scope (IT, DMZ, OT)? \\
Systems & HMIs, PLCs, RTUs, historians? \\
Protocols & Which industrial protocols? \\
Active/passive & Can exploitation be performed? \\
Physical & Physical security testing? \\
Time & Testing windows, blackout periods? \\
\end{tabular}
\end{center}

\section{\textcolor{accent}{\faIcon{list-ol}}\hspace{0.4em}Testing Phases}

\subsection{\textcolor{accent}{\faIcon{eye}}\hspace{0.3em}Phase 1: Passive Reconnaissance}

\begin{itemize}
    \item \textcolor{info}{\faIcon{globe}}\hspace{0.2em}\textbf{OSINT} -- Public info about systems, vendors
    \item \textcolor{info}{\faIcon{network-wired}}\hspace{0.2em}\textbf{Network capture} -- Passive traffic analysis (SPAN/TAP)
    \item \textcolor{accent}{\faIcon{code}}\hspace{0.2em}\textbf{Protocol ID} -- Identify industrial protocols in use
    \item \textcolor{accent}{\faIcon{sitemap}}\hspace{0.2em}\textbf{Asset mapping} -- Map devices from observed traffic
\end{itemize}

\posterinfo{
\textbf{Spend adequate time on passive recon.} It provides significant intelligence with minimal risk. This phase should be thorough before any active testing begins.
}

\subsection{\textcolor{accent}{\faIcon{search}}\hspace{0.3em}Phase 2: Active Reconnaissance}

\begin{itemize}
    \item \textcolor{warning}{\faIcon{bullseye}}\hspace{0.2em}\textbf{Controlled scanning} -- Slow, targeted port scans
    \item \textcolor{info}{\faIcon{tag}}\hspace{0.2em}\textbf{Service ID} -- Banner grabbing on known ports
    \item \textcolor{accent}{\faIcon{list-alt}}\hspace{0.2em}\textbf{Protocol enumeration} -- Query using native protocols
    \item \textcolor{danger}{\faIcon{search}}\hspace{0.2em}\textbf{Vuln scanning} -- OT-aware scanners only
\end{itemize}

\posterwarning{
\textcolor{warning}{\faIcon{exclamation-triangle}}\hspace{0.2em}\textbf{Standard IT vulnerability scanners can crash OT devices.} Use only scanners designed for industrial environments or perform manual testing. Always test tools in lab first.
}

\subsection{\textcolor{accent}{\faIcon{bug}}\hspace{0.3em}Phase 3: Vulnerability Assessment}

\begin{itemize}
    \item \textcolor{danger}{\faIcon{key}}\hspace{0.2em}Default and weak credentials
    \item \textcolor{danger}{\faIcon{bug}}\hspace{0.2em}Unpatched vulnerabilities
    \item \textcolor{warning}{\faIcon{cog}}\hspace{0.2em}Insecure protocol configurations
    \item \textcolor{warning}{\faIcon{project-diagram}}\hspace{0.2em}Network segmentation gaps
\end{itemize}

\subsection{\textcolor{accent}{\faIcon{crosshairs}}\hspace{0.3em}Phase 4: Controlled Exploitation}

\begin{itemize}
    \item \textcolor{info}{\faIcon{flask}}\hspace{0.2em}\textbf{Test replicas first} -- Validate exploits in lab
    \item \textcolor{success}{\faIcon{undo}}\hspace{0.2em}\textbf{Reversible only} -- No destructive testing
    \item \textcolor{warning}{\faIcon{heartbeat}}\hspace{0.2em}\textbf{Continuous monitoring} -- Watch process during tests
    \item \textcolor{danger}{\faIcon{stop-circle}}\hspace{0.2em}\textbf{Immediate rollback} -- Stop at first sign of issues
\end{itemize}

\section{\textcolor{accent}{\faIcon{vial}}\hspace{0.4em}Testing Techniques}

\begin{center}
\rowcolors{2}{lightgray}{white}
\begin{tabular}{p{2.2cm}p{4.3cm}}
\rowcolor{primary}
\textcolor{white}{\faIcon{vial}\hspace{0.2em}\bfseries Test} & \textcolor{white}{\faIcon{bullseye}\hspace{0.2em}\bfseries Purpose} \\
\midrule
Segmentation & Verify zone boundaries enforced \\
Firewall rules & Identify overly permissive rules \\
VLAN hopping & Test Layer 2 isolation bypasses \\
Traffic intercept & Assess encryption/authentication \\
Wireless & Identify rogue/insecure wireless \\
\end{tabular}
\end{center}

\subsection{\textcolor{accent}{\faIcon{stream}}\hspace{0.3em}Protocol-Specific Testing}

\begin{itemize}
    \item \textcolor{accent}{\faIcon{microchip}}\hspace{0.2em}\textbf{Modbus} -- Read/write coils and registers, function codes
    \item \textcolor{accent}{\faIcon{broadcast-tower}}\hspace{0.2em}\textbf{DNP3} -- Command injection, authentication bypass
    \item \textcolor{accent}{\faIcon{plug}}\hspace{0.2em}\textbf{OPC} -- Enumeration, unauthorized data access
    \item \textcolor{accent}{\faIcon{industry}}\hspace{0.2em}\textbf{EtherNet/IP} -- Discovery, configuration changes
\end{itemize}

\section{\textcolor{accent}{\faIcon{hard-hat}}\hspace{0.4em}Safety Guidelines}

\posterdanger{
\textcolor{warning}{\faIcon{exclamation-triangle}}\hspace{0.2em}\textbf{Never without explicit approval:} Write commands to production PLCs/RTUs. Modify safety system configs. Test during active production without operations. Use DoS techniques on OT networks. Exploit vulnerabilities with physical impact potential.
}

\subsection{\textcolor{accent}{\faIcon{shield-alt}}\hspace{0.3em}Safe Practices}

\begin{enumerate}
    \item \textcolor{info}{\faIcon{eye}}\hspace{0.2em}Start passive -- observation before interaction
    \item \textcolor{accent}{\faIcon{flask}}\hspace{0.2em}Test replicas first -- use lab environments
    \item \textcolor{success}{\faIcon{comments}}\hspace{0.2em}Coordinate continuously -- real-time ops communication
    \item \textcolor{warning}{\faIcon{heartbeat}}\hspace{0.2em}Monitor impacts -- watch process variables
    \item \textcolor{danger}{\faIcon{undo}}\hspace{0.2em}Have rollback plans -- know how to undo changes
\end{enumerate}

\section{\textcolor{accent}{\faIcon{wrench}}\hspace{0.4em}OT-Specific Tools}

\begin{center}
\rowcolors{2}{lightgray}{white}
\begin{tabular}{p{2.2cm}p{4.3cm}}
\rowcolor{primary}
\textcolor{white}{\faIcon{wrench}\hspace{0.2em}\bfseries Tool} & \textcolor{white}{\faIcon{cogs}\hspace{0.2em}\bfseries Purpose} \\
\midrule
Wireshark & Protocol analysis (Modbus, DNP3) \\
PLCScan & PLC discovery and enumeration \\
mbtget & Modbus protocol testing \\
GRASSMARLIN & Passive OT network mapping \\
\end{tabular}
\end{center}

\postertip{
OT pentesting requires \textbf{safety-first methodology}. Start with passive reconnaissance---it provides high value at low risk. Always test exploits in lab environments before production. Coordinate continuously with operations and have rollback plans ready. Reports should include OT-specific risk ratings (not just CVSS), attack path diagrams, and compensating controls for unpatchable systems. Never test safety systems without explicit written authorization.
}

\end{multicols}

\end{document}
