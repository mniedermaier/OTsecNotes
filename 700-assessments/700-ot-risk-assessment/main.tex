% ============================================================================
%  OT Risk Assessment - OT Security Learning Resource
% ============================================================================

\documentclass[11pt,a4paper]{article}
\usepackage{otsec-template}

\hypersetup{
    pdftitle={OT Risk Assessment},
    pdfsubject={Assessing Cybersecurity Risk in Industrial Environments},
}

\begin{document}

% ----------------------------------------------------------------------------
%  TITLE PAGE
% ----------------------------------------------------------------------------

\maketitlepage
    {OT Risk Assessment}
    {Assessing Cybersecurity Risk in Industrial Environments}
    {OT Security Learning Series}
    {Document 700 \quad|\quad January 2026}
    {Matthias Niedermaier}

% ----------------------------------------------------------------------------
%  TABLE OF CONTENTS
% ----------------------------------------------------------------------------

\tableofcontents
\newpage

% ----------------------------------------------------------------------------
%  INTRODUCTION
% ----------------------------------------------------------------------------

\section{Introduction}

Risk assessment in OT environments must consider factors beyond traditional IT security---including physical safety, environmental impact, and operational continuity. The goal is to identify, analyze, and prioritize risks to enable informed security decisions.

\begin{infobox}
OT risk assessment bridges cybersecurity and process safety. It requires collaboration between security professionals, engineers, and operations staff to accurately evaluate threats and their potential consequences.
\end{infobox}

\subsection{OT vs IT Risk Considerations}

\begin{center}
\small
\rowcolors{2}{lightgray}{white}
\begin{tabular}{p{3.5cm}p{4.5cm}p{5cm}}
\rowcolor{primary}
\textcolor{white}{\bfseries Factor} & \textcolor{white}{\bfseries IT Focus} & \textcolor{white}{\bfseries OT Focus} \\
\midrule
Primary concern & Data confidentiality & Safety and availability \\
Impact scope & Business operations & Physical world effects \\
Threat actors & Cybercriminals, APTs & Nation-states, insiders \\
Asset lifetime & 3--5 years & 15--25+ years \\
Patching & Regular updates & Constrained by operations \\
\end{tabular}
\end{center}

% ----------------------------------------------------------------------------
%  RISK FORMULA
% ----------------------------------------------------------------------------

\section{Risk Fundamentals}

\subsection{Risk Equation}

\begin{conceptbox}{Risk Calculation}
\begin{center}
\textbf{Risk = Threat $\times$ Vulnerability $\times$ Consequence}
\end{center}

\begin{itemize}
    \item \textbf{Threat:} Who might attack and how likely?
    \item \textbf{Vulnerability:} What weaknesses can be exploited?
    \item \textbf{Consequence:} What is the impact if successful?
\end{itemize}
\end{conceptbox}

\subsection{Consequence Categories}

OT risk assessment must consider multiple impact types:

\begin{itemize}
    \item \textbf{Safety:} Potential for injury or loss of life
    \item \textbf{Environmental:} Spills, emissions, contamination
    \item \textbf{Operational:} Production loss, equipment damage
    \item \textbf{Financial:} Direct costs and business impact
    \item \textbf{Reputational:} Customer and public trust
    \item \textbf{Regulatory:} Fines, compliance violations
\end{itemize}

% ----------------------------------------------------------------------------
%  IEC 62443 APPROACH
% ----------------------------------------------------------------------------

\section{IEC 62443 Risk Assessment}

\subsection{Zone and Conduit Model}

IEC 62443-3-2 defines a systematic approach:

\begin{enumerate}
    \item \textbf{Identify system under consideration (SUC)}
    \item \textbf{Perform initial risk assessment}
    \item \textbf{Partition into zones and conduits}
    \item \textbf{Assign target security levels (SL-T)}
    \item \textbf{Document security requirements}
\end{enumerate}

\subsection{Security Levels}

\begin{center}
\small
\rowcolors{2}{lightgray}{white}
\begin{tabular}{p{1.5cm}p{3cm}p{8.5cm}}
\rowcolor{primary}
\textcolor{white}{\bfseries SL} & \textcolor{white}{\bfseries Threat Level} & \textcolor{white}{\bfseries Description} \\
\midrule
\slone & Casual/Accidental & Protection against unintentional errors \\
\sltwo & Intentional (simple) & Low skill, limited resources, general motivation \\
\slthree & Intentional (sophisticated) & Moderate skill, moderate resources, specific target \\
\slfour & Intentional (nation-state) & High skill, extensive resources, highly motivated \\
\end{tabular}
\end{center}

\subsection{Risk Matrix}

\begin{center}
\begin{tikzpicture}[scale=0.8]
    % Grid
    \draw[step=1.5cm, gray, thin] (0,0) grid (6,6);

    % Colors
    \fill[success!40] (0,0) rectangle (1.5,1.5);
    \fill[success!40] (0,1.5) rectangle (1.5,3);
    \fill[success!40] (1.5,0) rectangle (3,1.5);
    \fill[warning!40] (0,3) rectangle (1.5,4.5);
    \fill[warning!40] (1.5,1.5) rectangle (3,3);
    \fill[warning!40] (3,0) rectangle (4.5,1.5);
    \fill[warning!40] (0,4.5) rectangle (1.5,6);
    \fill[warning!40] (1.5,3) rectangle (3,4.5);
    \fill[warning!40] (3,1.5) rectangle (4.5,3);
    \fill[warning!40] (4.5,0) rectangle (6,1.5);
    \fill[danger!40] (1.5,4.5) rectangle (3,6);
    \fill[danger!40] (3,3) rectangle (4.5,4.5);
    \fill[danger!40] (4.5,1.5) rectangle (6,3);
    \fill[danger!40] (3,4.5) rectangle (4.5,6);
    \fill[danger!40] (4.5,3) rectangle (6,4.5);
    \fill[danger!60] (4.5,4.5) rectangle (6,6);

    % Labels
    \node[font=\scriptsize] at (0.75,0.75) {Low};
    \node[font=\scriptsize] at (2.25,2.25) {Med};
    \node[font=\scriptsize] at (3.75,3.75) {High};
    \node[font=\scriptsize] at (5.25,5.25) {Critical};

    % Axes
    \node[font=\small, rotate=90] at (-0.7,3) {Likelihood};
    \node[font=\small] at (3,-0.5) {Consequence};
\end{tikzpicture}
\end{center}

% ----------------------------------------------------------------------------
%  ASSESSMENT PROCESS
% ----------------------------------------------------------------------------

\section{Assessment Process}

\subsection{Step 1: Scope Definition}

\begin{itemize}
    \item Define boundaries of assessment (facility, system, zone)
    \item Identify stakeholders and their concerns
    \item Gather existing documentation (diagrams, procedures)
    \item Establish assessment criteria and risk tolerance
\end{itemize}

\subsection{Step 2: Asset Identification}

\begin{itemize}
    \item Inventory all assets in scope
    \item Classify by criticality and function
    \item Map communication flows and dependencies
    \item Identify data flows crossing zone boundaries
\end{itemize}

\subsection{Step 3: Threat Assessment}

\begin{itemize}
    \item Identify relevant threat actors (nation-state, criminal, insider)
    \item Consider threat capabilities and motivations
    \item Review industry-specific threat intelligence
    \item Assess likelihood of targeting your organization
\end{itemize}

\subsection{Step 4: Vulnerability Assessment}

\begin{warningbox}
\textbf{OT vulnerability assessment considerations:}
\begin{itemize}
    \item Avoid active scanning of production systems
    \item Use passive methods and configuration review
    \item Consider architectural weaknesses, not just CVEs
    \item Assess compensating controls effectiveness
\end{itemize}
\end{warningbox}

\subsection{Step 5: Risk Evaluation}

\begin{itemize}
    \item Calculate risk for each threat-vulnerability pair
    \item Consider existing controls and their effectiveness
    \item Prioritize risks based on consequence severity
    \item Document assumptions and uncertainties
\end{itemize}

% ----------------------------------------------------------------------------
%  RISK TREATMENT
% ----------------------------------------------------------------------------

\section{Risk Treatment}

\subsection{Treatment Options}

\begin{center}
\small
\rowcolors{2}{lightgray}{white}
\begin{tabular}{p{2.5cm}p{10.5cm}}
\rowcolor{primary}
\textcolor{white}{\bfseries Option} & \textcolor{white}{\bfseries Description} \\
\midrule
Mitigate & Implement controls to reduce likelihood or consequence \\
Transfer & Insurance, contractual arrangements with vendors \\
Accept & Document acceptance with management approval \\
Avoid & Eliminate the risk source (remove system, change process) \\
\end{tabular}
\end{center}

\subsection{Control Selection}

Prioritize controls that:
\begin{itemize}
    \item Address highest risks first
    \item Are compatible with operational requirements
    \item Can be implemented without safety impact
    \item Provide defense in depth
\end{itemize}

% ----------------------------------------------------------------------------
%  DOCUMENTATION
% ----------------------------------------------------------------------------

\section{Documentation}

\subsection{Required Outputs}

\begin{itemize}
    \item \textbf{Asset inventory} with criticality ratings
    \item \textbf{Zone and conduit diagram} with security levels
    \item \textbf{Risk register} with ratings and treatment plans
    \item \textbf{Security requirements} for each zone
    \item \textbf{Residual risk statement} with management sign-off
\end{itemize}

\subsection{Review Cycle}

\begin{itemize}
    \item \textbf{Annual review:} Full reassessment recommended
    \item \textbf{Trigger-based:} After incidents, major changes, new threats
    \item \textbf{Continuous:} Update as new vulnerabilities discovered
\end{itemize}

% ----------------------------------------------------------------------------
%  FURTHER READING
% ----------------------------------------------------------------------------

\section{Further Reading}

\subsection*{Standards}
\begin{itemize}
    \item \textbf{IEC 62443-3-2} -- Security Risk Assessment for System Design\\
          \url{https://www.isa.org/standards-and-publications/isa-standards/isa-iec-62443-series-of-standards}
    \item \textbf{NIST SP 800-82 Rev. 3} -- Guide to OT Security\\
          \url{https://csrc.nist.gov/pubs/sp/800/82/r3/final}
    \item \textbf{ISO 27005} -- Information Security Risk Management\\
          \url{https://www.iso.org/standard/80585.html}
\end{itemize}

\subsection*{Resources}
\begin{itemize}
    \item \textbf{NIST Cybersecurity Framework}\\
          \url{https://www.nist.gov/cyberframework}
\end{itemize}

\vfill
\begin{center}
\textcolor{mediumgray}{\rule{0.5\textwidth}{0.5pt}}\\[1em]
\textcolor{mediumgray}{\small Part of the OT Security Learning Series}
\end{center}

\end{document}
