% ============================================================================
%  NIST SP 800-82 - OT Security Learning Resource
% ============================================================================

\documentclass[11pt,a4paper]{article}
\usepackage{otsec-template}

\hypersetup{
    pdftitle={NIST SP 800-82},
    pdfsubject={Guide to Operational Technology Security},
}

\begin{document}

% ----------------------------------------------------------------------------
%  TITLE PAGE
% ----------------------------------------------------------------------------

\maketitlepage
    {NIST SP 800-82}
    {Guide to Operational Technology Security}
    {OT Security Learning Series}
    {Document 111 \quad|\quad January 2026}
    {Matthias Niedermaier}

% ----------------------------------------------------------------------------
%  TABLE OF CONTENTS
% ----------------------------------------------------------------------------

\tableofcontents
\newpage

% ----------------------------------------------------------------------------
%  INTRODUCTION
% ----------------------------------------------------------------------------

\section{Introduction}

NIST Special Publication 800-82 is the primary U.S. government guidance document for securing Operational Technology (OT) systems, including Industrial Control Systems (ICS), SCADA systems, and Distributed Control Systems (DCS).

\begin{infobox}
NIST SP 800-82 Revision 3 (2023) represents a significant update, expanding scope from ICS to all OT systems and aligning with the NIST Cybersecurity Framework and SP 800-53.
\end{infobox}

\subsection{Document Evolution}

\begin{center}
\small
\rowcolors{2}{lightgray}{white}
\begin{tabular}{p{2.5cm}p{2cm}p{8.5cm}}
\rowcolor{primary}
\textcolor{white}{\bfseries Version} & \textcolor{white}{\bfseries Year} & \textcolor{white}{\bfseries Key Changes} \\
\midrule
Original & 2006 & Initial ICS security guidance \\
Revision 1 & 2011 & Updated threats, expanded controls \\
Revision 2 & 2015 & Risk management, ISA/IEC 62443 alignment \\
Revision 3 & 2023 & OT scope, supply chain, cloud, CSF alignment \\
\end{tabular}
\end{center}

\subsection{Why NIST 800-82 Matters}

\begin{itemize}
    \item \textbf{Authoritative:} Primary U.S. federal guidance for OT security
    \item \textbf{Comprehensive:} Covers threats, vulnerabilities, and countermeasures
    \item \textbf{Practical:} Provides implementation guidance and overlay controls
    \item \textbf{Referenced:} Used by regulators, auditors, and industry worldwide
\end{itemize}

% ----------------------------------------------------------------------------
%  DOCUMENT STRUCTURE
% ----------------------------------------------------------------------------

\section{Document Structure}

NIST SP 800-82 Rev. 3 is organized into six main sections:

\begin{conceptbox}{Document Organization}
\begin{enumerate}
    \item \textbf{Introduction:} Purpose, scope, and document structure
    \item \textbf{OT Overview:} System types, architectures, and components
    \item \textbf{OT Risk Management:} Risk assessment and management approaches
    \item \textbf{OT Security Program:} Governance, policies, and procedures
    \item \textbf{OT Security Architecture:} Network design and segmentation
    \item \textbf{Security Controls:} OT-specific control recommendations
\end{enumerate}
\end{conceptbox}

% ----------------------------------------------------------------------------
%  OT SYSTEM OVERVIEW
% ----------------------------------------------------------------------------

\section{OT System Overview}

\subsection{System Types Covered}

\begin{center}
\small
\rowcolors{2}{lightgray}{white}
\begin{tabular}{p{3.5cm}p{9.5cm}}
\rowcolor{primary}
\textcolor{white}{\bfseries System Type} & \textcolor{white}{\bfseries Description} \\
\midrule
SCADA & Supervisory Control and Data Acquisition for distributed assets \\
DCS & Distributed Control Systems for process manufacturing \\
PLC/RTU & Programmable controllers for local automation \\
SIS & Safety Instrumented Systems for emergency shutdown \\
BAS/BMS & Building Automation and Management Systems \\
IIoT & Industrial Internet of Things devices and sensors \\
\end{tabular}
\end{center}

\subsection{Key Differences: IT vs OT}

\begin{warningbox}
NIST 800-82 emphasizes that OT environments have different priorities:
\begin{itemize}
    \item \textbf{Availability} is typically the highest priority (not confidentiality)
    \item \textbf{Safety} considerations may override security decisions
    \item \textbf{Legacy systems} often cannot be patched or upgraded
    \item \textbf{Real-time requirements} limit security control options
\end{itemize}
\end{warningbox}

% ----------------------------------------------------------------------------
%  RISK MANAGEMENT
% ----------------------------------------------------------------------------

\section{Risk Management}

\subsection{Risk Assessment Process}

NIST 800-82 recommends a structured risk assessment approach:

\begin{enumerate}
    \item \textbf{System characterization:} Identify OT assets and boundaries
    \item \textbf{Threat identification:} Determine relevant threat sources
    \item \textbf{Vulnerability identification:} Assess system weaknesses
    \item \textbf{Impact analysis:} Evaluate consequences of compromise
    \item \textbf{Risk determination:} Calculate risk levels
    \item \textbf{Control recommendations:} Select appropriate mitigations
\end{enumerate}

\subsection{Threat Sources}

\begin{center}
\small
\rowcolors{2}{lightgray}{white}
\begin{tabular}{p{3.5cm}p{9.5cm}}
\rowcolor{primary}
\textcolor{white}{\bfseries Threat Source} & \textcolor{white}{\bfseries Characteristics} \\
\midrule
Nation-states & Advanced capabilities, strategic objectives, persistent \\
Cybercriminals & Financial motivation, ransomware, extortion \\
Hacktivists & Political/social motivation, publicity-seeking \\
Insiders & Authorized access, knowledge of systems \\
Terrorists & Disruption and destruction goals \\
\end{tabular}
\end{center}

% ----------------------------------------------------------------------------
%  SECURITY ARCHITECTURE
% ----------------------------------------------------------------------------

\section{Security Architecture}

\subsection{Defense-in-Depth}

NIST 800-82 advocates for layered security:

\begin{successbox}
\textbf{Defense-in-Depth Layers:}
\begin{itemize}
    \item Physical security (access control, surveillance)
    \item Network security (segmentation, firewalls, DMZ)
    \item Host security (hardening, whitelisting, patching)
    \item Application security (secure coding, input validation)
    \item Data security (encryption, integrity checking)
\end{itemize}
\end{successbox}

\subsection{Network Segmentation}

The document recommends network architecture based on zones:

\begin{itemize}
    \item \textbf{Enterprise Zone:} Corporate IT network
    \item \textbf{DMZ:} Buffer zone with shared services
    \item \textbf{Control Zone:} OT network with control systems
    \item \textbf{Safety Zone:} Isolated safety systems
\end{itemize}

\begin{tipbox}
NIST 800-82 recommends using the Purdue Model or IEC 62443 zone concepts for network architecture design.
\end{tipbox}

% ----------------------------------------------------------------------------
%  SECURITY CONTROLS
% ----------------------------------------------------------------------------

\section{Security Controls}

\subsection{OT Overlay}

NIST 800-82 provides an OT overlay for SP 800-53 controls, with tailored guidance for:

\begin{center}
\small
\rowcolors{2}{lightgray}{white}
\begin{tabular}{p{2cm}p{4cm}p{7cm}}
\rowcolor{primary}
\textcolor{white}{\bfseries Family} & \textcolor{white}{\bfseries Name} & \textcolor{white}{\bfseries OT Considerations} \\
\midrule
AC & Access Control & Physical and logical access, remote access \\
AU & Audit & Logging without performance impact \\
CM & Configuration Mgmt & Change control, baseline configurations \\
CP & Contingency Planning & Backup, recovery, continuity \\
IA & Identification/Auth & Account management, strong authentication \\
IR & Incident Response & OT-specific procedures, coordination \\
MA & Maintenance & Secure remote maintenance, vendor access \\
SC & System/Comm Protection & Network segmentation, encryption \\
\end{tabular}
\end{center}

\subsection{Key Control Recommendations}

\begin{dangerbox}
\textbf{High-Priority Controls for OT:}
\begin{itemize}
    \item Network segmentation and traffic filtering
    \item Secure remote access with multi-factor authentication
    \item Application whitelisting on critical systems
    \item Comprehensive logging and monitoring
    \item Regular vulnerability assessments
    \item Incident response planning and testing
\end{itemize}
\end{dangerbox}

% ----------------------------------------------------------------------------
%  IMPLEMENTATION GUIDANCE
% ----------------------------------------------------------------------------

\section{Implementation Guidance}

\subsection{Phased Approach}

NIST 800-82 recommends implementing security in phases:

\begin{enumerate}
    \item \textbf{Assessment:} Inventory assets, assess current state
    \item \textbf{Planning:} Develop security plan, prioritize actions
    \item \textbf{Implementation:} Deploy controls incrementally
    \item \textbf{Operations:} Monitor, maintain, and improve
\end{enumerate}

\subsection{Integration with Other Standards}

\begin{infobox}
NIST 800-82 aligns with and references:
\begin{itemize}
    \item NIST Cybersecurity Framework (CSF)
    \item ISA/IEC 62443 series
    \item NERC CIP (for electric utilities)
    \item ISO 27001/27002
\end{itemize}
\end{infobox}

% ----------------------------------------------------------------------------
%  FURTHER READING
% ----------------------------------------------------------------------------

\section{Further Reading}

\subsection*{NIST Publications}
\begin{itemize}
    \item \textbf{NIST SP 800-82 Rev. 3} -- Guide to OT Security\\
          \url{https://csrc.nist.gov/publications/detail/sp/800-82/rev-3/final}
    \item \textbf{NIST Cybersecurity Framework 2.0}\\
          \url{https://www.nist.gov/cyberframework}
    \item \textbf{NIST SP 800-53 Rev. 5} -- Security Controls\\
          \url{https://csrc.nist.gov/publications/detail/sp/800-53/rev-5/final}
\end{itemize}

\subsection*{Related Standards}
\begin{itemize}
    \item \textbf{ISA/IEC 62443} -- Industrial Automation Security\\
          \url{https://www.isa.org/isa62443}
\end{itemize}

\vfill
\begin{center}
\textcolor{mediumgray}{\rule{0.5\textwidth}{0.5pt}}\\[1em]
\textcolor{mediumgray}{\small Part of the OT Security Learning Series}
\end{center}

\end{document}
