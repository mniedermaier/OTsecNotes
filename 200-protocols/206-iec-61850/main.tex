% ============================================================================
%  206-iec-61850 - OT Security Learning Resource
% ============================================================================

\documentclass[11pt,a4paper]{article}
\usepackage{otsec-template}

% Define colors for TikZ
\colorlet{otprimary}{primary}
\colorlet{otaccent}{accent}
\colorlet{otsuccess}{success}
\colorlet{otwarning}{warning}
\colorlet{otdanger}{danger}
\colorlet{otinfo}{info}

\begin{document}

\maketitlepage
    {IEC 61850 Protocol}
    {Communication standard for electrical substation automation}
    {OT Security Learning Series}
    {Document 206 \quad|\quad January 2026}
    {Matthias Niedermaier}

\tableofcontents
\newpage

% ============================================================================
\section{Introduction}
% ============================================================================

\begin{infobox}
IEC 61850 is the international standard for communication in electrical substations. It enables interoperability between protective relays, circuit breakers, transformers, and control systems in power grid infrastructure.
\end{infobox}

Key characteristics:
\begin{itemize}
    \item International standard for substation automation
    \item Object-oriented data modeling
    \item Multiple communication services (MMS, GOOSE, SV)
    \item Designed for power utility environments
    \item Critical infrastructure protocol
\end{itemize}

\begin{dangerbox}
IEC 61850 controls protective relays and circuit breakers in the power grid. Attacks on this protocol can cause equipment damage, widespread outages, or safety hazards. The Industroyer/CrashOverride malware targeted IEC 61850.
\end{dangerbox}

% ============================================================================
\section{Protocol Architecture}
% ============================================================================

\subsection{Data Model}

IEC 61850 uses a hierarchical object model:

\begin{itemize}
    \item \textbf{Physical Device} -- The actual hardware
    \item \textbf{Logical Device} -- Functional grouping within physical device
    \item \textbf{Logical Node} -- Specific function (e.g., XCBR for circuit breaker)
    \item \textbf{Data Object} -- Attributes of the logical node
    \item \textbf{Data Attribute} -- Individual values
\end{itemize}

\subsection{Logical Node Examples}

\begin{table}[h]
\centering
\begin{tabular}{|l|l|}
\hline
\textbf{Logical Node} & \textbf{Function} \\
\hline
XCBR & Circuit breaker \\
XSWI & Disconnector/switch \\
PDIS & Distance protection \\
PTOC & Overcurrent protection \\
MMXU & Measurement unit \\
CSWI & Switch controller \\
\hline
\end{tabular}
\caption{Common IEC 61850 Logical Nodes}
\end{table}

% ============================================================================
\section{Communication Services}
% ============================================================================

\subsection{Service Types}

\begin{definitionbox}{IEC 61850 Communication Services}
\begin{itemize}
    \item \textbf{MMS (Manufacturing Message Specification)} -- Client/server for configuration, reporting
    \item \textbf{GOOSE (Generic Object Oriented Substation Event)} -- Fast multicast for protection events
    \item \textbf{SV (Sampled Values)} -- Real-time measurement streaming
    \item \textbf{Time Sync} -- Precision time protocol (IEEE 1588 PTP)
\end{itemize}
\end{definitionbox}

\begin{figure}[h]
\centering
\begin{tikzpicture}[
    device/.style={rectangle, draw, thick, minimum height=0.8cm, minimum width=1.6cm, rounded corners=2pt, font=\tiny\bfseries, align=center},
    bus/.style={thick},
    arrow/.style={->, thick, >=stealth}
]

% Station level
\node[device, fill=otinfo!20] (hmi) at (-1,3) {HMI};
\node[device, fill=otinfo!20] (gateway) at (4,3) {Gateway};

% Station bus
\draw[bus, otprimary] (-3,2) -- (6,2);
\node[font=\tiny, otprimary, anchor=west] at (6.2,2) {Station Bus (MMS)};

% Bay level - IEDs
\node[device, fill=otaccent!20] (ied1) at (-1.5,1) {IED};
\node[device, fill=otaccent!20] (ied2) at (1.5,1) {IED};
\node[device, fill=otaccent!20] (ied3) at (4.5,1) {IED};

% Process bus
\draw[bus, otdanger] (-3,0) -- (6,0);
\node[font=\tiny, otdanger, anchor=west] at (6.2,0) {Process Bus (GOOSE/SV)};

% Field devices
\node[device, fill=otsuccess!20] (ct1) at (-1.5,-1) {CT/VT};
\node[device, fill=otsuccess!20] (cb) at (1.5,-1) {Breaker};
\node[device, fill=otsuccess!20] (ct2) at (4.5,-1) {CT/VT};

% Connections
\draw[arrow] (hmi) -- (-1,2);
\draw[arrow] (gateway) -- (4,2);
\draw[arrow] (ied1) -- (-1.5,2);
\draw[arrow] (ied2) -- (1.5,2);
\draw[arrow] (ied3) -- (4.5,2);
\draw[arrow] (ied1) -- (-1.5,0);
\draw[arrow] (ied2) -- (1.5,0);
\draw[arrow] (ied3) -- (4.5,0);
\draw[arrow] (ct1) -- (-1.5,0);
\draw[arrow] (cb) -- (1.5,0);
\draw[arrow] (ct2) -- (4.5,0);

% GOOSE between IEDs (arrows above the boxes)
\draw[arrow, otdanger, dashed] (-0.6,1.5) -- node[above, font=\tiny] {GOOSE} (0.6,1.5);
\draw[arrow, otdanger, dashed] (2.4,1.5) -- node[above, font=\tiny] {GOOSE} (3.6,1.5);

\end{tikzpicture}
\caption{IEC 61850 Substation Architecture}
\end{figure}

\subsection{Network Ports}

\begin{table}[h]
\centering
\begin{tabular}{|l|l|l|}
\hline
\textbf{Service} & \textbf{Port/Type} & \textbf{Purpose} \\
\hline
MMS & TCP/102 & Configuration, reporting \\
GOOSE & EtherType 0x88B8 & Protection events (Layer 2) \\
SV & EtherType 0x88BA & Sampled measurements (Layer 2) \\
PTP & UDP/319, 320 & Time synchronization \\
\hline
\end{tabular}
\caption{IEC 61850 Network Services}
\end{table}

% ============================================================================
\section{GOOSE Protocol}
% ============================================================================

\subsection{GOOSE Characteristics}

\begin{itemize}
    \item Layer 2 multicast (no IP routing)
    \item Sub-4ms transmission for protection
    \item Publisher/subscriber model
    \item Carries binary status and commands
    \item Critical for protection coordination
\end{itemize}

\subsection{GOOSE Message Structure}

\begin{itemize}
    \item \textbf{goID} -- GOOSE identifier
    \item \textbf{gocbRef} -- Control block reference
    \item \textbf{datSet} -- Dataset reference
    \item \textbf{stNum} -- State number (increments on change)
    \item \textbf{sqNum} -- Sequence number
    \item \textbf{allData} -- Actual data values
\end{itemize}

\begin{warningbox}
GOOSE operates at Layer 2 with no authentication. An attacker on the same network segment can inject GOOSE messages to trip breakers or block protection signals.
\end{warningbox}

% ============================================================================
\section{Security Vulnerabilities}
% ============================================================================

\begin{dangerbox}
The base IEC 61850 standard (Edition 1 and 2) includes no security mechanisms. GOOSE and SV are particularly vulnerable as they operate at Layer 2 and must be processed within milliseconds.
\end{dangerbox}

\subsection{Protocol Weaknesses}

\begin{itemize}
    \item \textbf{No authentication} -- Messages accepted from any source
    \item \textbf{No encryption} -- All traffic in cleartext
    \item \textbf{Layer 2 protocols} -- GOOSE/SV bypass IP firewalls
    \item \textbf{Time-critical} -- Security checks may impact performance
    \item \textbf{Multicast} -- Easy to sniff and replay
\end{itemize}

\subsection{Attack Vectors}

\begin{itemize}
    \item \textbf{GOOSE injection} -- Send fake trip commands to breakers
    \item \textbf{GOOSE blocking} -- Prevent legitimate protection signals
    \item \textbf{SV manipulation} -- Inject false measurements
    \item \textbf{MMS exploitation} -- Unauthorized configuration changes
    \item \textbf{Time sync attacks} -- Disrupt PTP to desynchronize protection
    \item \textbf{Replay attacks} -- Capture and replay GOOSE messages
\end{itemize}

\subsection{Industroyer/CrashOverride}

The 2016 Ukraine power grid attack used IEC 61850:
\begin{itemize}
    \item Targeted IEC 61850 and other substation protocols
    \item Opened circuit breakers causing outages
    \item Demonstrated real-world IEC 61850 attack capability
\end{itemize}

% ============================================================================
\section{IEC 62351 Security Standard}
% ============================================================================

\begin{successbox}
IEC 62351 provides security extensions for IEC 61850 and other power system protocols. It addresses authentication, integrity, and confidentiality requirements.
\end{successbox}

\subsection{IEC 62351 Components}

\begin{itemize}
    \item \textbf{Part 3} -- TLS for TCP-based protocols (MMS)
    \item \textbf{Part 4} -- Security for MMS specifically
    \item \textbf{Part 6} -- Security for GOOSE and SV (digital signatures)
    \item \textbf{Part 8} -- Role-based access control
    \item \textbf{Part 9} -- Key management
\end{itemize}

\subsection{GOOSE Security (IEC 62351-6)}

\begin{itemize}
    \item Digital signatures for message authentication
    \item Requires hardware acceleration for timing
    \item Adds ~20 bytes overhead per message
    \item Limited vendor implementation
\end{itemize}

\begin{warningbox}
IEC 62351 adoption is slow. Most installed IEC 61850 devices lack security features, requiring network-based protection.
\end{warningbox}

% ============================================================================
\section{Security Mitigations}
% ============================================================================

\subsection{Network Architecture}

\begin{itemize}
    \item \textbf{Physical isolation} -- Separate process bus from station bus
    \item \textbf{VLAN segmentation} -- Isolate GOOSE/SV traffic
    \item \textbf{Managed switches} -- Port security, disable unused ports
    \item \textbf{No remote access} -- Or strictly controlled VPN only
\end{itemize}

\subsection{Layer 2 Security}

Since GOOSE/SV are Layer 2:
\begin{itemize}
    \item MAC address filtering
    \item 802.1X port authentication
    \item Private VLANs
    \item Physical access control to network equipment
\end{itemize}

\subsection{Monitoring}

\begin{itemize}
    \item Monitor for unauthorized GOOSE publishers
    \item Alert on stNum/sqNum anomalies
    \item Detect unexpected MMS connections
    \item Log all configuration changes
    \item Watch for time synchronization issues
\end{itemize}

% ============================================================================
\section{Substation Architecture}
% ============================================================================

\subsection{Network Zones}

\begin{itemize}
    \item \textbf{Process Bus} -- GOOSE/SV between bays (most critical)
    \item \textbf{Station Bus} -- MMS to station controller
    \item \textbf{Remote Access} -- Connection to control center
\end{itemize}

\subsection{Defense in Depth}

\begin{itemize}
    \item Physical perimeter security
    \item Network segmentation between buses
    \item Firewall between substation and control center
    \item Intrusion detection for IEC 61850 protocols
    \item Secure engineering access procedures
\end{itemize}

% ============================================================================
\section{Summary}
% ============================================================================

\begin{definitionbox}{Key Takeaways}
\begin{itemize}
    \item \textbf{Substation standard} -- Controls breakers, relays in power grid
    \item \textbf{Multiple services} -- MMS (TCP), GOOSE/SV (Layer 2)
    \item \textbf{Time-critical} -- GOOSE requires sub-4ms response
    \item \textbf{No native security} -- Base standard lacks authentication
    \item \textbf{Industroyer target} -- Real attacks have occurred
    \item \textbf{IEC 62351} -- Security extensions, limited adoption
    \item \textbf{Layer 2 isolation} -- Critical for GOOSE/SV protection
\end{itemize}
\end{definitionbox}

% ============================================================================
\section{Further Reading}
% ============================================================================

\subsection*{Standards}

\begin{itemize}
    \item \textbf{IEC 61850} -- Communication networks in substations\\
          \url{https://webstore.iec.ch/publication/6028}
    \item \textbf{IEC 62351} -- Power systems security\\
          \url{https://webstore.iec.ch/publication/6912}
\end{itemize}

\subsection*{Resources}

\begin{itemize}
    \item \textbf{NERC CIP Standards}\\
          \url{https://www.nerc.com/standards/reliability-standards/cip}
    \item \textbf{CISA -- Energy Sector Security}\\
          \url{https://www.cisa.gov/topics/industrial-control-systems}
\end{itemize}

\vfill
\begin{center}
\textit{Part of the OT Security Learning Series}
\end{center}

\end{document}
