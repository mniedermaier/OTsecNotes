% ============================================================================
%  203-ethernet-ip - OT Security Learning Resource
% ============================================================================

\documentclass[11pt,a4paper]{article}
\usepackage{otsec-template}

% Define colors for TikZ
\colorlet{otprimary}{primary}
\colorlet{otaccent}{accent}
\colorlet{otsuccess}{success}
\colorlet{otwarning}{warning}
\colorlet{otdanger}{danger}
\colorlet{otinfo}{info}

\begin{document}

\maketitlepage
    {EtherNet/IP Protocol}
    {Industrial Ethernet protocol for Rockwell and Allen-Bradley systems}
    {OT Security Learning Series}
    {Document 203 \quad|\quad January 2026}
    {Matthias Niedermaier}

\tableofcontents
\newpage

% ============================================================================
\section{Introduction}
% ============================================================================

\begin{infobox}
EtherNet/IP (Ethernet Industrial Protocol) is one of the most widely deployed industrial Ethernet protocols, particularly dominant in North American manufacturing. It is the primary protocol for Rockwell Automation and Allen-Bradley control systems.
\end{infobox}

Key characteristics:
\begin{itemize}
    \item Uses standard Ethernet and TCP/IP infrastructure
    \item Built on Common Industrial Protocol (CIP)
    \item Supports both discrete and process control
    \item Real-time I/O and messaging capabilities
    \item Managed by ODVA (Open DeviceNet Vendors Association)
\end{itemize}

% ============================================================================
\section{Protocol Architecture}
% ============================================================================

\subsection{Protocol Stack}

\begin{figure}[h]
\centering
\begin{tikzpicture}[
    layer/.style={rectangle, draw, thick, minimum height=0.8cm, minimum width=6cm, font=\small},
    arrow/.style={->, thick, >=stealth}
]

\node[layer, fill=otaccent!30] (app) at (0,3.2) {CIP (Common Industrial Protocol)};
\node[layer, fill=otinfo!20] (trans) at (0,2.4) {TCP (explicit) / UDP (implicit)};
\node[layer, fill=otsuccess!20] (net) at (0,1.6) {IP};
\node[layer, fill=otwarning!20] (link) at (0,0.8) {Ethernet (802.3)};
\node[layer, fill=lightgray] (phys) at (0,0) {Physical (copper / fiber)};

\node[font=\tiny, anchor=west] at (3.3,3.2) {Application};
\node[font=\tiny, anchor=west] at (3.3,2.4) {Transport};
\node[font=\tiny, anchor=west] at (3.3,1.6) {Network};
\node[font=\tiny, anchor=west] at (3.3,0.8) {Data Link};
\node[font=\tiny, anchor=west] at (3.3,0) {Physical};

\end{tikzpicture}
\caption{EtherNet/IP Protocol Stack}
\end{figure}

\subsection{Common Industrial Protocol (CIP)}

\begin{definitionbox}{CIP - The Foundation}
CIP is an application layer protocol shared by EtherNet/IP, DeviceNet, and ControlNet. It provides:
\begin{itemize}
    \item Object-oriented device modeling
    \item Standardized device profiles
    \item Common services across networks
    \item Producer/consumer communication model
\end{itemize}
\end{definitionbox}

\subsection{Communication Types}

\begin{figure}[h]
\centering
\begin{tikzpicture}[
    box/.style={rectangle, draw, thick, minimum height=1.2cm, minimum width=2.5cm, rounded corners=3pt, font=\small\bfseries, align=center},
    arrow/.style={->, thick, >=stealth}
]

% Devices
\node[box, fill=otaccent!20] (scanner) at (0,0) {Scanner\\(PLC)};
\node[box, fill=otsuccess!20] (adapter) at (7,0) {Adapter\\(I/O Device)};

% Explicit messaging (TCP)
\draw[arrow, otinfo, dashed] (1.5,0.3) -- node[above, font=\tiny] {Explicit (TCP:44818)} (5.5,0.3);
\draw[arrow, otinfo, dashed] (5.5,0.1) -- node[below, font=\tiny] {Configuration, Diagnostics} (1.5,0.1);

% Implicit messaging (UDP)
\draw[arrow, otdanger] (1.5,-0.3) -- node[above, font=\tiny] {Implicit (UDP:2222)} (5.5,-0.3);
\draw[arrow, otsuccess] (5.5,-0.5) -- node[below, font=\tiny] {Real-time I/O Data} (1.5,-0.5);

\end{tikzpicture}
\caption{EtherNet/IP Communication Types}
\end{figure}

\begin{itemize}
    \item \textbf{Explicit Messaging} -- TCP-based, connection-oriented, for configuration and diagnostics
    \item \textbf{Implicit Messaging} -- UDP-based, for real-time I/O data exchange
    \item \textbf{Unconnected} -- Single request/response transactions
    \item \textbf{Connected} -- Established sessions for ongoing communication
\end{itemize}

% ============================================================================
\section{Network Ports and Services}
% ============================================================================

\begin{table}[h]
\centering
\begin{tabular}{|l|l|l|}
\hline
\textbf{Port} & \textbf{Protocol} & \textbf{Purpose} \\
\hline
TCP/44818 & CIP Explicit & Configuration, diagnostics \\
UDP/44818 & CIP Explicit & Unconnected messages \\
UDP/2222 & CIP Implicit & Real-time I/O data \\
TCP/80 & HTTP & Web interface (if enabled) \\
UDP/67-68 & DHCP/BootP & Address assignment \\
\hline
\end{tabular}
\caption{EtherNet/IP Network Ports}
\end{table}

% ============================================================================
\section{CIP Objects and Services}
% ============================================================================

\subsection{Object Model}

EtherNet/IP devices are modeled as collections of objects:

\begin{itemize}
    \item \textbf{Identity Object (0x01)} -- Device information, serial number, vendor
    \item \textbf{Message Router (0x02)} -- Routes requests to appropriate objects
    \item \textbf{Assembly Object (0x04)} -- Groups I/O data for transmission
    \item \textbf{Connection Manager (0x06)} -- Manages connections
    \item \textbf{TCP/IP Interface (0xF5)} -- Network configuration
    \item \textbf{Ethernet Link (0xF6)} -- Ethernet statistics
\end{itemize}

\subsection{Common Services}

\begin{table}[h]
\centering
\begin{tabular}{|l|l|l|}
\hline
\textbf{Code} & \textbf{Service} & \textbf{Purpose} \\
\hline
0x01 & Get\_Attribute\_All & Read all object attributes \\
0x0E & Get\_Attribute\_Single & Read single attribute \\
0x10 & Set\_Attribute\_Single & Write single attribute \\
0x4C & Forward\_Open & Establish connection \\
0x4E & Forward\_Close & Terminate connection \\
0x52 & Unconnected\_Send & Send without connection \\
\hline
\end{tabular}
\caption{Common CIP Services}
\end{table}

% ============================================================================
\section{Security Vulnerabilities}
% ============================================================================

\begin{dangerbox}
EtherNet/IP was designed for reliability and interoperability, not security. Like most industrial protocols, it lacks built-in authentication and encryption.
\end{dangerbox}

\subsection{Protocol Weaknesses}

\begin{itemize}
    \item \textbf{No authentication} -- Any device can send commands
    \item \textbf{No encryption} -- All traffic is plaintext
    \item \textbf{No integrity protection} -- Messages can be modified
    \item \textbf{Predictable ports} -- Easy to identify on network
    \item \textbf{Information disclosure} -- Identity object reveals device details
\end{itemize}

\subsection{Known Attack Vectors}

\begin{itemize}
    \item \textbf{Device enumeration} -- Query Identity objects to map network
    \item \textbf{Unauthorized configuration} -- Change device settings via Set\_Attribute
    \item \textbf{I/O manipulation} -- Inject or modify implicit messages
    \item \textbf{Connection hijacking} -- Take over established connections
    \item \textbf{Denial of service} -- Flood with connection requests
    \item \textbf{Firmware manipulation} -- Upload malicious firmware
\end{itemize}

% ============================================================================
\section{Security Mitigations}
% ============================================================================

\subsection{Network-Level Controls}

\begin{successbox}
Since EtherNet/IP lacks native security, protection must come from network architecture and external controls.
\end{successbox}

\begin{itemize}
    \item \textbf{Network segmentation} -- Isolate EtherNet/IP traffic in dedicated VLANs
    \item \textbf{Firewall rules} -- Restrict access to ports 44818 and 2222
    \item \textbf{Industrial firewalls} -- Deep packet inspection for CIP
    \item \textbf{Access control lists} -- Limit which devices can communicate
\end{itemize}

\subsection{CIP Security (Recent Addition)}

ODVA has developed CIP Security extensions:
\begin{itemize}
    \item TLS/DTLS for encrypted communications
    \item X.509 certificates for device authentication
    \item Integrity protection for messages
    \item Requires newer devices with CIP Security support
\end{itemize}

\begin{warningbox}
CIP Security adoption is still limited. Most installed devices do not support it, requiring network-based protection.
\end{warningbox}

\subsection{Monitoring and Detection}

\begin{itemize}
    \item Monitor for unauthorized CIP service requests
    \item Alert on Identity object queries (reconnaissance)
    \item Detect configuration changes via Set\_Attribute services
    \item Baseline normal I/O patterns, alert on anomalies
    \item Watch for connections from unauthorized IP addresses
\end{itemize}

% ============================================================================
\section{Common Implementations}
% ============================================================================

\begin{itemize}
    \item \textbf{Rockwell/Allen-Bradley} -- ControlLogix, CompactLogix PLCs
    \item \textbf{Drives} -- PowerFlex variable frequency drives
    \item \textbf{I/O} -- POINT I/O, FLEX I/O modules
    \item \textbf{HMI} -- PanelView terminals
    \item \textbf{Third-party} -- Many vendors support EtherNet/IP
\end{itemize}

% ============================================================================
\section{Summary}
% ============================================================================

\begin{definitionbox}{Key Takeaways}
\begin{itemize}
    \item \textbf{Dominant protocol} -- Primary for Rockwell/Allen-Bradley systems
    \item \textbf{CIP-based} -- Shares application layer with DeviceNet, ControlNet
    \item \textbf{Standard Ethernet} -- Uses TCP/UDP on ports 44818, 2222
    \item \textbf{No native security} -- Authentication and encryption absent
    \item \textbf{CIP Security} -- New extension, limited adoption
    \item \textbf{Network protection} -- Segmentation and firewalls essential
\end{itemize}
\end{definitionbox}

% ============================================================================
\section{Further Reading}
% ============================================================================

\subsection*{Standards}

\begin{itemize}
    \item \textbf{ODVA -- EtherNet/IP Specification}\\
          \url{https://www.odva.org/}
    \item \textbf{IEC 61158} -- Industrial communication networks\\
          \url{https://webstore.iec.ch/}
\end{itemize}

\subsection*{Resources}

\begin{itemize}
    \item \textbf{CISA -- ICS Advisories}\\
          \url{https://www.cisa.gov/topics/industrial-control-systems}
    \item \textbf{Rockwell Automation Security}\\
          \url{https://www.rockwellautomation.com/en-us/capabilities/industrial-cybersecurity.html}
\end{itemize}

\vfill
\begin{center}
\textit{Part of the OT Security Learning Series}
\end{center}

\end{document}
