% ============================================================================
%  OT Network Segmentation - OT Security Learning Resource
% ============================================================================

\documentclass[11pt,a4paper]{article}
\usepackage{otsec-template}

\hypersetup{
    pdftitle={OT Network Segmentation},
    pdfsubject={Implementing Defense in Depth for Industrial Networks},
}

\begin{document}

% ----------------------------------------------------------------------------
%  TITLE PAGE
% ----------------------------------------------------------------------------

\maketitlepage
    {OT Network Segmentation}
    {Implementing Defense in Depth for Industrial Networks}
    {OT Security Learning Series}
    {Document 300 \quad|\quad January 2026}
    {Matthias Niedermaier}

% ----------------------------------------------------------------------------
%  TABLE OF CONTENTS
% ----------------------------------------------------------------------------

\tableofcontents
\newpage

% ----------------------------------------------------------------------------
%  INTRODUCTION
% ----------------------------------------------------------------------------

\section{Introduction}

Network segmentation is the practice of dividing a network into smaller, isolated segments to limit the spread of attacks and control traffic flow. In OT environments, proper segmentation is one of the most effective security controls available.

\begin{infobox}
Network segmentation is consistently identified as a critical missing control in post-incident analyses. Flat networks allow attackers to move laterally from initial compromise directly to critical control systems.
\end{infobox}

\subsection{Why Segmentation Matters}

\begin{itemize}
    \item \textbf{Limits blast radius:} Compromises contained to single segment
    \item \textbf{Controls data flow:} Only authorized traffic between zones
    \item \textbf{Enables monitoring:} Choke points for traffic inspection
    \item \textbf{Supports compliance:} Required by IEC 62443, NIST, NERC CIP
\end{itemize}

% ----------------------------------------------------------------------------
%  SEGMENTATION MODELS
% ----------------------------------------------------------------------------

\section{Segmentation Models}

\subsection{Purdue Model Application}

The Purdue Model provides the conceptual framework; segmentation is the implementation:

\begin{center}
\begin{tikzpicture}[scale=0.8, every node/.style={transform shape}]
    % Zones
    \node[rectangle, rounded corners, minimum width=8cm, minimum height=1cm, fill=zone4!30, draw=zone4] at (0,3) {Enterprise Zone (Levels 4-5)};
    \node[rectangle, rounded corners, minimum width=8cm, minimum height=0.8cm, fill=zone35!30, draw=zone35, line width=1.5pt] at (0,1.8) {Industrial DMZ (Level 3.5)};
    \node[rectangle, rounded corners, minimum width=8cm, minimum height=1cm, fill=zone3!30, draw=zone3] at (0,0.6) {Operations Zone (Level 3)};
    \node[rectangle, rounded corners, minimum width=8cm, minimum height=1cm, fill=zone2!30, draw=zone2] at (0,-0.6) {Control Zone (Levels 1-2)};
    \node[rectangle, rounded corners, minimum width=8cm, minimum height=0.8cm, fill=zone0!30, draw=zone0] at (0,-1.8) {Field Zone (Level 0)};

    % Firewalls
    \draw[red, line width=1pt, dashed] (-4.5,2.4) -- (4.5,2.4);
    \draw[red, line width=1pt, dashed] (-4.5,1.2) -- (4.5,1.2);
    \node[font=\tiny, text=red] at (5,2.4) {FW};
    \node[font=\tiny, text=red] at (5,1.2) {FW};
\end{tikzpicture}
\end{center}

\subsection{Zone and Conduit Model (IEC 62443)}

\begin{conceptbox}{Zones and Conduits}
\begin{itemize}
    \item \textbf{Zone:} Grouping of assets with common security requirements
    \item \textbf{Conduit:} Controlled communication path between zones
    \item \textbf{Security Level:} Target security capability for each zone (SL 1-4)
\end{itemize}
\end{conceptbox}

% ----------------------------------------------------------------------------
%  IMPLEMENTATION
% ----------------------------------------------------------------------------

\section{Implementation Approaches}

\subsection{Physical Segmentation}

\begin{itemize}
    \item \textbf{Separate physical networks:} Different cables, switches, infrastructure
    \item \textbf{Air gaps:} Complete physical isolation (increasingly rare)
    \item \textbf{Data diodes:} Hardware-enforced unidirectional communication
\end{itemize}

\begin{tipbox}
Physical segmentation provides the strongest isolation but is costly and inflexible. Use for the most critical systems where risk justifies the expense.
\end{tipbox}

\subsection{Logical Segmentation}

\begin{itemize}
    \item \textbf{VLANs:} Layer 2 separation on shared infrastructure
    \item \textbf{Firewalls:} Layer 3/4 traffic filtering between segments
    \item \textbf{ACLs:} Router-based access control lists
    \item \textbf{Software-defined networking:} Centralized policy enforcement
\end{itemize}

\subsection{Comparison}

\begin{center}
\small
\rowcolors{2}{lightgray}{white}
\begin{tabular}{p{3cm}p{3cm}p{3cm}p{3.5cm}}
\rowcolor{primary}
\textcolor{white}{\bfseries Method} & \textcolor{white}{\bfseries Security} & \textcolor{white}{\bfseries Cost} & \textcolor{white}{\bfseries Flexibility} \\
\midrule
Air Gap & Highest & High & Very Low \\
Data Diode & Very High & High & Low \\
Physical Separation & High & Medium-High & Low \\
Firewalls + VLANs & Medium-High & Medium & High \\
VLANs Only & Low-Medium & Low & High \\
\end{tabular}
\end{center}

% ----------------------------------------------------------------------------
%  DMZ DESIGN
% ----------------------------------------------------------------------------

\section{Industrial DMZ Design}

\subsection{Purpose}

The Industrial DMZ (Level 3.5) serves as a buffer zone:

\begin{itemize}
    \item \textbf{No direct IT-OT connections:} All traffic proxied through DMZ
    \item \textbf{Services hosted in DMZ:} Jump servers, historians, patch servers
    \item \textbf{Dual-firewall architecture:} Separate firewalls on each side
\end{itemize}

\begin{dangerbox}
Never allow direct connections from enterprise networks (Level 4-5) to control systems (Level 0-2). All communication must pass through the DMZ.
\end{dangerbox}

\subsection{Common DMZ Services}

\begin{center}
\small
\rowcolors{2}{lightgray}{white}
\begin{tabular}{p{4cm}p{9cm}}
\rowcolor{primary}
\textcolor{white}{\bfseries Service} & \textcolor{white}{\bfseries Purpose} \\
\midrule
Jump Server / Bastion Host & Controlled remote access point \\
Historian Mirror & Replicated process data for business use \\
Patch Repository & Staging area for OT software updates \\
AV Update Server & Antivirus signatures for OT systems \\
Remote Access Gateway & VPN termination and access control \\
\end{tabular}
\end{center}

% ----------------------------------------------------------------------------
%  FIREWALL RULES
% ----------------------------------------------------------------------------

\section{Firewall Rule Design}

\subsection{Principles}

\begin{successbox}
\textbf{Firewall rule design principles:}
\begin{enumerate}
    \item \textbf{Default deny:} Block all traffic not explicitly permitted
    \item \textbf{Least privilege:} Allow only required protocols and ports
    \item \textbf{Source/destination specific:} No ``any'' rules
    \item \textbf{Direction matters:} OT initiates outbound; block inbound
    \item \textbf{Log denied traffic:} For detection and troubleshooting
\end{enumerate}
\end{successbox}

\subsection{Example Rule Set}

DMZ to OT Zone (Level 3) rules:

\begin{center}
\small
\rowcolors{2}{lightgray}{white}
\begin{tabular}{p{2cm}p{2.5cm}p{2.5cm}p{1.5cm}p{1.5cm}p{2cm}}
\rowcolor{primary}
\textcolor{white}{\bfseries Action} & \textcolor{white}{\bfseries Source} & \textcolor{white}{\bfseries Dest} & \textcolor{white}{\bfseries Port} & \textcolor{white}{\bfseries Proto} & \textcolor{white}{\bfseries Purpose} \\
\midrule
Allow & Historian-DMZ & Historian-OT & 1433 & TCP & DB replication \\
Allow & Jump-Server & Eng-WS & 3389 & TCP & Remote access \\
Allow & Patch-Server & OT-Servers & 445 & TCP & Updates \\
Deny & Any & Any & Any & Any & Default deny \\
\end{tabular}
\end{center}

% ----------------------------------------------------------------------------
%  COMMON MISTAKES
% ----------------------------------------------------------------------------

\section{Common Mistakes}

\begin{warningbox}
\textbf{Segmentation pitfalls to avoid:}
\begin{itemize}
    \item \textbf{VLANs without firewalls:} VLANs alone don't filter traffic
    \item \textbf{Overly permissive rules:} ``Allow any'' defeats the purpose
    \item \textbf{Dual-homed systems:} Workstations bridging zones
    \item \textbf{Forgotten connections:} Vendor modems, cellular gateways
    \item \textbf{No monitoring:} Segmentation without visibility
    \item \textbf{Static rules:} Never reviewed or updated
\end{itemize}
\end{warningbox}

% ----------------------------------------------------------------------------
%  VALIDATION
% ----------------------------------------------------------------------------

\section{Validation and Maintenance}

\subsection{Testing Segmentation}

\begin{itemize}
    \item \textbf{Network scanning:} Verify only expected hosts reachable
    \item \textbf{Rule review:} Audit firewall rules regularly
    \item \textbf{Traffic analysis:} Confirm actual flows match policy
    \item \textbf{Penetration testing:} Validate segmentation effectiveness
\end{itemize}

\subsection{Ongoing Maintenance}

\begin{itemize}
    \item \textbf{Change management:} Document and approve all rule changes
    \item \textbf{Regular audits:} Quarterly review of firewall rules
    \item \textbf{Asset inventory:} Keep zone assignments current
    \item \textbf{Incident integration:} Update rules based on lessons learned
\end{itemize}

% ----------------------------------------------------------------------------
%  FURTHER READING
% ----------------------------------------------------------------------------

\section{Further Reading}

\subsection*{Standards}
\begin{itemize}
    \item \textbf{IEC 62443-3-2} -- Security Risk Assessment and Zone/Conduit Design\\
          \url{https://www.isa.org/standards-and-publications/isa-standards/isa-iec-62443-series-of-standards}
    \item \textbf{NIST SP 800-82 Rev. 3} -- Guide to OT Security\\
          \url{https://csrc.nist.gov/publications/detail/sp/800-82/rev-3/final}
\end{itemize}

\subsection*{Resources}
\begin{itemize}
    \item \textbf{CISA} -- Network Segmentation Fact Sheet\\
          \url{https://www.cisa.gov/resources-tools/resources}
\end{itemize}

\vfill
\begin{center}
\textcolor{mediumgray}{\rule{0.5\textwidth}{0.5pt}}\\[1em]
\textcolor{mediumgray}{\small Part of the OT Security Learning Series}
\end{center}

\end{document}
