% ============================================================================
%  Industrial DMZ Design - OT Security Learning Resource
% ============================================================================

\documentclass[11pt,a4paper]{article}
\usepackage{otsec-template}

\hypersetup{
    pdftitle={Industrial DMZ Design},
    pdfsubject={Designing Secure DMZ Architecture for OT Networks},
}

\begin{document}

% ----------------------------------------------------------------------------
%  TITLE PAGE
% ----------------------------------------------------------------------------

\maketitlepage
    {Industrial DMZ Design}
    {Secure Buffer Zone Architecture for OT Networks}
    {OT Security Learning Series}
    {Document 302 \quad|\quad January 2026}
    {Matthias Niedermaier}

% ----------------------------------------------------------------------------
%  TABLE OF CONTENTS
% ----------------------------------------------------------------------------

\tableofcontents
\newpage

% ----------------------------------------------------------------------------
%  INTRODUCTION
% ----------------------------------------------------------------------------

\section{Introduction}

An Industrial Demilitarized Zone (IDMZ) is a network segment that acts as a secure buffer between enterprise IT networks and operational technology (OT) networks. It provides controlled connectivity while preventing direct communication between business and control systems.

\begin{infobox}
The IDMZ is one of the most critical security controls in OT environments. A properly designed DMZ can prevent lateral movement from compromised IT systems to critical control networks.
\end{infobox}

\subsection{Why IDMZ Matters}

\begin{itemize}
    \item \textbf{Isolation:} Prevents direct IT-to-OT communication paths
    \item \textbf{Control point:} Centralizes security monitoring and enforcement
    \item \textbf{Data exchange:} Enables safe sharing of operational data
    \item \textbf{Defense-in-depth:} Adds critical security layer
\end{itemize}

% ----------------------------------------------------------------------------
%  DMZ CONCEPTS
% ----------------------------------------------------------------------------

\section{DMZ Concepts}

\subsection{Traditional IT DMZ vs Industrial DMZ}

\begin{center}
\small
\rowcolors{2}{lightgray}{white}
\begin{tabular}{p{3.5cm}p{4.5cm}p{5cm}}
\rowcolor{primary}
\textcolor{white}{\bfseries Aspect} & \textcolor{white}{\bfseries IT DMZ} & \textcolor{white}{\bfseries Industrial DMZ} \\
\midrule
Primary purpose & Expose services to internet & Separate IT from OT \\
Traffic direction & Inbound from untrusted & Bidirectional, controlled \\
Services hosted & Web, email, DNS & Historians, jump servers \\
Trust model & Internet is untrusted & IT is semi-trusted \\
Availability needs & High & Critical (production impact) \\
\end{tabular}
\end{center}

\subsection{Purdue Model Context}

In the Purdue Model, the IDMZ sits between Levels 3 and 4:

\begin{conceptbox}{DMZ in Purdue Architecture}
\begin{itemize}
    \item \textbf{Level 4/5:} Enterprise network (IT)
    \item \textbf{Level 3.5:} Industrial DMZ (IDMZ)
    \item \textbf{Level 3:} Site operations (OT)
    \item \textbf{Levels 0--2:} Control and field devices
\end{itemize}
\end{conceptbox}

% ----------------------------------------------------------------------------
%  ARCHITECTURE PATTERNS
% ----------------------------------------------------------------------------

\section{Architecture Patterns}

\subsection{Single-Firewall DMZ}

The simplest pattern uses one firewall with three zones:

\begin{itemize}
    \item Enterprise zone (IT network)
    \item DMZ zone (shared services)
    \item Control zone (OT network)
\end{itemize}

\begin{warningbox}
Single-firewall designs create a single point of failure. If the firewall is compromised, attackers gain access to all zones. Use only for small, low-risk environments.
\end{warningbox}

\subsection{Dual-Firewall DMZ}

The recommended pattern uses two firewalls:

\begin{successbox}
\textbf{Dual-Firewall Architecture:}
\begin{itemize}
    \item \textbf{Outer firewall:} Faces enterprise network
    \item \textbf{DMZ segment:} Between the firewalls
    \item \textbf{Inner firewall:} Protects OT network
    \item \textbf{Different vendors:} Prevents single vulnerability from compromising both
\end{itemize}
\end{successbox}

\subsection{Data Diode Integration}

For highest security, add hardware-enforced unidirectional gateways:

\begin{itemize}
    \item \textbf{Outbound data diode:} OT to DMZ (historian replication)
    \item \textbf{No inbound path:} Physically prevents attacks from IT
    \item \textbf{Use cases:} Nuclear, critical infrastructure, high-security environments
\end{itemize}

% ----------------------------------------------------------------------------
%  DMZ SERVICES
% ----------------------------------------------------------------------------

\section{DMZ Services}

\subsection{Typical DMZ Components}

\begin{center}
\small
\rowcolors{2}{lightgray}{white}
\begin{tabular}{p{3.5cm}p{9.5cm}}
\rowcolor{primary}
\textcolor{white}{\bfseries Service} & \textcolor{white}{\bfseries Purpose} \\
\midrule
Historian Mirror & Read-only copy of process data for business access \\
Jump Server & Secure remote access point for OT administration \\
Patch Server & Staging area for tested updates before OT deployment \\
AV/Update Server & Antivirus definitions and software updates \\
File Transfer & Secure exchange of files between IT and OT \\
Remote Access Gateway & VPN termination for authorized remote users \\
Log Collector & Aggregation point for OT security logs \\
\end{tabular}
\end{center}

\subsection{Service Placement Rules}

\begin{tipbox}
\textbf{Key principles for DMZ service placement:}
\begin{itemize}
    \item No service should have simultaneous connections to IT and OT
    \item Data should ``break'' in the DMZ (no direct tunnels)
    \item OT systems should initiate connections outbound to DMZ
    \item IT systems connect only to DMZ, never directly to OT
\end{itemize}
\end{tipbox}

% ----------------------------------------------------------------------------
%  TRAFFIC FLOW DESIGN
% ----------------------------------------------------------------------------

\section{Traffic Flow Design}

\subsection{Allowed Traffic Patterns}

\begin{center}
\small
\rowcolors{2}{lightgray}{white}
\begin{tabular}{p{3cm}p{3cm}p{7cm}}
\rowcolor{primary}
\textcolor{white}{\bfseries Source} & \textcolor{white}{\bfseries Destination} & \textcolor{white}{\bfseries Purpose} \\
\midrule
OT $\rightarrow$ DMZ & Historian mirror & Push process data for replication \\
OT $\rightarrow$ DMZ & Log collector & Send security events \\
DMZ $\rightarrow$ OT & Patch server & Pull tested updates (scheduled) \\
IT $\rightarrow$ DMZ & Historian mirror & Query operational data \\
IT $\rightarrow$ DMZ & Jump server & Remote administration access \\
\end{tabular}
\end{center}

\subsection{Prohibited Traffic}

\begin{dangerbox}
\textbf{Never allow these traffic flows:}
\begin{itemize}
    \item IT directly to OT (any protocol)
    \item OT directly to IT (any protocol)
    \item Internet directly to DMZ or OT
    \item Broad ``any-any'' rules through the DMZ
\end{itemize}
\end{dangerbox}

% ----------------------------------------------------------------------------
%  FIREWALL RULES
% ----------------------------------------------------------------------------

\section{Firewall Configuration}

\subsection{Rule Design Principles}

\begin{enumerate}
    \item \textbf{Default deny:} Block all traffic not explicitly permitted
    \item \textbf{Least privilege:} Allow only required ports and protocols
    \item \textbf{Specific sources/destinations:} No ``any'' in rules
    \item \textbf{Application awareness:} Use OT-aware deep packet inspection
    \item \textbf{Logging:} Log all traffic, especially denied connections
\end{enumerate}

\subsection{Example Rule Structure}

\begin{center}
\small
\rowcolors{2}{lightgray}{white}
\begin{tabular}{p{0.5cm}p{2.5cm}p{2.5cm}p{2cm}p{2.5cm}p{1.5cm}}
\rowcolor{primary}
\textcolor{white}{\bfseries \#} & \textcolor{white}{\bfseries Source} & \textcolor{white}{\bfseries Dest} & \textcolor{white}{\bfseries Port} & \textcolor{white}{\bfseries Purpose} & \textcolor{white}{\bfseries Action} \\
\midrule
1 & OT-Historian & DMZ-Mirror & 1433 & DB replication & Allow \\
2 & IT-Analysts & DMZ-Mirror & 443 & Data queries & Allow \\
3 & Admins & DMZ-Jump & 3389 & Remote admin & Allow \\
4 & Any & Any & Any & Default & Deny \\
\end{tabular}
\end{center}

% ----------------------------------------------------------------------------
%  MONITORING AND MANAGEMENT
% ----------------------------------------------------------------------------

\section{Monitoring and Management}

\subsection{Security Monitoring}

\begin{itemize}
    \item \textbf{Traffic analysis:} Monitor for anomalous patterns
    \item \textbf{IDS/IPS:} Deploy at DMZ boundaries
    \item \textbf{Log correlation:} Aggregate logs from all DMZ systems
    \item \textbf{Alerting:} Immediate notification of policy violations
\end{itemize}

\subsection{Change Management}

\begin{infobox}
All DMZ changes should follow strict change control:
\begin{itemize}
    \item Document business justification for rule changes
    \item Test changes in non-production environment
    \item Require approval from both IT and OT stakeholders
    \item Maintain audit trail of all modifications
\end{itemize}
\end{infobox}

% ----------------------------------------------------------------------------
%  FURTHER READING
% ----------------------------------------------------------------------------

\section{Further Reading}

\subsection*{Standards and Guidelines}
\begin{itemize}
    \item \textbf{IEC 62443-3-2} -- Security Risk Assessment and Zone Design\\
          \url{https://www.isa.org/standards-and-publications/isa-standards/isa-iec-62443-series-of-standards}
    \item \textbf{NIST SP 800-82 Rev. 3} -- OT Security Guide\\
          \url{https://csrc.nist.gov/pubs/sp/800/82/r3/final}
\end{itemize}

\subsection*{Resources}
\begin{itemize}
    \item \textbf{CISA} -- Recommended Practices for OT\\
          \url{https://www.cisa.gov/topics/industrial-control-systems}
\end{itemize}

\subsection*{Books}
\begin{itemize}
    \item Pascal Ackerman -- \textit{Industrial Cybersecurity} (Packt)
\end{itemize}

\vfill
\begin{center}
\textcolor{mediumgray}{\rule{0.5\textwidth}{0.5pt}}\\[1em]
\textcolor{mediumgray}{\small Part of the OT Security Learning Series}
\end{center}

\end{document}
