% ============================================================================
%  304-data-diodes - OT Security Learning Resource
% ============================================================================

\documentclass[11pt,a4paper]{article}
\usepackage{otsec-template}
\usepackage{float}

% Define colors for TikZ
\colorlet{otprimary}{primary}
\colorlet{otaccent}{accent}
\colorlet{otsuccess}{success}
\colorlet{otwarning}{warning}
\colorlet{otdanger}{danger}
\colorlet{otinfo}{info}

\begin{document}

\maketitlepage
    {Data Diodes}
    {Unidirectional security gateways for OT environments}
    {OT Security Learning Series}
    {Document 304 \quad|\quad January 2026}
    {Matthias Niedermaier}

\tableofcontents
\newpage

% ============================================================================
\section{Introduction}
% ============================================================================

\begin{infobox}
Data diodes (also called unidirectional security gateways) are hardware-enforced, one-way data transfer devices. They physically prevent any data from flowing back to the source network, providing the highest level of network isolation while still allowing data export.
\end{infobox}

Key benefits:
\begin{itemize}
    \item \textbf{Hardware-enforced} -- Cannot be bypassed by software attacks
    \item \textbf{Air-gap with data flow} -- Isolation without losing visibility
    \item \textbf{No return path} -- Physically impossible to send commands back
    \item \textbf{Regulatory compliance} -- Meets strict isolation requirements
\end{itemize}

\begin{warningbox}
Data diodes are not firewalls. They provide absolute one-way data flow, not filtered bidirectional communication. Choose the right tool for the requirement.
\end{warningbox}

% ============================================================================
\section{How Data Diodes Work}
% ============================================================================

\subsection{Physical Principle}

\begin{definitionbox}{Unidirectional Communication}
Data diodes use hardware that physically supports only one-way transmission. The most common implementation uses fiber optic cables with a transmitter on one side and a receiver on the other---with no transmitter on the receiving side.
\end{definitionbox}

Components:
\begin{itemize}
    \item \textbf{TX (Transmit) appliance} -- Connects to source network, sends data
    \item \textbf{Optical fiber} -- One-way light transmission medium
    \item \textbf{RX (Receive) appliance} -- Receives data, connects to destination
    \item \textbf{No return fiber} -- Physical absence of backward path
\end{itemize}

\begin{figure}[H]
\centering
\begin{tikzpicture}[
    box/.style={rectangle, draw, thick, rounded corners=3pt, minimum width=2.5cm, minimum height=1.5cm, align=center, font=\small},
    fiber/.style={->, very thick, >=stealth},
    nofiber/.style={dashed, thick, color=otdanger}
]

% Source Network
\node[box, fill=otsuccess!20] (source) at (0,0) {OT Network\\(Source)};

% TX Appliance
\node[box, fill=otaccent!20] (tx) at (3.5,0) {TX\\Appliance};

% Fiber (one-way)
\draw[fiber, color=otsuccess, line width=2pt] (5.2,0.2) -- (7.3,0.2);
\node[above, font=\scriptsize] at (6.25,0.3) {Fiber (TX only)};

% No return path
\draw[nofiber] (7.3,-0.2) -- (5.2,-0.2);
\node[below, font=\scriptsize, color=otdanger] at (6.25,-0.3) {No return path};

% RX Appliance
\node[box, fill=otaccent!20] (rx) at (9,0) {RX\\Appliance};

% Destination Network
\node[box, fill=otinfo!20] (dest) at (12.5,0) {IT/DMZ\\(Destination)};

% Connections
\draw[<->, thick] (source) -- (tx);
\draw[<->, thick] (rx) -- (dest);

% Data flow arrow
\draw[->, very thick, color=otprimary, dashed] (0,-1.2) -- (12.5,-1.2);
\node[below, font=\small] at (6.25,-1.2) {Data flows one direction only};

\end{tikzpicture}
\caption{Physical Data Diode Architecture}
\end{figure}

\subsection{Protocol Handling}

Since TCP/IP requires bidirectional communication (ACKs), data diodes must handle protocols specially:

\begin{table}[H]
\centering
\small
\begin{tabular}{|l|l|l|}
\hline
\textbf{Protocol} & \textbf{Native Support} & \textbf{Diode Handling} \\
\hline
UDP & Yes (stateless) & Direct transfer \\
TCP & No (requires ACKs) & Protocol break/proxy \\
File transfer & No (bidirectional) & Store-and-forward \\
Database sync & No & Specialized replication \\
Video streams & Yes (UDP-based) & Direct or buffered \\
\hline
\end{tabular}
\caption{Protocol Handling in Data Diodes}
\end{table}

% ============================================================================
\section{Use Cases}
% ============================================================================

\subsection{Common OT Applications}

\begin{table}[H]
\centering
\small
\begin{tabularx}{\textwidth}{|l|X|}
\hline
\textbf{Use Case} & \textbf{Description} \\
\hline
Historian replication & Export process data to enterprise without inbound risk \\
Log export & Send security logs to corporate SIEM \\
SCADA to business & Share production data with ERP systems \\
Regulatory reporting & Export compliance data to external systems \\
Backup export & Send backups out without allowing restore commands \\
Video surveillance & Export camera feeds from secure areas \\
\hline
\end{tabularx}
\caption{Data Diode Use Cases}
\end{table}

\subsection{Industry Applications}

\begin{itemize}
    \item \textbf{Nuclear} -- Regulatory requirement for safety system isolation
    \item \textbf{Defense} -- Classified network boundaries
    \item \textbf{Energy} -- NERC CIP compliance for critical assets
    \item \textbf{Manufacturing} -- Protecting proprietary processes
    \item \textbf{Water/Utilities} -- Critical infrastructure protection
\end{itemize}

% ============================================================================
\section{Architecture Patterns}
% ============================================================================

\subsection{Basic Data Export}

\begin{successbox}
\textbf{Pattern: OT to IT data export}
\begin{enumerate}
    \item OT historian collects process data
    \item Data diode TX reads from OT historian
    \item One-way transfer to RX appliance
    \item RX writes to replica historian in IT/DMZ
    \item Enterprise applications read from replica
\end{enumerate}
\end{successbox}

\subsection{Deployment Locations}

\begin{figure}[H]
\centering
\begin{tikzpicture}[
    zone/.style={rectangle, draw, thick, rounded corners=5pt, minimum width=3.5cm, minimum height=0.8cm, align=center, font=\small},
    device/.style={rectangle, draw, thick, minimum width=0.8cm, minimum height=0.5cm, font=\tiny}
]

% Zones - stacked vertically
\node[zone, fill=otinfo!15] (enterprise) at (0,6) {Enterprise Zone};
\node[zone, fill=otwarning!15] (dmz) at (0,4) {Industrial DMZ};
\node[zone, fill=otsuccess!15] (control) at (0,2) {Control Zone};
\node[zone, fill=otdanger!15] (safety) at (0,0) {Safety Zone};

% Left side: Firewalls (between Enterprise/DMZ and DMZ/Control)
\node[device, fill=otinfo!30] (fw1) at (-3,5) {FW};
\node[device, fill=otinfo!30] (fw2) at (-3,3) {FW};

% Right side: Data Diodes (between each zone going up)
\node[device, fill=otwarning!40] (d1) at (3,5) {DD};
\node[device, fill=otwarning!40] (d2) at (3,3) {DD};
\node[device, fill=otwarning!40] (d3) at (3,1) {DD};

% Firewall vertical lines with bidirectional arrows
\draw[<->, thick] (fw1.north) -- (fw1.north |- enterprise.south);
\draw[<->, thick] (fw1.south) -- (fw1.south |- dmz.north);
\draw[<->, thick] (fw2.north) -- (fw2.north |- dmz.south);
\draw[<->, thick] (fw2.south) -- (fw2.south |- control.north);

% Data Diode vertical lines (one-way UP)
\draw[thick, color=otsuccess] (d1.south) -- (d1.south |- dmz.north);
\draw[->, thick, color=otsuccess] (d1.north) -- (d1.north |- enterprise.south);
\draw[thick, color=otsuccess] (d2.south) -- (d2.south |- control.north);
\draw[->, thick, color=otsuccess] (d2.north) -- (d2.north |- dmz.south);
\draw[thick, color=otsuccess] (d3.south) -- (d3.south |- safety.north);
\draw[->, thick, color=otsuccess] (d3.north) -- (d3.north |- control.south);

% Labels for data flows
\node[font=\scriptsize, right] at (3.5,5) {Logs};
\node[font=\scriptsize, right] at (3.5,3) {Process Data};
\node[font=\scriptsize, right] at (3.5,1) {Safety Status};

% Legend labels
\node[font=\scriptsize, left] at (-4.5,6) {Bidirectional};
\node[font=\scriptsize, color=otsuccess, right] at (5,6) {One-way (up)};

\end{tikzpicture}
\caption{Data Diode Deployment in Zone Architecture}
\end{figure}

\begin{table}[H]
\centering
\small
\begin{tabularx}{\textwidth}{|l|l|X|}
\hline
\textbf{Boundary} & \textbf{Direction} & \textbf{Data Flow} \\
\hline
Control $\rightarrow$ DMZ & Outbound & Process data, logs, alarms \\
Safety $\rightarrow$ Control & Outbound & Safety status (read-only) \\
OT $\rightarrow$ Enterprise & Outbound & Business intelligence data \\
OT $\rightarrow$ Cloud & Outbound & Analytics, monitoring \\
\hline
\end{tabularx}
\caption{Data Diode Deployment Locations}
\end{table}

% ============================================================================
\section{Data Diode vs Firewall}
% ============================================================================

\begin{table}[H]
\centering
\small
\begin{tabular}{|l|c|c|}
\hline
\textbf{Capability} & \textbf{Firewall} & \textbf{Data Diode} \\
\hline
Bidirectional communication & Yes & No \\
Software configurable & Yes & Limited \\
Can be misconfigured & Yes & Minimal \\
Hackable via software & Possible & No (hardware) \\
Supports TCP natively & Yes & Requires proxy \\
Allows remote access & Yes & No \\
Cost & Lower & Higher \\
Complexity & Moderate & Higher initially \\
\hline
\end{tabular}
\caption{Data Diode vs Firewall Comparison}
\end{table}

\begin{dangerbox}
\textbf{When NOT to use data diodes:}
\begin{itemize}
    \item When bidirectional communication is required
    \item For remote access or engineering connections
    \item When cost is prohibitive for the risk level
    \item If protocol support is not available
\end{itemize}
\end{dangerbox}

% ============================================================================
\section{Implementation Considerations}
% ============================================================================

\subsection{Protocol Support}

Verify the data diode supports your required protocols:
\begin{itemize}
    \item OPC UA/DA (requires special handling)
    \item Modbus (typically UDP mode)
    \item Database replication (vendor-specific)
    \item File transfer (FTP/SFTP proxy)
    \item Syslog (native UDP support)
    \item SNMP traps (outbound only)
\end{itemize}

\subsection{Performance}

\begin{itemize}
    \item \textbf{Throughput} -- Ranges from 10 Mbps to 10 Gbps
    \item \textbf{Latency} -- Protocol conversion adds delay
    \item \textbf{Buffering} -- TX side must buffer if RX cannot keep up
    \item \textbf{Reliability} -- No ACKs means potential data loss
\end{itemize}

\subsection{Reliability Without ACKs}

\begin{warningbox}
Since there's no return path for acknowledgments, data diodes use techniques to ensure reliability:
\begin{itemize}
    \item Forward Error Correction (FEC)
    \item Redundant transmission (send data multiple times)
    \item Application-level verification on receiving side
    \item Buffering and store-and-forward
\end{itemize}
\end{warningbox}

% ============================================================================
\section{Deployment Best Practices}
% ============================================================================

\begin{enumerate}
    \item \textbf{Define data requirements} -- What data must flow and in which direction
    \item \textbf{Verify protocol support} -- Ensure diode handles your protocols
    \item \textbf{Plan for no return path} -- Applications must work without bidirectional communication
    \item \textbf{Size appropriately} -- Consider throughput and latency requirements
    \item \textbf{Test thoroughly} -- Validate all data flows before production
    \item \textbf{Document architecture} -- Clear diagrams showing data flow direction
    \item \textbf{Plan maintenance} -- How to update/maintain systems on both sides
\end{enumerate}

% ============================================================================
\section{Summary}
% ============================================================================

\begin{definitionbox}{Key Takeaways}
\begin{itemize}
    \item \textbf{Hardware-enforced} -- Cannot be bypassed by software attacks
    \item \textbf{One-way only} -- Data export without inbound risk
    \item \textbf{Air-gap alternative} -- Maintain isolation while sharing data
    \item \textbf{Protocol handling} -- Requires proxies for TCP-based protocols
    \item \textbf{High assurance} -- Meets strictest regulatory requirements
    \item \textbf{Not for remote access} -- Cannot replace VPN/jump servers
\end{itemize}
\end{definitionbox}

% ============================================================================
\section{Further Reading}
% ============================================================================

\subsection*{Standards}

\begin{itemize}
    \item \textbf{IEC 62443-3-3} -- System security requirements\\
          \url{https://webstore.iec.ch/publication/7033}
    \item \textbf{NIST SP 800-82} -- Guide to ICS Security\\
          \url{https://csrc.nist.gov/publications/detail/sp/800-82/rev-3/final}
\end{itemize}

\subsection*{Resources}

\begin{itemize}
    \item \textbf{CISA -- Data Diodes}\\
          \url{https://www.cisa.gov/topics/industrial-control-systems}
\end{itemize}

\vfill
\begin{center}
\textit{Part of the OT Security Learning Series}
\end{center}

\end{document}
