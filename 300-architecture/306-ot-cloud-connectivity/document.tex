% ============================================================================
%  306-ot-cloud-connectivity - OT Security Learning Resource
% ============================================================================

\documentclass[11pt,a4paper]{article}
\usepackage{otsec-template}
\usepackage{float}

% Define colors for TikZ
\colorlet{otprimary}{primary}
\colorlet{otaccent}{accent}
\colorlet{otsuccess}{success}
\colorlet{otwarning}{warning}
\colorlet{otdanger}{danger}
\colorlet{otinfo}{info}

\begin{document}

\maketitlepage
    {OT Cloud Connectivity}
    {Securely Connecting Industrial Systems to Cloud Services}
    {OT Security Learning Series}
    {Document 306 \quad|\quad January 2026}
    {Matthias Niedermaier}

\tableofcontents
\newpage

\section{Introduction}

\begin{warningbox}
Cloud connectivity for OT environments should be \textbf{highly avoided}. When business requirements make it unavoidable, connectivity must be strictly limited to \textbf{outbound data flows for monitoring only}. The traditional air-gap model exists for good reason---every cloud connection introduces attack surface.
\end{warningbox}

Before implementing any cloud connectivity, organizations must establish compelling business justification. The default position is \textbf{no cloud connectivity}. When absolutely necessary, connectivity must be architected for read-only monitoring and data egress only.

\begin{dangerbox}
\textbf{Absolute Rule:} Remote control, command execution, or \textbf{any data flow from cloud into OT control systems is unacceptable} and creates critical safety and security risks. There are no exceptions. If a use case requires cloud-to-OT control, the use case must be rejected or redesigned.
\end{dangerbox}

\section{Use Cases}

\subsection{Common Cloud Integration Scenarios}

\begin{table}[H]
\centering
\small
\rowcolors{2}{lightgray}{white}
\begin{tabular}{p{4cm}p{9cm}}
\rowcolor{primary}
\textcolor{white}{\bfseries Use Case} & \textcolor{white}{\bfseries Description} \\
\midrule
Remote Monitoring & View OT data from anywhere without VPN to OT network \\
Predictive Maintenance & ML models analyzing equipment data for failure prediction \\
Cloud Historian & Scalable storage for long-term process data \\
Multi-Site Aggregation & Centralized view across geographically distributed facilities \\
Digital Twin & Cloud-based simulation models fed by real-time OT data \\
Supply Chain Integration & Sharing production data with partners/customers \\
\end{tabular}
\caption{Common OT-to-cloud use cases}
\end{table}

\subsection{Data Types}

Not all OT data carries the same risk when exposed to cloud:

\begin{itemize}
    \item \textbf{Low Risk} -- Aggregated metrics, historical trends, equipment health
    \item \textbf{Medium Risk} -- Real-time process values, production rates
    \item \textbf{High Risk} -- Control commands, setpoints, configuration data
\end{itemize}

\begin{dangerbox}
Control commands, setpoints, and configuration data must \textbf{never} flow from cloud to OT. Cloud connectivity is for \textbf{monitoring and analytics only}. Any architecture that allows cloud-initiated control of OT systems fundamentally violates safe OT security principles.
\end{dangerbox}

\section{Architecture Patterns}

\subsection{Pattern 1: Edge Gateway}

The most common and recommended pattern uses an edge gateway as intermediary:

\begin{figure}[H]
\centering
\begin{tikzpicture}[
    zone/.style={rectangle, draw=otprimary, thick, fill=otprimary!5,
                 rounded corners=5pt, minimum width=3cm, minimum height=1cm,
                 align=center, font=\small},
    device/.style={rectangle, draw=otaccent, thick, fill=otaccent!10,
                   rounded corners=3pt, minimum width=2.5cm, minimum height=0.8cm,
                   align=center, font=\scriptsize},
    arrow/.style={->, thick, >=stealth}
]
    % Cloud
    \node[zone, fill=otinfo!10, draw=otinfo] (cloud) at (0,4) {Cloud Platform};

    % DMZ
    \node[zone, fill=otwarning!10, draw=otwarning] (dmz) at (0,2) {Industrial DMZ};
    \node[device] (edge) at (0,2) {Edge Gateway};

    % OT Network
    \node[zone, fill=otsuccess!10, draw=otsuccess, minimum width=8cm] (ot) at (0,0) {OT Network};
    \node[device] at (-2.5,0) {Historian};
    \node[device] at (2.5,0) {SCADA};

    % Arrows
    \draw[arrow] (0,0.6) -- (0,1.5);
    \draw[arrow] (0,2.5) -- (0,3.5);

    % Labels
    \node[font=\scriptsize, right] at (0.2,1) {Data pull};
    \node[font=\scriptsize, right] at (0.2,3) {Outbound only};
\end{tikzpicture}
\caption{Edge gateway architecture pattern}
\end{figure}

\textbf{Key characteristics:}
\begin{itemize}
    \item Edge gateway initiates all connections (outbound only)
    \item Data aggregation and filtering at the edge
    \item No direct cloud-to-OT connectivity
    \item Gateway placed in DMZ, not OT network
\end{itemize}

\subsection{Pattern 2: Data Diode with Cloud Relay}

For higher security requirements, combine data diodes with cloud connectivity:

\begin{itemize}
    \item Hardware-enforced unidirectional flow from OT
    \item Relay server in DMZ receives diode output
    \item Relay forwards to cloud over encrypted connection
    \item Physically impossible for cloud to send commands to OT
\end{itemize}

\subsection{Pattern 3: Store-and-Forward}

Batch transfer pattern for non-real-time requirements:

\begin{itemize}
    \item Data collected and stored locally
    \item Periodic uploads during maintenance windows
    \item Manual approval option for sensitive data
    \item Air gap maintained except during transfers
\end{itemize}

\section{Security Challenges}

\begin{figure}[H]
\centering
\begin{tikzpicture}[
    challenge/.style={rectangle, draw=otdanger, thick, fill=otdanger!10,
                      rounded corners=3pt, minimum width=6cm, minimum height=0.7cm,
                      align=left, text width=5.8cm, font=\small},
    num/.style={circle, fill=otprimary, text=white, font=\small\bfseries,
                minimum size=0.6cm}
]
    \node[num] at (0,0) {1};
    \node[challenge, anchor=west] at (0.8,0) {Expanded attack surface};
    \node[num] at (0,-1.0) {2};
    \node[challenge, anchor=west] at (0.8,-1.0) {Cloud provider as trusted third party};
    \node[num] at (0,-2.0) {3};
    \node[challenge, anchor=west] at (0.8,-2.0) {Data sovereignty and compliance};
    \node[num] at (0,-3.0) {4};
    \node[challenge, anchor=west] at (0.8,-3.0) {Credential management complexity};
    \node[num] at (0,-4.0) {5};
    \node[challenge, anchor=west] at (0.8,-4.0) {Visibility gaps in encrypted traffic};
    \node[num] at (0,-5.0) {6};
    \node[challenge, anchor=west] at (0.8,-5.0) {Dependency on internet availability};
\end{tikzpicture}
\caption{Key security challenges for OT cloud connectivity}
\end{figure}

\subsection{Attack Surface Expansion}

Cloud connectivity creates new attack vectors:

\begin{itemize}
    \item Compromised cloud credentials provide path to OT data
    \item Vulnerable edge gateways become pivot points
    \item Cloud misconfigurations expose industrial data
    \item Supply chain attacks through cloud services
\end{itemize}

\subsection{Shared Responsibility}

\begin{table}[H]
\centering
\small
\rowcolors{2}{lightgray}{white}
\begin{tabular}{p{4cm}p{4.5cm}p{4.5cm}}
\rowcolor{primary}
\textcolor{white}{\bfseries Layer} & \textcolor{white}{\bfseries Cloud Provider} & \textcolor{white}{\bfseries Customer} \\
\midrule
Physical infrastructure & Provider & -- \\
Network controls & Shared & Shared \\
Identity \& access & Provider (platform) & Customer (config) \\
Data protection & -- & Customer \\
Application security & -- & Customer \\
OT integration & -- & Customer \\
\end{tabular}
\caption{Shared responsibility model for OT cloud}
\end{table}

\section{Security Controls}

\subsection{Network Security}

\begin{itemize}
    \item \textbf{Outbound-Only Connections} -- OT/edge initiates all cloud connections
    \item \textbf{Firewall Rules} -- Whitelist specific cloud endpoints only
    \item \textbf{TLS 1.3} -- Encrypt all data in transit
    \item \textbf{Certificate Pinning} -- Prevent MITM attacks
    \item \textbf{Network Monitoring} -- Alert on unexpected destinations
\end{itemize}

\subsection{Identity and Access}

\begin{itemize}
    \item \textbf{Device Identity} -- Unique certificates per edge gateway
    \item \textbf{Managed Identities} -- Avoid storing credentials on devices
    \item \textbf{Least Privilege} -- Minimal cloud permissions for OT connections
    \item \textbf{MFA} -- For all human access to cloud OT data
    \item \textbf{Regular Rotation} -- Automated credential rotation
\end{itemize}

\subsection{Data Protection}

\begin{table}[H]
\centering
\small
\rowcolors{2}{lightgray}{white}
\begin{tabular}{p{3.5cm}p{9.5cm}}
\rowcolor{primary}
\textcolor{white}{\bfseries Control} & \textcolor{white}{\bfseries Implementation} \\
\midrule
Data Classification & Tag OT data by sensitivity before cloud upload \\
Filtering at Edge & Remove sensitive data before transmission \\
Encryption at Rest & Customer-managed keys for cloud storage \\
Access Logging & Audit all access to OT data in cloud \\
Retention Policies & Automated deletion per compliance requirements \\
\end{tabular}
\caption{Data protection controls}
\end{table}

\begin{successbox}
Apply the principle of data minimization: only send data to the cloud that is actually needed for the use case. Filter, aggregate, and anonymize at the edge wherever possible.
\end{successbox}

\subsection{Edge Gateway Hardening}

The edge gateway is a critical security boundary:

\begin{itemize}
    \item Harden OS, disable unnecessary services
    \item Apply security patches promptly
    \item Enable secure boot and firmware validation
    \item Implement local logging and monitoring
    \item Physical security at installation location
\end{itemize}

\section{Compliance Considerations}

\begin{table}[H]
\centering
\small
\rowcolors{2}{lightgray}{white}
\begin{tabular}{p{3cm}p{10cm}}
\rowcolor{primary}
\textcolor{white}{\bfseries Regulation} & \textcolor{white}{\bfseries Cloud Connectivity Impact} \\
\midrule
NERC CIP & Electronic Security Perimeter extends to cloud connections \\
NIS2 & Supply chain risk includes cloud providers \\
IEC 62443 & Zone/conduit model must account for cloud boundary \\
GDPR & OT data containing personal info subject to data residency \\
\end{tabular}
\caption{Compliance implications of OT cloud connectivity}
\end{table}

\begin{dangerbox}
Some regulations may prohibit or restrict cloud connectivity for certain OT systems. Verify compliance requirements before implementing cloud integration for critical infrastructure.
\end{dangerbox}

\section{Implementation Checklist}

\begin{enumerate}
    \item \textbf{Document business justification}---why is cloud connectivity needed?
    \item Verify no requirement for cloud-to-OT control (reject if required)
    \item Classify data sensitivity---what goes to cloud?
    \item Select architecture pattern appropriate to risk
    \item Implement \textbf{outbound-only} connectivity (block all inbound)
    \item Deploy edge gateway in DMZ (never in OT network)
    \item Configure firewalls with minimal cloud endpoints
    \item Establish device identity and credential management
    \item Enable encryption in transit and at rest
    \item Implement monitoring and alerting
    \item Document in security architecture and compliance evidence
\end{enumerate}

\section{Summary}

\begin{definitionbox}{Key Takeaways}
\begin{itemize}
    \item \textbf{Highly Avoid:} Cloud connectivity to OT should be avoided; default position is no connectivity
    \item \textbf{No Cloud-to-OT Control:} Remote control, commands, or any inbound data flow is unacceptable
    \item \textbf{Monitoring Only:} If connectivity is unavoidable, limit strictly to read-only data egress
    \item \textbf{Outbound Only:} All connections initiated from OT side; block all inbound from cloud
    \item \textbf{Architecture:} Edge gateway in DMZ with strict outbound-only firewall rules
    \item \textbf{Data Minimization:} Filter at edge; send only what is absolutely necessary
\end{itemize}
\end{definitionbox}

\section{Further Reading}

\subsection*{Industry Resources}
\begin{itemize}
    \item \textbf{ICS-CERT Recommended Practices} -- Defense in depth for ICS\\
          \url{https://www.cisa.gov/topics/industrial-control-systems}
    \item \textbf{NIST Cybersecurity for IoT} -- Considerations for connected devices\\
          \url{https://www.nist.gov/programs-projects/nist-cybersecurity-iot-program}
\end{itemize}

\subsection*{Standards}
\begin{itemize}
    \item \textbf{IEC 62443-3-3} -- System security requirements including remote access\\
          \url{https://webstore.iec.ch/publication/7033}
\end{itemize}

\subsection*{Books}
\begin{itemize}
    \item Knapp \& Langill -- \textit{Industrial Network Security} (Syngress)
    \item Ackerman -- \textit{Industrial Cybersecurity} (Packt)
\end{itemize}

\vfill
\begin{center}
\textit{Part of the OT Security Learning Series}
\end{center}

\end{document}
