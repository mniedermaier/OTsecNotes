% ============================================================================
%  Secure Remote Access - OT Security Learning Resource
% ============================================================================

\documentclass[11pt,a4paper]{article}
\usepackage{otsec-template}

\hypersetup{
    pdftitle={Secure Remote Access for OT},
    pdfsubject={Implementing Safe Remote Connectivity to Industrial Systems},
}

\begin{document}

% ----------------------------------------------------------------------------
%  TITLE PAGE
% ----------------------------------------------------------------------------

\maketitlepage
    {Secure Remote Access}
    {Implementing Safe Remote Connectivity to Industrial Systems}
    {OT Security Learning Series}
    {Document 301 \quad|\quad January 2026}
    {Matthias Niedermaier}

% ----------------------------------------------------------------------------
%  TABLE OF CONTENTS
% ----------------------------------------------------------------------------

\tableofcontents
\newpage

% ----------------------------------------------------------------------------
%  INTRODUCTION
% ----------------------------------------------------------------------------

\section{Introduction}

Remote access to OT environments is increasingly necessary for operations, maintenance, and vendor support. However, improperly implemented remote access has been the entry point for numerous high-profile attacks on industrial systems.

\begin{dangerbox}
Remote access was the attack vector in the Ukraine power grid attacks (2015/2016). Attackers used hijacked VPN credentials to access SCADA systems and open circuit breakers remotely.
\end{dangerbox}

\subsection{Business Drivers}

\begin{itemize}
    \item \textbf{Centralized operations:} Monitor multiple sites from control center
    \item \textbf{Vendor support:} Third-party maintenance and troubleshooting
    \item \textbf{Expert access:} Specialists supporting remote facilities
    \item \textbf{Emergency response:} After-hours incident support
\end{itemize}

\begin{warningbox}
Every remote access connection is a potential attack path. The convenience must be balanced against the risk of unauthorized access to critical systems.
\end{warningbox}

% ----------------------------------------------------------------------------
%  ARCHITECTURE
% ----------------------------------------------------------------------------

\section{Architecture Principles}

\subsection{Key Requirements}

\begin{successbox}
\textbf{Secure remote access must provide:}
\begin{enumerate}
    \item \textbf{Strong authentication:} Multi-factor, not just passwords
    \item \textbf{Encrypted transport:} Protect data in transit
    \item \textbf{Access control:} Limit what users can reach
    \item \textbf{Session monitoring:} Record and audit all activity
    \item \textbf{Controlled entry point:} Single, monitored gateway
\end{enumerate}
\end{successbox}

\subsection{DMZ-Based Architecture}

Remote access should terminate in the Industrial DMZ, never directly into OT:

\begin{center}
\begin{tikzpicture}[scale=1.1, every node/.style={transform shape}]
    % External
    \node[rectangle, rounded corners, fill=gray!20, draw=gray, minimum width=2.5cm, minimum height=0.8cm] (ext) at (-4,0) {Remote User};

    % DMZ
    \node[rectangle, rounded corners, fill=zone35!30, draw=zone35, minimum width=3cm, minimum height=2cm] (dmz) at (0,0) {};
    \node[font=\scriptsize, anchor=north] at (0,0.9) {DMZ};
    \node[rectangle, rounded corners, fill=white, draw=zone35, minimum width=2cm, minimum height=0.5cm, font=\tiny] (jump) at (0,0.2) {Jump Server};
    \node[rectangle, rounded corners, fill=white, draw=zone35, minimum width=2cm, minimum height=0.5cm, font=\tiny] (vpn) at (0,-0.5) {VPN Gateway};

    % OT
    \node[rectangle, rounded corners, fill=zone2!30, draw=zone2, minimum width=2.5cm, minimum height=0.8cm] (ot) at (4,0) {OT Systems};

    % Arrows
    \draw[->, thick] (ext) -- (vpn);
    \draw[->, thick] (vpn) -- (jump);
    \draw[->, thick] (jump) -- (ot);

    % Firewalls
    \draw[red, line width=1pt] (-1.8,-1.2) -- (-1.8,1.2);
    \draw[red, line width=1pt] (1.8,-1.2) -- (1.8,1.2);
    \node[font=\tiny, text=red] at (-1.8,-1.5) {FW};
    \node[font=\tiny, text=red] at (1.8,-1.5) {FW};
\end{tikzpicture}
\end{center}

% ----------------------------------------------------------------------------
%  COMPONENTS
% ----------------------------------------------------------------------------

\section{Key Components}

\subsection{Jump Servers (Bastion Hosts)}

\begin{conceptbox}{Jump Server Functions}
\begin{itemize}
    \item \textbf{Single entry point:} All remote sessions pass through
    \item \textbf{Session brokering:} Connects users to authorized systems
    \item \textbf{No direct OT access:} Users cannot bypass the jump server
    \item \textbf{Hardened system:} Minimal services, fully patched
\end{itemize}
\end{conceptbox}

\subsection{VPN Technologies}

\begin{center}
\small
\rowcolors{2}{lightgray}{white}
\begin{tabular}{p{3cm}p{10cm}}
\rowcolor{primary}
\textcolor{white}{\bfseries Type} & \textcolor{white}{\bfseries Considerations} \\
\midrule
IPsec & Strong encryption, complex configuration, site-to-site or client \\
SSL/TLS VPN & Easier deployment, browser-based options, client-based \\
WireGuard & Modern, lightweight, good performance \\
\end{tabular}
\end{center}

\subsection{Multi-Factor Authentication (MFA)}

\begin{itemize}
    \item \textbf{Something you know:} Password, PIN
    \item \textbf{Something you have:} Hardware token, mobile app, smart card
    \item \textbf{Something you are:} Biometrics (less common in OT)
\end{itemize}

\begin{tipbox}
MFA is essential for remote access. Password-only authentication is insufficient---stolen credentials were used in the Ukraine attacks.
\end{tipbox}

% ----------------------------------------------------------------------------
%  ACCESS CONTROL
% ----------------------------------------------------------------------------

\section{Access Control}

\subsection{Principle of Least Privilege}

\begin{itemize}
    \item \textbf{Role-based access:} Define roles with specific permissions
    \item \textbf{System-level restrictions:} Limit which systems each role can access
    \item \textbf{Time-based access:} Restrict access to business hours or maintenance windows
    \item \textbf{Purpose-specific accounts:} Separate accounts for different functions
\end{itemize}

\subsection{Vendor Access Management}

\begin{warningbox}
\textbf{Third-party access requires additional controls:}
\begin{itemize}
    \item Dedicated vendor accounts (no shared credentials)
    \item Access enabled only when needed, disabled by default
    \item Explicit approval workflow for each session
    \item All sessions monitored and recorded
\end{itemize}
\end{warningbox}

% ----------------------------------------------------------------------------
%  MONITORING
% ----------------------------------------------------------------------------

\section{Monitoring and Audit}

\subsection{Session Recording}

\begin{itemize}
    \item \textbf{Video recording:} Capture screen activity for review
    \item \textbf{Keystroke logging:} Record commands entered
    \item \textbf{File transfer logging:} Track all files moved
    \item \textbf{Retention:} Store recordings per compliance requirements
\end{itemize}

\subsection{Real-Time Monitoring}

\begin{itemize}
    \item \textbf{Active session visibility:} See who is connected now
    \item \textbf{Anomaly detection:} Alert on unusual activity patterns
    \item \textbf{Session termination:} Ability to kill suspicious sessions
    \item \textbf{Geographic restrictions:} Block access from unexpected locations
\end{itemize}

\subsection{Audit Trail}

\begin{itemize}
    \item \textbf{Authentication logs:} All login attempts (success and failure)
    \item \textbf{Authorization logs:} Access requests and approvals
    \item \textbf{Activity logs:} Actions performed during sessions
    \item \textbf{Log integrity:} Protect logs from tampering
\end{itemize}

% ----------------------------------------------------------------------------
%  COMMON MISTAKES
% ----------------------------------------------------------------------------

\section{Common Mistakes}

\begin{dangerbox}
\textbf{Remote access security failures:}
\begin{itemize}
    \item \textbf{Direct VPN to OT:} Bypassing DMZ and jump servers
    \item \textbf{Shared credentials:} Multiple users with same account
    \item \textbf{No MFA:} Password-only authentication
    \item \textbf{Always-on access:} Vendor connections left enabled
    \item \textbf{No monitoring:} Sessions not recorded or reviewed
    \item \textbf{Forgotten access:} Former employees/vendors still enabled
\end{itemize}
\end{dangerbox}

% ----------------------------------------------------------------------------
%  IMPLEMENTATION CHECKLIST
% ----------------------------------------------------------------------------

\section{Implementation Checklist}

\begin{successbox}
\textbf{Secure remote access checklist:}
\begin{enumerate}
    \item[$\square$] Remote access terminates in DMZ, not directly in OT
    \item[$\square$] Jump server required for all OT system access
    \item[$\square$] Multi-factor authentication enforced
    \item[$\square$] Role-based access control implemented
    \item[$\square$] All sessions recorded and logged
    \item[$\square$] Vendor access disabled by default, enabled per-request
    \item[$\square$] Regular access reviews conducted
    \item[$\square$] Incident response plan includes remote access scenarios
\end{enumerate}
\end{successbox}

% ----------------------------------------------------------------------------
%  FURTHER READING
% ----------------------------------------------------------------------------

\section{Further Reading}

\subsection*{Standards and Guidelines}
\begin{itemize}
    \item \textbf{NIST SP 800-82 Rev. 3} -- Guide to OT Security (Chapter 5)\\
          \url{https://csrc.nist.gov/pubs/sp/800/82/r3/final}
    \item \textbf{IEC 62443-3-3} -- System Security Requirements\\
          \url{https://www.isa.org/standards-and-publications/isa-standards/isa-iec-62443-series-of-standards}
\end{itemize}

\subsection*{Resources}
\begin{itemize}
    \item \textbf{CISA} -- Remote Access Guidance\\
          \url{https://www.cisa.gov/resources-tools/resources}
    \item \textbf{SANS ICS} -- Securing Remote Access\\
          \url{https://www.sans.org/blog/}
\end{itemize}

\vfill
\begin{center}
\textcolor{mediumgray}{\rule{0.5\textwidth}{0.5pt}}\\[1em]
\textcolor{mediumgray}{\small Part of the OT Security Learning Series}
\end{center}

\end{document}
