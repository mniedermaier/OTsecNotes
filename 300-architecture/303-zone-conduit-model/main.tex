% ============================================================================
%  303-zone-conduit-model - OT Security Learning Resource
% ============================================================================

\documentclass[11pt,a4paper]{article}
\usepackage{otsec-template}
\usepackage{float}

% Define colors for TikZ
\colorlet{otprimary}{primary}
\colorlet{otaccent}{accent}
\colorlet{otsuccess}{success}
\colorlet{otwarning}{warning}
\colorlet{otdanger}{danger}
\colorlet{otinfo}{info}

\begin{document}

\maketitlepage
    {Zone \& Conduit Model}
    {IEC 62443 network segmentation architecture}
    {OT Security Learning Series}
    {Document 303 \quad|\quad January 2026}
    {AI Assistant}

\tableofcontents
\newpage

% ============================================================================
\section{Introduction}
% ============================================================================

\begin{infobox}
The Zone and Conduit Model is the foundational architecture concept in IEC 62443 for segmenting industrial control systems. It provides a structured approach to grouping assets by security requirements and controlling communication between groups.
\end{infobox}

Key principles:
\begin{itemize}
    \item Group assets with similar security requirements into \textbf{zones}
    \item Control all communication between zones through \textbf{conduits}
    \item Assign Security Levels (SL) to each zone based on risk
    \item Apply appropriate controls at conduit boundaries
\end{itemize}

% ============================================================================
\section{Core Concepts}
% ============================================================================

\subsection{Zones}

\begin{definitionbox}{Security Zone}
A grouping of logical or physical assets that share common security requirements. All assets within a zone have the same Security Level (SL) target.
\end{definitionbox}

Zone characteristics:
\begin{itemize}
    \item \textbf{Clear boundary} -- Defined perimeter with controlled entry/exit
    \item \textbf{Common SL} -- All assets share the same target Security Level
    \item \textbf{Trust relationship} -- Assets within a zone trust each other
    \item \textbf{Managed independently} -- Each zone has defined ownership
\end{itemize}

\subsection{Conduits}

\begin{definitionbox}{Conduit}
A logical grouping of communication channels that share common security requirements, connecting two or more zones.
\end{definitionbox}

Conduit characteristics:
\begin{itemize}
    \item \textbf{Controlled path} -- All traffic between zones flows through conduits
    \item \textbf{Security controls} -- Firewalls, data diodes, or other mechanisms
    \item \textbf{Defined protocols} -- Only approved communication is permitted
    \item \textbf{Monitored} -- Traffic is logged and inspected
\end{itemize}

% ============================================================================
\section{Zone Architecture}
% ============================================================================

\begin{figure}[H]
\centering
\begin{tikzpicture}[
    zone/.style={rectangle, draw, thick, rounded corners=5pt, minimum width=3cm, minimum height=2cm, align=center},
    conduit/.style={rectangle, draw, thick, fill=otwarning!30, minimum width=0.8cm, minimum height=0.5cm, font=\tiny},
    arrow/.style={<->, thick, >=stealth}
]

% Enterprise Zone
\node[zone, fill=otinfo!15, minimum width=10cm, minimum height=1.5cm] (enterprise) at (0,5.5) {Enterprise Zone (SL1)\\ERP, Email, Business Systems};

% DMZ
\node[zone, fill=otwarning!15, minimum width=10cm, minimum height=1.5cm] (dmz) at (0,2.5) {Industrial DMZ\\Historians, Jump Servers, Patch Management};

% Manufacturing Zone
\node[zone, fill=otsuccess!15, minimum width=4.5cm, minimum height=2cm] (mfg1) at (-2.5,-1.5) {Manufacturing\\Zone A (SL2)\\HMI, Engineering};

\node[zone, fill=otsuccess!15, minimum width=4.5cm, minimum height=2cm] (mfg2) at (2.5,-1.5) {Manufacturing\\Zone B (SL2)\\HMI, Engineering};

% Control Zones
\node[zone, fill=otdanger!15, minimum width=4.5cm, minimum height=1.5cm] (ctrl1) at (-2.5,-5) {Control Zone A (SL3)\\PLCs, RTUs, I/O};

\node[zone, fill=otdanger!15, minimum width=4.5cm, minimum height=1.5cm] (ctrl2) at (2.5,-5) {Control Zone B (SL3)\\PLCs, RTUs, I/O};

% Safety Zone
\node[zone, fill=otdanger!30, minimum width=2cm, minimum height=1cm] (safety) at (0,-7) {Safety (SL4)\\SIS};

% Conduits - positioned with more space from zones
\node[conduit] (c1) at (0,4) {FW};
\node[conduit] (c2) at (-2.5,0.5) {FW};
\node[conduit] (c3) at (2.5,0.5) {FW};
\node[conduit] (c4) at (-2.5,-3.25) {FW};
\node[conduit] (c5) at (2.5,-3.25) {FW};

% Arrows
\draw[arrow, otprimary] (enterprise) -- (c1);
\draw[arrow, otprimary] (c1) -- (dmz);
\draw[arrow, otprimary] (dmz.south -| c2) -- (c2);
\draw[arrow, otprimary] (c2) -- (mfg1);
\draw[arrow, otprimary] (dmz.south -| c3) -- (c3);
\draw[arrow, otprimary] (c3) -- (mfg2);
\draw[arrow, otprimary] (mfg1) -- (c4);
\draw[arrow, otprimary] (c4) -- (ctrl1);
\draw[arrow, otprimary] (mfg2) -- (c5);
\draw[arrow, otprimary] (c5) -- (ctrl2);

\end{tikzpicture}
\caption{Zone and Conduit Architecture Example}
\end{figure}

% ============================================================================
\section{Zone Types}
% ============================================================================

\subsection{Common Zone Classifications}

\begin{table}[H]
\centering
\small
\begin{tabularx}{\textwidth}{|l|l|X|}
\hline
\textbf{Zone Type} & \textbf{Typical SL} & \textbf{Contents} \\
\hline
Enterprise & SL1 & Corporate IT, ERP, email, internet access \\
Industrial DMZ & SL2 & Historians, patch servers, jump hosts \\
Manufacturing & SL2 & HMI, engineering workstations, SCADA \\
Control & SL2--SL3 & PLCs, RTUs, DCS controllers \\
Safety & SL3--SL4 & SIS, emergency shutdown systems \\
Field & SL1--SL2 & Sensors, actuators, field devices \\
\hline
\end{tabularx}
\caption{Common Zone Types and Security Levels}
\end{table}

\subsection{Zone Sizing Considerations}

\begin{itemize}
    \item \textbf{Too large} -- Difficult to manage, broad attack surface
    \item \textbf{Too small} -- Excessive complexity, operational burden
    \item \textbf{Right size} -- Based on function, criticality, and connectivity
\end{itemize}

\begin{warningbox}
Avoid creating zones purely based on vendor or equipment type. Zone boundaries should reflect security requirements and operational dependencies.
\end{warningbox}

% ============================================================================
\section{Conduit Design}
% ============================================================================

\subsection{Conduit Security Controls}

\begin{table}[H]
\centering
\small
\begin{tabularx}{\textwidth}{|l|X|}
\hline
\textbf{Control Type} & \textbf{Description} \\
\hline
Firewall & Stateful packet filtering, application awareness \\
Data Diode & Hardware-enforced unidirectional flow \\
Jump Server & Controlled access point for administration \\
Protocol Proxy & Protocol break, inspection, and translation \\
VPN Gateway & Encrypted tunnel for remote connectivity \\
IDS/IPS & Traffic inspection and threat detection \\
\hline
\end{tabularx}
\caption{Conduit Security Control Options}
\end{table}

\subsection{Conduit Rules}

\begin{successbox}
Every conduit should have explicitly defined rules specifying:
\begin{itemize}
    \item Permitted protocols and ports
    \item Allowed source and destination addresses
    \item Direction of data flow
    \item Authentication requirements
\end{itemize}
\end{successbox}

\subsection{Conduit SL Requirements}

The conduit must meet the \textbf{higher} SL of the two zones it connects:

\begin{table}[H]
\centering
\begin{tabular}{|c|c|c|}
\hline
\textbf{Zone A SL} & \textbf{Zone B SL} & \textbf{Conduit SL} \\
\hline
SL1 & SL2 & SL2 \\
SL2 & SL2 & SL2 \\
SL2 & SL3 & SL3 \\
SL3 & SL4 & SL4 \\
\hline
\end{tabular}
\caption{Conduit Security Level Determination}
\end{table}

% ============================================================================
\section{Implementation Steps}
% ============================================================================

\begin{enumerate}
    \item \textbf{Asset Inventory} -- Identify all assets in the IACS
    \item \textbf{Group by Function} -- Cluster assets by operational role
    \item \textbf{Identify Dependencies} -- Map communication requirements
    \item \textbf{Define Zone Boundaries} -- Draw logical/physical perimeters
    \item \textbf{Assign Security Levels} -- Based on risk assessment
    \item \textbf{Design Conduits} -- Define allowed flows between zones
    \item \textbf{Implement Controls} -- Deploy firewalls, monitoring
    \item \textbf{Validate} -- Test that zones are properly isolated
    \item \textbf{Document} -- Maintain zone/conduit diagrams and rules
\end{enumerate}

% ============================================================================
\section{Common Mistakes}
% ============================================================================

\begin{dangerbox}
\textbf{Anti-patterns to avoid:}
\begin{itemize}
    \item Flat networks with no segmentation
    \item Single firewall between IT and all of OT
    \item Zones based on physical location only
    \item Conduits with ``allow all'' rules
    \item Undocumented or forgotten connections
    \item Mixing different SL requirements in one zone
\end{itemize}
\end{dangerbox}

% ============================================================================
\section{Summary}
% ============================================================================

\begin{definitionbox}{Key Takeaways}
\begin{itemize}
    \item \textbf{Zones} -- Group assets with same security requirements
    \item \textbf{Conduits} -- Controlled paths between zones
    \item \textbf{Security Levels} -- Drive control requirements per zone
    \item \textbf{Defense in depth} -- Multiple zone layers increase security
    \item \textbf{Documentation} -- Maintain accurate zone/conduit diagrams
    \item \textbf{IEC 62443-3-2} -- Standard for zone/conduit design
\end{itemize}
\end{definitionbox}

% ============================================================================
\section{Further Reading}
% ============================================================================

\subsection*{Standards}

\begin{itemize}
    \item \textbf{IEC 62443-3-2} -- Security risk assessment and zone design\\
          \url{https://webstore.iec.ch/publication/7032}
    \item \textbf{IEC 62443-3-3} -- System security requirements\\
          \url{https://webstore.iec.ch/publication/7033}
\end{itemize}

\subsection*{Resources}

\begin{itemize}
    \item \textbf{CISA -- Network Segmentation}\\
          \url{https://www.cisa.gov/topics/industrial-control-systems}
\end{itemize}

\vfill
\begin{center}
\textit{Part of the OT Security Learning Series}
\end{center}

\end{document}
